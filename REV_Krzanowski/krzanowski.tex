\begin{newrevengenv}{Roman Krzanowski}
	{Science and its social grounding}
	{Science and its social grounding}
	{Science and its social grounding}
	{Pontifical University of John Paul II in Krakow}
	{Stuart Ritchie, \textit{Science Fictions: Exposing Frauds, Bias, Negligence and Hype in Science}. Metropolitan Books Henry Holt and Company, New York 2020. pp.356.}





\lettrine[loversize=0.13,lines=2,lraise=-0.01,nindent=0em,findent=0.2pt]%
{W}{}e are constantly puzzled by the question of ``What is science?'' This comes despite the fact that most of our greatest achievements and failures are directly or indirectly related to science. The chaos of the COVID-19 pandemic stirred up various anti-scientific sentiments, intensifying the confusion and deep apprehension about science
%\label{ref:RNDRARpRqkDz6}(see e.g. Blaylock, 2022).
\parencite[see e.g.][]{blaylock_covid_2022}. %
 Nevertheless, whatever notions laypeople, journalists, bloggers, politicians, actors, and other influencers have been entertaining about science, they have not stopped or hindered scientific work, or at least so it seems. Science mostly just continued without paying too much attention to what people thought about it, like it was somewhat oblivious to social moods; the Aristotelian claim \textit{All men by nature desire to know} still holds firm. Now Stuart Richie comes along with a~claim: ``Science is a~social construct.'' This is his opening statement in the book \textit{Science Fictions: Exposing Frauds, Bias, Negligence and Hype in Science} 
%\label{ref:RNDjnPiE9LBse}(2020),
\parencite*[][]{ritchie_science_2020}, %
 but who is Stuart Richie, and what is he actually saying?

Stuart Richie lectures in Social Genetic \& Developmental Psychiatry at King's College London, and he has authored several publications for experimental studies in cognitive research, brain studies, intelligence, publication bias, meta studies, and secondary data analysis, among other things.\footnote{See Stuart Richie on Google Scholar: https://scholar.google.com/citations?user=9TsCy3IAAAAJ} Thus, he is an insider and therefore seems qualified to comment on the nature of scientific enterprise.

The usual claim that science is a~social phenomenon
%\label{ref:RNDBar4yDkHsT}(see e.g. Bloor, 1976; 1991; MacKenzie, 1981; Longino, 2002)
\parencites*[see e.g.][]{bloor_knowledge_1976}[][]{bloor_knowledge_1991}[][]{mackenzie_notes_1981}[][]{longino_social_2002} %
 is that there is no direct link between science and nature. For Richie, science remains a~quest search for objective truth. Its social dimension means that science as a~profession is a~social construct made by society and embedded into our social structures, so it is subject to social ills, just like any human activity is. Its social aspect comes into play not in the scientific methodology \textit{per se} but rather in everything around science; science—as a~complex of cultural systems comprising academia, papers, grants, tenures, hierarchies, competitions, and so on—is a~social enterprise.

Richie begins his story by asking how does science make itself known? This generally happens through publications, with top scientific journals like \textit{Science, Nature, The Lancet}, and such like carrying particular weight. These journals are supposed to present significant, top-quality research that is thoroughly vetted for coherence, verifiable results (which usually take the form of breakthrough discoveries or illuminating insights), and a~strict adherence to accepted methodologies and the explicit and implicit ethical principles of science. It therefore comes as a~surprise that many of the studies published in top journals, when put under close scrutiny, turn out to be fraudulent or at the minimum, poorly designed with misleading conclusions. This was the fate of Zimbardo's famous, or more accurately infamous, Stanford Prison Experiment, and the fate of the study of Stanley Milgram that was supposed to demonstrate that people will blindly follow orders that go against their moral principles. The problem is that these are not isolated cases.

A~study from 2019 reported that out of 3,000 papers published in the top medical journals, 396 studies that advised a~significant change in medical practices were completely unfounded. The findings of the studies were never replicated (i.e., confirmed), even though the golden rule is that scientific results should be replicable to be deemed reliable. Furthermore, the flawed claims these studies made were not trivial matters. For example, they suggested changes in practice for childbirth, allergy treatment, and the treatment of heart attacks and strokes, so these changes were at worst harmful to patients or ineffective at best. A~review published by the Cochrane Collaboration,\footnote{Cochrane is an independent, diverse, global organization that collaborates to produce trusted synthesized evidence, making it accessible to all: \url{https://www.cochrane.org}.} an organization dedicated to the quality of medical research, reassessed the published work and concluded that 45\% of medical studies lacked sufficient evidence to support their conclusions. In other words, the findings of these studies could not have been replicated (i.e., verified).

The evil ways in which science gets distorted do not register as capital vices, but they still comprise an impressive collection. Among them, Richie lists fraud, bias, negligence, and hype. Fraud in science comes in many forms. It may be as simple as falsifying results for breakthrough surgical procedures, as was the case with Paolo Macchiarini of The Karolinska Institute, whose results were published in \textit{The Lancet} as being based on solid evidence. As one may guess, following investigation into his work, his supposedly life-saving procedures turned out to be ineffective if not outright harmful. Fraud also occurred in the case of Woo-Suk Hwang, who published a~paper in \textit{Science} about successfully cloning human stem cells. Unfortunately for him, further investigation revealed that the images used in his papers to prove his claim were doctored.

Next on the list of evils is bias. In publications, bias refers to papers focusing on reporting positive, significant results, because these are generally the ones that get published. Papers reporting null or negative results are usually desk-rejected, although things are slowly changing here, as Richie points out. After all, who is interested in a~seemingly unsuccessful study? Successful studies, meanwhile, report significant experimental results, and they attract citations and readership, so ratings go up. Who wants to read about null results from an unsuccessful study? Actually, some may. Studies with negative or null results can still provide valuable information, such as by challenging the results of a~successful study. Anyway, what is a~``significant result''? It's generally regarded as one that confirms an experimental hypothesis with a~p-value of less than 0.05.\footnote{A~statistical measurement used to validate a~hypothesis against observed data.} This convention for evaluating the significance of experimental results was conceived by Ronald Fisher in the 1920s, but it has no deeper meaning. Indeed, it is arbitrary. The required p-value is arbitrary, yet it is a~deciding factor in whether a~paper gets accepted for publication or not. (Note that in high-energy physics a~significant result must pass the five-sigma test, which is also a~kind of p-test but much more stringent in that the required p-value is 3x10\textsuperscript{{}-7}, which would be a~killer outside physics.)

Thus, an arbitrary number has become an arbiter of an experimental science. Like any statistic, however, a~p-value can be manipulated in many ways (i.e., p-hacking) by changing data sets, altering experimental hypotheses, and manipulating experimental data. A~hacked p-value means hacked research, meaning that science has been hacked. If you need successful results from a~study, and the p-value is not cooperating with you, simply change the statistics you use. You need to be expert at massaging data, which is what p-hacking requires, so only an expert can unravel these practices. Where does this leave us, the public, and science, though? How can the public trust experts when the experts cannot trust each other?

Everything has its causes, so fraud, bias, negligence, and hype in science must as well. The main cause is us, and the main victims are us, the public, scientists, trust in science, and science itself, or at least our idea of it. But why does this happen? In the scientific world, like everywhere else, everyone wants to get ahead. There is nothing wrong with this, and it is what makes us human, at least for many of us in some way. It is how we go about doing it that may cause problems. Richie says that the system of incentives in the scientific world ``incentivisises scientists not to practice science but to meet its own perverse demand.'' Simply put, they are perverse incentives. The scientific establishment imposes a~simple measure of progress, namely the number of publications, which translates into the now infamous prime directive: publish or perish. It matters little what you publish as long as it is published in prestigious journals and in great quantity. Of course, there is a~tacit assumption that these publications are the best stories produced by science, but Richie's book showed that this is often not the case, maybe too often. But what does ``too often'' mean in this case, you may well wonder?

Incentives are originally provided to stimulate and support good research, and there is nothing wrong with that, but in reality, these incentives often work contrary to expectations. There are many ways in which incentives can lead to perverse outcomes. Grant-dispensing institutions, profit-oriented private companies, big and small pharmaceutical companies, and the chemicals industry tend to support research that benefits their bottom line. Studies showing anything contrary to this—such as harmful side effects, placebo effects, or poor efficacy—are rejected or filed away in a~basement somewhere. Studies that align with founding agencies' objectives receive support (i.e., incentives),\footnote{One can also read about the famous Eddington experiment for ``confirming'' Einstein's general relativity theory to see how the human element enters science
%\label{ref:RND5Y9ea8ol2S}(Strevens, 2022).
\parencite[][]{strevens_knowledge_2022}. %
 } while other studies get nothing. It is therefore unsurprising when studies show the expected results 
%\label{ref:RNDcp4ZPj6rFm}(see also Strevens, 2022).
\parencite[see also][]{strevens_knowledge_2022}. %
 If you publish more, you get awarded more funding, so you do whatever it takes to publish more. The tenure-track system also favors those who publish frequently in high-flying journals, rightly so in principle but without considering the essence of what they publish. Thus, you publish as much as you can, whatever and wherever you can. To aid this effort, predatory journals will publish anything for those who can pay the publication fee. This in turn encourages the so-called salami slicing of research. The term ``salami slicing'' relates to a~practice were a~large study whose findings should ideally be published as a~single paper is instead divided into a~series of shorter papers (i.e., salami slicing) covering differ areas of the main study. When the number of publications counts, this is the way to increase it, but if what counts are reporting significant findings, this is a~counterproductive practice. Measuring of academic productivity based solely on a~range of publication metrics 
%\label{ref:RNDCtxunOlihu}(see e.g. Carpenter, Cone and Sarli, 2014)
\parencite[see e.g.][]{carpenter_using_2014} %
 loses the sight of what science is about and how science is actually done 
%\label{ref:RNDlJ7hvJAQNb}(Polak, 2011).
\parencite[][]{}.%


Can we do anything about it? It seems, according to Richie, that the chances of healing are, let's say, slim. We are what we are, and our science practices reflect this. (We need to be careful to distinguish scientific practices from the methodology of science.) The whole system of science and its social construct—which is dependent on, and embedded in, the economic and political structures—does not bode well for the future. Science in this respect is an expression of our modern society with its drive for success and recognition, as well as its competitive culture of winner takes all. It does not mean that science will inevitably fail, though, because we will still have amazing discoveries and make progress in developing mind-boggling inventions. A~sea of dirt will also be there, however, with tsunamis of hype swaying the public and feeding into all kinds of anti-whatever camps of less sensible but highly vocal groups. Is it therefore a~surprise that ``people do not believe in science,'' as we have often heard during the pandemic? If people dismiss the Apollo moon landings as fake, why would they not question the COVID-19 vaccines? Are we surprised? However, do not blame science or the mythical ``them,'' because the fault lies with us, the ones doing the science.

Of course, there is much more detail to every story that Richie narrates. There are more examples of nuanced transgressions committed in scientific publications and research, fraud, and personal bias, both out of stupidity and arrogance—the list is long. Indeed, it reads like a~crime novel, and it makes little difference that it is a~white-collar crime. Again, you have to read about these in the book, but they certainly make for interesting and rewarding study. Now, does Richie's book have any import for science outside of experimental research? After all, Richie focuses mostly on clinical, social, psychological studies? It seems that lessons in Richie's book are for everyone ``doing'' some research, whatever it may be. The prime directive is real, and it works in every corner of academia, so the game of publications and grants is what today's science is all about. Studies that do not rely on experiments unfortunately lack the truth serum of reproducibility, which is why it is difficult to spot the bogus research and ideas within them. Only time will ultimately reveal them.

So, what lessons does Richie's story have for all of us? First, science as practiced profession, is a~survival game, and it must be played once you are in it. Some are good at science, some are good at games, and some are good at gaming science, but who wins? It's not necessarily those who are good at science. But in the long run, science seems to possess some self-cleansing properties. Second, do not read newspaper reports about miraculous new cures, foods, or whatever. These ‘news' are certainly not science-based facts. So do not blame science for them. Thirds, in the long run, maybe too long for some of us, the truth, despite many doubting its value or existence, always eventually prevails in science, philosophy, and humanity, just maybe not during your lifetime.

So, who should read Stuart Richie's book? Anyone thinking about a~career in science should certainly consider it. Indeed, young, idealistic scholars should realize they are not entering an idealistic field purely dedicated to the pursuit of truth and knowledge but rather a~fierce neoliberal game of survival where anything is permitted as long as you do not get caught. In this world, you better learn the tricks of the trade or you will soon be gone unless you are independently wealthy. This is a~perverse point of view, of course—it is infinitely preferable to invest in honest studies because they age much better, even if they are less spectacular.

Should anyone else read Stuart Richie's book? The universities and their governing bodies certainly should before they turn the screw to tighten the publishing requirements of their staff. Indeed, they should think twice about it and maybe start promoting research quality rather than quantity. Unfortunately, there is little hope for this. Quality research requires time, money, patience and a~tolerance of failure, something that managers have very little time for. The managers responsible for the rules usually have nothing to do with science and generally only care about the stats, because this is what they understand, if anything, and this is how they win the game that they are playing.

And what about Richie's original claim that science is a~social construct? We believe that Richie proves his case. Science is embedded in societal structures, so it inherits all of their characteristics, both good and bad. This only applies to the operational aspects of science, however, and science in itself as the search for truth is unaffected in principle by this social dependency. The social dependencies hinder the reception of science in society, but the idea of science does not. This is why we still believe in it, at least some of us.

The optimistic message to take away is that science is still science, and it represents the greatest intellectual achievement of the human race so far, at least when done right. Science in its essence is not a~playground of changing moods and social trends, as some would like to have us believe. In the absence of political and economic pressures, if such an ideal world could exist, or with sufficient time, science will refocus itself on its original objectives. This reflects what science was meant to be, and always will be: the search for truth
%\label{ref:RNDTGHdziDXIS}(e.g. Grayling, 2022).
\parencite[e.g.][]{grayling_frontiers_2022}. %
 The problem remains how to reconcile the objectives of science (as a~search for truth) with external irrational influences (e.g., social, political, and cultural); the topic discussed in details by Liana 
%\label{ref:RNDdWF1gYgQBN}(2019; 2020).
\parencites*[][]{liana_nauka_2019}[][]{liana_jozefa_2020}.%




%---------------



\vspace{15mm}%
{\subsubsectit{\hfill Abstract}}\\
{Praca nie ma abastraktu i słów kulczowych, proszę o dosłanie.}\par%
\vspace{2mm}%
{\subsubsectit{\hfill Keywords}}\\
{Praca, nie, ma, abastraktu i słów kulczowych, proszę o dosłanie.}%



\end{newrevengenv}
