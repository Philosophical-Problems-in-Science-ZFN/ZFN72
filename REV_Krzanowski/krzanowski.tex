\begin{newrevengenv}{Roman Krzanowski}
	{Novacene: The coming age of hyperintelligence---James Lovelock's vision of posthumanism}
	{Novacene: The coming age of hyperintelligence\ldots}
	{Novacene: The coming age of hyperintelligence---James Lovelock's vision of posthumanism}
	{Pontifical University of John Paul II in Krakow}
	{James Lovelock, \textit{Novacene: The Coming Age of Hyperintelligence}, Penguin Books, London 2020 [2019], pp.140.}
	


\lettrine[loversize=0.13,lines=2,lraise=-0.03,nindent=0em,findent=0.2pt]%
{J}{}ames Lovelock
%\label{ref:RNDXpOGSrjkfz}(2021),
\parencite*[][]{lovelock_james_2021}, %
 who is famous for the Gaia hypothesis,\footnote{\textit{Gaia: A~New Look at Life on Earth} 
%\label{ref:RNDUguvO8y4So}(1979, 3rd ed. 2000)
\parencites*[][]{lovelock_gaia:_1979}[][3rd~ed.]{lovelock_gaia_2000} %
 and the series of subsequent books developing the topic of Gaia. The microbiologist Lynn Margulist is a~co-creator of the Gaia hypothesis.} has written a~new book called \textit{Novacene: The Coming Age of Hyperintelligence} 
%\label{ref:RNDwoM2Y82YT7}(2019).
\parencite*[][]{lovelock_novacene_2019}. %
 It is an extended argument about an impending new epoch on Earth called Novacene in which biological life as we know it will evolve into lifeforms based on cyber technology (i.e., cyborgs) built from non-biological materials. Novacene may be seen as a~development of the ideas presented in Lovelock's earlier book \textit{A~Rough Guide to The Future} 
%\label{ref:RNDCfMvJB9ja4}(2014).
\parencite*[][]{lovelock_rough_2014}. %
 In \textit{Novacene}, the Earth will be populated by cyborgs, which are self-replicating and self-improving mechanical systems that will eventually dominate and rule the Earth. These cyborgs will possess intelligence and knowledge beyond our understanding. To quote Lovelock, ``cyborgs… will design and build themselves from the artificial intelligence systems we have already [sic] constructed. These will soon become thousands then millions of times more intelligent than us'' (p. 29). Despite the vast difference in intellectual power, the relationship between cyborgs and humans will be peaceful, at least in the early stages of Novacene, as Gaia (i.e., the Earth) will need to maintain biological life (including us) to maintain the thermal balance of its (her?) ecosphere. In other words, cyborgs and we will need each other, but only at least at the beginning of Novacene 
%\label{ref:RNDc7Et3Dl7r7}(Lovelock, 2019, p.30).
\parencite[][p.30]{lovelock_novacene_2019}. %
 Eventually, however, we will be ``no more masters of our creations than our much-loved pet is in charge of us'' 
%\label{ref:RND5qE976QP8j}(Lovelock, 2019, p.119).
\parencite[][p.119]{lovelock_novacene_2019}. %
 All this posthumanism or transhumanism, or choose your own term, is a~product of natural evolution (i.e., Darwinian-like) of Gaia, similar to how we claim that evolution created us out of a~primordial soup of simple elements and compounds with some help from natural electricity and the Sun. The idea behind Novacene is a~bit mind-numbing at both the first and subsequent readings, so it needs a~book to explain it. (As a~reminder, in the first approximation of Gaia in Lovelock's philosophy, the Earth is conceived as a~living organism directing its own evolution 
%\label{ref:RNDzNm6wJKkS9}(e.g. Lovelock, 2019, pp.12–17; 70).
\parencites[e.g.][pp.12–17; 70]{lovelock_novacene_2019}.%


James Lovelock is known for his far reaching ideas, breakthrough inventions
%\label{ref:RNDVGd0nwKfLH}(e.g. the electron capture detector (ECD)
\parencite[e.g., the electron capture detector (ECD)][p.38]{lovelock_novacene_2019} %
% Lovelock, 2019, p.38),
and research
%\label{ref:RNDFCWYk6rnjK}(e.g. chlorofluorocarbons or CFC presence Lovelock, 2019, p.38).
\parencite[e.g., chlorofluorocarbons or CFC presence][p.38]{wylie_mindfck_2019}.%

Thus, the prophetic visions of a~new human–cyber future may not be a~complete surprise coming from him, especially when we take the book on Novacene as the conclusion of Lovelock's series of books about Gaia. One may ask, however, how good is Lovelock's argument for the coming of Novacene, knowing that any predictions of a~radical new future (and Novacene really is quite radical) are always rather dubious? This is maybe especially true when they come from creative individuals---think about Ernest Rutherford's ``moonshine'' or Thomas Watson's computers
%\label{ref:RNDiZZflR4yDY}(see Strohmeyer, 2008; or Singer and Franco, 2021).
\parencites[see][]{strohmeyer_7_2008}[or][]{singer_hindsight_2021}. %
 So, is the idea of Novacene elaborated in Lovelock's book a~form of moonshine (like Rutherford's), science-fiction, conjecture, or a~well-argued prophecy based on scientific facts?

Okay, what is Novacene? Novacene is a~new epoch of life on Earth, one that follows the current epoch of Anthropocene, during which human activity has had a~global impact
%\label{ref:RNDbwHQDUk2MD}(2019, pp.33–44).
\parencite*[][pp.33–44]{lovelock_novacene_2019}. %
 In Novacene, cybernetic life will dominate the Earth. These organisms \textit{will} be, as they do not currently exist, living systems constructed out of electronic rather than organic components. These ``systems'' will evolve from their initial state, just as biological life evolved from primary elements billions of years ago 
%\label{ref:RNDYITDZE5Nuo}(2019, p.27).
\parencite*[][p.27]{lovelock_novacene_2019}. %
 Exactly how this evolution will occur is not explained. By ``living system,'' Lovelock refers to a~system that is self-reproducing, self-improving, and autonomous (i.e., self-controlling, self-directing, etc.).

The current epoch Anthropocene is dominated by human activities, particularly the harnessing of the Sun's energy trapped in fossil fuels
%\label{ref:RNDdQzPd8xyG5}(2019, p.33).
\parencite*[][p.33]{lovelock_novacene_2019}. %
 We are also at the threshold of harnessing pure information, the very stuff the cosmos is made of. In fact, fossil fuels did not just trap energy from the Sun but also information carried in light rays. Thus, information processing is more fundamental than the use of energy. As Lovelock claims, information is the stuff the universe is made of 
%\label{ref:RNDOg2xRWIXwK}(2019, pp.87–89),
\parencite*[][pp.87–89]{lovelock_novacene_2019}, %
 and life does not evolve toward energy-hungry, information-poor forms (i.e., humans) but rather toward information-saturated forms of life, like cyborgs. Lovelock's argument for the coming of Novacene goes as follows.

\myquote{
(1) The Earth is Gaia, a~living organism of which we are products. Gaia is unique in the universe in that it has produced biological life to sustain itself. There are no Goldilocks criteria for habitable planets
%\label{ref:RNDA9n8RBMiX1}(2019, p.11)
\parencite*[][p.11]{lovelock_novacene_2019} %
 that lead to life. Gaia had to create life and the living conditions needed to sustain that biological life: ``The truth is that the Earth's environment has been massively adapted to sustain habitability'' 
%\label{ref:RNDvhdqaYMRX3}(Lovelock, 2019, p.11).
\parencite*[][p.11]{lovelock_novacene_2019}.%


(2) As information rather than energy is the primary stuff of the cosmos, Gaia drives evolution from energy-based (i.e., us) life forms to information-based ones (e.g., cyborgs).

(Comment) As we construct complex information systems like AlphaZero, it is only a~matter of time until these systems become self-aware and self-reproductive. Eventually, their created systems (cyborgs) will be capable of self-improvement, resulting in (almost?) fault-free silicon or diamond-based mechanical organisms, at least of sorts.

(3) Mechanical systems, at least at the beginning, will need us because they will not be capable of self-assembly from the ground up like biological life was. At some point, however, they will become capable of ``doing it on their own.'' This will be the Novacene epoch.
}
But why would this evolution of life on Earth from biological organisms to cyborgs happen? It will happen because the essence of the cosmos is information, and natural evolution on Gaia is directed toward the perfection of intelligence. (We human beings were just a~stage in this process, and we are very imperfect.) Or it may be that by creating information-saturated life, Gaia, or even the cosmos, gains a~perfect understanding of itself. Of course, we may in hindsight ask why the cosmos would even try to understand itself. To even ask such a~question, the cosmos would need to already have some understanding of itself, or at least an urge to achieve it! Nevertheless, such questions do not taint Lovelock's narrative and argument.

Lovelock claims that ``only we are the way in which the cosmos has awoken to self-knowledge''
%\label{ref:RNDkLCQCarfPj}(2019, p.12).
\parencite*[][p.12]{lovelock_novacene_2019}. %
 We are not sure how to interpret how ``the cosmos has awoken to self-knowledge,'' and Lovelock does not explain this further. What is more, cyborgs will no doubt need more energy than humans, so they will not be such angelic beings either. If we need so much energy, how much more will cyborgs require? Information processing generally needs energy, a~lot of it 
%\label{ref:RND0t3IqxsYb1}(Landauer, 1961; or Bremermann, 1982; see also Shendruk and McDonnell, 2021).
\parencites[][]{landauer_irreversibility_1961}[or][]{bremermann_minimum_1982}[see also][]{shendruk_how_2021}.%


So, does Lovelock's Novacene idea make sense? Probably not in the literal sense. Gaia is a~beautiful concept, and it helps us to appreciate nature better, but it is hard to view Gaia as a~living organism in the sense of biological life, as Lovelock suggests. After all, this is the only form of life we know. Are other forms of life possible? Maybe they are, and such a~possibility is not logically impossible, but reality does not necessarily follow purely logical possibilities. One could say, ``That's too bad for reality,'' but this does not resolve the argument.

The Earth is certainly a~complex system of interlocking dependencies and relations between living and non-living systems. We have known this for some time, whether from native wisdom that we disregarded
%\label{ref:RNDghrFmIu1jH}(see for example Abram, 1996)
\parencite[see for example][]{abram_spell_1996} %
 or from what Alexander von Humboldt concluded about the Earth in his studies of the cosmos some 200 years ago, which we also disregarded. We also now have first-hand experience as we face the results of our ignorance and hubris. Yet this Earth-system is a~physical–biological–chemical complex, not a~living thing by our standards. The argument for Gaia's self-directed evolution toward some self-awareness, implying some sort of teleology on a~planetary scale, seems like a~myth, albeit a~nice one. It would certainly be welcomed by New Age enthusiasts! Gaia may not be alive, but the self-regulating ability of the Earth's ecosystem is rather evident, as is its limited capacity to maintain equilibrium beyond a~certain point. Indeed, humanity's impact on nature has now been well documented and understood.

Producing cyber systems with the capacity to exceed general human intelligence seems rather dubious in the near term for several reasons. For a~start, we do not even know what the bases of human intelligence are, and we are not sure whether intelligence, meaning human intelligence because this is the only intelligence we know, is a~medium-neutral property that can be replicated in any physical system. With some current AI systems like AlphaZero, or even a~handheld calculator, we have already exceeded some capacities of the human mind, and we will continue to progress in narrow, selective domains. But we are talking about the whole thing here, not a~game of chess.

A~rather important observation is worthwhile here: AlphaZero is amazing, but it is not even close to human intelligence
%\label{ref:RNDsqdlig9cAb}(e.g. Wooldridge, 2021, pp.79–85; see also Krzanowski, 2021).
\parencites[e.g.][pp.79–85]{wooldridge_road_2021}[see also][]{krzanowski_road_2021}. %
 Lovelock is missing this point when presenting AlphaZero as a~defining evolutionary step toward Novacene.

We also need to remember that replicating and improving the human mind may also lead to producing an unlimited capacity for stupidity, which is part and parcel of human intelligence, in cyborgs. Thus, the whole idea of electronic systems (i.e., cyborgs) being totally informed, benevolent life forms who take over their own evolution and tend to us like (happy) plants to maintain the stability of ecosystem, as is the role of early life in Lovelock's narrative, seems too farfetched and nebulous to argue about in any systematic way. It seems that Lovelock's Novacene, as a~vision of posthumanism, is rather a~dream, while all the environmental problems we cause for Gaia, whatever it/she is, are not.

One would be well advised to compare Lovelock's idea of cyborg life with the future of AI systems that is envisioned by AI expert Michael Wooldridge in his book \textit{The Road to Conscious Machines}
%\label{ref:RNDBhhOWui9Mv}(2021; cf. Krzanowski, 2021).
(\cite*[][]{wooldridge_road_2021}; \cite[cf.][]{krzanowski_road_2021}). %
 Wooldridge is quite skeptical about artificial systems taking over the planet or dominating humanity in the not-so-distant future. This would implicitly include any kind of human–computer, self-improving, self-replicating cyborg. Predictions about how the future may look in the distant future are always very risky, and Wooldridge, being a~scientist and not a~fortune teller, does not venture far into the Wonderland of Lovelock's transhumanism, Ray Kurzweil's singularity 
%\label{ref:RNDedjiQC2fAb}(Kurzweil, 2005),
\parencite[][]{kurzweil_singularity_2005}, %
 or \textit{The Terminator} 
%\label{ref:RNDlwExTnK3mw}(2019, p.112)
%\parencite*[][p.112]{lovelock_novacene_2019} %
 and the related stories that have grown around this movie 
%\label{ref:RNDpF5a7LCBc5}(The Terminator, 2021).
\parencite[][]{noauthor_terminator_2021}. %
 Wooldridge clearly sees the increasing role of digital technology because it surpasses humans in some aspects of information processing. However, in no foreseeable future does he envision these systems attaining human-like awareness or general intelligence, let alone an ability to procreate or replicate themselves. So who should we believe? Is it an AI expert with a~deep hands-on understanding of digital technology or the unbound prophecies of an AI enthusiast? Lovelock is certainly the latter, because he is recognized as an accomplished inventor, atmospheric scientist, geochemist, and scientific visionary but not an expert on AI. Nice, enticing ideas are not necessarily well-conceived ones.

Setting aside the believability of Gaia and Novacene, Lovelock's book is a~treasure trove of intriguing concepts. Lovelock is an ideas man, and we have usually seen these ideas run counter to popular entrenched knowledge, which is always a~refreshing but risky business. These ideas are worth reading for themselves, in addition to the main topic of \textit{Novacene}, and here follows a~selection of them.

Humans are the only ``understanders'' existing in the Universe. Nothing else understands the cosmos
%\label{ref:RNDyGcOZFixUT}(2019, p.3).
\parencite*[][p.3]{lovelock_novacene_2019}. %
 We are alone, like it or not. We have only the Earth, and migrating to Mars is just a~dangerous phantasm 
%\label{ref:RNDgtOWtfvQR1}(2019, pp.8–9).
\parencite*[][pp.8–9]{lovelock_novacene_2019}. %
 Information is ``the innate property of the universe and therefore conscious beings must come into existence…we are the tools by which the cosmos would explain itself'' 
%\label{ref:RNDbIRLIHSLKs}(2019, p.26).
\parencite*[][p.26]{lovelock_novacene_2019}. %
 We are too reliant on logical thinking 
%\label{ref:RNDfuMOQ3N6V8}(2019, p.13),
\parencite*[][p.13]{lovelock_novacene_2019}, %
 but ``with intuition we can now know far more than we can see'' 
%\label{ref:RNDRkEy3Uv0q2}(2019, p.22).
\parencite*[][p.22]{lovelock_novacene_2019}. %
 Also, ``the [our] misuse of science is the greatest form of sin'' 
%\label{ref:RNDkYnxEI0BWP}(2019, p.49),
\parencite*[][p.49]{lovelock_novacene_2019}, %
 and ``…the discovery of new sources of cheap fossil fuel would not be any better than the discovery of a~mine full of heroin'' 
%\label{ref:RNDoDttFtbFX6}(2019, p.49).
\parencite*[][p.49]{lovelock_novacene_2019}. %
 We cannot even measure intelligence beyond our own, because we have nothing to compare it with. In other words, we do not know how much better than humans AlphaZero is, for example 
%\label{ref:RNDpWzXNPwFhC}(2019, p.80),
\parencite*[][p.80]{lovelock_novacene_2019}, %
 so we cannot know how much smarter cyborgs would be. We can only speculate. In hindsight, measuring intelligence among humans is like a~primitive form of counting: one, two, three, and plenty.

These are just a~few samples of Lovelock's sparks of wisdom. His book is worth reading if only to encounter these uncommon intriguing thoughts. We may not agree with all of them, but we should at least recognize them because they bring rare perspectives on the human condition and maybe make us rethink some of our perceived concepts. We read them at the risk of upsetting our comfortable world view, which is an exercise worth doing in itself.

So, should you read the book? Yes, for several reasons. The book is engaging and thought-provoking. It discusses concepts that we are grappling with today, such as the essence of humanity, climate change, transhumanism, the future of our planet, travel to Mars and subsequent colonization, and the future of information technology and its impact on our civilization. Lovelock's take on these ideas is far-reaching and thought-provoking, going counter to the received or popular wisdom. We may find Lovelock's argument about Novacene a~bit too nebulous, too stretched to be taken seriously. Yet the core of his ideas is solid: The future of our life on Earth is threatened by us, because we have made a~mess of the gift that is our planet, and there is no place to hide (e.g., Mars!). The time to solve these problems is running out. And what about the cyborgs of Novacene? Only the future will tell, so we should worry more about the AI systems we are deploying now
%\label{ref:RND58jhpOo2yx}(see for example Zuboff, 2019; O'Neil, 2016; or Wylie, 2019).
\parencites[see for example][]{zuboff_age_2019}[][]{oneil_weapons_2016}[or][]{wylie_mindfck_2019}.%





%---------------



\vspace{15mm}%
{\subsubsectit{\hfill Abstract}}\\
{James Lovelock,
%\parencite*{lovelock_james_2021}
who is famous for the Gaia hypothesis \parencite*{lovelock_gaia:_1979}, has written a new book entitled \textit{Novacene: The Coming Age of Hyperintelligence} \parencite*{lovelock_novacene_2019}. It is an extended argument about an impending new epoch on Earth called Novacene in which biological life as we know it will evolve into lifeforms based on cyber technology (i.e., cyborgs) built from non-biological materials. Novacene may be seen as a development of the ideas presented in Lovelock’s earlier book \textit{A~Rough Guide to The Future} \parencite*{lovelock_rough_2014}. In Novacene, the Earth will be populated by cyborgs, which are self-replicating and self-improving mechanical systems that will eventually dominate and rule the Earth. These cyborgs will possess intelligence and knowledge beyond our understanding.}\par%
\vspace{2mm}%
{\subsubsectit{\hfill Keywords}}\\
{Novacene, posthumanity, Gaia, James Lovelock.}%



\end{newrevengenv}