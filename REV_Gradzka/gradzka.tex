\begin{newrevengenv}{Ewelina Grądzka}
	{A~companion to Kazimierz Twardowski}
	{A~companion to Kazimierz Twardowski}
	{A~companion to Kazimierz Twardowski}
	{Pontifical University of John Paul II in Kraków}
	{J. Jadacki, ed., \textit{Rozum i~wola. Kazimierz Twardowski i~jego wpływ na kształt kultury polskiej XX wieku}, Wydawnictwo Academicon, Lublin 2021, pp.573.}




Kazimierz Twardowski is considered one of the influential Polish philosophers, if not the most influential, having founded a~famous Central European analytical school, namely the Lvov–Warsaw School (LWS), which Alfred Tarski came from. However, anyone interested in learning more about Twardowski and the current state of the research will encounter some difficulty in finding an adequate source,\footnote{It is worth noting that there is a~wide range of publications about Twardowski and his school
%(e.g. Woleński, 1985; 1989; Poli, Coniglione and Woleński, 1993; Brandl and Woleński, 1999; Chybińska et al., 2016; Brożek, Stadler and Woleński, 2017; Drabarek, Woleński and Radzki, 2019),
\parencites[e.g.,][]{wolenski_filozoficzna_1985}{wolenski_logic_1989}{poli_polish_1993}{brandl_actions_1999}{chybinska_tradition_2016}{brozek_significance_2017}{drabarek_interdisciplinary_2019},
but each focuses on some particular aspects. } something like ``A companion to…'' Fortunately, Professor Jacek Jadacki has provided us with a~book that, thanks to chapters authored mostly by established experts in the field, aims to guide the reader through the works and ideas of Twardowski and their influence on Polish and international philosophy. The book can therefore be recommended to anyone who has never encountered Twardowski's \textit{ouvré}, as well as those interested in the trends and shifts in the research of this.

The book under review is \textit{Rozum i~wola. Kazimierz Twardowski i~jego wpływ na kształt kultury polskiej XX wieku} [Reason and will. Kazimierz Twardowski and his influence on the shape of Polish culture in the 20th century], and it resulted from an interesting project called ``Kazimierz Twardowski's place in Polish culture and European philosophy,'' which was financed by the National Science Center of Poland. It is also the fourth volume in a~series of publications
%(\textit{cf.} Brożek et al., 2020; Brożek, 2020; Jadacki and Cullen, 2020)
\parencites[cf.][]{brozek_analiza_2020}{brozek_anti-irrationalism_2020}{jadacki_stanislaw_2020}
by the Lvov–Warsaw School Research Center, which was established in 2020 at the Faculty of Philosophy, University of Warsaw\footnote{For more on \parencite{brozek_analiza_2020} and the center, see \parencites{gradzka_putting_2021}{gradzka_report_2021}.}
%\footnote{For more on (Brożek, 2020) and the center, see (Grądzka, 2021a; 2021b)}
%\parencites[for more on][]{brozek_analiza_2020}[and the center, see][]{gradzka_putting_2021}{gradzka_report_2021}
with the aim of promoting and developing research related to the LWS. The book can also be considered a~great supplement to compilations of Twardowski's texts'
%(Twardowski, 1927; 1992; 2013; 2014; Brandl and Woleński, 1999; Brożek et al., 2020).
\parencites{twardowski_rozprawy_1927}{decewicz_wybor_1992}{twardowski_mysl_2013}{twardowski_mysl_2014}{brandl_actions_1999}{brozek_anti-irrationalism_2020}.
It is also a~book that---next to those of
%(Jadczak, 1993; Kleszcz, 2013; and Brożek, 2015)
\parencites{jadczak_czlowiek_1993}{kleszcz_metoda_2013}[and][]{brozek_kazimierz_2015}---aims to present Twardowski's ideas in a~comprehensive and critical way, with it going further than just focusing on one area of his activity.

%%%tu

The editor specializes in ontology and epistemology, logical semiotics, and the history of Polish philosophy, especially the Lvov–Warsaw School. Jadacki claims to be a~follower of Jan Łukasiewicz's program of philosophizing, so therefore considers himself a~continuator of the LWS. The problem of LWS members and followers is an engaging and still debatable issue that leads to another discussion about what it means to belong to a~school, both in general and for this particular example. More on this can be found in the work of
%(Woleński, 1985).
\parencite{wolenski_filozoficzna_1985}.

We first question whether the book can be referred to as a~companion to Twardowski. Taking a~look at the offerings of publishers like Cambridge, Oxford, Routledge, and Springer, one can find books where the editors attempt to refer step by step to particular, significant aspects of a~philosopher's ideas and publications. They are then later put into the context of their epoch and how they were received by philosophy/science. In this way, the reader is generally guided through the difficulties in understanding the philosopher's \textit{œuvre } and helped to appreciate the questions or solutions that were presented by the philosopher. Giving such an extensive background enables the reader to comprehend the significance of the thoughts and critique them. So, does the book under review meet these aims? Let's take a~closer look.

The book is divided into seventeen chapters grouped into six sections, each entitled ``Twardowski as…'' It seems the editor aims to present the Lvov philosopher as a~personality who played multiple roles, which is indeed true.

The first section is titled ``Twardowski jako klasyk'' [Twardowski as a~classic], and it refers to his role in Polish and global philosophy. It was written by some of the most significant Polish specialists on Twardowski, namely Jan Woleński, Jacek Jadacki, and Ryszard Kleszcz, who have great experience in this subject and the LWS in general
%(\textit{cf.} Woleński, 1985; Jadacki, 1989; 2003; 2009; 2015; Brandl and Woleński, 1999; Jadacki and Paśniczek, 2006; Kleszcz, 2013; Chybińska et al., 2016; Brożek, Stadler and Woleński, 2017).
\parencites[cf.][]{wolenski_filozoficzna_1985}{jadacki_semiotyka_1989}{jadacki_viewpoint_2003}{jadacki_polish_2009}{jadacki_polish_2015}{brandl_actions_1999}{jadacki_lvov-warsaw_2006}{kleszcz_metoda_2013}{chybinska_tradition_2016}{brozek_significance_2017}
Their contribution seeks to present Twardowski as an international personality, one that was important for his contemporaries (i.e., students, continuators, and opponents), and a~person who applied his method of philosophizing to issues in daily life and all professional engagements.

They manage to convince the reader that Twardowski is ``a classic'' worth knowing. On the one hand, he had an impact on global philosophy (as described by Woleński on pp.17–47) thanks to his \textit{Habilitationsschrift} entitled \textit{On the Content and Object of Presentations}~\parencite*{twardowski_1894},
%(1894),
in which he distinguishes between act, content, and object of representation. It inspired Alexius Meinong's theory of objects and Alois Höfler's and Edmund Husserl's concepts of intentionality. On the other hand, Twardowski's thought was integrated into global philosophy by his followers, particularly Jan Łukasiewicz and Alfred Tarski. It is continuously emphasized that many problems that the LWS members developed had originally taken root in Twardowski's ideas. (He is said to have not been a~publish-or-perish sort of person, instead preferring to openly share his ideas with students and be pleased at seeing them advanced or refined.)

Woleński refers here to Twardowski's style of teaching logic, which was introduced by Tarski to the University of California, Berkeley, where it helped develop the Californian school of logic. The concept of absolute truth is also claimed to have provided the philosophical background for the many-valued logic of Jan Łukasiewicz, who later worked at the University of Dublin, and Tarski's logical semiotics. Woleński concludes that Twardowski's pedagogical success, with him having educated over thirty professors, would be enough alone to count as a~great contribution to global philosophy.

How this achievement was made possible is explained in the next chapter by Kleszcz (pp.49–96). Twardowski, unlike other analytic philosophers, was able to raise ``an army of intellectuals'' thanks to his unprecedented engagement in working with students through proseminars, seminars, a~library, the \textit{Polish Philosophical Society}, and the \textit{Polish Philosophical Congress}. He also established the journal \textit{Ruch Filozoficzny} [Philosophical Movement], made himself accessible every day at particular hours, introduced a~consistent philosophical curriculum, and demanded high methodological standards (i.e., clarity of expression, criticism, and application of logic and reasoning).

This was evidently so effective that the most fruitful and reasonable criticisms of Twardowski's works came from his students, which, according to Jadacki, is proof of his didactical success (Jadacki, pp.97–199). It is an interesting contribution of Jadacki to present such criticism, because it gives the reader a~feeling of how discussions took place within the LWS. The editor also touches upon another important issue, namely the state of Polish philosophy in the post-war period, when a~new political and ideological system was imposed by the Soviet Union. He describes how Twardowski and his students were slandered and defamed, with them being accused of having bourgeois, anti-proletarian, imperialistic, reactionary, obscurantist, fideistic, and speculative ideas by prominent personalities like Adam Schaff, Henryk Holland, and Leszek Kołakowski.

Additionally, Władysław Tatarkiewicz, Roman Ingarden (the attacks were also directed against phenomenology and neotomism), and Izydora Dąmbska were banned from teaching students. A~short Appendix II with citations from \textit{Krótki słownik filozoficzny} [A short philosophical dictionary] helps understand the meanings of some typical Marxism–Leninism concepts, and this is a~great idea, especially for the younger generation.

The communist persecution of Polish philosophers is still relatively fresh, however, and it is difficult to remain moderate and present the issue in a~restrained way, which is understandable, especially for those with experience of this system. Jadacki's work is relevant, but he does not manage to be neutral, and this can unfortunately be discouraging, like it was in the case of the books by Kuliniak, Pandura, and Ratajczak (2018, 2019). Maybe it will be the task of the next generation to prepare material that will expose the brutal and shameful truth but in a~reserved, detached manner.

Twardowski was a~great inspirer, and Woleński's efforts (pp.203–235) to emphasize the importance of \textit{Action and Products}
%(\textit{cf.} Brandl and Woleński, 1999)
\parencite[cf.][]{brandl_actions_1999}
are especially significant nowadays, when we face a~sort of resurgence of neopositivism in the form of ``neuroscientism,'' a~belief that all our behavior and awareness can be explained by research into the functioning of the brain.

Meanwhile, Alicja Chybińska offers the intriguing hypothesis that Twardowski's methodological requirements of clarity of expression influenced Tadeusz Kotarbiński's reism (pp.237–252). As stated before, Twardowski's students were strongly influenced by their master, but they also often resisted his solutions. It is claimed here that Kotarbiński was ``inspired negatively'' in the sense that he was motivated to contradict his teacher by rejecting ``hypostasis.''

Nevertheless, the methodological postulate of clarity, so emphasized by Twardowski, is tricky and challenging to follow, even for the members of the LWS. Łukasiewicz, one of the most famous of these, is considered to have been a~master of precision in terms of the theories he created, but he failed while considering the relations between logic and philosophy. He even claimed that during his investigations, he faced some kind of ideal construction that a~philosopher-believer would call God's thought. This is surprisingly close to the statements of people like Michael Heller
%(Heller, 2019)
\parencite*{heller_god_2019}
and Albert Einstein (the mind of God). The problem of the relation of logic to philosophy or even theology is a~question for debates that can be followed in publications like those of Dadaczyński
%(2014)
\parencite*{dadaczynski_remarks_2014}
and Heller and Awodey
%(2020).
\parencite*{awodey_homunculus_2020}.

The importance of logic and analysis, not just for philosophers but also those in other disciplines, can be gleaned in the texts of the third part of the book. Anna Brożek, clear and bright as ever
%\footnote{(Brożek, 2020; \textit{cf.} Grądzka, 2021a})
\parencites{brozek_analiza_2020}[cf.][]{gradzka_putting_2021}
introduces how Twardowski conceived his analysis of concepts (pp.271–297). This provides significant background for the next chapter, which exposes a~rarely mentioned aspect of Twardowski's engagement, namely his teaching of logic to students of medicine. It turns out that the Lvov philosopher also influenced the Polish School of the Philosophy of Medicine and one of its main contributors, Władysław Szumowski (Aleksandra Horecka, pp.299–323). One again gets a~feeling that some groundbreaking ideas were introduced long ago, ones that we are ignorant of now. Modern medicine is far removed from recognizing the importance of logic and philosophy for making progress, reflecting over its methods, and reconsidering the modern form of materialism as a~paradigm, and no one seems to care if ``illness'' really exists or not. This chapter is therefore especially recommended to doctors and other healthcare professionals, especially in the wake of the recent pandemic, when so many paradigms were challenged.

In his introduction to the book, Jadacki admits that the publication has a~problem in repeating some issues in different texts. He considers this not just a~weakness but also a~strength, because it allows readers to follow each article independently. This impression prevails on reaching some chapters, namely Jadacki's ``On Twardowski's contributions to development of logic in Poland'' (pp.325–343) and ``Twardowski's postulate of Clarity in the Eyes of an Anti-Irrationalist'' (pp.385–406), as well as Dariusz Łukasiewicz's ``Analysis of the relationship between Philosophy, Science and the Worldview in Terms of Twardowski'' (pp.347–383). However, it must be admitted that despite some basic information being repeated in the texts, the authors manage to present a~fairly systematic and thorough analysis of the problems, and there are new aspects that are worth reading, such as the comparison of Twardowski's, Salamucha's, and Kleszcz's ideas on the worldview and Twardowski's theory of induction, which comes thanks to investigating some unpublished lectures about logic.

Finally, we reach the two final chapters, which may be somewhat controversial because they attempt to interpret Twardowski's activities as an expression of his political (Ryszard Mordarski pp.409–428, Jacek J. Jadacki pp.429–459) and pedagogical (Tadeusz Czeżowski pp.463–468, Wojciech Rechlewicz pp.469–530, Jacek J. Jadacki pp.531–562) ideas and theories. The Lvov philosopher lived and worked in an interesting time when political science and pedagogy were gaining independence from philosophy. He himself was motivated not just by his way of philosophizing but also his upbringing, and he engaged in many projects related to building a~well-educated nation of rational patriots.

Nowadays, we would say he was a~theoretician and a~practitioner, an interdisciplinary one like in so many other cases. However, methodological questions remain as to whether he should be considered a~philosopher of politics or education, or possibly a~political scientist or pedagogist in modern terms. It is fair to say, however, that we should judge the thinker by the standards of his times. Indeed, ``the case of Twardowski'' is captivating because it makes the reader or investigator deliberate with much precision, but also meditation, on what the fundamentals or differentiations are between disciplines or fields. This might be challenging for beginners and even some advanced researchers, but it is a~good exercise for anyone interested in the philosophy of science.

In conclusion, the answer to the question of whether the book meets the requirements of being a~companion to Kazimierz Twardowski's intellectual, pedagogical, and practical heritage is a~positive one. The editor brought together a~valuable group of researchers who then managed to inspire their readers to further investigate the works and accomplishments of the Lvov philosopher. It will be a~challenge for a~future publisher to go further in preparing a~subsequent companion to Twardowski, because such work is never completed. Meanwhile, it would be useful to translate the book into English, so non-Polish speakers can also better appreciate Tarski's background.\footnote{For now, there is a~valuable introduction to Twardowski in the Stanford Encyclopedia
%(Betti, 2019).
\parencite{betti_kazimierz_2019}.
}




%---------------



\vspace{15mm}%
{\subsubsectit{\hfill Abstract}}\\
{
There have been many significant publications on Kazimierz Twardowski. Jacek Jadacki intends to add to this list another book \textit{Rozum i wola. Kazimierz Twardowski i jego wpływ na kształt kultury polskiej XX wieku}. In the review it is appraised whether it can be called ``a companion to…''. It provides introductory information that can help readers better understand the role of Twardowski in Polish philosophy and culture. Updated findings by contemporary scholars are also included. The quality of the articles is guaranteed by such authors as J.~Woleński, R.~Kleszcz, A.~Brożek and J.~Jadacki. However, new authors are also present as well as less common topics like Twardowski's influence on the Polish School of Philosophy of Medicine and his roles as political scientist, educational theorist, and historian of Ancient philosophy. The authors manage to convince the reader that Twardowski is ``a classic'' worth knowing, in consequence the book can be treated as a ``companion to Twardowski''. It also inspires readers to further investigate the works and accomplishments of the Lvov philosopher.    
%Kazimierz Twardowski is considered one of, if not the most influential Polish philosophers and founder of the famous Central European analytical school, the Lvov-Warsaw School (LWS), that gave rise among others to Alfred Tarski. However, those interested in getting to know him better and the current state of the research might have had some difficulty to find an adequate source---something like ``a companion to''. Professor Jacek Jadacki seems to take that task seriously and provides us with a book that intends to guide the reader (thanks to chapters authored mostly by established experts in the field) through the works, ideas and influences of Twardowski on Polish culture and beyond. Therefore, the book can be recommended to beginners as well as those interested in trends and shifts in the research.
}\par%
\vspace{2mm}%
{\subsubsectit{\hfill Keywords}}\\
{Kazimierz Twardowski, Lvov-Warsaw School, analythic philosophy, Alfred Tarski.}%



\end{newrevengenv}
