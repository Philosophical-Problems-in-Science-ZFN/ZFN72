\begin{newrevengenv}{Ewelina Grądzka}
	{Putting analysis and construction of concepts into its legitimate position}
	{Putting analysis and construction of concepts\ldots}
	{Putting analysis and construction of concepts into its legitimate\\position}
	{Pontifical University of John Paul II in Krakow}
	{Aanna Brożek, \textit{Analiza i~konstrukcja: o~metodach badania pojęć w~Szkole Lwowsko-Warszawskiej}, Copernicus Center Press, Kraków 2020, pp.241.}
	
	



\lettrine[loversize=0.13,lines=2,lraise=-0.03,nindent=0em,findent=0.2pt]%
{A}{}nalytic description---according to members of the Lvov-Warsaw School (LWS) like Czeżowski, Ajdukiewicz, Ossowska, Tarski---is a~powerful and an indispensable tool not only in philosophy, but also in any natural science---in psychology especially. It should be equally respected together with empirical analysis and even it is recommended that it precedes any further research. Therefore, the book \textit{Analiza i~konstrukcja: o~metodach badania pojęć w~Szkole Lwowsko-Warszawskiej} [\textit{Analysis and construction: on the methods of researching concepts in the Lvov-Warsaw School}] can be recommended to philosophers and scientists as well.

The reviewed book is a~consequence of an interesting project called ``Philosophy from a~methodological point of view. The condition and perspectives of philosophy in the light of the Lvov-Warsaw School paradigm'', financed by the National Science Center (Poland). It is a~second publication related to this grant, as the first one was \textit{Antyirracjonalizm. Metody filozoficzne w~Szkole Lwowsko-Warszawskiej} [\textit{Anti-irrationalism. Philosophical Methods in the Lvov-Warsaw School}]
%\label{ref:RNDOovUZQcSmD}(\textit{Bro}żek et al., 2020b, eng. transl. a).
(\cite[][]{brozek_antyirracjonalizm_2020}, \cite*[Eng. transl.][]{brozek_anti-irrationalism_2020}). %
%\parencites[][]{brozek_antyirracjonalizm_2020}[Eng. transl.][]{brozek_anti-irrationalism_2020}.%
The book is also a~third volume in the publication series 
%\label{ref:RNDUEQO3CPswZ}(Brożek et al., 2020a; Jadacki and Cullen, 2020)
\parencites[][]{brozek_anti-irrationalism_2020}[][]{jadacki_stanislaw_2020} %
 of The Lvov-Warsaw School Research Center\footnote{Among members of the Center there are Johannes Brandl (Salzburg University), Francesco Coniglione (University of Catania), Guillaume Frechette (Salzburg University), Stepan Ivanyk (Kazimierz Twardowski Philosophical Society for Lviv), Jacek Jadacki (University of Warsaw), Ryszard Kleszcz (University of Lodz), Maria van der Schaar (Leiden University), Peter Simons (Trinity College Dublin), Friedrich Stadler (University of Vienna), Jan Woleński (University of Information, Technology and Management, Rzeszów).} established in 2020 at the Faculty of Philosophy, University of Warsaw. According to the information on the website ``The aim of the Center is to stimulate and coordinate research into the tradition of the Lvov-Warsaw School.'' 
%\label{ref:RNDtY1bhiPMyb}(LWS Research Center, 2021)
\parencite[][]{noauthor_lws_nodate} %
 Indeed, in the recent years we can observe a~growing interest in the work and heritage of the LWS and a~significant increase in the number of publications referred to the School's achievements, also in English 
%\label{ref:RNDkLHZkS11xu}(Drabarek, Woleński and Radzki, 2019; Schaar, 2016; Woleński, 1989; Brandl and Woleński, 1999; Brożek, Stadler and Woleński, 2017; Twardowski, Brożek and Jadacki, 2014; Simons, 1992; Poli, Coniglione and Woleński, 1993; Kijania-Placek and Woleński, 1998; Szaniawski, 1989; Chybińska et al., 2016).
\parencites[][]{drabarek_interdisciplinary_2019}[][]{schaar_kazimierz_2016}[][]{wolenski_logic_1989}[][]{brandl_actions_1999}[][]{brozek_significance_2017}[][]{twardowski_prejudices_2014}[][]{simons_philosophy_1992}[][]{poli_polish_1993}[][]{kijania-placek_lvov-warsaw_1998}[][]{szaniawski_vienna_1989}[][]{chybinska_tradition_2016}. %
 The Center focuses on research but also on the organization of various events\footnote{The most recent events are: \textit{Roman Ingarden and the Lvov-Warsaw School}. International online symposium October 22-24, 2020; \textit{Philosophy Workshop}, 1st Edition Online, February 11-14, 2021 and \textit{The World of Values in the Lvov-Warsaw School}. International Symposium, October 21-23, 2021.}---seminars, symposia, conferences. The Center also offers an ``institutional support for publications on the LWS.'' 
%\label{ref:RNDM4yPmr2W8C}(LWS Research Center, 2021)
\parencite[][]{noauthor_lws_nodate}%


The author, Anna Brożek works at the Department of Logical Semiotics at the Institute of Philosophy, University of Warsaw and is also the Head of the Lvov-Warsaw School Research Center. She specializes in logical semiotics, methodology, ontology, history of Polish philosophy as well as theory and philosophy of music.

As the author rightly points out in the introduction, ``Analyzing and constructing concepts, although it is a~creative and thrilling occupation, rarely brings brilliant results that have a~wide response''
%\label{ref:RNDpm8jUV8MhJ}(Brożek et al., 2020a, p.7).
\parencite[][p.7]{brozek_anti-irrationalism_2020}. %
 The more we should appreciate the willingness to devote research to this issue and to prepare a~book that would enable to appreciate this exquisite work. Undoubtedly, the fact that the author identifies with the described method adds the value to the publication. She emphasizes that ``in my philosophical research I~use the analysis and construction of concepts in the style appropriate to the representatives of the Lvov-Warsaw School'' 
%\label{ref:RNDyDTK9pS8IP}(Brożek et al., 2020a, p.7).
\parencite[][p.7]{brozek_anti-irrationalism_2020}. %
 It seems significant that Brożek has learned LWS analytical method ``inclusively from the examples in this book.'' This gives additional credibility to the usefulness of the chosen examples and their effectiveness in conveying the intended content. This has an epistemic, methodological and didactical value. The author's belief that ``a gradual progress in philosophical disciplines is feasible thanks to painstaking analytical and constructional efforts'' 
%\label{ref:RNDhtGwlw1Ju3}(Brożek et al., 2020a, p.7)
\parencite[][p.7]{brozek_anti-irrationalism_2020} %
 encourages interest in the content of the following chapters. Moreover, the title itself, which refers to the title of Jan Łukasiewicz's book \textit{Analiza i~konstrukcja pojęcia przyczyny} [\textit{Analysis and Construction of the Concept of Cause}] 
%\label{ref:RNDPCJj5QWdY4}(Łukasiewicz, 1906),
\parencite[][]{lukasiewicz_analiza_1906}, %
 confirms the inclination towards the tradition of the LWS.

The author is right to discern that the ``analytic era'' did not start in the 20th century. Indeed, it can be dated back as far as Aristotle himself, followed later on by Medieval Philosophy. However, it was the development of logic and its new tools that made a~difference. Although the Lvov-Warsaw School is considered one of the branches of the 20\textsuperscript{th} century analytic philosophy, it is often not recognized in companions
%\label{ref:RNDy27IqAsZgR}(cf. Beaney, 2015; Dainton and Robinson, 2015; Martinich and Sosa, 2006)
\parencites[cf.][]{beaney_oxford_2015}[][]{dainton_bloomsbury_2015}[][]{martinich_companion_2006} %
 or interest into the School's heritage is quite neglected in the international philosophical community.\footnote{However, there are some international scholars interested in the research on the works of the LWS. 
%\label{ref:RNDsKDET0NfGH}(cf. Brandl and Woleński, 1999; Poli, Coniglione and Woleński, 1993; Schaar, 2016; Simons, 1992).
\parencites[cf.][]{brandl_actions_1999}[][]{poli_polish_1993}[][]{schaar_kazimierz_2016}[][]{simons_philosophy_1992}.%
} There is still a~lot of research to be done into logical tools used by its members, which is probably a~consequence of a~long period of a~Communist rule in Poland that at the beginning openly attacked the LWS scholars (eliminating professors from chairs at top Polish universities, closing academic journals etc.) or, after the thaw, there was simply no interest in the accomplishments of the School that did not go in line with the official Marxist-Leninist philosophy 
%\label{ref:RNDenQX9V3nkX}(cf. Kuliniak, Pandura and Ratajczak, 2018, 2019).
\parencites[cf.][]{kuliniak_filozofia_2018}[][]{kuliniak_filozofia_2019}. %
 Nevertheless, some important work has been done by a~few members who were alive like I. Dąmbska, her student J. Woleński 
%\label{ref:RNDr3hJ5yNFik}(1985, 1986, 1989).
\parencites*{wolenski_filozoficzna_1985}{wolenski_filozofia_1986}[][]{wolenski_logic_1989}. %
 Professor J. Jadacki's work is also significant 
%\label{ref:RND0tNZpKlAwd}(Jadacki, 1987, 1989).
\parencites[][]{jadacki_semiotyka_1987}[][]{jadacki_semiotyka_1989}. %
 A~new wave of investigations and efforts into the LWS heritage has recently been noted as a~favorable indicator.

The author's intention is to participate in the general discussion on the metaphilosophical issues, that in her opinion just recently gained a~considerable attention. However, interest in this subject can be traced to the works of Plato or Aristotle. Interestingly, in the 1970 a~peer-reviewed journal \textit{Metaphilosophy} was established. It is understandable that the author refers to
\parencites{williamson_philosophy_2007}{rescher_metaphilosophy_2014}[or][]{doro_cambridge_2017}
%%\label{ref:RNDduBe9zscfR}(Williamson, 2007),
%\parencite[][]{williamson_philosophy_2007}, %
%%\label{ref:RNDOyw0a3F45t}(Rescher, 2014)
%\parencite[][]{rescher_metaphilosophy_2014} %
% or 
%%\label{ref:RNDwkQjnVDZ2C}(D'Oro and Overgaard, 2017)
%\parencite[][]{doro_cambridge_2017} %
 and therefore her conclusion. Although the book does not cover all methodological procedures used in the LWS, it focuses on one of the most fundamental one, that is conceptual analysis. Brożek focuses on the reconstruction of their methodological thesis expressed \textit{implicite} in their research (not omitting their direct works in this field) as it is claimed to be neglected. It is also thought to be a~more effective method of learning. Therefore, in the book we can find a~lot of citations from LWS members' analysis which gives the book a~valuable, practical format and allows to pervade with the method. It is probably the greatest advantage of this book. Conceptual analysis was often used in LWS practice. One of Kazimierz Twardowski's (student of Brentano, founder of the LWS,) first students was Władysław Witwicki, whose PhD thesis was entitled \textit{Analiza psychologiczna ambicji} [\textit{Psychological analysis of ambition}] 
%\label{ref:RND0kM5TBey59}(Witwicki, 1934)
\parencite[][]{witwicki_analiza_1934} %
and it was entirely devoted to analysis of this concept without any additional remarks.

The book consists of three parts divided into sixteen chapters. In part one, that contains two chapters, we can find a~theoretical introduction to the subject of analysis of concepts and formulation of definitions. The second part, divided into twelve chapters, is focused on the presentation of examples of analysis from the works of the most representative members of the LWS. Brożek states that the key was to present a~variety of branches and philosophical disciplines. The last part, made of two chapters, intends to critically summarize the way analysis was done in the LWS and presents the perspectives for the usage of analysis of concepts in the methodology of philosophy. Such a~division of the book seems clear and allows to follow easily the author's reasoning.

In the introduction a~short background to the idea of the book and the hypotheses that guided the author are presented. Later, concise history, branches, methodological foundations of the LWS along with current bibliography worth consideration are given. Interesting is a~remark that the most significant results in philosophy were achieved by those members (K. Ajdukiewicz, T. Czeżowski, T. Kotarbiński) who engaged equally into the investigations related to psychology and logic. Nowadays such an attitude is considered a~novelty and promoted as an interdisciplinary research which is gaining importance and prestige (p.17).

Part one can be considered a~brief introduction or even a~short textbook in logical semiotics---a~term coined by K. Ajdukiewicz (an outstanding Polish logician)
%\label{ref:RNDVNJ4lWTIlS}(cf. Pelc, 1979).
\parencite[cf.][]{pelc_logical_1979}. %
 Chapter one in a~clear and simple manner familiarizes the reader with the concept of the analysis (its objects, tools, method, results). Next, the analysis of ‘the concept' is presented. It is noticeable that the author being plain in her explanations refers to the noble tradition of the LWS where clearance of expression was of greatest value and majority of the works written, especially by Kazimierz Twardowski, are famous for simplicity together with the extensiveness of the explanation. Analysis is considered a~way to acquire new knowledge as it enables understanding of the object's structure which, in turn, clarifies the world view. The analysis, especially in philosophy, is sometimes regarded as an opposition to the synthesis. Nevertheless, a~valuable analysis should not lose a~broader perspective. It might be even a~good advice, especially for representatives of natural sciences who tend to forget about the general picture of what they analyze. It is worth remembering that it is philosophical speculation rather than ``synthetic philosophy'' that analytical philosophy stands against. Chapter two is thought to describe the result of the process of analysis which is a~formulation of definition. Brożek claims that definition should be at the heart of analytic philosophy, but very often it is neglected due to the level of complexity of the analyzed concept. Nowadays, a~growing interest into ``conceptual engineering'' is observed and the book fits the discussion 
%\label{ref:RNDa4My0hI6E6}(cf. Chalmers, 2020; Koch, 2020).
\parencites[cf.][]{chalmers_what_2020}[][]{koch_engineering_2020}. %
 Also \textit{Philosophical Problems in Science} encourage such type of investigation 
%\label{ref:RND7IVZNlNfn4}(cf. Awodey and Heller, 2020; Piechowicz, 2020).
\parencites[cf.][]{awodey_homunculus_2020}[][]{piechowicz_formal_2020}.%


Part two is a~textbook of its own kind, which distinguishes this book as it gives examples and refers to the works of the LWS like the Ajdukiewicz's concept of meaning, Tarski's concept of truth or Tatarkiewicz's concept of happiness
%\label{ref:RNDJJBDYKEVbe}(Brożek et al., 2020a, p.52).
\parencite[][p.52]{brozek_anti-irrationalism_2020}. %
 It can be especially helpful in academic didactics in Poland, but also abroad, where students can get a~better understanding of the achievements of the LWS in context. 
% However, the biggest disadvantage is the fact that the book is written in Polish. 
 It is recommended to translate the book into English so that it can participate in the international discourse as a~valuable source. It might be questioned whether in a~monograph should have such a~textbook-like section. Nevertheless, it seems useful and concise with the whole concept of the book which hence forms a~united and complete entity.

If someone feels 
%at ease with 
does not care at all about
the theoretical part, certainly it can be advised to start from the second one---a~kind of historical and practical part, in which results of the research are presented. Each chapter is dedicated to a~different significant member of the LWS and his/her analysis of one concept (or two in case of Ajdukiewicz). In the beginning each philosopher is presented with a~short biography as well as a~photo portrait which creates an impression of closeness.

The presentation of concepts begins in chapter three which is dedicated to the founder of the Lvov-Warsaw School, Kazimierz Twardowski. His migration from highly modernized Vienna, the center of Austria-Hungary Empire to the provincial Lvov to take the chair at the Faculty of Philosophy is considered the beginning of the LWS. His unprecedented effort to establish high-quality research at the university soon contributed with the growing number of exceptional students who finally formed a~unique School of modern analytic philosophy in Central Europe. In the book, his analysis of the concept of concept [sic!] is presented and critically evaluated. It comes from a~paper called \textit{O~istocie pojęć} [\textit{The Essence of Concepts}]
%\label{ref:RNDS4rcIKGUiw}(Twardowski, 1965, eng. transl. 1999)
(\cite[][]{twardowski_o_1965}, \cite*[Eng. transl.][]{brandl_essence_1999}) %
 first published in German in 1902. Concepts are fundamental to philosophy and science, so it is crucial to understand its nature. Brożek suggests that Twardowski's intention might have been to fill the gap between the understanding of the concept of concept in psychological and logical sense. She points to shortcomings and offers a~valid clarification on what is missing in the definition. It is a~neat and skillful work.

Chapter four is dedicated to W. Witwicki, who participated also in the creation of the Lvov Psychology School. Here Brożek focuses on presenting his analysis of the concept of ambition, which was his PhD dissertation. For Witwicki, ambition is a~disposition to have feelings (positive or negative) based on judgements. Steps taken to reach such a~conclusion are intriguing to read.

Chapter five introduces Jan Łukasiewicz's, famous Polish logician, co-founder of the Warsaw Logic School with Leśniewski and Kotarbiński and one of the first authors of the many-valued logic. This time it was a~habilitation work. Łukasiewicz, being in opposition to the psychological approach in the LWS, focused on logical analysis. Despite the fact, Brożek notices, that he stood strongly against psychologization, yet his own concept of concept was more psychological than Twardowski's. However, she reveals something not mentioned elsewhere. Both philosophers agreed on the usage of the inductive-deductive method for construction of real concepts. Next, Brożek presents stages of Łukasiewicz's analysis and adds her critical remarks.

Chapter six demonstrates the analysis of the concept of deed prepared by T. Kotarbiński, famous for the (ontological and semantical) theory of reism and praxeology (part of its task is an analysis of concepts related to the theory of action). Due to reism, he practiced analysis as a~part of the theory of semiotic functions of expressions. His analysis comes from the first chapter of his book \textit{Traktat o~dobrej robocie} [\textit{Praxiology. An Introduction to the Science of Efficient Action}]
%\label{ref:RNDL0DIUxVUnB}(eng. transl. Kotarbiński, 1958, 1965).
(\cite[][]{kotarbinski_traktat_1958}, \cite*[Eng. transl.][]{kotarbinski_praxiology_1965}). %
%\parencite[Eng. transl.][]{kotarbinski_praxiology_1965}. %
 Brożek presents thoroughly steps of his reasoning (and this time sums them up in a~schema). Inspiringly, she compares some of its aspects to Łukasiewicz's analysis of the concept of cause (as it is necessary for Kotarbiński's investigation) or Czeżowski's method of analytical description. It is a~valuable work, as it helps to observe how philosophers in the LWS differed in perspective on various issues.

Chapter seven exposes W. Tatarkiewicz's analysis of the concept of happiness presented in the book \textit{O~szczęściu} [\textit{Analysis of happiness}]
%\label{ref:RNDRFVsKezQKY}(Tatarkiewicz, 1947, eng. transl. 1976).
(\cite[][]{tatarkiewicz_o_1947}, \cite*[Eng. transl.][]{tatarkiewicz_analysis_1976}). %
%\parencite[][Eng. transl. 1976]{tatarkiewicz_o_1947}. %
 Although he did not study directly under Twardowski's supervision, he identified himself as a~member of the LWS.\footnote{The problem of belonging to the LWS is one of the highly discussed methodological issues related to the LWS and research on their heritage 
%\label{ref:RND2n6475gEcD}(cf. Woleński, 1985, p.338).
\parencite[cf.][p.338]{wolenski_filozoficzna_1985}.%
} He is famous for the books \textit{Historia filozofii} [\textit{History of philosophy}] and \textit{Historia estetyki} [\textit{History of aesthetics}]. The knowledge he gained while preparing and analyzing the materials for that books made him conclude that there was a~need to differentiate between definition and theory. Additionally, his analysis of the concept of happiness is a~consequence of historical analysis of all the literature related to that issue and accessible to him. He intended to prepare \textit{Summa de beatitudine}. Appealingly, Brożek emphasized that he wrote the book during the tragic war period of 1939--1943.\footnote{Interestingly, R. Ingarden, who worked in phenomenology but was partly related to the LWS as K. Twardowski was his teacher, also despite the brutality of the IIWW prepared his significant work \textit{Spór o~istnienie świata} [\textit{Controversy over the Existence of the World}] 
%\label{ref:RNDeKNwoCqiLH}(Ingarden, 1947, eng. transl. 2013).
(\cite[][]{ingarden_spor_1947}, \cite*[Eng. transl.][]{ingarden_controversy_2013}). %
%\parencite[][Eng. transl. 2013]{ingarden_spor_1947}. %
 In both cases, the philosophers managed to create great work despite the times of turmoil.} Tatarkiewicz confessed that his definition of happiness is an idealization. Importantly, Brożek indicates that although idealization is commonly used in natural sciences, in other areas of human activity it is reluctantly accepted.

Chapter nine deals with analysis of the concept of analysis created by T. Czeżowski, who contrary to others from LWS turned from mathematical logic to descriptive psychology and was called ``Polish Brentanist''. He believed that analytic description is still a~powerful and necessary tool in any academic domain, in psychology especially, although we can observe an overwhelming domination of experiment. Hence this chapter (and, of course, the whole book) can be recommended especially to scientists. Brożek reconstructs the procedure of analytic description mainly from Czeżowski's lecture entitled \textit{O~metodzie opisu analitycznego} [\textit{On the method of analytic description}]
%\label{ref:RND9HHJWIFr2h}(Czeżowski, 1958, eng. transl. 2000).
(\cite[][]{czezowski_o_1958}, \cite*[Eng. transl.][]{czezowski_method_2000}). %
%\parencite[][Eng. transl. 2000]{czezowski_o_1958}. %
 She recalls Twardowski's analysis of concept of concept (which encaptures his idea here even better than in the chapter three) as it was a~starting point for Czeżowski. Next, she successfully takes up the task to improve organization of the stages of Czeżowski's work and later offers valuable steps of commentary. This chapter gives an impression of the biggest engagement of the author. Brożek admits later that the whole book as well as the final reconstruction in the third part is inspired by Czeżowski's work.

Chapter nine is dedicated to Kazimierz Ajdukiewicz, one of Twardowski's favorite students and later his son-in-law, founder of \textit{Studia Logica}. This time Brożek makes an exception and presents analysis of two concepts: meaning and justice. Although the author is the same, they differ due to the subject (ethical and semiotic), tools and level of precision. Probably, this chapter could be treated as a~crash course into the analysis of concepts used in the LWS for those with limited time. Ajdukiewicz opposed Vienna Circle's idea that only formal and empirical methods can be called ``scientific''. He believed that analysis of concepts is equally important. Surprisingly, the paper Brożek exposes here \textit{Język i~znaczenie} [\textit{Language and meaning}]
%\label{ref:RNDAAbyBFy26M}(Ajdukiewicz, 1960, eng. transl. 1978)
(\cite[][]{ajdukiewicz_jezyk_1960}, \cite*[Eng. transl.][]{giedymin_language_1978}). %
%\parencite[][Eng. transl. 1978]{ajdukiewicz_jezyk_1960} %
 was published in \textit{Erkenntnis}, journal of the Vienna Circle.

Chapter ten introduces the analysis of the concept of expression prepared by Maria Ossowska, one of many female members of the LWS. She visited England on a~few occasions to meet Russell, Moore and Malinowski. Brożek claims that Ossowska's paper is a~good example of the change of interest that happened inside the LWS from descriptive psychology toward semiotics.

Chapter eleven is dedicated to probably the most internationally recognizable member of the LWS, Alfred Tarski and his analysis of the concept of truth. As it has already been visible his work did not appear in vacuum but is part of the rich heritage of the LWS. His teachers at the University of Warsaw were Łukasiewicz, Leśniewski (his PhD supervisor) and Kotarbiński. He was actively engaged in the international community of logicians, visiting Vienna Circle or USA (during his visit in 1939 the WWII outbroke, Poland was invaded by Nazi Germany and he never came back to his homeland). Although he rarely engaged in philosophical problems, he made an exception for the analysis of the concept of truth. He believed there was a~lot of vagueness in the philosophical concepts, but unlike other members of the LWS he did not think this was unsolvable. Tarski assumed that scientists should be more trustful in the worth of the analysis and changes in the concepts they use for the good of scientific precision and progress.

Chapter twelve is dedicated to the analysis of the most fundamental concept in the logical semiotics that is the concept of sign. It was conducted by another woman among the LWS, Janina Kotarbińska, wife of Tadeusz Kotarbiński. She luckily survived the WWII (saved by the ``White Buses'' operation from the Ravensbrück concentration camp). She was interested in the theory of definition she claimed indispensable to make language a~better ``mirror of the reality''. Contrary to allegations that analytic philosophers ``sharpen logical tools'' in vain, she believed in the explanatory role of such practices. In this chapter Brożek presents her analysis of the concept of a~sign. Interestingly, Kotarbińska formulated analyses not of one but a~few concepts related to ``sign'' and offers definitions of various kinds of signs.

Chapter thirteen is dedicated to J.M. Bocheński's analysis of the concept of authority
%\label{ref:RND7b6Djctb4U}(cf. Bocheński, 1974).
\parencite[cf.][]{bochenski_analysis_1974}. %
 He was a~Dominican Friar and did not directly belong to the LWS (as he did not write a~PhD under the supervision of any of its members). He did, however, make contact with Łukasiewicz and other logicians from Warsaw. In the 1930s together with Rev. J. Salamucha and F. Drewnowski he established the Cracow Circle motivated to reform Catholic theology with modern tools of logic. He believed that criticism is useful to fight with superstitions (understood as unjustified and commonly accepted ideas that are not necessarily religious). Bocheński complained on the lack of adequate textbook to philosophical analysis and claimed that there were three types of analysis of concepts---Russell style, linguistic and categorical (supposed to be Polish specialty). As Brożek states he also complained that often the consequence of analysis of concepts and philosophical problems is its banalization 
%\label{ref:RNDMtedYp4nkl}(Brożek et al., 2020a, p.184).
\parencite[][p.184]{brozek_anti-irrationalism_2020}. %
 His famous analysis of the concept of authority is presented in stages together with the final distinction between the epistemic and deontic authority. However, Brożek suggests that his works lacks demarcation between objective and subjective aspects of authority. More on that can be found in \textit{Bocheński on authority} 
%\label{ref:RND3PQ7VXQMCi}(Brożek, 2013).
\parencite[][]{brozek_bochenski_2013}.%


The last chapter is devoted to Izydora Dąmbska, a~favorite student of Twardowski and his devoted assistant. During the communist Polish People's Republic, she was twice rejected from university for political reasons. It is claimed it was a~very painful experience for her as she was strongly dedicated to didactical work. Brożek focuses on her analysis of the concept of understanding. This concept is problematic. It is used in common language in various contexts. Dąmbska investigates those usages and tries to identify what unites them so that she can form a~definition.

The final part of the book is a~concise summary, where Brożek interestingly compares and contrasts all the analysis mentioned in the book. Although the matter is extensive the work is done very skillfully. Undoubtedly, it helps the reader to better appreciate the approach of the LWS to analysis. In chapter fifteen, Brożek concisely reports each authors' idea. Next, an important reflection is shared---that not many philosophers uncover the backstage of their work. Usually, we can observe the result, but the whole process itself is inspiring and significant. Brożek maintains that it is almost certain that a~lot of the analysis presented in the book arrived at another destination than initially assumed. That shows the beauty of sincere analysis and exposes the dispositions of the character that are necessary for the work of a~philosopher. Later, she intends to reconstruct the model for the analysis of concepts that can be treated as symptomatic for the whole LWS in three major stages: an apropriate choice of the corpus for the analysis, the analysis itself and construction of the definition. Finally, she presents how a~critique of the analysis could be conducted. Chapter sixteen offers a~final reflection on the role of the analysis of concepts in the methodology of philosophy. Here Brożek confronts the LWS with other modern analytic traditions. She decisively differentiates between the methods used in the LWS and by Moore's analysis of concept or Carnap's idea of explication. It might be especially interesting for the philosophers related to the Anglo-American analytic tradition as well as the whole international community. For instance, the LWS did not postulate ``naturalization of philosophy'' or understood logic as a~combination of formal logic, semiotics and methodology of science. Moreover, Brożek criticizes the disappointment with the analysis of concepts in philosophy and claims that there are neither simply natural languages nor pure formal languages. However, there are ``languages of various levels of ‘formality'''
%\label{ref:RND9IIhnI5UD0}(Brożek et al., 2020a, p.220).
\parencite[][p.220]{brozek_anti-irrationalism_2020}. %
 The good examples can be language of biology, chemistry or sociology. She also refers to the recent revival of the thought experiment and the procedure in the so-called experimental philosophy. Brożek maintains that they were already commonly used by the LWS.

In the summary of the book Brożek repeats parts of her reconstruction of the analysis of concepts used by the LWS and reaffirms her strong belief that although that method is not spectacular it is efficient.

Evidently, chapters like four (on ambition, W. Witwicki), seven (on happiness, W. Tatarkiewicz), nine (part related to concept of justice), thirteen (on authority, J.M. Bocheński) or fourteen (on understanding, I. Dąmbska) would be more accessible to beginners or people who are less interested in sophisticated semantical analysis. For sure, they can be recommended to the psychologists. Chapter eight (on analysis, T. Czeżowski) together with chapter fifteen seem to be a~valuable introduction into the idea of the analysis in the LWS. The book should interest academic teachers and students of philosophy as it makes a~great example of how theory and practice can be combined in the academic work. Researchers who did not go deeper into the problems of analysis in the LWS would find an interesting reconstruction with valuable remarks and commentaries. Due to competent introduction into the general history and ideas of the LWS and, later, short biographies of the key members one can get also an introduction to a~very challenging and intriguing part of history of Polish philosophy.

One drawback could relate to the layout used inside each chapter of part two. Each section is numbered but maybe a~better idea would be to give them also titles. Besides, sometimes the division into sections seems not necessary whereas in other cases the division would help to distinguish a~new issue. However, it does not unable to follow the main intention of the author.

Definitely, the book is recommendable especially to those who appreciate the beauty of clarity and investigations of the reality. It is an important contribution to the research on the Lvov-Warsaw School and in logical semiotics. Hopefully, the book will encourage more advanced readers, but not only them, to face the analysis of the authors presented in the book directly by reading their original works.




%---------------



\vspace{15mm}%
{\subsubsectit{\hfill Abstract}}\\
{Analytic description, according to members of the Lvov-Warsaw School (LWS) like Czeżowski, Ajdukiewicz, Ossowska, Tarski is a powerful and an indispensable tool, not only in philosophy but also in any natural science – in psychology especially. It should be equally respected together with empirical analysis and even it is recommended that it should precede any further research. Therefore, the book \textit{Analiza i konstrukcja: o metodach badania pojęć w Szkole Lwowsko-Warszawskiej} [\textit{Analysis and construction: on the methods of researching concepts in the Lvov-Warsaw School}] can be recommended to philosophers as well as scientists.}\par%
\vspace{2mm}%
{\subsubsectit{\hfill Keywords}}\\
{Lvov-Warsaw School, analysis of concepts, conceptual engineering, Alfred Tarski, Kazimierz Ajdukiewicz, Tadeusz Kotarbiński, Kaziemierz Twardowski.}%



\end{newrevengenv}
