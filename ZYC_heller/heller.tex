\begin{artplenv}{Michał Heller}
	{Józefa Życińskiego zmagania z~językiem o~Bogu}
	{Józefa Życińskiego zmagania z~językiem o~Bogu}
	{Józefa Życińskiego zmagania z~językiem o~Bogu}
	{Centrum Kopernika Badań Interdyscyplinarnych}
	{Joseph Życiński's struggle with the language about God}
	{In the two-volume work \textit{Theism and the Analytical Philosophy} \parencites*{zycinski_teizm_1985}{zycinski_teizm_1988} Joseph Życiński took up the challenge of renewing Christian metaphysics so that it could appear as a~full-fledged partner in the dialogue with other streams of contemporary philosophy. This renewal should use two sources: the methodological principles of analytic philosophy, especially its philosophy of language, and certain elements of Whitehead's process philosophy. This study presents a~critical reconstruction of Życiński's arguments contained in the first two chapters of \parencite*[][]{zycinski_teizm_1985}, which are devoted to the problem of language. Main results of this part of Życiński's work are negative, that is, they refute the arguments and interpretations of those analytical philosophers who show the meaninglessness of the theistic language or try to assimilate it to other standard languages, depriving it of a~reference to the transcendent reality.
	
	How can a~positive part of the Życiński program be developed? It seems that only by formulating specific problems in the field of philosophy of God, or even theology, and choosing the right linguistic tools to drill down on a~given problem and seek its solution. This is in line with Wittgenstein's concept of language games. Życiński tries to do this in \parencite*[][]{zycinski_teizm_1988}.
	
	Życiński turned out to be a~precursor of nowadays increasingly developing analytical theology.}
	{philosophy of language, process philosophy, analytical theology, language game, Józef Życiński, Ludwig Wittgenstein, Alfred North Whitehead, Willard Van Orman Quine.}



\section{Wstęp}
\lettrine[loversize=0.13,lines=2,lraise=-0.01,nindent=0em,findent=0.2pt]%
{W}{}~twórczości Józefa Życińskiego trudno jest nakreślić ostrą linię graniczną pomiędzy artykułem naukowym, esejem i~prasową polemiką. Gdy autora ponosi pisarski temperament (a zdarza się to często), elementy tych stylów mieszają się ze sobą, tworząc całość z~daleka rozpoznawalną jako charakterystyczny dla niego ,,rodzaj literacki''. Życiński jest jednak także głębokim filozofem, choć nie zawsze łatwo jest spod jego stylu wydobyć oryginalną myśl. Zapewne z~tego powodu jest on bardziej znany jako pisarz-polemista i~działacz na polu społeczno-kościelnym niż jako filozof, a~jego praktycznie jedyna monografia naukowa\footnote{Może obok \textit{Struktury rewolucji metanaukowej}
%\label{ref:RNDg6b31UyyUq}(Życiński, 2013).
\parencite[][]{zycinski_struktura_2013}. %
 Oryginał w~języku angielskim 
%\label{ref:RND0j27544F0Z}(Życiński, 1988b).
\parencite[][]{zycinski_structure_1988}.%
}, dwutomowy \textit{Teizm i~filozofia analityczna} 
%\label{ref:RND9g3CZUsL5Z}(Życiński, 1985, 1988a),
\parencites{zycinski_teizm_1985}[][]{zycinski_teizm_1988}, %
 przeszła właściwie niezauważona. Myślę, że jesteśmy winni jego pamięci, by dzieło to ,,ocalić od zapomnienia'', tym bardziej, że kryją się w~nim nowatorskie myśli, które nie tylko warto przypomnieć, ale także twórczo rozwijać.

Zamiarem Życińskiego, gdy pisał to dzieło, było -- ni mniej, ni więcej -- tylko odnowienie, czy uwspółcześnienie, metafizyki, na której mogłaby się oprzeć chrześcijańska filozofia i~teologia. Odnowienie to, jego zdaniem, powinno wykorzystać dwa źródła: dorobek współczesnej filozofii analitycznej i~pewne elementy filozofii procesu Whiteheada. Pierwsze z~tych źródeł miałoby wprowadzić do myśli chrześcijańskiej element ścisłości i~krytycyzmu (temu poświęcony jest pierwszy tom \textit{Teizmu}); drugie wypełnić metodę analityczną niezbędną treścią ontologiczną. \textit{Teizm i~filozofia analityczna} zawiera dość szczegółowo zarysowany projekt i~wizję całości. Niestety, Życiński nigdy bardziej systematycznie nie wrócił do tego tematu. Obowiązki biskupie pozwalały mu na pracę filozoficzną jedynie ,,z doskoku''.

Wobec ciążącego na nas obowiązku przypomnienia filozoficznego dorobku Józefa Życińskiego, cel niniejszego opracowania jest raczej skromny. Pragnę wziąć na warsztat jedynie dwa podrozdziały pierwszego rozdziału pierwszego tomu \textit{Teizmu}
%\label{ref:RND8afvfEyW6I}(Życiński, 1985, s.~13–36),
\parencite[][s.~13–36]{zycinski_teizm_1985}, %
 poświęcone problemowi języka, i~dokonać rekonstrukcji przebiegu rozumowań (argumentacji) autora, a~tym samym wyraźnie sformułować wnioski, do jakich jest uprawniony. Nie jest to zadanie trywialne, ponieważ i~tu Życiński niekiedy daje się ponieść literackiej swadzie, a~jego własne argumenty są wplatane w~liczne odwoływania się do poglądów innych autorów i~polemik między nimi. W~efekcie czytelnik otrzymuje efektowny pejzaż różnych opinii oraz autorskich komentarzy i~trzeba niemałego wysiłku i~uwagi, by wyłowić nić logicznych powiązań, która prawdopodobnie autorowi wydawała się oczywistą.
 \enlargethispage{1.5\baselineskip}

\section{Problem}
Problem, z~jakim Życiński pragnie się zmierzyć w~tych dwu podrozdziałach, można ująć w~postaci następującego pytania: Czy język teizmu może być sensowny? W~związku z~tym pytaniem nasuwają się dwie uwagi:

Pierwsza: z~przebiegu dyskusji, jaką w~dalszym ciągu prowadzi Życiński, wynika, że chodzi również o~język metafizyki (nie tylko teizmu), ponieważ teizm -- jak zobaczymy -- jakąś metafizykę zakłada.

Druga: należy zapytać, o~jaką sensowność chodzi. Odpowiedź na to pytanie nie jest trudna -- taką, o~jakiej mówi się we współczesnej filozofii języka. W~pierwszej połowie lat osiemdziesiątych zeszłego stulecia, wtedy, gdy książka była pisana, filozofię języka kształtowała filozofia analityczna, wywodząca się z~empiryzmu logicznego Koła Wiedeńskiego. Wpływy samego Koła Wiedeńskiego były wówczas znacznie mocniejsze niż obecnie. W~kontekście tak rozumianej filozofii języka pytanie o~sensowność języka teizmu (czy ogólniej -- metafizyki, a~w dalszej perspektywie również teologii) jest szczególnie doniosłe, gdyż bezpośrednie stosowanie standardowych kryteriów sensowności, wyznawanych przez empiryzm logiczny, do języka teizmu prowadzi wprost do odmówienia mu sensowności. Celem dzieła Życińskiego jest więc stworzenie takiego języka, który mógłby być językiem teizmu, ale równocześnie korzystałby w~jak największym stopniu z~narzędzi analizy językowej wypracowanych przez współczesną filozofię języka -- stworzenie takiego języka, lub przynajmniej ukazania kierunku, w~jakim należy zmierzać, by taki język stworzyć. Życińskiemu przyświeca więc cel apologetyczny -- dostarczenie filozofii Boga takiego wyrazu językowego, w~którym mogłaby się ona stać pełnoprawnym partnerem dialogu na scenie współczesnych kierunków filozoficznych.

\section{Język dyskursu}
Życiński rozpoczyna swoje analizy od wyróżnienia w~nurcie analitycznym dwu głównych koncepcji języka:

\begin{enumerate}
\item Koncepcja empiryzmu logicznego -- według niej istnieje ,,radykalne cięcie między naukowymi i~sensownymi zdaniami, które można sprawdzić empirycznie i~między pozbawionymi sensu czysto ekspresywnymi zdaniami poezji, metafizyki czy filozofii Boga''
%\label{ref:RNDNjJT74WuYy}(Życiński, 1985, s.~13).
\parencite[][s.~13]{zycinski_teizm_1985}. %
 Koncepcja ta nawiązuje do poglądów ,,pierwszego Wittgensteina'', wyrażonych w~jego \textit{Traktacie logiczno-filozoficznym}. Chcąc tworzyć sensowny język teizmu, nie można oczywiście podążać tą drogą.
\item Konkurencyjna w~stosunku do poprzedniej, koncepcja ,,gier językowych'', pochodząca od ,,drugiego Wittgensteina'', głównie z~jego \textit{Dociekań filozoficznych}. Zdaniem Wittgensteina w~języku należy wyróżnić strukturę powierzchniową i~strukturę głęboką. Struktury powierzchniowej można nauczyć się ze słownika i~podręcznika gramatyki danego języka. Strukturę głęboką można opanować jedynie ,,grając danym językiem'', czyli posługując się nim w~codziennych sytuacjach. Każdy język ma swoją własną ,,grę językową''. Życiński w~tym miejscu (ani gdzie indziej w~tej książce) nie wyjaśnia dokładniej, na czym polega koncepcja gier językowych. Najwidoczniej zakłada, że czytelnik powinien to wiedzieć. Zaznacza tylko, swoim dziennikarskim stylem, że ,,w interpretacji tej za bezsensowne uznane zostały próby poszukiwania idealnego języka. Uznano je za nierealistyczne i~bezcelowe, choćby z~tej racji, iż nawet w~różnych działach fizyki możliwości empirycznej weryfikacji poszczególnych twierdzeń są bardzo zróżnicowane''
%\label{ref:RNDIhZyh7H3j1}(Życiński, 1985, s.~13).
\parencite[][s.~13]{zycinski_teizm_1985}.%

\end{enumerate}
Istnieje oczywiście wiele wariantów tych dwu koncepcji, niektóre z~nich pojawią się w~dalszych rozważaniach, zwłaszcza przy referowaniu poglądów różnych autorów.

Już na tym etapie rozważań łatwo się domyślić, że myśl Życińskiego będzie zmierzać w~kierunku języka teizmu jako właściwej mu gry językowej. Jednakże poprzestanie na tym stwierdzeniu oznaczałoby właściwie zaniechanie tego tematu i~pozostawienie go w~stanie, w~jakim obecnie się znajduje, wzbogaconym tylko o~formułkę ,,gry językowe''. Posługując się tak ogólnikowym rozumieniem koncepcji gier językowych, można by uzasadniać dowolny żargon jako język jakiejś ,,nauki''. Życiński podkreśla: ,,W perspektywie takiej teorii języka wiary można by szukać usprawiedliwienia również dla irracjonalno-fideistycznych postaw wobec Absolutu  [\ldots]''
%\label{ref:RNDZhNk70QFIX}(Życiński, 1985, s.~17).
\parencite[][s.~17]{zycinski_teizm_1985}. %
 Należy więc dążyć do takiej koncepcji gry językowej, w~której określone byłyby ,,specyficzne reguły sensowności dyskursu filozofii Boga i~precyzowane warunki metodologiczne jego poprawności'' 
%\label{ref:RND2jsKjJJhA4}(Życiński, 1985, s.~14).
\parencite[][s.~14]{zycinski_teizm_1985}.%


Następuje teraz dłuższy wywód na temat, czy sam Wittgenstein uznawał istnienie Boga i~ewentualnie, w~jakim sensie. Swoim zwyczajem Życiński rozgrywa ten problem, przytaczając opinie różnych autorów. Faworyzuje przy tym opinie, które przychylają się do teistycznych interpretacji poglądów Wittgensteina. Oczywiście -- dodajmy od siebie -- to, czy Wittgenstein był teistą, czy nie, nie ma istotnego znaczenia dla interesującego nas problemu, to znaczy poszukiwania właściwego języka o~Bogu. Nie znaczy to jednak, że taka dyskusja jest nie na miejscu. Podczas niej padają bowiem argumenty, które już takie znaczenie mają.

I~tak z~dyskusji tej wynika ,,zasadnicza kwestia, którą trzeba rozstrzygnąć w~teorii gier językowych''. Można ją sformułować w~postaci następującego pytania: ,,Czy poza pojętym szeroko językiem wiary, językiem, którego wyrazem są zarówno słowa modlitw, gesty kulturowe, jak i~postawy etyczne, istnieje nieredukowalna do tych elementów rzeczywistość transcendentna opisywana w~dyskursie religii?''
%\label{ref:RNDNxP1BFGqNF}(Życiński, 1985, s.~16).
\parencite[][s.~16]{zycinski_teizm_1985}. %
 Takie sformułowanie ,,zasadniczej kwestii'' wymaga komentarza. Analizując język, nie można wyjść poza język i~stwierdzić, czy to, o~czym język mówi, istnieje, czy nie. A~więc nie można tego pytania rozstrzygnąć ,,w teorii gier językowych''\footnote{Język matematyki może bardzo precyzyjnie opisywać jakieś zjawisko fizyczne, na przykład kwanty pewnego pola, ale aby stwierdzić, czy kwanty te istnieją jako cząstki fizyczne, niezbędny jest eksperyment, a~więc wyjście poza język.}. Można co najwyżej, wiedząc skądinąd, że dany przedmiot istnieje, \textit{przyporządkować} mu pewne wyrażenie językowe. Jest to oczywiście wyjście poza język do problemu istnienia, a~więc do problemu ontologicznego. Życiński jest tego świadomy, chociaż w~jego wypowiedziach granica między ontologią a~filozofią języka niekiedy się zaciera. Przykładem, w~którym tej nieostrości nie ma, jest następujący cytat: ,,[\ldots] trzeba odpowiedzieć na pytanie, do jakiej rzeczywistości pozajęzykowej odnosi się jej [danej teorii gier językowych] dyskurs i~jakie \textit{pozjęzykowe implikacje ontologiczne} można mu przyporządkować'' 
%\label{ref:RNDQzeO9Ewi5e}(Życiński, 1985, s.~18)
\parencite[][s.~18]{zycinski_teizm_1985}%
[podkreślenia Życińskiego].

To rozróżnienie warstwy językowej i~pozajęzykowej nie przeszkadza jednak Życińskiemu mówić o~,,ontologicznych implikacjach teorii \textit{language--game}''
%\label{ref:RNDVnNa25b5wL}(Życiński, 1985, s.~18).
\parencite[][s.~18]{zycinski_teizm_1985}. %
 Wobec niemożności wyjścia poza język, jak więc należy rozumieć ,,ontologiczne implikacje języka''? Istnieje tylko jedna odpowiedź na to pytanie -- może tu być mowa jedynie o~ontologii w~sensie Quine'a. Willard Van Orman Quine w~słynnym eseju ,,O tym, co istnieje'' 
%\label{ref:RNDKWImJoM3iS}(Quine, 1969b)
\parencite[][]{quine_o_1969} %
 rozpatrywał kwestię ,,ontologicznych zaangażowań języka''. Zaangażowania takie mają miejsce, gdy badając język, przeprowadzamy analizę nie po to, ,,by dowiedzieć się, co istnieje, lecz po to, by dowiedzieć się, co dana wypowiedź lub teoria -- nasza, czy też sformułowana przez kogoś innego -- \textit{uznaje} za istniejące''
%\label{ref:RNDawgpbqnxqQ}(Quine, 1969b, s.~29).
\parencite[][s.~29]{quine_o_1969}. %
 Innymi słowy, tego rodzaju ontologiczna analiza języka zmierza do tego, by wyodrębnić minimum bytów, założenie istnienia których jest niezbędne, by zagwarantować sensowność danej wypowiedzi. Na przykład wiem, że Pegaz nie istnieje\footnote{Mam oczywiście na myśli Pegaza -- mitycznego konia, a~nie szpiegowskie urządzenie elektroniczne o~tej nazwie, które na pewno istnieje (w sensie ontologicznym).}, ale jeżeli mówię coś o~Pegazie i~jeżeli moja wypowiedź ma mieć sens, to musi ona traktować Pegaza jakby istniał naprawdę. Dla niej (dla wypowiedzi), nie dla mnie, Pegaz istnieje. Moja wypowiedź jest zaangażowana w~istnienie Pegaza.

W~tym kontekście Życiński nie wspomina Quine'a, ale z~całego przebiegu jego argumentacji jasno wynika, że od języka teizmu wymaga on czegoś więcej niż zaangażowania tego języka (w sensie Quine'a) w~istnienie Boga, a~więc przyjęcia istnienia Boga jedynie ,,na potrzeby tego języka''; chodzi mu o~istnienie Boga jako rzeczywistości pozajęzykowej. Kilka stronic dalej Życiński polemizuje z~poglądem W. Zuurdeega, który utrzymywał, że ,,w filozofii analitycznej niemożliwe jest postawienie kwestii prawdziwości sformułowań tego języka [chodzi o~język teizmu]. Co najwyżej można tylko stwierdzić, iż sformułowania te odnoszą się do swoistej ‘rzeczywistości' uznawanej przez grupę posługującą się tym językiem''
%\label{ref:RNDllG9T9w9jf}(Życiński, 1985, s.~28).
\parencite[][s.~28]{zycinski_teizm_1985}. %
 Zauważmy, że jest tu mowa o~,,rzeczywistości'' zrelatywizowanej do języka, jakim posługuje się dana grupa. Jest to więc w~gruncie rzeczy ontologia w~sensie Quine'a. ,,W ujęciu takim -- pisze Życiński -- uderza przede wszystkim \textit{zrównanie statusu wszystkich przekonań} i nadanie im charakteru pozaracjonalnego'' 
%\label{ref:RNDbOI2dlR9pE}(Życiński, 1985, s.~28).
\parencite[][s.~28]{zycinski_teizm_1985}. %
 Jest to wymowny dowód tego, że w~dyskursie religijnym Życiński nie zadowala się ontologią w~sensie Quine'a.

A~więc, według niego, w~rozważaniach o~języku teizmu analizy językowe (w duchu filozofii analitycznej) należy uzupełnić o~składową (autentycznie, nie w~duchu Quine'a) metafizyczną. Chcąc jakoś usprawiedliwić ten istotny ,,dodatek'', Życiński polemizuje z~autorami, którzy Wittgensteinowskie gry językowe, związane z~językiem o~Bogu, interpretują ściśle wedle reguł sensu logicznego pozytywizmu.


\begin{center}
* * *
\end{center}


Wnioski, jakie Życiński wyprowadza z~analiz, przeprowadzonych w~tym podrozdziale, są oszczędne:

\begin{enumerate}
\item ,,[\ldots] odmienne koncepcje teizmu można łączyć z~Wittgensteinowską teorią gier językowych''
%\label{ref:RNDdyHF2lyICY}(Życiński, 1985, s.~22).
\parencite[][s.~22]{zycinski_teizm_1985}. %
 Ściślej byłoby: ,,z różnymi wersjami Wittgensteinowskich teorii gier językowych''.
\item Czynnikiem decydującym o~wyborze którejś z~koncepcji teizmu ,,jest podzielana koncepcja języka wiary oraz koncepcja metafizyki''
%\label{ref:RND4GmVMrqeTy}(Życiński, 1985, s.~22).
\parencite[][s.~22]{zycinski_teizm_1985}. %
 Tu konieczne jest następujące uściślenie.
\end{enumerate}
W~tym podrozdziale kilkakrotnie pojawia się określenie ,,język wiary''. Częściej jednak Życiński mówi po prostu o~,,języku teizmu'' albo o~,,języku o~Bogu''. Proponowałbym, aby uwypuklić to rozróżnienie: z~jednej strony ,,język dyskursu o~Bogu'' lub ,,język filozofii Boga'' i~z~drugiej ,,język wiary''. Na początku następnego podrozdziału Życiński wyjaśnia: 
\myquote{
Wyrażenie ,,język wiary'' ma szerszy zakres niż wyrażenie ,,język filozofii Boga''. Jakkolwiek Bóg filozofii i~Bóg wiary jest tym samym Bogiem, to jednak w~pierwszym przypadku jest On poznawany wyłącznie za pośrednictwem inferencji logicznych. W~przypadku drugim jawi się On jako Osoba, w~stosunku do której możliwa jest nie tylko refleksja filozoficzna, lecz także ułatwiane przez prawdy objawione przeżycie fascynacji, miłości, czci, czy zależności egzystencjalnej
%\label{ref:RNDkmZjm3QQpP}(Życiński, 1985, s.~22).
\parencite[][s.~22]{zycinski_teizm_1985}.
}
 W~analizowanym podrozdziale Życiński mówi w~zasadzie o~języku filozofii Boga, choć określenie ,,język wiary'' pojawia się kilkakrotnie i~nie zawsze granica pomiędzy nimi jest wyraźna. Powyższy wniosek (2) należy odnieść do języka dyskursu o~Bogu, a~nie do języka wiary. Nie znaczy to, że podobnego wniosku nie można wyciągać również dla języka wiary, ale wydaje się, że w~tym miejscu nie było to intencją autora.

Nasuwa się wreszcie uwaga co do sposobu, w~jaki Życiński wykorzystał poglądy Wittgensteina. Zrobił to w~sposób mocno wybiórczy, czerpiąc z~nich tylko to, co stanowiło podatny materiał do budowania zrębów własnego systemu (taki jest niewątpliwie cel obydwu tomów \textit{Teizmu}). Oczywiście, każdy autor ma prawo korzystać z~dorobku innych myślicieli w~sposób, jaki mu odpowiada, ale pewnego rodzaju lojalność wobec swojego ,,źródła'' wymagałaby, by nie sprawiać wrażenia, że ,,źródło'' we wszystkim zgadza się z~autorem.

\section{Język wiary}
Drugi podrozdział nosi tytuł ,,Specyfika języka wiary''. Życiński omawia w~nim i~poddaje krytyce ,,współczesne dyskusje z~zakresu teorii języka'', które ,,znalazły swe odbicie w~nowych propozycjach dotyczących zasadniczej reinterpretacji języka wiary''
%\label{ref:RNDtxEj2Fkqf7}(Życiński, 1985, s.~22–23).
\parencite[][s.~22–23]{zycinski_teizm_1985}. %
 Zważywszy antyreligijne nastawienie większości ówczesnych filozofów analitycznych, nie dziwi fakt, iż na ogół propozycje te zmierzają do takich interpretacji, które by ukazywały bądź bezsensowność, bądź jedynie ekspresywność języka wiary.

Wśród tych prób są i~takie, które usiłują tak przykroić język religijny, aby w~jakimś stopniu spełniał on neopozytywistyczne kryteria. Środkiem do tego celu byłoby tego rodzaju przeformułowanie zasady weryfikacji, które by odnosiło się do eksperymentalnych przejawów religijności, na przykład do ,,specyficznego ujmowania świata innych ludzi i~osobistego życia na podstawie przesłanek religijnych''
%\label{ref:RND60YbS77un8}(Życiński, 1985, s.~23),
\parencite[][s.~23]{zycinski_teizm_1985}, %
 lub wręcz ,,redukcja języka wiary do języka obserwowanych zachowań'' 
%\label{ref:RNDPw2E8vgArm}(Życiński, 1985, s.~25)
\parencite[][s.~25]{zycinski_teizm_1985} %
 (P. van Buren).

Niejako kontrprzykładem w~stosunku do tego rodzaju redukcjonizmu jest, zdaniem Życińskiego, fideizm, przejawiający się w~poglądach D.Z. Phillipsa, który będąc człowiekiem religijnym, utrzymywał, że ,,filozofia nie jest ani za, ani przeciw wierzeniom religijnym. Jej rola kończy się na wprowadzaniu porządku do gramatyki tych wierzeń''
%\label{ref:RND6f82IiMwGx}(Życiński, 1985, s.~26).
\parencite[][s.~26]{zycinski_teizm_1985}.%


Zauważmy, że Życiński stosuje tu swoją ulubioną strategię: chcąc zneutralizować poglądy jakiegoś myśliciela (van Buren), przeciwstawia mu poglądy innego autora (Phillips), umiejętnie przy tym rozkłada akcenty w~ten sposób, żeby opinia, której nie sprzyja, okazała się fałszywa lub przynajmniej nieatrakcyjna.

Nie wnikając w~szczegóły (jest ich naprawdę dużo), dokonajmy pobieżnego zestawienia poglądów, z~którymi Życiński polemizuje. Są to:

\begin{itemize}
\item Wspomniane wyżej poglądy van Burena: modyfikacja zasady weryfikacji, redukcja języka wiary do języka obserwowalnych zachowań.
\item Koncepcja ,,weryfikacji eschatologicznej'', ,,według której empiryczna weryfikacja teizmu jest możliwa i~dokonuje się po biologicznej śmierci''
%\label{ref:RNDWV6W6lHoj2}(Życiński, 1985, s.~27)
\parencite[][s.~27]{zycinski_teizm_1985} %
 (J. Hick).
\item Wspomniane wyżej poglądy W. Zuurdeega, w~zasadzie powtarzające klasyczne zarzuty neopozytywistów pod adresem języka wiary, związane z~kryterium sensu.
\item ,,Interpretacja języka wiary w~kategoriach autorytetu i~posłuszeństwa'' (A. MacIntyre). ,,U podstaw tej koncepcji znajduje się założenie, iż dowolne wierzenia religijne usprawiedliwiamy jako całość, ‘odnosząc je do autorytetu'''
%\label{ref:RNDOO7e4RnEgC}(Życiński, 1985, s.~29).
\parencite[][s.~29]{zycinski_teizm_1985}.%

\item Redukowanie języka wiary do języka etyki (R.B. Braithwaite).
\item ,,Próby redukowania funkcji języka wiary do ekspresji postawy czci, kultu i~szacunku, opisu doświadczeń numinotycznych czy przeżywania \textit{sacrum} jako wartości''
%\label{ref:RNDkZXwzfUKfz}(Życiński, 1985, s.~31).
\parencite[][s.~31]{zycinski_teizm_1985}.%

\end{itemize}
Ogólne konkluzje, jakie Życiński wyciąga z~tej obszernej dyskusji, są następujące:

\begin{enumerate}
\item ,,[\ldots] trzeba definitywnie odrzucić możliwość zredukowania [języka wiary] do jego innych poziomów bez uwzględniania jego składowej ontycznej odnoszącej się do rzeczywistości transcendentnej''
%\label{ref:RNDOb81h2f9eK}(Życiński, 1985, s.~32).
\parencite[][s.~32]{zycinski_teizm_1985}.%

\item ,,Ten specyficzny wymiar [ontyczny] dyskursu teizmu sprawia, iż w~języku wiary pojawiają się nowe uwarunkowania epistemologiczne. Zarówno status empiryczny, jak i~kryteria racjonalności tego języka są częściowo różne od analogicznych cech innych języków''
%\label{ref:RNDohpzsSxGqT}(Życiński, 1985, s.~32).
\parencite[][s.~32]{zycinski_teizm_1985}.%

\item ,,Odmienność ta nie przesądza \textit{a~priori} niczego o~wartości dyskursu teistycznego'', także ,,nie implikuje niemożności stosowania do teizmu zwykłych kryteriów racjonalności''
%\label{ref:RNDOla1PvKxs8}(Życiński, 1985, s.~32).
\parencite[][s.~32]{zycinski_teizm_1985}.%

\end{enumerate}
Jak wiadomo, neopozytywiści uznawali tylko dwa rodzaje sensownych wypowiedzi: tautologie i~zdania empiryczne. Co więcej, podział ten jest, według nich, dychotomiczny. Zdania, które nie są ani tautologiami, ani zdaniami empirycznymi, tworzą zbiór zdań bezsensownych, ,,zawierający jako podzbiór właściwy klasę zdań metafizyki i~teologii''
%\label{ref:RNDRYbdwlO4BK}(Życiński, 1985, s.~34).
\parencite[][s.~34]{zycinski_teizm_1985}. %
 Odpowiadając na ten zarzut, Życiński, całkiem słusznie, przywołuje pracę Quine'a, który podważył tę dychotomiczność 
%\label{ref:RNDTe7eaxGMW0}(Quine, 1969a),
\parencite[][]{quine_dwa_1969}, %
 ale powołuje się także na drugi argument, a~mianowicie na ,,brak ścisłej i~przyjmowanej powszechnie definicji zdań analitycznych'' 
%\label{ref:RNDKNK8VSCwiC}(Życiński, 1985, s.~34).
\parencite[][s.~34]{zycinski_teizm_1985}. %
 Na dowód tego przytacza listę aż dziewięciu różnych tego rodzaju definicji. Jego zdaniem, ukazuje to ,,nierealistyczny charakter propozycji Koła Wiedeńskiego'' 
%\label{ref:RNDhr9x1YvHek}(Życiński, 1985, s.~34).
\parencite[][s.~34]{zycinski_teizm_1985}. %
 Jest to jednak argument nietrafny, ponieważ zdanie ,,Bóg istnieje'' nie jest ani zdaniem empirycznym, ani tautologią w~żadnym z~tych dziewięciu znaczeń. Co więcej, wobec trafności argumentacji Quine'a, jest to argument zbyteczny.

\section{Wnioski i~dalszy rozwój}
Pamiętajmy, jest rok 1985 (rok wydania pierwszego tomu \textit{Teizmu}). Życiński proponuje reformę języka o~Bogu, a~w perspektywie całej teologii, w~ten sposób, by mogła ona przemówić do tej części społeczeństwa, która mówi językiem kształtowanym przez nauki ścisłe. Jest to krok nowatorski i~prekursorski. Teologia katolicka już raz niedawno zdobyła się na językową transformację: po Soborze Watykańskim II porzuciła język scholastyki na rzecz języka kształtowanego przez różne filozofie typu egzystencjalno-fenomenologicznego i~nauki humanistyczne, ale pozostała całkowicie obojętna na falę przemian wywołaną stopniową ewolucją empiryzmu logicznego w~różne odmiany filozofii analitycznej. Wkrótce ta fala ogarnęła cały świat anglojęzyczny i~zaczęła sobie torować drogę w~Europie. Życiński, jako jeden z~pierwszych, i~na pewno pierwszy w~Polsce, dostrzegł wagę problemu. Dziś, po trzydziestu pięciu latach, na naszych oczach rodzi się nowy kierunek w~teologii -- teologia analityczna. Wprawdzie dzieje się to niezależnie od prekursorskich projektów Życińskiego, ale jego praca nie zdezaktualizowała się, mieści się w~niej bowiem szereg elementów, na które do dziś nie zwrócono baczniejszej uwagi. Żeby je dostrzec, spójrzmy jeszcze raz krytycznie na wyniki naszych analiz.

Jak widzieliśmy, Życiński wyróżnił język dyskursu teistycznego i~język wiary, choć w~trakcie swoich rozważań nie zawsze dość rygorystycznie przestrzegał tego rozróżnienia. W~teologii praktyka religijna odgrywa ważną rolę: jako \textit{locus theologicus} może być ona źródłem teologicznej argumentacji. Nie należy więc pomniejszać znaczenia analiz języka wiary, ale trzeba mieć również na uwadze, że dla projektu reformy języka teologicznego pierwszorzędne znaczenie ma przede wszystkim laboratoryjna praca nad językiem dyskursu.

Główna część tej pracy, jaką wykonał Życiński, ma charakter negatywny, to znaczy obala ona argumenty i~interpretacje tych analityków, którzy wykazują bezsensowność języka teistycznego lub starając się ,,uanalitycznić'' ten język, pozbawiają go odniesienia do rzeczywistości transcendentnej. Nie jest to robota błaha: sama w~sobie ma charakter analityczny i~toruje drogę do bardziej konstruktywnej części projektu.

Powstaje pytanie: w~jaki sposób można wypracować pozytywną część programu Życińskiego? Wydaje się, że jedynie w~konkretnej pracy wykonywanej tym językiem, to znaczy formułując konkretne problemy z~zakresu filozofii Boga, lub wręcz teologii i~dobierając właściwe narzędzia językowe, drążyć dany problem i~poszukiwać jego rozwiązania. Zgodnie z~tezą Wittgensteina, trzeba tak długo grać danym językiem, aż się on wykrystalizuje. Życiński próbuje to robić w~drugim tomie \textit{Teizmu}
%\label{ref:RNDchPzveDuKZ}(Życiński, 1988a),
\parencite[][]{zycinski_teizm_1988}, %
 w~którym stara się wypracować zręby chrześcijańskiej metafizyki opartej na filozofii Whiteheada. Inna sprawa, czy mu się to udało i~w jaki sposób ewentualnie kontynuować jego przedsięwzięcie.

Takie postawienie sprawy jest w~pełni zgodne z~innym, bardzo doniosłym wnioskiem, jaki wyłania się z~negatywnej części programu Życińskiego (sam ten wniosek już nie jest negatywny): wprawdzie analizy czysto językowe nie pozwalają wyjść do rzeczywistości pozajęzykowej, ale język dyskursu teistycznego musi być tego rodzaju, żeby nie wykluczał jego odniesienia do rzeczywistości transcendentnej czyli Boga. ,,Do rozstrzygnięcia tej kwestii konieczne jest przeanalizowanie epistemologicznego statusu metafizyki i~filozofii Boga''
%\label{ref:RND9NHjcmU8Xa}(Życiński, 1985, s.~36).
\parencite[][s.~36]{zycinski_teizm_1985}. %
 Ale to jest już przedmiotem następnego rozdziału pierwszego tomu \textit{Teizmu}.

Życiński pod wieloma względami okazał się prekursorem, zwłaszcza na terenie myśli polskiej, gdzie -- o~ile mi wiadomo -- nie było przed nim prób ,,uanalitycznienia'' filozoficznego systemu White\-heada i~wykorzystania go w~filozofii Boga. Z~czasem teologia dostrzegła możliwości, jakie kryją się w~metodach analitycznych. Zrobiła to na innej drodze niż proponował Życiński, ale efekt zmierza w~tym samym kierunku, jaki on proponował. Życiński zaczął od sformułowania projektu i~drążenia pytania, w~jakim stopniu metody analityczne są w~stanie sprostać wymaganiom teistycznego języka. Dzisiejsi teologowie po prostu biorą na warsztat konkretne problemy teologiczne i~starając się je rozwiązać, naśladują to, co analitycy robią ,,u siebie''
%\label{ref:RNDpFCEQTABHh}(Crisp, Rea, 2009).
\parencites[por. np.][]{crisp_analytic_2009}[pojawiła się także polska ,,jaskółka'':][]{holda_kochac_2021}. %
 Jeżeli po drodze pojawiają się jakieś problemy metafizyczne -- a~są one nieuniknione -- to trzeba sobie z~nimi także jakoś radzić. Oczywiście, nie brakuje również czynionej przy okazji refleksji metodologicznej. Jest to na pewno strategia owocna, gdyż metoda zawsze najlepiej wyostrza się w~działaniu.

Nie można jednak zapominać, że projekt Życińskiego był znacznie bardziej ambitny: najpierw przeorać fundamenty -- dokładnie rozeznać to, co analitycy już w~tej dziedzinie zrobili, także ich krytykę języka filozofii Boga i~teologii, wybrać z~tego i~rozwinąć to, co można odpowiednio przystosować do konstrukcji odnowionego języka teizmu, a~dopiero potem, przy pomocy takiego narzędzia, budować chrześcijańską metafizykę i~w dalszej perspektywie teologię. Projekt Życińskiego jest nadal wielkim wyzwaniem. Czy ktoś je podejmie?


%\begin{flushright}
%Tarnów, 20 sierpnia 2021 roku
%\end{flushright}


\end{artplenv}
