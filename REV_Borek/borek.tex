\begin{newrevplenv}{Michał Borek}
	{Duchowość bez ducha}
	{Duchowość bez ducha}
	{Duchowość bez ducha}
	{Uniwersytet Papieski Jana Pawła II w Krakowie}
	{Wesley J. Wildman, Kate J. Stockly, \textit{Spirit Tech: The Brave New World of Consciousness Hacking and Enlightenment Engineering}, St. Martin's Press, New York 2021, ss.~400.}





\lettrine[loversize=0.13,lines=2,lraise=-0.03,nindent=0em,findent=0.2pt]%
{W}{}~rozwiniętych krajach świata, pomimo ogólnego spadku zainteresowania i~uczestnictwa w~tradycyjnych, zinstytucjonalizowanych obrzędach religijnych, nie brak zaangażowania w~indywidualne praktyki duchowe. Taki indywidualny rozwój praktyk religijnych silnie wspomagany jest rozwojem technologii. Już szybki przegląd App Store ujawnia tysiące aplikacji, które pomogą nam odmawiać różaniec, utrzymać oddech w~czasie medytacji, okiełznać moc kryształów, rzucić zaklęcie, uczestniczyć w~wirtualnym darshanie, monitorować swoje czakry lub czytać dowolną liczbę świętych tekstów, w~tym Torę, Biblię, Koran, Bhagavad Gitę lub Guru Granth Sahib. Zagadnienie interakcji technologii i~duchowości postanowili zbadać w~swojej książce \textit{Spirit Tech: The Brave New World of Consciousness Hacking and Enlightenment Engineering}
%\label{ref:RND5uZxhYXFkk}(Wildman and Stockly, 2021)
\parencite*[][]{wildman_spirit_2021} %
 Wesley J. Wildman i~Kate J. Stockly.

Wesley J. Wildman jest wykładowcą filozofii, religii, teologii i~etyki na Uniwersytecie w~Bostonie. Jest członkiem International Society for Science and Religion. Jest również założycielem i~dyrektorem wykonawczym Centrum Umysłu i~Kultury, a~także jest zaangażowany w~program Liberal Evangelical Project. Poza projektami badawczymi związanymi z~duchowością i~psychologią, zajmuje się również polityką, zdrowiem publicznym, czy szeroko pojmowaną równością.

Natomiast Kate J. Stockly łączy badania z~dziedziny afektywnej neuronauki, kognitywistyki i~biologii ewolucyjnej, próbując konstruować biokulturowe teorie ucieleśnionego rytuału religijnego. Bada również różnice wynikające z~płci w~przeżywaniu religii i~duchowości z~perspektywy biopsychospołecznej.

Książka \textit{Spirit Tech\ldots} jest zbiorem badań, analiz, doświadczeń i~prognoz dotyczących interakcji zachodzących między technologią i~duchowością. Swoje dzieło autorzy traktują interdyscyplinarnie, dotykając w~nim zagadnień neuronauki, inżynierii medycznej, psychologii, religii i~etyki. Nie stronią od wielkich filozoficznych, psychologicznych i~etycznych pytań, w~sposób szczególny koncentrują się nad pytaniami o~autentyczność, znaczenie, bezpieczeństwo i~odpowiedzialność społeczną wynikającą z~mariażu technologii i~duchowości.

Z~psychologii poznawczej dowiadujemy się o~wbudowanych ludzkich tendencjach do wiary w~istoty nadprzyrodzone, ustabilizowanych w~długim procesie ewolucji. Jednak z~punktu widzenia neuronauk takie stwierdzenie jest niewystarczające i~domaga się pogłębionej analizy i~badań. Analizując doświadczenie duchowe, zauważamy neurologiczne efekty, które ono powoduje, a~które są możliwe do uzyskania poprzez odpowiednie technologie, substancje halucynogenne, czy rytuały.

Faktem jest, że historia religii pełna jest przykładów ludzi używających wszelkiego rodzaju materiałów, ćwiczeń, artystycznej ekspresji czy różnorakich substancji, aby przygotować swoje serca i~umysły dla duchowych doświadczeń. W~tradycji duchowości powszechne były mantry, muzyka, kadzidła, ikony, relikwie, taniec, posty itp.

Dzięki postępowi w~technologii, neuronauce i~psychologii, jesteśmy w~stanie zmierzyć takie doświadczenia, opisywać je i~badać. Potrafimy dziś ocenić ich funkcje społeczne, konsekwencje behawioralne, jak i~wpływ na zdrowie i~poczucie szczęścia. Jesteśmy nawet zdolni do wywoływania takich doświadczeń duchowych, hamowania ich lub wzmacniania.

Wildman i~Stockly nie zatrzymują się jednak na tradycyjnych bodźcach stymulujących doświadczenie duchowe, ale badają najnowsze trendy w~stymulacji mózgu. Prowadząc nas przez kolejne rozdziały poświęcone nowym technologiom, każdorazowo zaczynają od opisu osobistego doświadczenia uczestników badań, ich przeżyć oraz wpływu tych technologii na ich życie. Stanowią one wprowadzenie do części zdecydowanie bardziej technicznej i~naukowej. Tej części trudno jest cokolwiek zarzucić. Przeprowadzone analizy i~badania przedstawione są w~sposób bardzo konkretny i~rzeczowy. Autorzy nie uciekają od głębszych wyjaśnień oddziaływań i~funkcjonowania ludzkiego mózgu. Wiele uwagi poświęcają również opisom, poglądowym ilustracjom i~zdjęciom urządzeń i~stymulatorów, niezbędnych czytelnikowi w~pracy o~tak szerokim zakresie.

W~kolejnych rozdziałach autorzy analizują: stymulatory mózgu (bazujące na falach elektrycznych, magnetycznych\footnote{W~wybranych opisach działania stymulacji mózgu falami elektrycznymi i~magnetycznymi naukowcy stawiają hipotezę, że wystarczająco silne pola elektromagnetyczne bezpośrednio wpływają na ruch jonów, które przenoszą sygnały elektryczne przez neurony.}, świetlnych i~ultradźwiękowych\footnote{Możliwe, że wibracje ultradźwiękowe oddziałują na kanały jonowe potasu w~neuronach, wpływając na szybkość i~napięcie neuronów, albo powodują oscylacje w~dwóch warstwach ścian komórkowych, zwiększając napięcie, co wywołuje tymczasowe zmiany w~ścianach komórkowych, ponieważ warstwy te są rozciągane
%\label{ref:RNDYJUVSvEem7}(Wildman and Stockly, 2021, s.~37).
\parencite[][s.~37]{wildman_spirit_2021}.%
}), metodę neurofeedback (obrazującą aktywność mózgu w~EEG o~długościach fal delta, theta, alfa, beta i~gamma), inżynierię poczucia wspólnotowości, technologię VR, substancje psychotropowe (LSD, MDMA ecstasy, psylocybina, ayahuasca, kaktus San Pedro) oraz technologiczne kierownictwo duchowe. Ostatnie trzy rozdziały poświęcone są kryterium autentyczności, bezpieczeństwie i~perspektywach rozwoju technologii duchowości. Książka zawiera również bogaty indeks fotografii, diagramów, wykresów i~rysunków, a~także dwa dodatki: pierwszy dotyczący historii rozwoju technologii duchowości oraz drugi dotyczący sposobu wykonywania pomiarów i~opisu doświadczenia duchowego.

Autorzy dokonali bardzo subiektywnego wyboru opisanych technologii, określanych przez nich za najbardziej renomowane, najlepiej przebadane oraz mające największe szanse na rozpowszechnienie i~odciśnięcie swojego piętna na przyszłość technologii duchowości
%\label{ref:RNDg4bOehZpr7}(Wildman and Stockly, 2021, s.~250).
\parencite[][s.~250]{wildman_spirit_2021}. %
 Otrzymany jednak w~ten sposób zbiór robi wrażenie niezwykle przypadkowego. Poza tym ciężko utrzymać nawet takie kryteria autorów, umieszczających na liście powszechnie zakazane substancje psychotropowe. Trudno uznać je za renomowane czy przyszłościowe.

W~podejmowanych zagadnieniach w~poglądach autorów bije duży optymizm w~kwestii rozwoju i~stosowania w~przyszłości różnorakich nowinek technicznych mających wspierać rozwój duchowy. Przewidują nawet, że technologia duchowości będzie podążać za trendem podobnym do masowego ruchu wellness, wieszcząc jej nadzwyczaj wielką popularność.

Analizowane w~książce badania w~dużej części potwierdzają oddziaływanie starożytnych praktyk duchowych na strukturę mózgu, np. medytacja spowalnia jego starzenie i~zwiększa plastyczność
%\label{ref:RNDwoT8AC4PW2}(Xiong, Doraiswamy, 2009).
\parencite[][]{xiong_does_2009}. %
 Autorzy na różnych poziomach badawczych wykazują, że nowe formy technologii duchowości wydają się umożliwiać czerpanie tych samych korzyści zdrowotnych i~duchowych z~tradycyjnych praktyk, bez poświęcania im ogromnych nakładów czasu, energii czy kosztów 
%\label{ref:RNDjAkvZN0omy}(Bærentsen i~in., 2010; Hasenkamp, Barsalou, 2012; Xiong, Doraiswamy, 2009).
\parencites[][]{baerentsen_investigation_2010}[][]{hasenkamp_effects_2012}[][]{xiong_does_2009}. %
 Uzyskanie efektów, które kiedyś wymagały wspinania się po wysokich górach dla spotkania kierownika duchowego, a~następnie domagały się lat medytacji i~podporządkowania całego życia, teraz dostępne jest w~sposób bardzo prosty i~przystępny. Ukazanie rozwoju techniki w~tej materii i~potwierdzenie skuteczności i~korzyści płynących z~praktyk duchowych są niewątpliwą zaletą tej książki. Ostatnie osiągnięcia dodatkowo potwierdzają postęp, jaki poczyniliśmy wpływając technologią na funkcjonowanie ludzkiego mózgu. W~październiku świat obiegła wiadomość o~wyleczeniu 38-letniej Sarah z~ciężkiej depresji za pomocą eksperymentalnego implantu mózgu. Daje to wielką nadzieję osobom cierpiącym na nieuleczalne choroby psychiczne 
%\label{ref:RNDBVjgCJdNqy}(Wilson, 2021).
\parencite[][]{wilson_womans_2021}.%


Trzeba jednak zauważyć, że wyleczenie z~depresji, stymulowanie mózgu dla osiągnięcia pewnych korzystnych zmian w~jego strukturze, czy nawet wywołanie jakiegoś doświadczenia utożsamianego z~doświadczeniem duchowym jest w~tych przypadkach celem samym w~sobie. Autorzy jednak bardzo często idą o~krok dalej, używają tych terminów synonimicznie z~pojęciem duchowości. O~ile istnieją pewne przesłanki do takiego podejścia w~kontekście religii, w~których pojęcie Boga osobowego odgrywa drugorzędne (bądź nawet żadne) znaczenie, o~tyle w~kontekście pozostałych religii wydaje się to nadużyciem. Należy pamiętać, że zmiany w~strukturze mózgu czy subiektywne doświadczenia nie są ani celem duchowości, ani jej centralnym punktem.

Ten problem wybrzmiewa również w~kwestii autentyczności i~weryfikacji doświadczenia duchowego. Oceniając autentyczność naszych doświadczeń, koncentrujemy się na jakości doświadczenia, jego skutkach i~przyczynach. Autorzy akurat w~kwestii przyczyn doświadczenia duchowego zachowują dużą pokorę i~pomimo przeprowadzonych tak obszernych i~dokładnych badań zauważają:

\myquote{
Wiemy, że mózg pośredniczy we wszystkich doświadczenia, ale procesy neurologiczne są tak skomplikowane, że nie jesteśmy w~stanie powiedzieć, jak zaczyna się każde doświadczenie
%\label{ref:RNDiFKgkD2oqz}(Wildman and Stockly, 2021, s.~230).
\parencite[][s.~230]{wildman_spirit_2021}%
\footnote{,,We know the brain mediates all experiences, but neurological processes are so intricate that we can't really tell how any experience gets started''.}.
}
Przyczyny w~doświadczaniu duchowym nie da się zamknąć wyłącznie do siły wyższej, a~trzeba zauważyć, że nawet najbardziej tradycyjne i~historycznie celebrowane doświadczenia duchowe często wiążą się z~zaplanowanym ludzkim działaniem poprzez: muzykę, rytuały, taniec, ułożenie ciała, koncentrację, regulację oddechu itp. I~chociaż w~niektórych religiach cały czas podkreślane jest kryterium spontaniczności, to wydaje się być ono w~tym kontekście przeceniane. Podobnie zresztą jest w~kwestii konieczności ponoszenia wysiłku, gdyż nawet tradycyjnie stosowane metody miały pomagać i~ułatwiać osiągnięcie doświadczenia duchowego.

Duży sceptycyzm autorzy wykazują również w~kwestii konieczności duchowego przywódcy. Analizując poszczególne warunki doświadczenia duchowego widać, jak kolejne kryteria mocno podporządkowują technologii i~nawet takie zagadnienia, jak duchowy przewodnik czy szeroko pojęte kierownictwo duchowe, w~wizji autorów zastąpione będą sztuczną inteligencją.

Co do jakości doświadczenia duchowego, wiąże się ono z~uczuciami, treściami, obrazami i~wrażeniami. Tutaj technologia w~dużej mierze pomaga opisać je na poziomie neuronów, a~techniki takie jak neurofeedback pozwalają nauczyć się wchodzenia w~stan doświadczenia duchowego i~dodatkowo je stymulują. Podobnie skutki doświadczenia duchowego wspomaganego technologią są jasno widoczne i~zdaniem autorów to one przesądzają o~autentyczności takiego doświadczenia.

Autorzy będąc świadomi, że technologia jest obecnie postrzegana jako zasadniczo różniąca się od duchowości lub nawet jako będąca z~nią sprzeczna, ewidentnie starają się oswoić czytelników z~myślą o~ich mariażu. Niestety, czasami robią to aż za bardzo. I~pomimo że we wstępie \textit{Spirit Tech\ldots} zapowiadają, że książka jest zestawieniem różnego rodzaju technologii pomocnych w~badaniu i~rozwoju duchowości, ostatecznie jednak stają się orędownikami nowego rodzaju ,,duchowości''; duchowości pozbawionej założeń dotyczących wiary lub niewiary w~to, co nadprzyrodzone, pozbawionej relacji do sfery \textit{sacrum}, pozbawionej jakiejkolwiek osobowej więzi z~bóstwem lub Bogiem i~odcinającej się od tradycyjnych światopoglądów religijnych. Taka ,,duchowość koncentruje się jedynie na specyficznej sferze odczuć podmiotowych. Zamiast więc badać typowo rozumianą duchowość, poprzestają na śledzeniu, jak zmiany w~systemie psychicznym i~fizycznym wpływają na siebie oraz starają się te zmiany wywoływać i~postawić w~swoim centrum zagadnienia religijności bądź duchowości.

Książka Wesleya Wildmana i~Kate Stockly \textit{Spirit Tech: The Brave New World of Consciousness Hacking and Enlightenment Engineering} to ciekawe zestawienie i~analiza aktualnych badań nad technologiami duchowego doświadczenia. Książka oferuje najnowszą wiedzę w~zakresie neuronauk, bardzo odważne prognozy rozwoju technologii duchowości i~bardzo daleko idące wnioski w~kwestii rozwoju religii. Traktując o~technologii duchowości, niestety staje się w~kolejnych rozdziałach coraz bardziej księgą technologicznych substytutów duchowości, będącej synkretyzmem efektów i~praktyk starających się je wywoływać. Duchowości bez ducha. Widać, że współczesne neuronauki wraz z~inżynierią wytworzyły już nowe obszary relacji nauka-wiara i~technika-wiara. Technologiczne substytuty duchowości stają się też coraz wyraźniej nowym wyzwaniem dla tradycyjnej duchowości i~dla religii. Szczególnie zastanawiające w~tej pasjonującej książce jest to, że technologiczne substytuty wiary promuje teolog chrześcijański.



%-------------------------




\selectlanguage{english}
\vspace{5mm}%
\begin{flushright}
{\chaptitleeng\color{black!50}{Spirituality without spirit}}
\end{flushright}

%\vspace{10mm}%
{\subsubsectit{\hfill Abstract}}\\
{Wesley Wildman's and Kate Stockly's book \textit{Spirit Tech: The Brave New World of Consciousness Hacking and Enlightenment Engineering} is an interesting compilation and analysis of current research on technologies of spiritual experience. The book offers the latest advances in neuroscience, very bold predictions about the development of spirituality technology, and very far-reaching conclusions about the development of religion. The authors take an interdisciplinary approach to their work, touching on neuroscience, medical engineering, psychology, religion, and ethics. They do not shy away from the big philosophical, psychological, and ethical questions, and they specifically focus on questions of authenticity, meaning, safety, and social responsibility arising from the marriage of technology and spirituality. Modern neuroscience along with engineering has already created new areas of science-faith and technology-faith relationships. Spiritual alternatives in the form of technology are increasingly posing a fresh challenge to traditional spirituality and religion.}\par%
\vspace{2mm}%
{\subsubsectit{\hfill Keywords}}\\%
{spirituality, technology, spirit tech, spiritual experience, brain, neuroscience.}%

\selectlanguage{polish}




\end{newrevplenv}