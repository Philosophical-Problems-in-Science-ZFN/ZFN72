\begin{newrevengenv}{Marek Kuś}
	{Quantum mechanics of identical particles}
	{Quantum mechanics of identical particles}
	{Quantum mechanics of identical particles}
	{Center for Theoretical Physics, Polish Academy of Sciences}
	{Tomasz Bigaj, \textit{Identity and Indiscernibility in Quantum Mechanics}, Palgrave MacMillan, Cham 2022, pp.XV+262.}





\lettrine[loversize=0.13,lines=2,lraise=-0.01,nindent=0em,findent=0.2pt]%
{S}{}ystems of identical particles pose a~major ontological problem in quantum physics. In the classical world, the situation is much simpler. Here Leibniz's \textit{principium} \textit{identitatis indiscernibilium} states that two objects having exactly the same features are identical, so they are the same object. With its contraposition, stating that two identical objects have exactly the same features, they form one of the most important metaphysical principles, indispensable e.g. in the discussion about individuals
%\label{ref:RNDj02KsADImV}(Guay and Pradeu, 2016).
\parencite[][]{guay_individuals_2016}. %
 Although one can contest the universality of the Leibniz Principle even for macroscopic objects 
%\label{ref:RNDzBCOouFRCo}(Black, 1952),
\parencite[][]{black_identity_1952}, %
 the real danger is posed by quantum mechanics. Here the existence of objects identical in every respect, nevertheless, in some sense, different is not only possible but even postulated. Thus all electrons are identical in every respect, and yet their number may be greater than one. It should be thus clear that the Leibniz Principle is violated in the quantum world 
%\label{ref:RND1aXrT4Wfwl}(Cortes, 1976; Ginsberg, 1981; French and Redhead, 1988; Castellani and Mittelstaedt, 2000).
\parencites[][]{cortes_leibnizs_1976}[][]{ginsberg_quantum_1981}[][]{french_quantum_1988}[][]{castellani_leibnizs_2000}.%


Let us be more precise. States of physical systems, such as elementary particles (electrons, protons, mesons) are described by vectors in a~Hilbert space (in fact, to achieve unambiguity of such representation we should go to the appropriate projective space, however, it is not important here). Physical quantities, such as energy, spin, etc., are represented by Hermitian operators acting in a~given Hilbert space. Quantum mechanics is a~statistical theory and to relate the formalism to reality we use average values. There is a~precise recipe how, knowing the state of the system, to find the mean value of an observable to compare it with experimental result obtained by performing many runs of an experiment.

In case of more particles (or, in general, more components of the whole system) the appropriate Hilbert space is the tensor product of the single-particle spaces. Thus, to each of the particles there corresponds the number of its Hilbert space in the product, according to the assumed order of numbering the spaces. Thus, e.g., in the case of two particles the appropriate Hilbert space $H_S$ of the system equals $H_1 \otimes H_2$. To refer to a~single (say, first) component and measure its properties, we apply an observable of the form $A \otimes I$, where $A$ is an operator acting in $H_1$. This recipe works well when we deal with distinguishable particles possessing different physical properties, e.g. electron and proton creating a~binary system in a~hydrogen atom. But when all particles are of the same type\footnote{This is the term used by the author of the book to avoid the terms ``identical particles'' or ``indistinguishable particles'' commonly used in physics, supposing in advance properties of these objects.} (e.g. they are all electrons), $H_1 = H_2 = H$ and, consequently, $H_S  = H \otimes H$. However, quantum mechanics requires that the average value of each observable does not change with any exchange (permutation) of the components. This is the so-called ``indistinguishability postulate''. Its requirements will be met by adopting the so-called ``symmetry postulate''—the states of the whole system must be completely symmetric or completely anti-symmetric under the exchange of particles. Consequently, the set of states of a~complex system is not the whole ``available'' Hilbert space—the product of the spaces of individual particles, but its symmetric (antisymmetric) subspace. Whatever happens to such a~complex system leaves it in this subspace. It follows that also observables have to be permutation invariant, otherwise the action may produce a~state not belonging to the appropriate symmetric (antisymmetric) subspace. Still, at least in the orthodox approach to quantum mechanics, the numbers of spaces correspond to the numbers of individual particles, in other words, each particle is ``attached'' to its own Hilbert space. Such a~postulate is called the ``factorism postulate''
%\label{ref:RNDrwIJFxPDao}(Caulton, 2014).
\parencite[][]{caulton_qualitative_2014}. %
 A~particular constituent (particle) is distinguished and individuated by applying a~measurement that acts only in its space, but it is not allowed in light of the previous discussion; such an operator, like $A \otimes I$ in the two-particle case is not permutationally invariant.

Compatible with the Leibniz's principle, would be thus a~view that particles such as electrons cannot be treated as individuals in the sense to which we are attached in the physics of macroscopic systems described by classical mechanics, where all objects can be individuated by the totality of their properties
%\label{ref:RND07z55IeZcZ}(Redhead and Teller, 1992).
\parencite[][]{redhead_particle_1992}. %
 However, for a~physicist, treating quantum systems of particles of the same type in this way is not attractive. Results of many experiments concern ``single particles'', or e.g. objects like ``all electrons occupying a~given energy shell in an atom''. It is far from clear in what sense such particles could not be individuals.

Tomasz Bigaj's book
%\label{ref:RNDkwyrAVucEF}(Bigaj, 2022)
\parencite[][]{bigaj_identity_2022} %
 discusses the problems described above and does it in a~masterly way. The main thread of the book is a~discussion of whether particles of the same type, like electrons, protons, neutrons, etc., can be distinguished from one another using their physical properties. And if so, how to do it.

Chapter 2 of the book is a~very good introduction to the problems discussed further. It contains a~clear formulation of the quantum mechanical formalism of many-particle systems, in particular systems of particles of the same type. The most important parts are devoted to the permutation symmetry and its consequences. The role played by the symmetrization postulate stating that states of such systems are either fully symmetric or fully antisymmetric under permutations, (i.e. the appropriate Hilbert spaces are symmetric or antisymmetric subspaces of the tensor product) is thoroughly discussed, with a~special emphasis put on its consequence that observables have to be permutationally invariant. The above outlined ensuing problems concerning identity and discernibility in the context of the Leibniz Principle are analyzed.

In Chapter 3 the author presents a~deep analysis of the origins of the symmetrization postulate, in particular its relation to the indistinguishability postulate demanding that all that expectation values of all empirically accessible observables are invariant under permutations of the indices of particles.

From the philosopher's point of view, what is important are the ontological and epistemological consequences of the formal results described above. The author discusses in depth the relations between the above-mentioned postulates and the Leibniz Principle and their sources (primarily empirical). The discussion in this context of Leibniz's Principle concerns the features to be taken into account in establishing the indiscernibility of particles—so, for example, assuming that such a~feature is \textit{haecceity}, i.e. what makes each particular individual to be this very individual, trivializes Leibniz's Principle by reducing it to logical truth. Since within the interpretation of quantum mechanics of systems of particles of the same type described above, the Leibniz Principle is violated, the basic question remains whether it is necessary and sufficient for empirical purposes: the existence of the possibility of pointing to individual particles and establishing their numerical identity (e.g. to determine their number).

The most important result is that to secure the possibility of making reference to individual objects and grounding numerical identity in qualitative facts, one should consider ``relation-based'' types of discernibility. How to do it is the main topic of the rest of the book.

The story begins, in chapter 4, with a~discussion of various variants of discernibility. As it seems, the most important from the point of view of quantum mechanics is the so-called ``weak discernibility'' proposed by Simon Saunders
%\label{ref:RNDkkjMPdsc3D}(2003).
\parencite*[][]{brading_physics_2003}. %
 This concept has gained some popularity. Although it was clear that it does not lead to any possibility of referring to individual objects in the case of indistinguishability, it gave hope for establishing numerical differentiation of objects (e.g. to determine their number). Formally, this principle postulates that in the language we use in discourse about quantum particles of the same type, a~symmetric ($R(a,b) = R(b,a)$) and non-reflexive (${\sim}R(a,a)$) relation can be constructed. Such a~relation (weakly) distinguishes objects if it is satisfied by objects \textit{a} and \textit{b}. This way of distinguishing objects is criticized in Chapter 4 (it was also criticized earlier by several authors). However, I~think that the basic objection of circularity (we presuppose that there are two a~priori distinct objects) is rather misplaced, whether in this case or other similar ones, since it can be applied to Leibniz's Principle itself, making its discussion essentially pointless. The important point, however, is that weak discernibility does not ensure the possibility of referring to a~particular single object and no other one.

The core of the argument is the topic of Chapters 5 and 6. Let's enumerate once more the goals. So, firstly, from the point of view of experimental physics, we would like to have the possibility of relating to individual particles, secondly, the possibility of numerical individualization of particles (so, among other things, the possibility that the number of them in a~given system is, for example, equal to~2). Within the approach called by the author the ``orthodox'' one, we assign to each particle ``its'' Hilbert space (``factorism''), and then we would like to measure a~given observable (e.g., spin component) in one of these spaces, which would allow statements ``the spin component of the k-th particle in a~given direction has this particular value''. As it is shown in the previous chapters of the book, and shortly discussed above, such a~program is not viable, due to symmetrization postulate. Tomasz Bigaj calls the proposes and discusses in detail the ``heterodox'' approach, where individualization, i.e. possibility of assigning different properties to particular particles, is carried out using the only admissible observables, i.e. observables symmetric with respect to permutations. The detailed construction of such observables is presented in the book. The goal is achieved by abandoning factorizm in favor of such formulations for a~system of particles of the same type as, e.g. ``one of the particles has a~spin component of some value and the other of them of a~different value,'' with the terms ‘one' and ‘the other' not referring to specific Hilbert space numbers of individual particles but meaning ``one of two particles,'' etc.

In Chapter 7. the author compares the ``orthodox'' and ``unorthodox'' approaches in detail, and above all, it discusses the problems that the ``heterodox'' approach poses (we already know the problems of the ``orthodox'' approach). Two of them seem particularly important. First, the constructed symmetric observable, in addition to measuring specific values of the physical quantity associated with it, individuates particles; another observable, associated with another physical quantity, leads to another individuation (''different particles''). Second, to the well-known problems of diachronic identity in the macroscopic world, it adds specifically quantum problems, discussed in detail in the book. There is no doubt for me that these problems have not yet found a~definitive solution, as the author also states in several cases.

For the philosopher, the last chapter of the book devoted to the metaphysics of quantum objects should be particularly attractive. That the ontology of quantum mechanics can be fundamentally different from that of classical objects has been, more or less, clear since the dawn of quantum mechanics, To some extent, Bohr's description of measurement in quantum mechanics, emphasizing that we are condemned to use the language of classical theory to describe phenomena that are not subject to this theory, can be seen as an anticipation of problems arising when we try to force the ontology of the quantum world to fit into the known classical world (if only by using such concepts as ``particle''). A~departure from this treatment of quantum objects is offered by quantum field theory, which uses ``quanta'' instead of ``particles'', i.e. in this context, excitations of the quantum field. Such a~formalism is also discussed in Chapter 7, but its discussion left me somewhat unsatisfied. It was, however, largely reduced by reading the last chapter, where different ontological concepts of quantum objects are discussed and, above all, the metaphysical consequences of the ``orthodox'' and ``heterodox'' approaches discussed above are compared.

From my point of view, it will be interesting to include a~short analysis of the problem of a~part and the whole in the case of particles of the same type, e.g., in the mereological setting. I~know only one position in the literature
%\label{ref:RNDKfIVHvW7R7}(Caulton, 2015)
\parencite[][]{caulton_is_2015} %
 dealing with these issues, and definitely, more can be done here.

The potential reader should be warned that most of the reasonings carried out in the book require some knowledge of quantum mechanics and the formalism used therein. Fortunately, the author restricts himself mostly to quantum mechanics in finite-dimensional Hilbert spaces (which allows, for example, the analysis of spin systems), where the mathematical formalism of quantum mechanics uses only elementary linear algebra. The basic concepts of quantum mechanics are described in the Appendix that closes the book. Whether this is enough to make the work accessible to non-physicists is hard for me (as a~physicist) to say. I~think that readers not familiar with the basics of quantum mechanics will find it quite difficult since the conclusions drawn by the author are based on quantum-mechanical calculations. Good familiarity with the philosophical foundations of quantum mechanics is also expedient. So both for physicists interested in the basics of quantum mechanics and philosophers of science, reading the book can be quite a~challenge. However, the efforts of going through it will be highly rewarded. The book is the best and most comprehensive work on the physics and philosophy of quantum particle systems of the same type in world literature.



%---------------



\vspace{15mm}%
{\subsubsectit{\hfill Abstract}}\\
{Leibniz's \textit{principium identitatis indiscernibilium} excludes the existence of two different objects possessing all properties identical. Although perfectly acceptable for macroscopic systems, it becomes questionable in quantum mechanics, where the concept of identical particles is quite natural and has measurable consequences. On the other hand, Leibniz’s principle seems to be indispensable when we want to individuate an item and ascribe to it particular property (e.g. value of the projection of spin on a chosen axis). We may thus abandon the principle on the quantum level, claiming it falsity here, or (better) try to find other ways of individuation of objects, possibly by adopting appropriately the very concept of it. All these problems, and many other connected with identity and indiscernibility of quantum objects, are thoroughly discussed in the book of Tomasz Bigaj, unique in the world literature due to its  comprehensiveness.}\par%
\vspace{2mm}%
{\subsubsectit{\hfill Keywords}}\\
{quantum identical particle, indiscernibility, Leibniz Principle.}%



\end{newrevengenv}
