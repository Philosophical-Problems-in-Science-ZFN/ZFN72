\begin{artengenv}{Walter Block}
	{Response to Wysocki on indifference\footnotemark{}}
	{Response to Wysocki on indifference}
	{Response to Wysocki on indifference}
	{Loyola University New Orleans}
	{Nozick
%	(1977)
	\parencite*{nozick_austrian_1977}
	was a~critique of the view of Austrian economics which rejected the notion of indifference in human action. This author claimed that this stance was incompatible with the notion of the supply of a~good, and, also, with diminishing marginal utility, both of which were strongly supported by this praxeological school of thought. Block
%	(1980)
	\parencite*{block_pu1}
	was an attempt to rescue the Austrian school from this brilliant intellectual challenge. Hoppe
%	(2005, 2009A)
	\parencites*{hoppe_pu1}{hoppe_further_2009}
	rejected Nozick's challenge, and, also, Block's
%	(1980)
	\parencite*{block_pu1}
	response. Block
%	(2009A)
	\parencite*{block_rejoinder_2009}
	and Block and Barnett
%	(2010),
	\parencite*{block_rejoinder_2010},
	defended Block's
%	(1980)
	\parencite*{block_pu1}
	analysis of indifference. The latest contribution to this ongoing discussion is Wysocki
%	(2021)
	\parencite*{wysocki_problem_2021}
	who maintains that Hoppe was correct in his rejection of Nozick, while Block was not. The present paper is a~rejoinder to Wysocki
%	(2021).
	\parencite*{wysocki_problem_2021}.
	}
	{indifference, supply, diminishing marginal utility, praxeology.}


\footnotetext{I~wish to thank two referees of This Journal for helpful comments on an earlier version of this paper which when incorporated greatly improved it. The usual caveats of course always apply.}

\section{Introduction}
\lettrine[loversize=0.13,lines=2,lraise=-0.01,nindent=0em,findent=0.2pt]%
{W}{}ysocki
%\label{ref:RND6Uy0JUU7mC}(2021)
\parencite*[][]{wysocki_problem_2021} %
 is written in support of Hoppe 
%\label{ref:RNDWGhcRWeJic}(2005; 2009)
\parencites*[][]{hoppe_pu1}[][]{hoppe_further_2009} %
 in his debate with Block 
%\label{ref:RNDmWQ6OyRVB5}(2009a)
\parencite*[][]{block_rejoinder_2009} %
 and Block and Barnett 
%\label{ref:RNDqkpBujF28F}(2010).
\parencite*[][]{block_rejoinder_2010}. %
 This dispute concerns the best way to counter Nozick's 
%\label{ref:RNDXoqIUUtNTR}(1977)
\parencite*[][]{nozick_austrian_1977} %
 rejection of Austrian praxeological economics which involves indifference, the supply of a~good, and the law of diminishing marginal utility.

I~will be quoting widely from Wysocki
%\label{ref:RNDOJxzE3ycVb}(2021)
\parencite*[][]{wysocki_problem_2021}%
\footnote{All references to Wysocki, unless otherwise indicated, will be to this one article of his.} and then responding to his many, and important points. Let us begin.

According to Wysocki
%\label{ref:RNDHYmAGPUCkC}(2021):
\parencite*[][]{wysocki_problem_2021}:%


\myquote{
If he were forced to give up a~unit of apple juice or the (sic) one of mineral water, he would be indifferent between the two.
}

Before responding to this quote, let me try to give the reader the context. I'm going to be putting ``words into the mouth'' of this author, but, hopefully, I~will be accurately transmitting his overall viewpoint. In my understanding of what he is about, he is supporting Hoppe's criticism of my critique of Nozick. All three of us, Hoppe, Wysocki and me reject Nozick's criticism of Austrian economics; but Hoppe and Wysocki maintain that my rejection of Nozick is a~failure. The issue turns on indifference and the supply curve. All economists, Austrian or not, maintain that there is such a~thing as a~supply curve. If this is to exist, avers Nozick, then the economic actor such see every element of the supply as equivalent. If there is to be a~supply curve of apples, for example, the consumer must see all the apples as the same; he must be indifferent between them. If he is not, if he sees some apples as different from others, there cannot be one supply curve of apples; there must be two or more, depending upon how many types of apples are perceived. However, asserts Nozick, correctly, Austrians reject the concept of indifference; they are thus logically inconsistent in acquiescing as to the supply curve.

Let me now respond to this opening quote from Wysocki, who rejects Nozick's critique of Austrianism, and supports Hoppe's criticism of my critique of Nozick. My claim is that Austrians can have our cake and eat it too; we may acknowledge the legitimacy of the supply curve (the demand curve too), and still not buy into the concept of indifference, at least not when choices are being made.

I~just do not see how anything like this could be true. For if this economic actor gave up one of these economic goods, that would, surely, demonstrate
%\label{ref:RNDn4BKSh14VO}(Rothbard, 2011 [1956])
\parencite[][]{rothbard_toward_2011} %
 that he valued it less than the other; and vice versa. The point is, there is no way he could reveal, in action, that he was indifferent between these two items.

If he were truly ``indifferent between the two'' then of course, a~la Wysocki, he could not choose to set aside either. However, this contradicts this author's premise. He states that the economic actor was \textit{forced} to give up one unit of either the juice or the H\textsubscript{2}O. We much assume he complied. If so, he might well have tossed a~coin to determine which one he gave up, but, in going along with the coin toss, he is not acting in an indifferent manner. Logically, he cannot do any such thing.

Our author continues:

\myquote{
Specifically, indifference has such a~propositional content (believing that \textit{x} and \textit{y} are economically identical) that it cannot motivate an actor to act on it.
}
But this appears to be a~direct contradiction of his previous statement. If indifference ``cannot motivate'' someone to undertake an action of picking and choosing and setting aside, how can the person in question objectively reveal that he saw the apple juice and mineral water as equivalents?

Wysocki then avers as follows:

\myquote{
Specifically, indifference has such a~propositional content (believing that \textit{x} and \textit{y} are economically identical) that it cannot motivate an actor to act on it. By contrast, homogeneity is a~relation holding between economic goods. However, remember that economic goods are not mere physical goods.
}

Whenever a~person engages in human action, giving up a~gallon of water or purchasing an extra bottle containing this amount of H\textsubscript{2}O, he cannot be engaging in an act of indifference between his present stock and the additional unit. Indeed, there is no such thing as an ``act of indifference.'' There can only be acts of preference and of setting aside. People can only be preferring or setting aside; e.g., treating these units as \textit{different} from one another. A~supply curve, then, consists of goods that are seen as physically indistinguishable one from the other in the absence of any human action that takes place with regard to them.\footnote{That would be a~necessary condition. But not sufficient, for we must incorporate Machaj's
%\label{ref:RNDp032hAJ439}(2007)
\parencite*[][]{machaj_praxeological_2007} %
 insightful example of the same physical ring on the hand of one's fiancé, and another physically identical one in the jewelry shop   
%\label{ref:RND9f7aYwJ7kb}(Block, 2009b; also, 2012).
\parencites[see][]{block_rejoinder_2009-1}[also][]{block_response_2012}.%
}

In the view of Wysocki:

\myquote{
When asked how we should understand the concept of the same good presupposed by the universal law of time preference, an Austrian economist might reply in a~similar fashion: we can easily learn whether \textit{x}\textsubscript{1} (some economic good at \textit{t}\textsubscript{1}) and \textit{x}\textsubscript{2} (some economic good at \textit{t}\textsubscript{2}) are the units of the same good. We would do so by checking whether an actor would now necessarily prefer \textit{x}\textsubscript{1} to \textit{x}\textsubscript{2}.
}
But how are we to ``check'' whether that is true or not? All we can observe is human action: people choosing amongst alternatives. It is logic alone, not any presumably empirical ``checking'' that can make any such determination. But Wysocki rejects this as ``tautologies.'' He states in this regard:

\myquote{
But these two apodictically true statements come at a~price. For the consequence of the lack of the independent (of the laws in question) notion of the same good, would turn those laws into concealed tautologies.
}

This author continues:

\myquote{
Incidentally, similar remarks would apply to Austrian formulation of the universal law of time preference. Austrians hold that for one (and the same!) end, each actor would prefer to achieve it sooner rather than later. Note, this law also presupposes the notion of the same good—but this time in a~sort of a~temporal way for it is the same economic good that is carried over time (we may obtain it at \textit{t}\textsubscript{1}or at \textit{t}\textsubscript{2}). When asked how we should understand the concept of the same good presupposed by the universal law of time preference, an Austrian economist might reply in a~similar fashion: we can easily learn whether \textit{x}\textsubscript{1}(some economic good at \textit{t}\textsubscript{1}) and \textit{x}\textsubscript{2} (some economic good at \textit{t}\textsubscript{2}) are the units of the same good. We would do so by checking whether an actor would now necessarily prefer \textit{x}\textsubscript{1} to \textit{x}\textsubscript{2}. But these two apodictically true statements come at a~price. For the consequence of the lack of the independent (of the laws in question) notion of the same good, would turn those laws into concealed tautologies.

}
Our author continues in his footnote 13:

\myquote{
In fact, it was Rothbard
%\label{ref:RNDbm0hE646An}(2011, pp.15–16)
\parencite*[][pp.15–16]{rothbard_toward_2011} %
 himself who resorted to this tautologous defense of the universal law of time preference, which is evidenced by the following passage: ‘Time preference may be called the preference for present satisfaction over future satisfaction or present good over future good, provided it is remembered that it is the same satisfaction (or ``good'') that is being compared over the periods of time. Thus, a~common type of objection to the assertion of universal time preference is that, in the wintertime, a~man will prefer the delivery of ice the next summer (future) to delivery of ice in the present. This, however, confuses the concept ``good'' with the material properties of a~thing, whereas it actually refers to subjective satisfactions. Since ice-in-the-summer provides different (and greater) satisfaction than ice-in-the-winter, they are not the same, but different goods. In this case, it is different satisfactions that are being compared, despite the fact that physical property of the thing may be the same. Whereas in the body of the text we considered the possible Austrian rejoinder in the form of saying that the same good can be conceptualized as the one that obeys the universal law of time preference, Rothbard merely contraposes by saying that if there is an apparent preference for a~future good (ice cream in summer) over a~present good (ice-cream in winter now), these two cannot constitute one and the same economic good. So, not only is the Rothbardian solution clearly circular, but also it gives the impression of fudging the notion of the same good. It may seem that whatever counterexamples to the law of time preference one may possibly come up with, Rothbard would rebut it by claiming that his critic invokes two distinct economic goods. This is yet another indication that an independent concept of the same good is logically required.
}

There are problems with this analysis. A~tautology gives us no information about the real world. Rather, it merely indicates definitions. For example, ``bachelor'' is an ``unmarried man.'' This insight offered by Rothbard and rejected by Wysocki constitutes much more: it is a~synthetic apriori statement which is both necessarily true and, also, give us profound insight as to how the real world operates. As to ``fudging'' this seems an erroneous characterization. Let us consider a~different case. It is alleged that ``two things cannot be in the same place at the same time.'' Someone objects: but more than two people can both be in New York City at the same time. The ``fudger'' responds: that's too big a~``place.'' Whereupon the critic says: but more than two people can both be in Manhattan at the same time, and that borough is smaller than the entire city.'' What is the ``fudger'' to do but to ``fudge'' once again? And to keep on doing so until the critic gives up. But ``fudging'' is not the correct description of this process. ``Clarification'' would do much better.

Wysocki states:

\myquote{
First and foremost, Nozick's challenge may be construed as a~purely logical objection to Austrian repudiation of indifference. After all, remember, the gist of Nozick's objection was that without the concept of indifference, Austrians would be unable to formulate the law of diminishing marginal utility. Indubitably, in this respect Nozick is right. The law of diminishing marginal utility has it that when we deal with a~supply of economically same units, each additional unit we value less; or, in other words, each additional unit is of lower utility.
}

To my way of thinking, this all depends upon precisely what is meant by ``when we deal with.'' If ``dealing with'' means, merely, contemplating a~stock of goods, then ``indifference'' is quite acceptable. After all, it is a~perfectly good word in the English language. Mere contemplation is not a~human action, at least not vis a~vis the object of the contemplation.\footnote{It is a~human action insofar as it indicates that the person prefers to contemplate at this moment in time when he could have been doing something else, which he sees as a~lesser value to him.} But if ``dealing with'' is taken as acting with regard to, say, a~stock of goods, then, it depends upon the precise action. For example, suppose someone sells the entire stock to someone else. Then, again, indifference vis a~vis elements of the stock with each other, is again acceptable.\footnote{Here, there would be no question of indifference between the entire stock and the money the vendor receives in payment for it. Clearly he prefers the latter.} But, if he sells only one of these units, for example the 72\textsuperscript{nd} one of them out of a~holding of 100, then he demonstrates that he prefers the other elements to this one.

Here is our author once again:

\myquote{
Second, we must also concede to Nozick that pricing of the commodity also seems to rest on the notion of indifference. Then, if Austrians fail to somehow accommodate indifference into their theory, this would have disastrous consequences for their entire conceptual edifice. After all, it must be borne in mind that the market (equilibrium)price of a~given product is a~function of supply and demand. And the demand curve is but a~reflection of the diminishing marginal utility of a~given product. That is, the demand curve—rather unsurprisingly—slopes downwards because the more we have (of a~given product), the less we value marginal units. And it is precisely why we are ready to buy more(of pretty much anything)only when the successive units of the product in question cost less and less. Therefore, it is clear to see that the demand curve reflects the logic of diminishing marginal utility. Hence, Nozick is right.
}

I~do not agree with Nozick's attack on Austrian economics. A~supply curve consists of a~locus of points indicating prices and quantities. We refer here to quantities of goods, of course, such as apples, or shoes. Are we indifferent amongst all of these objects? Of course we are. We have not yet acted upon any of them. When and if we do, we can no longer be indifferent between them. If we choose one of them, we will have demonstrated preference, not indifference. Says Nozick (paraphrase on my part): ``Aha! You Austrians are logically inconsistent. You admit that indifference is required if the supply of a~good is to make any sense.'' But there is no inconsistency. Of course, indifference is all around us.\footnote{In the television series Mary Tyler Moore, ``love was all around us.''} It only breaks up when we actually \textit{do} anything with regard to this supply, such as buy or sell any of it. In like manner, the same holds true for diminishing marginal (ordinal!) utility. We have the proverbial three bottles of water. We are indifferent between them all. They are now just sitting there. They all have precisely the same chemical properties (H\textsubscript{2}O) and we have no psychic preferences amongst any of them. We cannot so much as tell the difference between them. But then, for some reason we have to give up one of them. A~robber demands this of us, and we decide to comply. We are still indifferent between them. So, we grab one, at random. Behold, we are no longer indifferent! By our own action, we reveal that we prefer the other two bottles to this one; that is why we are giving up this one, and none other, to the thief. Has Nozick upset any Austrian theory? I~cannot see how he has. Is this mode of refuting him fallacious as Hoppe contends, and, now Wysocki? I~cannot see my way clear to agreeing with that, either. I~fear that Wysocki is giving away too much of the Austrian store to Nozick in his concession: ``Hence, Nozick is right.''

Rather, we can have our cake and eat it too. When indifference is required, in the absence of human action, in order to undergird diminishing marginal utility, or the supply and demand curves, we can retain it. But when human action takes place, we can jettison this concept.

We need not concede anything whatsoever to Nozick
%\label{ref:RND7GdaAbhB3f}(1977)
\parencite*[][]{nozick_austrian_1977} %
 in this regard. He asserts that indifference is required to demonstrate diminishing marginal utility. It is not. Before action, yes, there is indifference. There are stocks, supplies of goods, after all. But, when action occurs, when a~unit of the good is given up, or added, there is preference, not indifference.

We now arrive, more specifically, at Wysocki's dissatisfaction with my
%\label{ref:RND1yWgqppOJP}(Block, 1980)
\parencite[][]{block_pu1} %
 attempted refutation of Nozick 
%\label{ref:RND17GJmsskmJ}(1977).
\parencite*[][]{nozick_austrian_1977}. %
 Wysocki is kind enough to credit me (and my friend and colleague Hoppe) with ``eloquence'' and for having ``contributed most to the entire debate in question'' and I~am grateful to him for that compliment.

But then he offers his critique:

\myquote{
Block's account\footnote{In Block
%\label{ref:RNDYCAKJC3fjs}(1980)
\parencite*[][]{block_pu1} %
 I~mentioned 100 pounds of butter, among and between which the owner was initially indifferent, but then for some reason had to give up one of these units. He chose the 72\textsuperscript{nd} one.} is doomed to conclude for it is inherently unable to square the two apparent facts: 1) that those units of butter are really (ex hypothesi) ‘equally useful, desirable, serviceable' and 2) that 72\textsuperscript{nd} one was ‘picked up'. As we are about to see, the whole problem trades on the concept of ‘picking up'. Block seems to be lured into thinking that the imagined actor does pick up the 72ndunit where he says: ``For if the person didn't really prefer to give up this (72\textsuperscript{nd}) one, why did he pick it to be given''. Fair enough, if we assume that he did pick up this very unit, he must have preferred giving up this one to giving up any other, which simply logically follows from the concept of ‘picking up' employed herein. And yet, why should we beg any questions? It is to be established first that the actor does indeed pick up the 72\textsuperscript{nd} unit. For settling this issue has a~bearing on whether he prefers giving this unit to any other or he does not. And this in turn determines whether the actor conceives of the 72\textsuperscript{nd} unit as the unit of the same supply (with all the other units of butter) or he conceives of the stock before him as consisting of two distinct classes: a) a~homogeneous class of 99 units (still intact) and b) a~singleton containing the very pound of butter given up. Therefore, it seems that something has to give here: either 100 units were not in fact perceived as equally useful or they indeed were but the actor did not (in a~relevant sense) choose to give up the 72\textsuperscript{nd} unit. So, Block cannot have it both ways.
}

It seems to me that I~can indeed ``have it both ways.'' Not, of course, necessarily, with the 72\textsuperscript{nd} unit, but with any pound of butter. In the example, the owner of the butter is just sitting there, perhaps, contemplating his butter stock, and congratulating himself upon his ownership of it. If he is of a~philosophical bent, and contemplates which of these butter units he likes best, he could readily admit that he sees them as a~homogeneous blob, and is indeed indifferent between them all. But then comes a~robber who tells him, at the point of a~gun, to select one of these pounds of butter and give it to him. Or perhaps, he faces a~customer, who wants to buy one of them. Now, the situation is going to hit the fan. The grocer must choose one of these one-pound packages, to give to the thief/customer. In my example, I~chose the 72\textsuperscript{nd} unit. But I~am hardly wedded to that number. It could be any unit. But, in the event, based on the example, he must choose one of them. Let us say he chooses the first one, because he closes his eyes, and just happens upon that one. Well, now, he is no longer indifferent. Even Wysocki admits the truth of this when he says: ``Fair enough, if we assume that he did pick up this very unit, he must have preferred giving up this one to giving up any other, which simply logically follows from the concept of ‘picking up' employed herein.'' I~insist I~can indeed ``have it both ways.'' There are two separate scenarios here: one, before the demand on the part of this other person for some butter, and the other after this takes place. In the first place there is indifference, plenty of it.\footnote{Don't ask me how much. There is no such thing as a~unit of indifference.} In the second place, there is no such thing. Wysocki in my view places insufficient attention to this two-stage situation. The two scenarios are very, very different.

Wysocki continues:

\myquote{
The problem with this contention is that Block must either invalidate his assumption that they were homogeneous before the choice in order to explain why the choice (i.e. picking up the least preferred unit of butter, as opposed to the remaining ones) took place. Alternatively, if he maintains that the units in question are indeed equally useful, then he cannot explain why this particular unit of butter was picked up because they were assumed to be equally valuable in the first place. Nozick's challenge comes with vengeance to Block and the reason is precisely that the latter author has a~distorted idea of choice.
}

This author then touches upon various subjects: the essence of choice, the infinite varieties thereof, probability, numerous types of bread, many routes to super markets, the fact that I~am ``left with no hope of intelligibly conceiving of the same commodity,'' why the 72\textsuperscript{nd} unit as opposed to any old ``a'' unit? My answer might be overly simplistic; let me try it on for size in any case. The choice of the 72\textsuperscript{nd} unit was random. It was totally and completely arbitrary, used for illustration purposes only. I~fear that Wysocki does not give full credence to the very different thoughts and behavior of the butter owner before and after he is called upon to give up one unit of butter. Before? Homogeneity, to be sure. After? Heterogeneity is the order of the day. Why this should be shocking, why impossible, why positing this to take place should be logically inconsistent, is beyond me. Time changes all sorts of things; why not this too?

We then arrive at Wysocki's second interpretation of my response to Nozick; it is true and correct, but merely trivial: ``On the second reading, Block's position may be rendered true but then it would amount to the mere restatement of the law of marginal utility.'' My first reaction to this is ``Please don't help me so much,'' or ``With friends like this, who needs (intellectual) enemies'' \parencite[in this regard see][]{riley_please_2016}. %
%\footnote{See
%%\label{ref:RNDlE6uwhcc55}(Riley, 2016)
%\parencite[][]{riley_please_2016} %
% in this regard.}
 On a~more serious note, I~fail to see how my distinction between before and after human action amounts, merely, to restating the law of diminishing marginal utility. It cannot be denied, of course, that with the 72\textsuperscript{nd} unit of butter out of the way, all the others are now of greater value, but this hardly is what my refutation of Nozick is all about.

It is more than passing curious that in the next section of his paper, where Wysocki claims that Hoppe's refutation of Nozick is vastly superior to my own feeble attempt, that we read this incisive and correct statement of his: ``[…] indifference cannot be demonstrated in action, as is usually reiterated by Austrians
%\label{ref:RNDkBDnhlt2jA}(Block, 2009a; Block and Barnett II, 2010; Rothbard, 2011).
\parencites[][]{block_rejoinder_2009}[][]{block_rejoinder_2010}[][]{rothbard_toward_2011}. %
 By no means can we deduce from any actual choice whether we were confronted with the units of the same good or with the ones of distinct goods.'' But this is precisely the point I~employed to launch my critique of Nozick, to which Wysocki gives the back of his hand.

In section 5 (``Hoppe's account as a~remedy for Block's shortcomings'') of his paper, Wysocki will demonstrate that while my disproof of Nozick's criticism of Austrian economics fails, Hoppe's succeeds. Given that this is his goal, he gets off on the wrong foot. He opens with this statement:

\myquote{
First and foremost, it must be noted that—unlike Block's solution—[…] (Hoppe's) […] involves both doing justice to indifference (at least admitting that a~man can be genuinely indifferent between some options) and barring it steadfastly from the realm of choice. Briefly speaking, Hoppe
%\label{ref:RND6dL2T23Yfz}(2005)
\parencite*[][]{hoppe_pu1} %
 maintains that one cannot make a~choice under indifference.
}
But I~too, do this. There is not a~dime's worth of difference between Hoppe and me on this issue. We both acknowledge that choice, or human action, is incompatible with a~state of indifference! We both, Hoppe and I, agree with, in Wysocki's words: ``the truth of the proposition that a~man cannot choose when indifferent.''

Wysocki's account gets curiouser and curiouser. He continues: ``Specifically, Hoppe defines choice in such a~way that it entails the lack of indifference. That is, if man chooses x~over y, he is not (and, logically speaking, cannot) indifferent between the two.'' Well, Hoppe is not the only one to point to this insight. That is precisely my position, too.

However, Wysocki does not continue down this path. Instead of failing to see that Hoppe and I~are on the same page, he now, correctly, notes a~deviation between the two of us, and error number two on his part, supports Hoppe's incorrect position; writes our author:

\myquote{
Likewise, a~mother who sees her equally loved sons Peter and Paul drown and who can only rescue one does not demonstrate that she loves Peter more than Paul if she rescues the former. Instead, she demonstrates that she prefers a~(one) rescued child to none. On the other hand, if the correct (preferred) description is that she rescued Peter, then she was not indifferent as regards her sons.
}
No, no, no. While it is of course true that this poor mother\footnote{The movie ``Sophie's choice'' subjects a~poor woman to this excruciating choice.} ``prefers one rescued child to none'' she also places a~higher value on Peter than Paul. She did rescue the former, when she could have chosen differently, and selected the latter for retrieval, did she not? I~simply cannot understand how it can be denied that she favored Peter over Paul, in the face of her action in rescuing Peter, not Paul.

Wysocki's account violates the spirit of praxeology. Now, it is quite alright to do so, at least in terms of congruence with mainstream economics. Our author is here on firm ground in this regard. My argument is that both Wysocki and most economists are in error here. I~simply do not see how it is possible to deny that since the mother chose Peter, not Paul, she was not indifferent between them. Even if she did this with her eyes closed, and just grabbed the nearest son, this I~think is the correct view.

Suppose she was offered the choice between a~chicken dinner and a~fish dinner. And, she selects the former. If Wysocki and Hoppe stick to their ``logic'' they would of course be correct in concluding that the woman preferred to eat dinner rather than go hungry. But they would be logically precluded from, also, maintaining that she preferred chicken to fish. The two of them, on the one hand, and I, on the other, live on different planets. Or, perhaps Mises was incorrect in his castigation of polylogism; there are two different ``logics'' at work in this case, mine and theirs.\footnote{I~am of course joking about this. Mises was correct, as am I~in this case. The same cannot be said for Hoppe and Wysocki.}

Wysocki misconstrues Buridan's ass in the same manner. This beast, let us say, chooses the bale of hay to the right. The correct interpretation of this is two fold: one, this creature preferred life to death, and, two, he favored the hay on the right to the hay on the left. In Wysocki's correct interpretation of Hoppe, and his own, only the first is true. The second, amazingly, is not. But, but, but, the donkey moved to his right, not his left! If this is not evidence that he preferred the right to the left bale, there can be no such thing as evidence, at least not in cases like this.

At least Wysocki is logically consistent. Let us give credit where it is due: he makes the same identical mistake when it comes to the butter example. He writes:

\myquote{
[…] since the actor did indeed give up the 72\textsuperscript{nd} pound of butter (while holding all of them equally serviceable), he must have preferred giving up a~unit of butter for some pecuniary equivalent. In other words, the actor preferred one unit of butter less, but some increment of money to retaining his entire stock of butter but depriving himself of an opportunity to earn this money. The second possibility is that the correct description of an action is that the actor really dispreferred that 72\textsuperscript{nd} unit that he actually gave up. If so, that unit was not the same economic good as all the other units in the first place and therefore, trivially, the original supply of 100 units was heterogeneous.

}
Again, I~say, no, no, no. Why is it so hard to realize that one and the same thing can be altogether different, given the passage of time? At time \textit{t}\textsubscript{1}, before any choice was made, yes, all units of butter were ``equally serviceable.'' Their owner was indifferent between all of them. They were homogeneous as far as he was concerned.\footnote{If he thought about it at all, which he almost certainly did not.} But then, at time \textit{t}\textsubscript{2}, when push came to shove, the grocer had a~decision to make: he valued the money more than a~unit of butter, any unit of this product, to be sure. In the event, he was compelled to choose one of them if he wanted to make the sale. If this does not establish that he valued this particular one, the 72\textsuperscript{nd} unit, less than the others, then there is no such thing as choice, utility, economic theory, common sense.

Wysocki characterizes the Peter and Paul example as ``the celebrated Hoppean thought.'' Sure.

Here we come to the very core of why our author favors Hoppe's analysis vis a~vis mine. Says Wysocki:

\myquote{
Block's point against Hoppe would be decisive if the act of saving a~particular child (under this description) instead of another were inherently preference-implying. That is, Block would succeed if we can infer from the fact of saving a~particular child (or from bringing about the event of a~particular child being saved) that this particular child was preferred to the other. Yet, there is a~deep distinction favored by Davidson
%\label{ref:RNDjKlYmL9Ybk}(2001)
\parencite*[][]{davidson_agency_2001} %
 between what an actor does (including his primitive action consisting in his bodily movements up to everything they cause) and what he does intentionally. As Davidson 
%\label{ref:RNDPGPvETu7dU}(2001, p.45)
\parencite*[][p.45]{davidson_agency_2001} %
 put it: ``[…] although intentionality implies agency, the converse does not hold''. Therefore, it would simply beg the question to say that by the act of saving Peter the mother demonstrated her preference for Peter over Paul. As established above by alluding to the Davidsonian insight, from the event that the mother authored, we cannot infer which aspects thereof were informed by her preference. Therefore, not to beg any questions, we should treat the act of saving Peter in the non-choice- (and hence also non-preference) -implying sense. Alternatively, just to remain neutral on whether the mother did actually choose to save Peter or chose to save a~child, we could say that what the mother in fact did was to save Peter. After all, to say that the mother saved Peter is only to attribute her agency to this event (in other words, it is to say that she authored the event of Peter having been saved), which does not imply that she saved Peter intentionally. And this is the key insight which, in our view, counts in favor of the Hoppean account.
}

I~have no problem with what Davidson writes. My difficulty concerns the manner in which Wysocki utilizes that insight in behalf of his own views and those of Hoppe. Why? Davidson's perception is a~matter of psychology, not praxeology. But we are now in the latter realm, not the former. Remember, this is Austrian economics we three are defending against Nozick's critique. As psychologists, we are undoubtedly interested in whether something is intentional or not. But I~insist, this is not the case in our role as praxeologists. Suppose someone shoves two plates of ice cream at you and demands you choose one. You are very busy with something else. You reach out and grab one. You are not aware of which flavor you now have at your disposal. You may be so concerned with these other matters that the actual taste of this treat does not even register with you. As psychologists, there is no objection, at least not stemming from the present quarter, in saying that you were totally unaware of what you were doing, at least ice cream wise. The other matter took all of your intention. But, in sharp contrast, we as praxeologists must note that you actually reached out and grabbed one of them, not the other. This is the essence of Hoppe's error, with support from Wysocki. That mother reached out and saved Peter not Paul. What might well have been on her mind had nothing to do with Peter nor Paul. It might well have been as Hoppe opined, she was just preferring to save one of her sons, rather than none. Who knows, she might have been thinking about ice cream, as far as we praxeologists are concerned. This does not matter in the slightest for the praxeologist. We see her grabbing Peter, not Paul, to safety, and we are compelled by praxis logic, e.g., praxeology, to note that she was not indifferent between her sons, she could not have been indifferent between them, given that she chose the one, not the other. In not seeing this, in maintaining the very opposite, Hoppe and Wysocki are guilty of what Rothbard
%\label{ref:RNDFnt7pkdLRp}(2011 [1956])
\parencite*[][]{rothbard_toward_2011} %
 labeled the error of psychologizing:

\myquote{
Psychologizing is a~common error in utility analysis. It is based on the assumption that utility analysis is a~kind of ‘psychology,' and that, therefore, economics must enter into psychological analysis in laying the foundations of its theoretical structure. Praxeology, the basis of economic theory, differs from psychology, however. Psychology analyzes the how and the why of people forming values. It treats the concrete content of ends and values. Economics, on the other hand, rests simply on the assumption of the existence of ends, and then deduces its valid theory from such a~general assumption. It therefore has nothing to do with the content of ends or with the internal operations of the mind of the acting man.
}

Consider an additional illustrative example. A~man chooses women who do not respect him. He has a~long trail of such psychologically unsatisfying and debilitating relationships. He knows he should select a~different kind of women the next time he is free to do so. But when the occasion again arises, he conforms to the same old sick pattern. A~psychologist might analyze this series of events by saying that the man full well knows what he is doing is counter-productive, and that he really wants a~better relationship with a~nice woman. But the Austrian praxeologist is precluded from making any such determination. He is required to look, only, at human action, actual behavior, and conclude that the man really prefers that type of woman.

Like a~good psychologist, Wysocki places great weight on ``what the mother intended to do'' regarding her two sons. But as praxeologists, we are forbidden to even take this into account to the slightest degree. Rather, we are (logically) required to ignore that entirely, and focus, instead, on the decision she actually made. She was \textit{not} indifferent between her two sons, as matters unfolded. What went through her mind at the actual point of choosing is entirely irrelevant. We praxeologists must focus, only, on the decision she actually made.

Nor will Parfit's
%\label{ref:RNDGya77ZS3Ig}(2011, p.289)
\parencite*[][p.289]{parfit_what_2011} %
 analysis save Wysocki's nor Hoppe's bacon. This philosopher points to

\myquote{
Two merchants […] who […] may both act on the policy ‘Never cheat my customers'. But these merchants act on different maxims if one of them never cheats his customers because he believes this to be his duty, while the other's motive is to preserve his reputation and his profits.
}
This 
is indeed, again, grist for the mill of the psychologist; but for the praxeologist it makes a~no never mind. We praxeologists could not be less interested in motivation. For us, the only, sole, issue is, what did the merchants actually do? They did exactly the same thing here. End of praxeological story!

Opines Wysocki: ``Finally, let us note that Nozick's challenge leaves the Hoppean position unscathed.'' Not so, not so, say I. Wysocki continues: ``when two -- economically identical -- goods are at stake, our goal (maxim) is satisfied to the same degree regardless of whether one good or the other is employed.'' I~repeat myself, not so, not so. Hoppe does not lay a~glove on Nozick in this instance.

Why did the grocer choose the 72\textsuperscript{nd} unit? He could have picked any of the other 99 pounds of butter. But he chose that one. That necessarily, logically, implies that at least at the moment of choice,\footnote{Not before, of course, when he was indifferent to the entire lot of them.} he preferred to jettison that one, not any of the others. He was indeed indifferent between them at time t\textsubscript{1}, but not at all at time \textit{t}\textsubscript{2}. In the latter epoch, he demonstrated preference, not indifference, \textit{a~la} Hoppe and Wysocki.

So far in this paper I~have been highly critical of Wysocki.\footnote{And, \textit{en passant}, of Hoppe.} I~have stressed the difference between the two of us. However, perhaps, there is hope for a~reconciliation, at least partially. For in his section 6 ``Extending the Hoppean framework: stating the law of diminishing marginal utility,'' we seem to be more in tandem with one another. He now acknowledges the importance of time; this is something I~have been emphasizing ever since Block 1980.

Wysocki avers: ``that a~given stock of units may be considered by an economic actor as a~supply of the same commodity only relative to a~given moment. Strictly speaking, it is a~matter of course that human action is sequential (in a~temporal sense) by nature; yet, an actor at t\textsubscript{1} may envisage the way he is going to employ consecutive units at later times. This double time indexation---one standing for a~given moment in which an actor envisages the employment of his successive means and the other standing for the actual time at which they are employed---is necessary.''

Perhaps, who knows, Wysocki may one day, give him time, come to the realization that one and the same stock at time t\textsubscript{1} may be totally homogeneous for a~grocer,\footnote{Thus, he is indifferent between all of the units.} and yet, later on, when push comes to shove and a~decision has to be made, this will cease, and one of these units, maybe even the 72\textsuperscript{nd} one just for illustration purposes, may seem to be more expendable than the others, in which case indifference between all of them no longer holds.

Here is yet another inkling of a~coming together of Wysocki and myself. He says of the ``Hoppean account'' that it ``accommodates indifference and keeps it steadfastly from the realm of choice -- very much in line with the demands of praxeology itself.'' This, too, is very much congruent with what I~have been saying since 1980.\footnote{Other hopes for reconciliation; the two of us are several times co-authors on similar matters:
%\label{ref:RNDVeJovUR3rV}(Wysocki and Block, 2018; 2019).
\parencites[][]{wysocki_analysis_2018}[][]{wysocki_homogeneity_2019}.%
}

It is time to conclude this paper. I~do so by quoting and responding to a~statement made by a~referee of This Journal to an earlier version of this paper:
\small{
\begin{itemize}
\item the author does not seem to provide a~satisfactory explanation (or reference, perhaps) of why indifference breaks up at a~moment of choice/action, especially if that action is not internally motivated but externally enforced; if it is enforced, it is not necessarily well reflected upon (especially under time-pressure), it could well be automatic and mean nothing for the agent's individual assessment of the value of the thing chosen;
\item what is also missing is some explanation of why passing of time in itself is enough for an agent to move from a~state of indifference to a~state of defined preference; from the examples used by Wysocki, and then commented on in this text, it seems that the apparent ‘preference' is enforced by external (and sometimes extreme) circumstances, e.g. the rescuing of one son only; what is puzzling here is the lack of interest in the role of intentionality in the act of ‘picking' as ‘choosing' (self-motivated) which is missing in an act of ‘picking' under someone's enforcement; there seems to be a~marked difference between internally and externally motivated choice; the author explains towards the end of the paper that inentionality (sic) or motivation does not interest praxeologists who are merely concerned with the end result of a~decision, but that seems to be the core of the misunderstanding between Wysocki and the author of the rejoinder.
\end{itemize}
}
I~thank this referee for giving me this opportunity to further clarify.

Indifference breaks up at a~moment of choice/action because that is the only precise time when it logically cannot exist. We can all be indifferent between the dozens if not scores or even hundreds of cans of Coca-Cola in the grocery store before selecting one of them. If not, then the word ``indifference'' has no meaning in the English language. But it clearly does. We all use this word upon occasion.\footnote{Economics is not the only discipline to utilize a~technical language for ordinary words. In physics, ``work'' means something very different than how this word is ordinarily employed.} But when push comes to shove, we select \textit{this} can of coke, and \textit{not} any of the others. How are we social scientists to account for this? Why, by saying that indifference is an accurate account for how we see large aspects of reality, with the exception of when choice/action takes place.

Does this apply, also, when the choice is made under duress? Yes, indeed, it does. It is certainly ``externally enforced'' when the mother can save only one of her children. It might well then be ``automatic.'' But this does not, cannot, change the primordial fact that even upon this occasion she chose this boy to save, and not the other. Can we really say, then, that she was indifferent between the two of them? Yes, of course we can: from a~psychological point of view. She loves them both; equally. However, Nozick, Hoppe and Wysocki do not properly occupy the psychological or ordinary world. My debate with these three scholars has nothing whatsoever to do with that realm. Rather, this discussion is a~matter of praxeology, or technical economics. My major criticism of these three in this debate is that they fail to make this distinction, and stick to it no matter what.

The ``passing of time'' is crucially important since it demarcates between the situation in which choice, or human action in Mises' terminology does not occur, and when, later on, it does take place. When looking out upon a~horde of coke cans, it is easy to be ``indifferent'' between them all. Not so, later on, when one has to grab one. But which \textit{one}? When one is selected, and the other is put aside, there can be no room for indifference in the analysis.

``Intentionality'' is a~psychological concept, not an (Austrian) economic one. We as economists are not privy to the internal thoughts of other people. We can only observe their ``human action,'' their actual behavior, the choices they make. We see someone selecting \textit{this} can of coke, not any of the others; we observe the mother saving \textit{this} son, not the other one. What are we to say about this? That indifference is now occurring? Not at all. Rather, that there is \textit{preference} taking place.

What of the claim that ``praxeologists [\ldots] are merely concerned with the end result of a~decision.'' This is not so. Rather, Austrian economists are concerned with the decision itself. As Mises
%\label{ref:RNDNeCpyIkgz0}(1998, p.97)
\parencite*[][p.97]{mises_human_1998} %
 wrote: Human ``Action is an attempt to substitute a~more satisfactory state of affairs for a~less satisfactory one…'' This is what choice is all about. There is simply no room in such behavior for indifference.

\end{artengenv}
