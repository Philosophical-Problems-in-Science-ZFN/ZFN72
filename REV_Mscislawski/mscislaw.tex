\begin{newrevengenv}{Łukasz Mścisławski}
	{Imagination, geniuses\\and thought collectives}
	{Imagination, geniuses and thought collectives}
	{Imagination, geniuses and thought collectives}
	{Wrocław University of Science and Technology}
	{Wojciech Sady, \textit{Struktura rewolucji relatywistycznej i~kwantowej w~fizyce}, Univeristas, Kraków 2020, pp.238.}
	
	


\lettrine[loversize=0.13,lines=2,lraise=-0.03,nindent=0em,findent=0.2pt]%
{T}{}alking about the relativistic and quantum revolution in physics that took place in the first two decades of the 20\textsuperscript{th} century has almost become a~common slogan. However, considering the momentousness of the consequences of this process for the development of the mentioned discipline and the philosophical implications appearing almost immediately, it is difficult to speak here of any exaggeration.

Wojciech Sady proposes, in his intentions at least, a~thorough study of one of the central issues in the philosophy of science, namely the question how the process called scientific revolution takes place. Its central aim, as the author declares, is an attempt to answer the question: ``how is it possible that scientists start thinking differently than they have been taught to think''
%\label{ref:RND0iwMHvnoP5}(Sady, 2020, p.26).
\parencite[][p.26]{sady_struktura_2020}. %
 He makes use of rich historical material in his undertaking, and conducts philosophical analyses mainly with reference to the main concepts proposed by K.R. Popper, T. Kuhn, I. Lakatos and L. Fleck. The last, drawing most extensively on the achievements of the latter, especially the concept of thought collectives, the process of socialisation and sources of new scientific inspiration in thought collectives.\footnote{These are of course not the only inspirations of W. Sady---the complete list is included in the Introduction  
%\label{ref:RNDoialNAH5d5}(Sady, 2020, pp.15--17).
\parencite[cf.][pp.15--17]{sady_struktura_2020}.%
} In relation to the first two (Popper and Kuhn) there is a~far-reaching polemic.

The historical material researched by the Author is impressive and constitutes an undoubted asset of his work. It can be divided into two parts, the first one connected with the formation of the special theory of relativity (chapters 1-4) and the second one in which the process of the formation of quantum mechanics is presented (chapters 5-7). Each stage of the presentation of this rich material is concluded with methodological remarks of the Author, which in his intentions are to convince the reader of the proposed answer to the previously mentioned central question.

A~brief and somewhat simplified reconstruction of this answer would be as follows: the source or charge of a~potential scientific revolution is the inconsistencies revealed in the juxtaposition of theoretical knowledge of a~given period with experimental data. The issues in which these inconsistencies reveal themselves become a~field of research for (relatively) young scholars who have not yet been so bound up with the scientific practice of a~given community of researchers that they feel obliged to prefer the classical system of established laws to a~problematic area. It is they---as Sady suggests, driven by unspecified factors, such as youthful fantasy---who choose this particular area (the anomalous parts of the protected belt) as the basis of their own. Thanks to this they were successful in the given field, which in turn led to systematic research, this time using new theoretical tools
%\label{ref:RNDYCV2psdoMT}(cf. Sady, 2020, pp.217--219).
\parencite[cf.][pp.217--219]{sady_struktura_2020}. %
 The author emphasises here the continuous development of knowledge in a~given discipline, which seems to be the main driver of the following more or less revolutionary changes. He also tries to deal with two myths: the myth of the role of creative imagination and the myth of genius or, in relation to specific people: geniuses, which does not play an exceptional role in the whole revolutionary process.\footnote{Some of his remarks on the role of imagination can be found in 
%\label{ref:RNDIR3K76Hq4Y}(Sady, 2020, pp.31--34, 71--78),
\parencite[][pp.31--34, 71--78]{sady_struktura_2020}, %
 and on the \textit{genius}: 
%\label{ref:RND7KLyIB55Ik}(Sady, 2020, pp.219--220).
\parencite[][pp.219--220]{sady_struktura_2020}.%
} The proposal seems attractive and well argued based on the material considered. The question is: Is it really so?

Above all, Wojciech Sady cannot be denied courage when it comes to presenting his own proposal for tackling the issue. Thus it is not another collection of quotations, but an attempt at a~genuine search for a~solution to the problem. Even if the reader does not share the Author's conclusions, or does not find his argumentation sufficiently convincing, he will undoubtedly have an opportunity to reflect on his own position in relation to the discussed issue of the structure of scientific revolutions. However, several issues arise that should be addressed here. It would be great if they would help the Author to improve the work and perhaps provoke him to write a~further sequel.

The first point worth noting concerns the object of Sady's research---that is, how the image of the world and the way of thinking about it among scholars have been radically restructured by the emergence of relativity theory and quantum mechanics. It seems that there is a~need for greater precision here. The Cracow philosopher refers to the special theory of relativity and to the old quantum theory.\footnote{The proper term for the stage described by author is \textit{old quantum theory}. In the Polish literature one can sometimes meet with the term primary quantum theory, cf.
%\label{ref:RND76z8PInKYT}(Średniawa, 1981, p.3).
\parencite[][p.3]{sredniawa_mechanika_1981}.%
} It may seem to be an unnecessary detail, nevertheless it has its far-reaching consequences which will be discussed a~little later. It is not clear for what reason the Author completely silent omitted in the relativistic revolution the role of Minkowski, who introduced the concept of space-time. The above mentioned clarification---distinguishing the special relativity from the general relativity is important here for the reason that---what is even more incomprehensible---the Author completely omitted in his considerations the general relativity and, what gives an impression of talking about both theories simultaneously.

He also presented as almost trivial the process of the formation of quantum mechanics as a~stabilised theory. To this last point he literally devotes several pages
%\label{ref:RNDdsBsO2NrY0}(cf. Sady, 2020, pp.203--207).
\parencite[cf.][pp.203--207]{sady_struktura_2020}.%
\footnote{The statement that reconstructing the process of the emergence of quantum mechanics would be much more difficult is undoubtedly correct. Unfortunately, it does not weaken the impression of trivialization of this process by the Author 
%\label{ref:RNDB2A0hYMjkM}(cf. Sady, 2020, p.219).
\parencite[cf.][p.219]{sady_struktura_2020}.%
} Considering that, when speaking about the revolution connected with quantum mechanics, it is necessary to note that the old quantum theory---to which Sady's analyses actually refer---is in a~sense an inherent prelude to it, but precisely: a~prelude. This applies both to the effect on the change in the world picture and to the radical change in the relation between the area of reality described by this theory and the means of description---that is, the mathematical formalism. This is a~change which is definitely deeper than that between non-relativistic and relativistic models.\footnote{In this case, Kopczyński and Trautman speak of a~cut to the viewability (\textit{cięcie poglądowości}) 
%\label{ref:RNDMD1RpNmtBX}(cf. Kopczyński and Trautman, 1984, p.24).
\parencite[cf.][p.24]{kopczynski_czasoprzestrzen_1984}.%
} It should also be noted here that the term ``quantum revolution'' most often refers to quantum mechanics as the theory presented by Heisenberg, Pauli, Jordan, Born, Dirac and Schrödinger.\footnote{A~good discussion of this process is the classic work 
%\label{ref:RNDlOdaLEJYem}(Jammer, 1966).
\parencite[][]{jammer_1966}.%
} The above lack of precision with regard to the subject under discussion is all too evident here, as is the statement that: ``the formalism of quantum mechanics was---and to this day remains---one'' 
%\label{ref:RNDpFSpjZwOBK}(cf. Sady, 2020, p.207).
\parencite[cf.][p.207]{sady_struktura_2020}. %
 It is not quite clear what the Author has in mind here, the more so, that among formulations of quantum mechanics one can enumerate (simplifying a~bit): a) formulation based on selfaddjoint operators acting on Hilbert spaces; b) path integral formulation; c) C*-algebra formalism... One can guess that the standard formulation a) is meant, which most often appears in philosophical discussions---but this statement is missing.

The lack of any mention of the general relativity and a~very brief reference to quantum mechanics gives the reader the impression that the author stopped halfway. This is a~great pity because, taking into account the history of the emergence of both theories, he could have found a~strengthening of some of his theses, especially as far as the role of mathematics in the formation of new physical theories is concerned. He could also possibly correct his other views on the interplay between background knowledge, experimental data, and well understood creative imagination---the topic of which will be touched upon later---taking into account the process of formation of the general theory of relativity. This lack of devoting even a~few words to Einstein's theory of gravitation should be considered a~serious shortcoming.

The next point relates to another two themes that Sady addresses in his work. Both are closely related, so they will be addressed together. The first thread concerns the role of imagination in the process of formation of scientific theories (here: in physics), while the second thread concerns the role of mathematics in this process. The author states: ``it is impossible to be ahead of one's time, to fill the gaps in our knowledge with products of the imagination. In science one should proceed step by step---also, as we shall see below, in revolutionary periods---and each time assert only as much as results from the existing knowledge and the results of experiments''
%\label{ref:RND8RTfvHZFeQ}(Sady, 2020, p.34)
\parencite[][p.34]{sady_struktura_2020}. %
 Leaving aside the apodictic-normative character of this statement, it is difficult to disagree with him. It seems, however, that it would be necessary at this point to introduce a~certain distinction, which seems to elude the author. For we should distinguish between the free creations of fantasy and specifically understood imagination, which seems to be necessary in practising a~given discipline, a~kind of intuition or intuition. There is also a~question whether by existing knowledge the Author means knowledge available only within a~given discipline---in this case physics or also disciplines with which a~given discipline is connected (here mathematics would come into play)? Assuming that this is the case, the quoted statement would be highly probable.\footnote{It would be problematic if mathematical structures were created, as it were, on an ongoing basis, thus creating a~body of knowledge that did not previously exist.} Creative imagination would then refer to the ability to identify the mathematical structures with which the scientist wishes to describe physical phenomena. As long as the imagination ``works in this spirit'', it seems that it can be allowed to go to any lengths, but not to fantasise. It seems that S. Kalinowski aptly put it, the point is that imagination should provide a~certain idea which organises the whole intellectual effort 
%\label{ref:RNDCAVujnruwF}(Kalinowski, 1916).
\parencite[][]{kalinowski_nauka_1916}. %
 The circumstance, repeatedly emphasized by Sady, that only adjusting to a~rigorous guidance through mathematical structures gave a~way out of troublesome situations 
%\label{ref:RNDBFU4baVvkd}(e.g. Sady, 2020, pp.71--79)
\parencite[e.g.][pp.71--79]{sady_struktura_2020} %
 is known, and is fully confirmed by the well-known methodological instruction of M. Heller: ``Theoretical physicist! When you have other views than your equations (confirmed by the agreement of their predictions with the results of measurements), do not move the equations, change your views!'' 
%\label{ref:RNDO2Jd7MsJJy}(Heller, 2011, p.113).
\parencite[][p.113]{heller_2011}. %
 It is also worth noting here, which in a~way also follows from the whole of Sady's book, that the history of physics becomes, as it were, the history of the adaptation of the imagination of physicists to the use of increasingly abstract formal structures. Hence the proposal to distinguish ``free fantasising'' from imagination that is shaped by mathematics. At this point, however, it is worth noting that sometimes the peculiar beauty and elegance of mathematical structures can be deceptive. After all, the physicist creates a~physical theory and if a~given mathematical structure---regardless of its beauty---does not produce the expected results (predictive or unifying), deducing the next steps with its help may turn out to be a~dead end.\footnote{A~classic example here seems to be string theory 
\parencites[][pp.291--323]{kragh_higher_2015}[cf. also][]{baggott_farewell_2013}[or][]{hossenfelder_lost_2018}. %
%%\label{ref:RNDXSrB38bVgw}(Kragh, 2015),
%\parencite[][]{kragh_higher_2015}, %
% pp. 291--323, cf. also 
%%\label{ref:RNDqcgaa6PQNV}(Baggott, 2013)
%\parencite[][]{baggott_farewell_2013} %
% or 
%%\label{ref:RNDt7fCFuiyZm}(Hossenfelder, 2018).
%\parencite[][]{hossenfelder_lost_2018}. %
 It is otherwise interesting to note that physicists were already aware of the illusory allure of mathematical structures at the beginning of the twentieth century, i.e. in the period covered by the author's publication  
%\label{ref:RND7smeYY6QMd}(cf. Kalinowski, 1916, pp.17--18).
\parencite[cf., e.g.][pp.17--18]{kalinowski_nauka_1916}.%
}

At this point, it seems appropriate to propose another clarification. The author carries out a~kind of demythologisation of genius as an individual who, in a~way, by creatio ex nihilo, introduces completely new theories and, at the same time, breaks the chains of established, hitherto existing thought patterns
%\label{ref:RND7UsJY7hapG}(Sady, 2020, p.219).
\parencite[][p.219]{sady_struktura_2020}. %
 Such a~naive image of genius is untenable. It seems, however, that the statement that a~genius is someone who, in the given state of science, undertook the right research at the right time, equipped with appropriate mathematical abilities, cognitive passion and desire for recognition, requires some reflection 
%\label{ref:RNDgG8QOorGjm}(Sady, 2020, p.220).
\parencite[][p.220]{sady_struktura_2020}. %
 It seems that being famous (often also in the media) is one thing and being a~genius in a~given discipline is another. On the one hand, one cannot disagree with the statement that a~genius is someone who undertakes the right research at the right time. It seems, however, that this statement is insufficient, as it requires a~more precise definition of what exactly is meant by the term `proper research' and, moreover, is this the only characteristic of someone considered a~genius? In this context, it is worth considering not only which scientists physicists would name as geniuses, but, perhaps above all, what criteria they would use to make this choice.

How is it that the greatest number of physicists who could be considered geniuses have been identified in this period and not another? This is undoubtedly an interesting research problem---is it just a~matter of the fact that not enough time has elapsed yet, and at the same time the amount of work by contemporary physicists is too great to be able to assess their achievements and contributions to science? Or are there other processes at play here? Maybe the point is that in order for there to be a~sufficient number of physicists who will be able to point out the right directions of development, there needs to be an adequate quality of the generally understood cultural background, as Staruszkiewicz
\parencite*[][]{staruszkiewicz_wspolczesny_2001} %
seems to suggest?
%\label{ref:RNDSgkYSSOPho}(Staruszkiewicz, 2001)
%\parencite[][]{staruszkiewicz_wspolczesny_2001}%


Sady also draws attention to the almost determinant role of the development of knowledge in the field of a~given discipline, which undoubtedly diminishes the role of brilliant scientists. He also poses the question: what would have happened if Einstein had not existed
%\label{ref:RNDaLZBuVG64Z}(Sady, 2020, pp.120--121)?
\parencite[][pp.120--121]{sady_struktura_2020}? %
 Of course, this question can be extended to other scientists who are commonly regarded as brilliant, or at least those who made a~significant contribution to the development of physics (or any other discipline). The answer is probably complex. From the fact that research problems, as a~result of the development of physics, are taken up in many places independently (or almost independently), the author seems to derive a~conclusion (referring at the same time to the law of large numbers) that in any case the history would unfold in a~very similar way 
%\label{ref:RNDLThfOQ6eSb}(cf. Sady, 2020, pp.120--121, 213).
\parencite[cf.][pp.120--121, 213]{sady_struktura_2020}.%


This seems to be a~fairly well argued conclusion.\footnote{It should be noted here that he is not alone; an analogous view seems to be shared by
%\label{ref:RNDaWuvCczwTA}(Szlachic, 2010, p.238).
\parencite[][p.238]{szlachic_czy_2010}.%
} Sady notes, however, that we are dealing with the history of physics as it happened, we do not have alternative histories to examine. However, it is worth making an observation here, especially in the context of remarks about the possibility of the appearance of (independently) formally identical solutions to important physical problems.\footnote{This type of situation can be traced back to the history of Einstein's publication of the special theory of relativity, when an analogous proposal was also made by Poincaré. } The problem, in physics, is not the appearance of a~formal explaining structure. The difficulty lies in the fact that a~formal structure still needs a~physical interpretation, i.e. an appropriate connection of mathematical expressions with physical (measurable) quantities 
%\label{ref:RNDJiv3TTUARA}(cf. Heller, 2006, p.109).
\parencite[cf.][p.109]{heller_einstein_2006}.%
\footnote{This, at least indirectly, is also the issue of seeing someone as a~genius, at least in physics. The right research work at the right time alone is not enough here---unless research work is understood very broadly. At least a~few words of comment from Sady would be needed here.} So it seems that the number of formal solutions, occurring more or less at the same time, is not yet sufficient justification to state something with absolute certainty about a~possible alternative history of physics. It seems that far-reaching caution is needed here, even if one gets the irresistible impression of the necessity of a~high convergence of such alternative histories. In this context, the occurrence of a~kind of ``determinism of development of knowledge'' seems almost apodictically suggested by the Author almost from the first pages of the book 
%\label{ref:RNDIUEvdKe8G3}(cf. Sady, 2020, p.26).
\parencite[cf.][p.26]{sady_struktura_2020}. %
 Such a~situation is controversial though the main idea of continous development of science does not seem to raise any major objections. The more so, as both the role of continuous development of knowledge, without Kuhn's revolutions, but with qualitative changes, is known, as exemplified by K. Szlachcic's analysis of P. Duhem's works 
%\label{ref:RNDzdOFUtnbYg}(Szlachic, 2010, pp.235--240)
\parencite[][pp.235--240]{szlachic_czy_2010}.%
\footnote{Interestingly, Szlachcic's work also contains an interesting juxtaposition of some of Sady's theses, especially about scholars not making hypotheses, which were also found in Sady's work under discussion.} The lack of any mention of the French physicist's views in this respect is puzzling to say the least, all the more so because an analogous approach, i.e. taking into account the continuity of the development of physics---thus emphasising the role of intermediate states between great discoveries---is the method used by the Author himself, who at the same time sees Kuhn's fundamental error precisely in the neglect of such an approach 
%\label{ref:RNDGhUXLTxULo}(cf. Sady, 2020, p.13)
\parencite[cf.][p.13]{sady_struktura_2020}.%
\footnote{In an era of an enormous amount of literature in the philosophy of science, it is almost impossible to take into account all the positions on a~given issue. Nevertheless, in aforementioned context, it would be appropriate to mention at least a~few Polish works on analogous issues, such as 
%\label{ref:RNDUosXPwAm0W}(Kokowski, 1993).
\parencite[][]{kokowski_proba_1993}.%
}

Somewhat problematic, not clearly explained and, it seems, not justified at all here is Sady's linking of the quality of scientific development with liberal democracy. One can easily see that two problems arise here. The first one is clear contradiction in theses posed by Sady. From one hand, there is a~kind of determinism of knowldege development, which is so fiercy defended by Sady and which seems to represent an internalist approach to issue of this developement. On the other hand however, Sady puts forward the thesis which refers to purely externalist approach to development of knowledge and -- in the way it has been presented -- is of doubtful justification. The second problem is a~certain difficulty in comprehending author's reasoning, because he does not explains what exactly he means when referring to liberal democratic system, while statements about its beneficial influence seem to be of an apodictic nature\footnote{This term however, would require an explanation.}
%\label{ref:RNDNsKA1W5Uvb}(cf. Sady, 2020, pp.20--21).
\parencite[cf.][pp.20--21]{sady_struktura_2020}. %
 Such way of their expression suggests also great importance of these statements however, this importance---particularly in context of difficulty mentioned---is nowhere explained. It is also completely unclear what would be meant by the statement that the attitude of scholars to the scientific environment is liberal-democratic 
%\label{ref:RNDx5nNphQHpD}(Sady, 2020, p.20).
\parencite[][p.20]{sady_struktura_2020}. %
 At the very least, the praise of liberal democracy in the context of the work described is questionable, especially since none of the discoveries described, nor any of the scholars referred to in the book, produced under such conditions. To say that it is no coincidence that the appearance of Newton's Principia and Lock's Second Treatise on Government 
%\label{ref:RNDVinD0EokVm}(Sady, 2020, p.21),
\parencite[][p.21]{sady_struktura_2020}, %
 does not seem to be a~good argument in favour of liberal democracy as the optimal environment for the development of science, since this was precisely not the social system that prevailed in the late 17\textsuperscript{th} century in the British Kingdom. Similarly, one can question this thesis in the case of Planck, Einstein or Bohr (the social system in which they functioned). One can agree, however, that a~certain intellectual freedom and a~certain well-being of a~given community (society, state) are necessary for the state of discipline to undergo a~fundamental change, by analogy with, for example, the history of philosophy and science in Ancient Greece.\footnote{Interesting insights on the link between the division of labour and the possibility of developing pure science and technical applications in Ancient Greece are provided by 
%\label{ref:RNDPf4Ja3PH3j}(Russo, 2004),
\parencite[][pp.185-202.]{russo_forgotten_2004}.%
} Perhaps, then, what is at stake is not so much this particular social system as the creation of a~certain quality of culture or environment, within which scholars live and work?

A~similar impression is given by the rather vague, yet strong statements made about social relations and the impact (positive or negative) that these relations have on the functioning of scientists significant for the development of physics. It is rather disingenuous to juxtapose Nazism and attachment to the homeland, as Sady does in the case of Lenard
%\label{ref:RND141YoitJxS}(Sady, 2020, p.154).
\parencite[][p.154]{sady_struktura_2020}. %
 However, the two attitudes are not the same. Similarly puzzling is the statement that: ``the progress of the sciences is fostered by systems of liberal democracy and a~sense of being a~citizen of the world rather than of a~small or large homeland'' 
%\label{ref:RNDx8Vw2GHCPD}(Sady, 2020, p.154)
\parencite[][p.154]{sady_struktura_2020}. %
 This is an example of, on the one hand, an apologia for liberal democracy (already mentioned above) and, on the other hand, a~statement that is at best only partially true. To what extent the sense of being a~citizen of the world influenced Einstein's pursuit of science---the flagship example in Sady's work 
%\label{ref:RNDBE8fjIBNIK}(Sady, 2020, p.117)
\parencite*[][p.117]{sady_struktura_2020}---at best, should be regarded as very vague.\footnote{It is true that, how Pais put it, giving up the German citizenship in 1895, moving to Italy and entering freer life and independent work transformed Einstein positively. It is also however, true that on February 21, 1901 he was granted the Swiss citizenship and for the rest of his life he remained a~citizen of Switzerland. It is the point which makes that statement of importance influence of being citizen of the world had great on his scientific work seems to be, at best, unconvincing. Seeing Einstein's genius in being revolutionary rebel resisitng authority and free-minded, as Sady seems suget, is also unjustified  
%\label{ref:RNDidfSxR4H95}(Pais, 2005),
\parencite[cf.][pp.38--45]{pais_subtle_2005}.%
} However, these Sady's theses seem completely inapplicable to other greats of physics, namely Bohr and, especially, Heisenberg.

It also seems that the rather apodictic nature of some of Sady's statements somewhat obscures the presentation of certain issues in history of science. A~good example is the passage in which the Author reconstructs the way in which Coulomb arrived at the dependence of the value of the electrostatic force on the distance between charges and their values. He first notes that we are unable to reconstruct the exact course of Coulomb's thought. The reconstruction of the whole reasoning does not seem to raise any objections, and the Author leads the Reader to the conclusion that Coulomb's reasoning was essentially deductive. He states at the same time that the thesis about the underdetermination of the theory by data does not apply here, and if it did---then it could be rejected on the basis of what was imposed on Coulomb
%\label{ref:RND1sQIoDKBcf}(Sady, 2020, pp.32--33).
\parencite[][pp.32--33]{sady_struktura_2020}. %
 The defence of the thesis of undetermination of scientific theory by evidence may, however, as the author remarks, be justified here by the fact that through any set of points any number of curves may be drawn. Hence the question why Coulomb chose the relation inversely proportional to the square of the distance? Sady's answer is as follows: he chose the simplest curve 
%\label{ref:RNDAG1nmZHykI}(Sady, 2020, pp.32--33).
\parencite[][pp.32--33]{sady_struktura_2020}. %
 A~difficulty arises here, because if it is a~choice, then an objection arises as to the deductive character of the whole reasoning. What is missing here is a~more subtle reconstruction of the connection between the formal element and the experimental data gathered by Coulomb. A~side issue is the Author's assertion that according to Poincaré the criterion here would be beauty, and according to him simplicity (the simplest curve). The problem is that one of Poincaré's criteria for choosing appropriate formal structures is, in addition to beauty, also simplicity.

It also seems that one of the central determinants of scientificity, which for Sady is the systematic nature of research, especially in the experimental field, should be more carefully discussed. On the one hand, it is difficult to deny that this feature is important for scientific research. However, the question arises as to how important it is, the more so that---unfortunately neglected---processes within the framework of which both the general relativity theory of relativity and quantum mechanics were formed could significantly weaken such a~strong emphasis on this very category.

All the above observations by no means undermine the initial observation that Sady's book is certainly worth recommending and constitutes an interesting proposal for a~look at scientific revolutions. As far as the historical layer is concerned, it remains to be wished that other presentations of issues from the history of physics will be equally interesting.

It is also obvious that it is impossible to include everything in any work---all the more so as the amount of literature produced every day is staggering, and its search and study has long outstripped one man's ability. It is therefore not surprising that the author did not manage to include everything.

As far as the philosophical, historical and physical content is concerned---if the Author had wished to discuss it earlier within the framework of some interdisciplinary seminar, it would undoubtedly have been a~much better position. One should, however, take into account the fact that its volume would probably have increased considerably. Therefore, it remains to wait for the second edition of the work and---if it turns out to be possible---for the second part devoted to the general relativity and quantum mechanics.

The reader will undoubtedly enjoy a~very inspiring position, forcing her or him to rethink her or his own views on the issues raised.




%---------------



\vspace{15mm}%
{\subsubsectit{\hfill Abstract}}\\
{In his book Wojciech Sady attempts to reconstruct the structure of the fundamental transformations that can be described as the relativistic and quantum revolution. Referring to rich historical material and Ludwik Fleck's reflections on the development of scientific knowledge, the author tries to explain how it is possible that ``scientists began to think differently than they had been taught.'' Sady's work, although not devoid of somewhat weaker points, is a~brave and thought-provoking attempt to propose his own explanation of the mechanisms of the aforementioned transformations.}\par%
\vspace{2mm}%
{\subsubsectit{\hfill Keywords}}\\
{scientific revolution, old quantum theory, special relativity, thought collective, physics, philosophy of science.}%



\end{newrevengenv}