\begin{newrevplenv}{Milena Cygan}
	{Nie-ludzkie emocje}
	{Nie-ludzkie emocje}
	{Nie-ludzkie emocje}
	{Uniwersytet Papieski Jana Pawła II w Krakowie}
	{Frans de Waal, \textit{Ostatni uścisk Mamy: emocje zwierząt i~co one mówią o~nas samych}, tłum. R.~Kosarzycki, Copernicus Center Press, Kraków 2019, ss.~392.}




\lettrine[loversize=0.13,lines=2,lraise=-0.03,nindent=0em,findent=0.2pt]%
{U}{}czucia, emocje to w~dzisiejszych czasach temat aktualny i~wielce pożądany. Nie było tak jednak zawsze. Oczywiście, błędem byłoby mniemać, że filozofia czy nauka lekceważyła lub próbowała wyrugować zagadnienie emocji ze swojego dyskursu. Przeciwnie, zainteresowanie filozofów emocjonalną sferą człowieka sięga daleko w~przeszłość, do czasów starożytnych. Niemniej, często postrzegano kierowanie się nimi jako coś ,,poniżej godności człowieka'', gdyż od zwierząt miał człowieka odróżniać rozum. Ideałem było więc kierowanie się i~życie zgodnie z~rozumem i~jego zasadami, nie zaś emocjami. Postrzeganie emocji zaczęło się zmieniać w~wieku XIX wraz ze zmierzchem oświeceniowego racjonalizmu, gdy kolejne pokolenia uczonych (filozofów, socjologów, psychologów, lekarzy) zaczęły poważniej przyglądać się emocjonalności, uznając jej istotną rolę w~ludzkiej egzystencji. Dziś już nikt nie ma wątpliwości, jak ważną funkcję pełni ta sfera w~życiu człowieka, i~jak katastrofalne bywa jej niedocenienie i~zaniedbanie
%\label{ref:RNDLgjysX9nLe}(Solomon, 2005; Głos, Załuski, 2017).
\parencites[][]{solomon_filozofia_2005}[][]{glos_negative_2017}. %
 Jednakże, co ciekawe, o~ile dawniej emocje traktowano jako element natury ,,zwierzęcej'' w~człowieku, o~tyle obecnie stały się one czymś tak ,,ludzkim'', że przestano zwracać uwagę na ich obecność u~zwierząt. Dlaczego tak łatwo przyszło nam odmówić zwierzętom czegoś, co wydaje się tak oczywiste? Jak wygląda świat zwierzęcych emocji? Czy są one inne niż te, których doświadczają ludzie? Czy zajmowanie się i~badanie zwierzęcej emocjonalności może nam dostarczyć także wiedzy o~naszym własnym gatunku?

Wokół powyższych pytań oscyluje treść najnowszej książki Fransa de Waala, jednego z~najbardziej rozpoznawalnych etologów i~prymatologów na świecie. Jest to już kolejna książka tego amerykańskiego profesora wydana przez CCPress\footnote{Do tej pory nakładem CCPress ukazały się następujące książki de Waala: \textit{Małpy i~filozofowie: skąd pochodzi moralność}?
%\label{ref:RNDx4MurmHGWb}(2013),
\parencite*[][]{waal_malpy_2013}, %
 \textit{Bonobo i~ateista: w~poszukiwaniu humanizmu wśród naczelnych} 
%\label{ref:RNDUKYRnnQUCd}(2014),
\parencite*[][]{waal_bonobo_2014}, %
 \textit{Małpa w~każdym z~nas: dlaczego seks, przemoc i~życzliwość są częścią natury człowieka}? 
%\label{ref:RNDI1reL44LM4}(2015),
\parencite*[][]{waal_malpa_2015}, %
 \textit{Bystre zwierzę: czy jesteśmy dość mądrzy, by zrozumieć mądrość zwierząt}? 
%\label{ref:RND2RVHHXgHmw}(2016),
\parencite*[][]{waal_bystre_2016}, %
 \textit{Wiek empatii: jak natura uczy nas życzliwości} 
%\label{ref:RNDtsG5PtsOjs}(2019b).
\parencite*[][]{waal_wiek_2019}. %
 Dwie z~nich były już recenzowane na łamach ZFN 
%\label{ref:RNDLwVkgDcdfM}(Sarosiek, 2018; Cygan, 2019).
\parencites[][]{sarosiek_odmiennosc_2018}[][]{cygan_empatia_2019}.%
}. W~recenzowanej pracy przedmiotem zainteresowań de Waala jest to, co można by nazwać ,,wewnętrznym światem zwierząt''. W~swoich badaniach i~publikacjach koncentruje się więc przede wszystkim na zachowaniach społecznych zwierząt. Badania, które u~początku naukowej kariery naszego autora dotyczyły agresji w~zwierzęcym świecie, doprowadziły go do takich zagadnień, jak problem zwierzęcej moralności czy empatii, czyniąc go jednym z~bardziej uznanych specjalistów w~tym obszarze.


Bodźcem do napisania książki \textit{Ostatni uścisk Mamy} stało się ostatnie, pełne emocji spotkanie i~pożegnanie profesora Jana van Hooffa z~umierającą Mamą, starą szympansicą w~Burgers Zoo. Zwierzę zachowywało się bardzo ,,po ludzku'': delikatnie obejmowało profesora, pocieszająco gładząc oraz poklepując go po głowie i~karku. Całe zdarzenie zostało nagrane, a~potem wyemitowane w~telewizji i~internecie. Komentarze i~reakcje ludzi po obejrzeniu filmu zdradzały wielkie poruszenie, a~nawet pewnego rodzaju szok, związany z~rozpoznaniem tak ,,ludzkich'' gestów i~reakcji zaprezentowanych przez przedstawicielkę odmiennego gatunku. Owo ,,zdumienie'' widzów, zdaniem de Waala, dowiodło tylko ,,niskiej opinii ludzi o~zdolnościach emocjonalnych i~umysłowych zwierząt''
%\label{ref:RNDcI91FgsPj7}(Waal, 2019a, s.~22)
\parencite[][s.~22]{waal_ostatni_2019} %
 oraz nieświadomości, że ,,gest, który jest tak bardzo ludzki, w~rzeczywistości, jest powszechny także u~wszystkich naczelnych'' 
%\label{ref:RND7ICQhoRTCb}(Waal, 2019a, s.~25).
\parencite[][s.~25]{waal_ostatni_2019}. %
 Kilka czułych i~wzruszających gestów Mamy wystarczyło, żeby pokazać ludziom, że nie wszystko, co ludzkie, jest też i~obce innym gatunkom, a~nas samych łączą z~nimi powiązania ewolucyjne o~wiele silniejsze niż się to powszechnie wydaje. Reakcje widzów na ,,ostatnie spotkanie'' sprowokowały więc de Waala do napisania książki, mającej jeszcze raz przypomnieć ludziom, że nie wszystko, co uważamy za specyficznie ludzkie, jest takie w~swojej istocie.

Na strukturę książki składa się osiem rozdziałów, z~których tylko pierwszy nawiązuje bezpośrednio do postaci Mamy i~jej wyjątkowej roli, jaką zajmowała w~kolonii szympansów tamtejszego zoo. Stosunki tej szympansiej matriarchini z~resztą stada posłużyły de Waalowi do przedstawienia złożonych relacji istniejących w~szympansich społecznościach. Analizując funkcje Mamy w~stadzie, autor niejako ,,reinterpretuje'' tezę głoszącą, że władza jest zawsze w~rękach najsilniejszego osobnika. Zwraca uwagę, że niekoniecznie chodzi tu o~siłę fizyczną. Ta bowiem może zapewnić wysoką pozycję w~hierarchii, ale to nie jest jeszcze tożsame z~posiadaniem realnej władzy. W~przypadku Mamy ,,wszystkie dorosłe samce przewyższały ją pozycją, ale jak przychodziło co do czego, wszystkie jej potrzebowały...''
%\label{ref:RNDlDAjDfmxXB}(Waal, 2019a, s.~51).
\parencite[][s.~51]{waal_ostatni_2019}. %
 Mama, pomimo niższej pozycji w~stadzie, dzierżyła faktyczną władzę, czyli miała realny wpływ na procesy, jakie zachodziły w~grupie. Szacunek i~poważanie, jakim cieszyła się stara szympansica, posłużył uczonemu do ciekawych obserwacji dotyczących reakcji naczelnych na śmierć innego członka stada. Okazuje się, że śmierć nie jest wydarzeniem ,,obojętnym'', wymaga specjalnych zachowań i~swego rodzaju etykiety. Jest ona tym silniej przeżywana, im większa była więź emocjonalna pomiędzy osobnikami. ,,Wszystko wskazuje na to -- konkluduje autor -- że przynajmniej niektóre zwierzęta uświadamiają sobie, iż martwy towarzysz już nigdy się nie poruszy. […] uświadomienie sobie nieodwracalności wskazuje na oczekiwania dotyczące przyszłości'' 
%\label{ref:RNDbBKZAAvg3J}(Waal, 2019a, s.~55).
\parencite[][s.~55]{waal_ostatni_2019}. %
 Uczony kojarzy zaobserwowane reakcje z~poczuciem ostateczności (oczywiście, nie jest ono tożsame z~ludzkim poczuciem śmiertelności).

Kolejne rozdziały książki przybliżają zwierzęce emocje i~zachowania, które zazwyczaj kojarzą się nam z~cechami wyróżniającymi nasz własny gatunek, jak: śmiech i~uśmiech, zdolność do empatii, intencjonalnego okrucieństwa, morderstwa z~premedytacją, odrazy, poczucia wstydu, sprawiedliwości, czy winy -- czyli do wszystkich tych emocji, które ,,czynią z~nas ludzi''. De Waal w~tych kwestiach właściwie przypomina, podsumowuje i~streszcza to, co już zawarł w~poprzednich swoich pracach. Niemniej tym, co może przyciągnąć uwagę, są refleksje na temat antropomorfizacji, czyli problemu, z~którym musi się zmierzyć każdy, kto podejmuje się rozważania, opisywania i~nazywania emocji zwierząt i~ich zachowania. Czy bowiem naukowiec powinien powstrzymać się przed użyciem antropomorfizmów? I~czy w~ten sposób uniknie się mylenia tego co ludzkie, z~tym co zwierzęce? Autor, wyjaśniając swoje podejście do tego zagadnienia, wyróżnia dwa typy antropomorfizmu, które (nieco upraszczając) można określić jako antropomorfizm użyteczny i~antropomorfizm szkodliwy. Pierwszy jest wręcz konieczny, drugi zaś nie do przyjęcia. Z~jednej strony uważa więc, że stosowanie antropomorfizmu jest szkodliwe i~pozbawione najmniejszego sensu. Świat natury odznacza się bowiem takim bogactwem i~różnorodnością, że nie można go zredukować, czy zamknąć w~jednym wymiarze. Proste porównania ludzi i~zwierząt nie zdają w~tym kontekście egzaminu. Frans de Waal protestuje więc przeciwko temu rodzajowi naukowego postępowania, które opierało się na porównywaniu poszczególnych gatunków, a~które to postępowanie miało swoje źródło w~arystotelesowskiej koncepcji \textit{scala naturae}, będącej drabiną hierarchicznie ustawionych bytów. De Waal twierdzi, że nie da się w~tak prosty sposób zhierarchizować wszystkich stworzeń, rzeczywistość jest bowiem bardziej złożona: ,,Czy nie powinno się oczekiwać, że każde zwierzę ma swoją własną inteligencję oraz emocje, dostosowane do własnych zmysłów i~historii naturalnej?''
%\label{ref:RNDBrbxPuvqN1}(Waal, 2019a, s.~64).
\parencite[][s.~64]{waal_ostatni_2019}. %
 Wzięcie tego pod uwagę czyni antropomorfizację nieuzasadnioną, a~przekładanie prostych, bezkrytycznych kalek nie jest najlepszym pomysłem: ,,Nieuzasadniony antropomorfizm jest zdecydowanie nieprzydatny'' 
%\label{ref:RND5bbBJw5EVP}(Waal, 2019a, s.~64).
\parencite[][s.~64]{waal_ostatni_2019}.%


Z~drugiej jednak strony, de Waal stwierdza, że antropomorfizm bywa użytecznym i~ważnym elementem metodologii badań nad zwierzętami. Biorąc pod uwagę nasze obserwacje, intuicje, a~także badania z~zakresu etologii porównawczej, genetyki i~neurobiologii, okazuje się, że ludzie i~zwierzęta (zwłaszcza naczelne) funkcjonują w~bardzo podobny, a~czasem nawet identyczny sposób. To zaś sprawia, iż, zdaniem de Waala, antropomorfizm niejednokrotnie może okazać się cennym narzędziem poznawczym. Z~tego względu dla de Waala większym zagrożeniem jest raczej odrzucenie antropomorfizacji niż jej stosowanie, co też przekłada się ostatecznie na jego pozytywne ustosunkowanie się do użyteczności antropomorfizacji w~badaniach nad (niektórymi) zwierzętami. ,,Osobiście -- pisze de Waal -- postrzegam odrzucanie podobieństwa ludzi i~zwierząt za większy problem niż jego zakładanie''
%\label{ref:RND6KhMyo9k0B}(Waal, 2019a, s.~64).
\parencite[][s.~64]{waal_ostatni_2019}. %
 Stwierdza: ,,Nasze mózgi posiadają tę samą budowę jak mózgi innych ssaków: nie mamy żadnych nowych części i~stosujemy te same neuroprzekaźniki'' 
%\label{ref:RNDFiIw4RhH32}(Waal, 2019a, s.~65).
\parencite[][s.~65]{waal_ostatni_2019}. %
 Autor podkreśla więc, że zakładane podobieństwo pomiędzy ludźmi i~zwierzętami leżące u~początków antropomorfizmu nie jest już tylko prostym założeniem, w~sukurs przychodzi mu neurobiologia, która zaciera linię pomiędzy zwierzęciem a~człowiekiem. Taki radykalny dualizm jest już, jak przekonuje de Waal, nie do utrzymania, a~,,antropomorfizm nie jest nawet w~części tak zły, jak ludzie uważają'' 
%\label{ref:RND60WpZliYyT}(Waal, 2019a, s.~65).
\parencite[][s.~65]{waal_ostatni_2019}.%


Najnowsza książka de Waala, jak wspomniano wcześniej i~jak sugeruje sam jej tytuł, nie ma na celu wyłącznie przybliżenia nam życia emocjonalnego zwierząt. Autor nie stara się ani przekonać czytelnika, ani mu udowodnić, że zwierzęta posiadają emocje czy uczucia (z tymi, sprawa jest nieco bardziej skomplikowana, ze względu na ich ,,ukryty'' i~,,wewnętrzny'' charakter), a~nawet bardzo bogaty i~zróżnicowany ich wachlarz. Książka, zgodnie z~intencją i~zamierzeniem autora, jest napisana nie tylko po to, żeby lepiej poznać zwierzęta, lecz również by lepiej zrozumieć człowieka. De Waal ze swoim naturalistycznym zorientowaniem chce pokazać nam, jak bardzo nie różnimy się od zwierząt. Oczywiście, nie znaczy to, że stawia prosty znak równości pomiędzy ludźmi i~zwierzętami. Wspomniany cel szczególnie uwidacznia się w~rozdziale, w~którym omawia inteligencję emocjonalną. Emocjonalność spełnia bowiem konkretne funkcje, nie jest sprzeczna z~racjonalnością i~tak zwaną ,,wolną wolą''. Pomiędzy tymi trzema elementami istnieje coś w~rodzaju ,,sprzężenia''. Wszystkie trzy siebie zakładają i~potrzebują. Są konieczne do tego, by dany osobnik mógł sprawnie funkcjonować, niezależnie czy jest on zwierzęciem, czy człowiekiem. Popularna narracja -- jak podkreśla de Waal -- ciągle jeszcze sprowadza zachowanie zwierząt do instynktów, uważając je za podstawowy mechanizm, jaki nimi steruje. Instynkty to jednak reakcje odruchowe i~-- jak przekonuje etolog -- w~większości bezużyteczne w~ciągle zmieniającym się świecie. Nie znaczy to, że ludzie i~zwierzęta nie działają instynktownie. To się zdarza, ale zachowanie zwierząt jest raczej zdominowane przez emocje, przez co i~zdecydowanie bardziej elastyczne od czysto instynktownego. De Waal nazywa emocje inteligentnymi instynktami, ponieważ w~pewnym sensie należą do odruchów, lecz zazwyczaj wywołują pożądaną zmianę zachowania po dokładnej, choć niekiedy bardzo krótkiej i~szybkiej ocenie sytuacji.

Co prawda w~pełni nie kontrolujemy emocji, ale i~nie jesteśmy także ich niewolnikami: ,,Emocjonalne zachowanie ma w~sobie pewien element dobrowolności''
%\label{ref:RNDBd9HguxcfU}(Waal, 2019a, s.~248).
\parencite[][s.~248]{waal_ostatni_2019}. %
 Emocje nigdy nie są ,,tylko emocjami i~nigdy nie są w~pełni automatyczne'' (tamże). Skorelowane zarówno z~ciałem jak i~umysłem są również elementarnym składnikiem naszego poznania, idącym w~parze z~rozumem i~w pewnym sensie udoskonalającym jego działanie, podobnie jak i~rozum bywa pomocny emocjom. ,,Czysty rozum'' wcale nam nie ułatwia życia, a~beznamiętność, jeśli się niekiedy zdarza, jest raczej upośledzeniem niż ideałem, do którego powinno się dążyć. Emocje nie są przeszkodą w~życiu, nie rujnują nam życia, przeciwnie -- niejednokrotnie mogą je uratować, a~na pewno usprawniają nasze działanie. De Waal podkreśla zatem ogromną poznawczą rolę emocji. Analizuje zdolność odczytywania emocji u~innych, wykorzystywania informacji przekazywanej przez emocje oraz kontrolowania własnych emocji, służące do osiągania złożonych celów. Ilustruje to wieloma badaniami i~przykładami, zarówno ze świata ludzi, jak i~zwierząt.

Sporo uwagi poświęca Autor zagadnieniu ,,panowania nad emocjami''. Dotyka ono bowiem problemu ,,wolnej woli'', o~którym wspomniano wyżej. De Waal, włączając się niejako w~wielowiekowy ,,spór o~wolną wolę'', stawia to pytanie także w~kontekście zwierząt. Czy zwierzęta posiadają ,,wolę'', a~w dodatku ,,wolną''? Na pewno nie są automatami, którymi kierują proste instynkty, ale jak radzą sobie z~emocjami? Czy zwierzęta potrafią je kontrolować? Czy potrafią robić to w~sposób świadomy? Czy zwierzę jest zdolne do dokonania wyboru w~sposób świadomy i~wolny? De Waal odpowiada na te pytania wskazując, że zwierzęta (a przynajmniej nie wszystkie) wcale nie są tak zdeterminowane przez swoje potrzeby, pragnienia i~popędy jak to się powszechnie wydaje. Posiadają one zdolność do samokontroli tak samo jak i~ludzie. Analizując przykłady zachowań zwierząt, które świadczą o~sprawowaniu kontroli nad popędami, de Waal stwierdza, że: ,,Zwierząt nie stać na to, aby ślepo gonić z~swoimi impulsami. Ich emocjonalne reakcje zawsze przechodzą przez ocenę sytuacji i~rozważanie dostępnych opcji. Dlatego też wszystkie mają zdolność samokontroli''
%\label{ref:RNDjnOvf4PDsJ}(Waal, 2019a, s.~279).
\parencite[][s.~279]{waal_ostatni_2019}. %
 Samokontrola i~połączona z~nią wola są ponadto sprzężone z~racjonalnością, gdyż ,,zwalczanie impulsu podążania jednym tokiem działania i~zmiana go na inny, który obiecuje lepszy wynik, jest oznaką racjonalnego myślenia'' 
%\label{ref:RNDpBCG1OHT8z}(Waal, 2019a, s.~279).
\parencite[][s.~279]{waal_ostatni_2019}.%


Książka Fransa de Waala \textit{Ostatni uścisk Mamy} ze względu na to, że zawiera w~sobie niejako streszczenie całego dotychczasowego dorobku autora, przynajmniej jeśli chodzi o~książki, może być uważana za prawdziwą ,,kopalnię'' tematów, zagadnień i~problemów, których nie sposób nawet wymienić, a~cóż dopiero omówić w~jednej recenzji. Bogate doświadczenie autora w~pracy ze zwierzętami przekłada się na bogatą treść książki i~dość swobodny styl narracji, za pomocą którego pragnie się on swoją szeroką wiedzą podzielić z~czytelnikiem. Jednym z~wniosków, do jakich dochodzi autor jest teza, że nie ma czegoś takiego jak unikatowe ludzkie emocje, których nie posiadałyby zwierzęta i~że są one czymś powszechnym, ponadgatunkowym i~uniwersalnym. Problem emocjonalności zwierząt niewątpliwie stanowi wezwanie do przemyśleń i~refleksji nad człowiekiem i~jego miejscem w~świecie. W~związku z~tym problemem wynikającym z~rozważań de Waala może być jego naturalistyczna wizja człowieka, opierająca się na założeniu ciągłości zmian ewolucyjnych. Różnica pomiędzy człowiekiem a~innymi gatunkami to dla de Waala różnica raczej w~stopniu rozwoju, nie zaś w~jakości, stąd pewnie jego stanowisko stanowi pole do polemik, zwłaszcza dla tych, którzy chcieliby widzieć w~człowieku byt o~zupełnie nowej jakości, czy wręcz supergatunek. Ponadto sama problematyka emocjonalności zwierząt związana jest z~kwestiami etyki i~moralności. I~nie chodzi tu nawet o~pytanie, czy zwierzęta mają moralność (tym de Waal zajmuje się w~innej swojej książce), ale raczej o~naszą perspektywę i~określenie bądź też zmianę naszego stosunku do zwierząt, sposobu ich traktowania, który wciąż przedstawia wiele do życzenia. Skoro zwierzęta nie są automatami pozbawionymi uczuć i~świadomości, jak niekiedy chcielibyśmy je postrzegać, to ciąży na nas wielki dylemat moralny, zwłaszcza jeśli weźmie się pod uwagę hodowlę przemysłową, praktyki laboratoryjne i~wszelkie inne czynności i nadużycia związane z~ich wykorzystywaniem. Książka de Waala może dostarczyć argumentów za tym, aby szanować wszystkie formy życia, dostrzec ich ,,godność'', którą mają przez sam fakt swojego istnienia, interes, jaki każda forma posiada we własnym istnieniu i~przetrwaniu, nie mówiąc już o~wrażliwości i~zdolności do cierpienia. Z~drugiej strony dostarcza także argumentów do tego, że nie musimy zbytnio idealizować i~ulegać sentymentalizmowi. Jesteśmy częścią tego świata, również świata zwierząt, wraz z~całym jego ,,kręgiem życia'', i~wszystkimi tego, niekiedy brutalnymi konsekwencjami. Z~biologicznego czy ewolucyjnego punktu widzenia człowiek nie jest ani wegetarianinem, ani weganinem. Mięso i~produkty pochodzenia zwierzęcego stanowiły istotną część naszej diety, bez czego być może nie bylibyśmy tym gatunkiem, którym jesteśmy. Czy musimy jeść aż tyle owych ,,produktów'', ile ich przeciętnie spożywamy, to już inna kwestia. Na pewno jest ona godna przemyślenia, jak i~wiele innych, do których prowokuje lektura niniejszej książki.

Czytając książkę Fransa de Waala przychodzi na pamięć zdanie, jakie zapisał w~swoich \textit{Rozmyślaniach} kilkanaście wieków temu Marek Aureliusz, że ,,nic nie powinno się nazywać cechą człowieka, co nie tyczy się człowieka jako człowieka''
%\label{ref:RND5Bo2giWckh}(Marek Aureliusz, 2003, s.~44).
\parencite[][s.~44]{marek_aureliusz_rozmyslania_2003}. %
 Z~pewnością, emocje, zdolność do odczuwania czy przeżywania nie są tymi cechami, które w~jakiś zasadniczy sposób wyróżniają człowieka. Niestety, uznanie tego, że zwierzęta doświadczają emocji, czy że sposób ich doświadczania niekoniecznie musi się tak bardzo różnić od naszego, budzi ciągle jeszcze spore kontrowersje. Być może przyczyną jest to, że zwierzęta nie informują nas o~swoich uczuciach, a~może dla wielu istnienie emocji i~uczuć pociąga za sobą pewien poziom świadomości, którego nie są zbyt skłonni przypisywać zwierzętom. W~każdym razie, \textit{Ostatni uścisk Mamy. Emocje zwierząt i~co one mówią o~nas samych}, nie jest próbą dowodzenia istnienia emocji zwierząt, lecz raczej refleksją nad tym, dlaczego ciągle tak mało o~nich wiemy: ,,[…] pytanie nigdy nie brzmiało: czy zwierzęta mają emocje, lecz w~jaki sposób nauka była w~stanie tak długo ich nie dostrzegać? Nie było tak od początku... Jakim cudem robiliśmy wszystko, by odmówić zwierzętom czegoś tak oczywistego?'' 
%\label{ref:RNDcVgdPhk8Yt}(Waal, 2019a, s.~335).
\parencite[][s.~335]{waal_ostatni_2019}. %
 Odpowiedź na te pytania, do których przemyślenia autor zaprasza czytelnika, jest więc zaproszeniem do rozważań, które mogą rzucić nieco nowego światła na nasze własne istnienie i~postawić nas w~obliczu nieustannie podnoszonego problemu o~naturę człowieka, o~to, ,,kim jest człowiek?''. Książka Fransa de Waala jest więc dobrym przykładem tego, jak zagadnienia naukowe mogą prowadzić do filozoficznych refleksji.



%-------------------------

\pagebreak


\selectlanguage{english}
\vspace{5mm}%
\begin{flushright}
{\chaptitleeng\color{black!50}{Non-human emotions}\nopagebreak[4]}
\end{flushright}

%\vspace{10mm}%
{\subsubsectit{\hfill Abstract}}\\
{This article is a review of Frans de Waal's book \textit{Mama's Lust Hugs. Animal Emotions and what They Tell Us about Ourselves}, which was released in Polish in 2019. The book deals with the problem of animal emotionality. One of the author's conclusions, which is underlined in the review, is that that there is no such thing as unique human emotions that animals would not have. Emotions are universal; they are shared by both humans and animals. Although the book is intriguing, it does not contain anything new in terms of content, as it deals with the topics that de Waal has previously addressed in his writings. The publication, on the other hand, can be an excellent starting point for reflection not just on animals (their emotionality and rights), but also on people and their place in the world, thanks to collecting and integrating topics relating to animal emotionality into a single book. It confronts the reader with the issue of human nature. As a result, as this review  attempts to demonstrate, Frans de Waal's book is an excellent example of how scientific issues lead to philosophical insights.
}\par%
\vspace{2mm}%
{\subsubsectit{\hfill Keywords}}\\%
{emotions, animals, human, behavior, ethology, philosophy, Frans de Waal.}%

\selectlanguage{polish}




\end{newrevplenv}
