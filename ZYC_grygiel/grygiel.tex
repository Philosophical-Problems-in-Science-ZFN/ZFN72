\begin{artengenv}{Wojciech Grygiel}
	{A~critical analysis of the philosophical motivations and development of the concept of the field of rationality as a~representation of the fundamental ontology of the physical reality}
	{A~critical analysis of the philosophical motivations\ldots}
	{A~critical analysis of the philosophical motivations and development of the concept of the field of rationality as a~representation of the fundamental ontology of the physical reality}
	{Pontifical University of John Paul II in Krakow}
	{The unusual applicability of mathematics to the description of the physical reality still remains a~major investigative task for philosophers, physicists, mathematicians and cognitive scientists. The presented article offers a~critical analysis of the philosophical motivations and development of a~major attempt to resolve this task put forward by two prominent Polish philosophers: Józef Życiński and Michał Heller. In order to explain this particular property of mathematics Życiński has first introduced the concept of the field of rationality together with the field of potentiality to be followed by Heller's formal field and the field of categories. It turns out that these concepts are fully intelligible once located within philosophical stances on the relations between mathematics and physical reality. IIt will be argued that in order to achieve more extended conceptual clarification of the precise meaning of the field of rationality, further advancements in the understanding of the nature of human mind are required.}
	{ontology, mathematics, platonism, category theory, Roger Penrose, Alfred North Whitehead.}



\section{Introduction}
\lettrine[loversize=0.13,lines=2,lraise=-0.01,nindent=0em,findent=0.2pt]%
{T}{}he problem of the ontology implied by the contemporary physical theories ties up closely with the philosophical issues pertaining to the nature of their fundamental language, that is, mathematics. The nature of purely mathematical objects and structures as well of how they can be known has been subject of lively debates among philosophers and mathematicians since the times of antiquity and it continues to inspire both ontological and epistemological reflection until the present day
%\label{ref:RNDfKXsM6WzEV}(e.g. Shapiro, 2000).
\parencite[e.g.][]{shapiro_thinking_2000}. %
 In order to provide philosophical justification of how the mathematical object exist and what the reasons for the great effectiveness of mathematics in the physical description of nature are, a~prominent Polish philosopher of science, Józef Życiński, coined out in 1987 the concept of the \textit{field of rationality} and its close correlate the field of potentiality 
%\label{ref:RNDq3D8oYGPDC}(Życiński, 1987; 1988).
\parencites[][]{zycinski_filozoficzne_1987}[][]{zycinski_teizm_1988}. %
 His most direct motivation to introduce the field of rationality as constitutive to the ontology of the Universe flows from a~careful analysis of the practice of science which shows that the generalization of the theoretical description of the physical reality effects a~radical shift from concrete things to abstract mathematical structures.

A~concept similar to that of the field of rationality has emerged from a~different approach assumed by another prominent Polish philosopher of science, cosmologist theologian and a~co-worker of Życiński, Michał Heller. Contrary to Życiński, however, Heller ventures out in his inquiry not from the nature of the physical world but from the nature of mathematics itself to arrive at the concept of \textit{the formal field}. Consequently, he considers the field of rationality to be an ontological interpretation of the formal field or, more properly, of its subfield
%\label{ref:RNDCx0KOzCoph}(Heller, 2014, p.442).
\parencite[][p.442]{heller_field_2014}. %
 Heller insists that the idea of the field of rationality is still very ,,fuzzy'' and is in need of further elaboration. In order to provide the necessary insight, he turns to the category theory as one of the most abstract and general theories of the contemporary mathematics often considered as a~firmer foundation of mathematics in comparison to the set theory. Heller's key claim in the effort of ``unfuzzying'' the concept of the field of rationality is to equate it with the \textit{field of categories} 
%\label{ref:RNDomiiKtJCrl}(Heller, 2014, p.453).
\parencite[][p.453]{heller_field_2014}.%


Despite of Heller's quite sophisticated ways of unpacking and sharpening of the conceptual content of the field of rationality much remains to be done especially in view of the relations between the meaning of the field of rationality with the formal field. The main scope and motivation of this article is to bring in the desired clarity into the meaning of the field of rationality as well as all its derivatives mentioned above. The inquiry will proceed in four stages. Firstly, a~detailed critical review of the justification of the concept of the field of rationality as proposed by Życiński will be carried out. Secondly, special attention will be devoted to some inconsistencies in how he handles the platonic doctrine of the ideal forms. In that prospective, the subsequent introduction of the concept of the \textit{noumenal structure} will be addressed. Thirdly, the alternative approach launched by Heller leading to the introduction of the field of categories will be surveyed. Fourthly, it will be concluded that although much better clarification of the concept of the field of rationality and its derivatives and their mutual relations has been achieved in the course of this study, further investigative efforts are still needed so that these concepts may achieve satisfactory account and clarity in their meaning.

\section{The Birth of the Field}
The development of the concept of the field of rationality as well as its derivatives is a~rather lengthy process which was initiated by Życiński in 1985 and developed by him through a~series of subsequent articles and books
%\label{ref:RNDc9iuC32sSY}(Życiński, 1985; 1987; 1988; 1991; 1995; 2006; 2013b).
\parencites[][]{zycinski_teizm_1985}[][]{zycinski_filozoficzne_1987}[][]{zycinski_teizm_1988}[][]{zycinski_poza_1991}[][]{zycinski_status_1995}[][]{zycinski_pole_2006}[][]{zycinski_swiat_2013}. %
 Although Życiński's \textit{post mortem} published platonic manifesto entitled \textit{Świat matematyki i~jej matematycznych cieni} 
%\label{ref:RNDWnCbfCOWsu}(Życiński, 2011; 2013b)
\parencites[][]{zycinski_swiat_2011}[][]{zycinski_swiat_2013} %
 evidently aspires to give full expression to the concept of the field of rationality, many arguments presented in this manifesto are at most extended elaborations on what had already appeared in the preceding articles. He declares expressly that the ontological assumptions on which the concept of the formal field rests are based on the philosophy of Plato and Alfred North Whitehead 
%\label{ref:RNDSwIsN24bsf}(Życiński, 1987, p.171).
\parencite[][p.171]{zycinski_filozoficzne_1987}. %
 These philosophers exert their marked influence on other areas of Życiński's philosophical reflection as well. Moreover, Życiński fell under considerable influence of a~renowned British theoretical physicist, mathematician, philosopher and the 2020 laureate of the Nobel Prize in physics, Roger Penrose and especially by his famous philosophical manifesto entitled \textit{The Emperor's New Mind} 
%\label{ref:RNDRDWRYVCvlw}(1989).
\parencite*[][]{penrose_emperors_1989}. %
 Although heavily criticized by experts from a~wide range of disciplines\footnote{For a~review of this critics see 
%\label{ref:RNDxVRvxdUQhz}(Grygiel and Hohol, 2009).
\parencite[][]{grygiel_rogera_2009}.%
}, Penrose made a~substantial effort to justify his platonic view of reality in which mathematical objects and structures exist in the world of ideal platonic forms.

The term ``field of rationality'' was formally introduced by Życiński with all due detail in his 1987 work. However, an important conceptual prelude appeared in his incisive treatment of the method of theology launched yet in 1985 in the first volume of the work entitled \textit{Teizm i~filozofia analityczna}
%\label{ref:RNDnxKq1SZJBl}(Życiński, 1985, pp.187–207).
\parencite[][pp.187–207]{zycinski_teizm_1985}. %
 In this prelude, Życiński takes up the issue of the \textit{ontic} and \textit{epistemic rationality} of nature with special emphasis on the mathematicity of nature as a~constitutive element of the ontic rationality. Consequently, Życiński's main reason for introducing the concept of the field of rationality is to provide the precise philosophical explanation of why the language of mathematics developed independently of the investigations of nature applies so accurately to its physical description 
%\label{ref:RNDXgSHAQnFUy}(Życiński, 1987).
\parencite[][]{zycinski_filozoficzne_1987}. %
 He argues that this applicability compels to assume that the fundamental mathematical structures are ontologically prior to their observable physical instantiations as well as to the development of mathematics itself revealing the existence of a~great number of structures that do not find their use in the study of nature. In short, ``the nature is mathematical because the level of the field of rationality is the fundamental level in its ontic structure'' 
%\label{ref:RNDwNyfKHAHnY}(Życiński, 1987, p.176)
\parencite[][p.176]{zycinski_filozoficzne_1987} %
 [author's translation]. Also, Życiński highlights the fact that that while the human psychological disposition favors the treatment of concrete things as the fundamental constituents of reality, the development of science forces radical departure from common sense perceptions of this kind towards abstract mathematical structures. These perceptions are conditioned by the phylogenetic conditions of the evolutionary development of the human cognitive apparatus proper to the level of reality at which this development had occurred.

While one can easily agree with Życiński that accepting the field of rationality as the fundamental level of reality implies ``certain version of platonism''
%\label{ref:RNDucQkVx6Xin}(Życiński, 1987, p.174),
\parencite[][p.174]{zycinski_filozoficzne_1987}, %
 considerable amount of further analysis will be necessary in order to establish to what degree this stance corresponds with the original ontological views of Plato. The central claim that Życiński builds into the concept of the field of rationality is that the abstract mathematical structures inherent in this field contain in themselves potentially all possible concrete objects and structures which can be actualized into existence at a~later stage of the development of the Universe. He also asserts that the field of rationality imposes certain restrictions on the ontology of the Universe which is particularly evident in very specific kinds of symmetries that strictly define the form of the fundamental equations describing the dynamics of particles and interactions 
%\label{ref:RNDy7k88aiahE}(Życiński, 1987, p.180).
\parencite[][p.180]{zycinski_filozoficzne_1987}.%


As the primary example illustrating this claim Życiński refers to the theoretical description of the physical fields and in particular to the concept of the quantum vacuum which is a~physically existing field with the lowest energy with no particles in it. The creation of a~particle occurs when a~creation operator acts on the wave function proper to the wave function of the vacuum thereby revealing obvious dependence of the reality of concrete physical particles on the mathematically abstract structure of the vacuum. Thus the designation of the field of rationality as the field of potentiality receives its proper explanation. This dependence in turn is intelligible only when the field of rationality represented by the vacuum is ontologically prior to concrete particles which are actualizations of the potentialites inherent in this vacuum through excitation to states of higher energy
%\label{ref:RNDgUj2QWCzOJ}(Życiński, 1987, pp.176–178).
\parencite[][pp.176–178]{zycinski_filozoficzne_1987}. %
 In this context Życiński provides justification as to why this ontologically primitive realm of abstract mathematical structures should be called a~field and not a~matrix or a~network. The physical field serves as a~suitable metaphor to illustrate the dynamic character of a~field out of which new entities can emerge as opposed to the static and invariant status of matrices and networks.

Some general references to the philosophy of Plato or more generally understood platonism are made in Życiński's 1987 article. Nevertheless in the same essay Życiński devotes more attention to show the importance of the legacy of Alfred North Whitehead
%\label{ref:RNDh3hBlm5Xcm}(Życiński, 1987, pp.181–185).
\parencite[][pp.181–185]{zycinski_filozoficzne_1987}. %
 He states that although Whitehead does not explicitly mention this concept in his writings, his main tenets in the interpretation of the mathematical character of nature coincide what is implied by the field of rationality. In particular, Życiński highlights that this interpretation bears markedly platonic character and what Whitehead has in mind is a~certain matrix of abstract structures which constitute the most fundamental ontology of the Universe and from which all concrete physical objects may emerge. Lastly, Życiński points to Whitehead's theological interpretation of this matrix by associating it with a~certain mode of the Divine presence in the Universe. This issue will receive its further clarification in Życiński's subsequent publications on the field of rationality.

It turns out that even on the contemporary scene Życiński is not at all alone in his insistence on the fundamental ontology of abstract mathematical structures. For instance, Penrose from whom he draws much inspiration in regards to the concept of the field of rationality promotes a~unique belief that the complex numbers and abstract mathematical structures based on them bearing the name of holomorphic structures underlie the physical reality at its most fundamental level. Penrose goes as far as to assert that ``we shall see something of the remarkable way in which complex numbers and holomorphic functions can exert their magic from behind the scenes''
%\label{ref:RNDvqre2EArPM}(Penrose, 2005, p.151).
\parencite[][p.151]{penrose_road_2005}. %
 This magic is well evidenced in the example of quantum mechanics which Penrose brings forth on numerous occasions. He takes every effort to emphasize that it is the complex character of the weightings in the linear combination of states composing an entangled state which is directly responsible for the quantum interference manifesting itself in the double slit experiment 
%\label{ref:RNDMUCBkTN1xb}(Penrose, 1989, pp.236–242; 1997, pp.50–92; 2005, pp.553–559).
\parencites[][pp.236–242]{penrose_emperors_1989}[][pp.50–92]{penrose_large_1997}[][pp.553–559]{penrose_road_2005}.%


\section{How Platonic is the Field?}
A~closer survey of Życiński's articles on the field of rationality following the original one discussed above
%\label{ref:RNDC0keEXgsUj}(Życiński, 1987)
\parencite[][]{zycinski_filozoficzne_1987} %
 reveals that in this article Życiński presented the majority of his key arguments to justify the meaningfulness of this concept. The articles following that of 1987 are mainly devoted to the search the philosophical interpretation of the field of rationality in agreement with the declaration that a~``certain version of platonism'' must be sought for this purpose. It must be remembered, however, that in reference to the ontological status of the abstract mathematical formalisms of the contemporary physical theories the term platonism is used in a~much broader sense as originally intended by Plato. The main difficulty consists in that the abstractness of the mathematical structures known today greatly exceeds the ancient mathematics of numbers and simple geometrical figures. As a~result, the references to the original platonic thought are never straightforward and demand particular care in relating the meanings of concepts developed in quite different intellectual environments.

In article
%\label{ref:RNDwWUY4pmXyB}(Życiński, 1991)
\parencite[][]{zycinski_poza_1991} %
 on the concept of field of rationality Życiński quite rightly locates its justification within the classical philosophical problem of the existence of abstract entities, that is, the problem of universals in which as a~radical form of realism platonism occupies an important position. In the following text on the field of rationality 
%\label{ref:RNDUhkvDHsF1v}(Życiński, 1995),
\parencite[][]{zycinski_status_1995}, %
 he evidently seeks further support for this concept by referring to the famous debate between two influential Polish philosophers Tadeusz Kotarbiński and Roman Ingarden on the fundamental ontology whether these are things (reism) or abstract objects or structures, respectively 
%\label{ref:RNDirp9yNxiV3}(Kotarbiński, 1920; Ingarden, 1972, pp.483–507).
\parencites[][]{kotarbinski_sprawa_1920}[][pp.483–507]{ingarden_z_1972}. %
 Unfortunately, Życiński addresses this debate in general terms only so that it is difficult to see how the argumentation for the field of rationality developed so far and merely restated here ties with the complexity of the debate in question.

As a~direct reference to the works of Plato Życiński picks the interpretation of Plato's statement from \textit{Phaedrus} 247 C~offered by G.M.A. Grube in light of which Plato should be read as implying that these are ideas that dwell above the heavens. Grube's claim
%\label{ref:RNDPI0JC2C4ns}(1958, pp.30–35)
\parencite*[][pp.30–35]{grube_platos_1958} %
 that the statement must be taken metaphorically serves as a~basis for Życiński to infer that the mathematical objects and structures contained in the field of rationality bear transcendent character in relation the realm of concrete physical objects because they exist beyond the spatio-temporal regime proper to what is termed as physical. Keeping in mind that the openness of metaphors allows for a~variety of interpretations, the interpretative path of the platonic thought assumed by Życiński shows inconsistencies with the basic tenets of Plato's doctrine on the ideal forms.

First of all, Życiński incorrectly places mathematical forms directly in the world of the platonic ideas. According to Plato, mathematics lies below the ideas in the hierarchy of being and, for instance, the idea of a~number can never enter into any computation
%\label{ref:RND3eecrVyWmV}(Shapiro, 2000, pp.52–60).
\parencite[][pp.52–60]{shapiro_thinking_2000}. %
 Next, 
%\label{ref:RNDlamrySj5AE}(Życiński, 1991)
\parencite[][]{zycinski_poza_1991} %
 makes a~lot of effort to demonstrate the existence of radical gap between the abstract realm of the field of rationality and the concrete physical objects. In addition to the aforementioned example of the theory of physical fields, Życiński proposes three new illustrations of this gap which include the Kepler laws, the DNA code and the algorithms. In particular, the example of algorithms attests to the inspirations that Życiński has drawn from the works of Penrose. For instance, in case of the Kepler's laws he suggests that although not physically instantiated before the birth of stars and galaxies, they ``somehow existed in the structures of the early Universe'' 
%\label{ref:RNDoU0IUCPDrv}(Życiński, 1991, p.71).
\parencite[][p.71]{zycinski_poza_1991}. %
 This means that despite of the radical separation on which Życiński insists, the abstract structures of the field of rationality may enter into causal relationships with other structures of the field to produce concrete physical objects. In other words, the fact that the abstract structures do not occupy the spatiotemporal realm does not imply that they must be relegated to the acausal and eternal realm of the platonic forms. However, the force with which Życiński suggests this separation does make an impression that such a~relegation is indeed intended. This would be still consistent with the interpretational freedom of the proposed metaphorical reading of Plato and would bring Życiński closer to the thought of Penrose who asserts that ,,I might baulk at actually attempting to identify physical reality within the abstract reality of the Plato's world'' 
%\label{ref:RNDvdKNxQjTHJ}(Penrose, 2005, p.1029).
\parencite[][p.1029]{penrose_road_2005}. %
 Interestingly enough, Penrose reveals a~slightly different understanding of physicality as compared to Życiński because in his ontology of the three worlds he evidently designates as physical the entire realm of what is constitutive to the structure of the Universe being a~subset of the platonic world of mathematical forms. Życiński's use of the term ``physical'' only in reference to concrete spatiotemporal instantiations makes his reading of Plato all the more difficult.

Another serious concern that arises on the grounds of Życiński's justification of the concept of the field of rationality is the exact meaning of the term ``concrete object'' or ``concrete thing''. Aside from standing in marked opposition to the abstract objects or structures and most likely occupying the spatiotemporal realm not much more can be asserted. For instance, when he states that the motion of stars follows the patters determined by Kepler laws and he considers the stars to be concrete and the laws abstract, what are the stars made out of? There is no doubt that this is an extremely intuitive and imprecise account which does not seem to differ much from the pre-scientific understanding of matter as a~chunk of stuff contained in a~given volume. By doing this, Życiński evidently falls into contradiction with himself as on one hand he heavily criticizes the use of the intuitive common sense concepts in science and, on the other, he makes them a~foundational concept in the process of defining the field of rationality. If the concrete structures and objects are actualized into existence from the field of rationality, from the point of view of the contemporary physical theory of matter it must be regarded as a~highly complex combination of fields and their excitations representing elementary particles that the molecules of the stuff are made out of. This is often referred to as the problem of the emergence of the classical world from quantum domain. For instance, it receives a~detailed treatment in the works of of Penrose who considers the reduction of the wave vector as a~real physical process induced by gravitational interactions
%\label{ref:RNDHTma0hFX1S}(e.g. Penrose, 2005, pp.816–868).
\parencite[e.g.][pp.816–868]{penrose_road_2005}. %
 Unfortunately, Życiński does not mention these issues in any of his works.

The interpretative difficulties associated with Życiński's attempt to locate the field of rationality in the original thought of Plato find at least partial solution in two his final works on the subject published in 2005 and \textit{post mortem} in 2013. As Heller rightly points out, Życiński has eventually abandoned platonic metaphysical view of the field of rationality and switched to its ontological interpretation by calling the field of rationality the \textit{noumenal structure} of the Universe
%\label{ref:RNDkf0EpbUso3}(Heller, 2014, p.442).
\parencite[][p.442]{heller_field_2014}. %
 Although Życiński still mentions Grube's metaphorical reading of Plato, he admits of the multiplicity of possible interpretations of Plato's thought and shifts his emphasis to different texts of Plato, namely \textit{Parmenides 132 D} and \textit{Philebus}, where the participation of the concrete physical objects in the abstract structures is clearly admitted 
%\label{ref:RNDJIpMWLEMEy}(Życiński, 2006, p.58).
\parencite[][p.58]{zycinski_pole_2006}. %
 Evidently Życiński begins to withdraw from his former stance of the radical separation between the two in favor of treating the field of rationality as the constitutive element of the fundamental ontology of the Universe. Hence comes the term ``noumenal structure'' which reflects close relationship of the field of rationality with the laws that govern the Universe 
%\label{ref:RNDoxsjndZ46v}(Życiński, 2006, pp.58–59).
\parencite[][pp.58–59]{zycinski_pole_2006}. %
 Życiński reaffirms his ontological approach to the field of rationality in his 2013 publication by asserting that the Kepler's laws had existed in the structures of the early Universe before they emerged with the formation of stars and galaxies 
%\label{ref:RNDns57m3kaEz}(Życiński, 2013b, p.161).
\parencite[][p.161]{zycinski_swiat_2013}. %
 This remains in good agreement with the theological interpretation of the field of rationality in which he refers to this field as the immanence of Logos in the Universe 
%\label{ref:RND2LKDvrBquD}(Życiński, 2006; 2013b, pp.170–172)
\parencites[][]{zycinski_pole_2006}[][pp.170–172]{zycinski_swiat_2013} %
 and that the actualization of potentialities inherent in this field does not require a~separate act of the Divine creation. It is worth mentioning that Życiński has also argued for the key role of the field of rationality in the origin and development of life in the Universe 
%\label{ref:RNDACYzTo7HLM}(Życiński, 2009).
\parencite[][]{zycinski_wszechswiat_2009}.%


In his introduction to Życiński's \textit{The World of Mathematics and Its Material Shades} Heller suggests a~novel interpretational approach to Plato's perspective on mathematics which in his opinion provides suitable philosophical setting for Życiński's ontological views expressed by the concept of the field of rationality
%\label{ref:RNDZ0ZaZfG0OJ}(Życiński, 2013b, pp.5–15).
\parencite[][pp.5–15]{zycinski_swiat_2013}. %
 This approach is marshaled by a~Polish philosopher Bogdan Dembiński who maintains that in its late period the platonic School was influenced by the Pythagoreans and led to the reshaping of the platonic understanding of the nature of mathematics 
%\label{ref:RND5HnopKUBJj}(Dembiński, 2003; 2010; 2015; 2017; 2019).
\parencites[][]{dembinski_pozna_2003}[][]{dembinski_pozny_2010}[][]{dembinski_o_2015}[][]{dembinski_u_2017}[][]{dembinski_theory_2019}. %
 This change was prompted mainly by the disciples of Plato: Speusippus (410--339 BC), Xenocrates (396--314~BC) and Eudoxos (408--355 BC) who turned mathematics into the main topic of discussions in the academy. Ultimately, these discussions resulted in the the belief that it is mathematics that constitutes the fundamental stuff of the Universe. While Speusippus assigned all the characteristics of the ideal numbers to the mathematical numbers, that is, the separate existence, eternity, unchangeability and objectivity 
%\label{ref:RNDam7tYV6bac}(Dembiński, 2010, pp.109–138),
\parencite[][pp.109–138]{dembinski_pozny_2010}, %
 Xenocrates turned mathematics into ontology 
%\label{ref:RNDYevdmedyyC}(Dembiński, 2010, pp.139–170).
\parencite[][pp.139–170]{dembinski_pozny_2010}. %
 According to Dembiński 
%\label{ref:RNDbYt0Rk1hAh}(2010, p.158),
\parencite*[][p.158]{dembinski_pozny_2010}, %
 Xenocrates is rightly named the forerunner of the concept of the mathematicity of the Universe. Undoubtedly, his ontological stance provides the most consistent philosophical environment for the proper foundation of Życiński's mature understanding of the concept of the field of rationality.

\section{Starting with mathematics}
It turns out that the path to the concept of the field of rationality does not have to commence with the physical reality as is the case of Życiński. As a~skilled mathematician, Heller approaches the field of rationality in abstraction to any physical instantiations by referring directly to the nature of mathematics itself
%\label{ref:RNDyKKVzPjLF3}(Heller, 1997, pp.216–238).
\parencite[][pp.216–238]{heller_uchwycic_1997}. %
 Heller's focal point in this regard is the famous Gödel theorem which stipulates that if any axiomatic system rich enough to contain arithmetic is complete then it must be contradictory 
%\label{ref:RNDoFTrnxRLw5}(e.g. Penrose, 1994, pp.66–116).
\parencite[e.g.][pp.66–116]{penrose_shadows_1994}. %
 The direct consequence of this theorem is that if one selects a~non-contradictory axiomatic system then in there will be theorems whose truth will not be provable within this system. To put things in short, provability cannot be equated with truth and mathematics cannot be reduced to an axiomatic system. Much rather one should think of axiomatisation as mere human means of capturing the complexity of all possible mathematical structures.

This is precisely the point where Heller clarifies his usage of the term ``field''. In doing so, he wishes to purposely avoid referring to all these structures as a~set of objects in the strict set-theoretic sense. What he rather has in mind is an ensemble of structures linked together with all possible paths of inference
%\label{ref:RNDdtR212ceeC}(Heller, 1997, pp.236–238).
\parencite[][pp.236–238]{heller_uchwycic_1997}. %
 Why then a~field? First of all, the concept of a~field takes into account all relations between the strictures and secondly, as Heller insists, this concept conveys the idea of potentiality, that is, the field contains not only already known mathematical structures both those who still await their discovery in the future. Since these are not only the structures but all possible relations of inference among them Heller proposes to qualify this field as the \textit{formal field}. This field provides a~good framework for the interpretation of the Gödel theorem because to construct an axiomatic system means to select a~certain small area in the formal field which is always too small to deductively grasp all theorems lying within this system.

Interestingly enough, while Heller does not invoke his famous distinction between two kinds of mathematics in this context: mathematics with the ``small m'' that evidently stands for axiomatised mathematics as a~human creative activity written down in academic books and mathematics with the ``capital M'' as the universe of objectively existing mathematical structures which the former purports to describe
%\label{ref:RNDX8siVWIOQG}(e.g. Heller, 2010).
\parencite[e.g.][]{heller_co_2010}. %
 Moreover, it is worth stressing that Heller leaves the meaning of the potentiality of a~mathematical structure somewhat vague as he merely asserts that ``these are the structures that have not been discovered yet or structures that will never be discovered but are in some sense possible'' 
%\label{ref:RNDDgNHPh7FgA}(Heller, 1997, pp.236–237).
\parencite[][pp.236–237]{heller_uchwycic_1997}. %
 Since the existence of any mathematical structure is guaranteed only by its inner non-contradiction, there are no reasons to think of a~possible mathematical structure as non-actualized from the point of view of mathematics alone. Much rather the problem lies in their not being known yet.

Ultimately, however, Heller shifts his attention to the ontological application of the formal field to explain the effectiveness of mathematics in modeling of the physical world
%\label{ref:RNDEhpoGyQJqz}(Heller, 1997, p.237).
\parencite[][p.237]{heller_uchwycic_1997}. %
 As he rightly notices, this demands the existence of an intimate connection between the formal field and the fabric of the Universe. On such an interpretation, however, Heller suggests to replace the term ``formal field'' with ``field of rationality''. And this is precisely the point where Heller's thought meets with the thought of Życiński. The concept of the field of rationality carries with it much more philosophical implications for it directly refers to the idea of the rationality of the Universe which occupies the central position in the works of both of these thinkers 
%\label{ref:RNDpzCu5UPuKs}(e.g. Heller, 2006; Życiński, 2013a).
\parencites[e.g.][]{heller_czy_2006}[][]{zycinski_granice_2013}.%


The concept of the field of rationality finds its further development in the thought of Heller as he turns his scientific interest to a~highly abstract mathematical theory known as the \textit{category theory}
%\label{ref:RNDoAmpaX4Idy}(Heller, 2014).
\parencite[][]{heller_field_2014}. %
 He is aware that a~mathematical theory that aspires to better capture the concept of the field of rationality must bear some marks of fundamentality for mathematics as a~whole. In this regard he relies on the opinion of one of the authors of the category theory, Saunders Mac Lane who claims that this theory has foundational significance 
%\label{ref:RND8WUibvNR9V}(Mac Lane, 1992).
\parencite[][]{mac_lane_protean_1992}. %
 This goal of Heller finds its somewhat humorous expression as he sees the need to make the concept of the field of rationality ``less fuzzy''. Quite surprisingly, as the interpretational perspective for this task Heller selects the ontology in the sense Willard V.O. Quine ``which does not aspire to establish what exists, but rather what a~given theory or doctrine assumes there exists''. The import of this choice will become more evident in the critical conclusion of this study 
%\label{ref:RND8eRhkJmDcu}(Heller, 2014, p.442).
\parencite[][p.442]{heller_field_2014}.%


Since a~rigorous presentation of the category theory cannot be offered in this article (for a~suitable source see for example
%\label{ref:RNDiuUalUzJnK}(Simmons, 2011)
\parencite[][]{simmons_introduction_2011}%
), several general points will suffice to illuminate the desired philosophical import. The category theory is not just another branch of mathematics like calculus, linear algebra or set theory, for instance. Instead, the theory sees these and other branches as separate categories whereby it provides an overview ``from above'' and reveals possible connections among them. A~category is a~collection of objects connected by means of arrows which are called \textit{morphism.} The most important aspect of the application of the category theory to the study of the structure of the Universe is that each major physical theory such as quantum mechanics or general relativity supervenes on a~certain area of mathematical discourse and, consequently, on a~certain category. In other words, a~separate category may be selected to represent a~section of the field of rationality that constitutes a~matrix for the functioning of a~given region of the physical reality. Moreover, since different logics govern different categories, different logics may apply to different physical theories, as it is seen in the case of quantum mechanics. Also, Heller gives several other arguments of more formal nature in support of matching the field of rationality with \textit{the field of categories}. Since their complexity would require lengthy explanations reaching beyond the scope of this paper, only the hierarchical structure of the category theory is worth mentioning here. In other words, mathematics cannot be captured into one axiomatic system but it should be looked upon as an ensemble of structures and structures of these structures, part of which may constitute the matrix of the functioning of the Universe.

\section{Knowing the Field: A~Critical Conclusion}
As the inquiry into the philosophical motivations and development of the concept of the field of rationality nears its conclusion, it is fitting to voice some closing critical remarks as well as to indicate possible areas of further research. The suitable point of departure is the seemingly uncontroversial matching of the formal field with the field of rationality suggested both by Życiński and by Heller. With this match in force, two fundamental difficulties arise: (1) how to explain the nature of mathematics as a~necessary and \textit{a~priori} knowledge which can be acquired entirely independently from the empirical method and (2) how to explicate the ``excessive'' status of mathematics evident in the predictive power of the laws of physics such as the Einstein field equation to contain information on phenomena not imagined by their authors. In short, this issue focus around the central problem in philosophy of mathematics—how the mathematical structures exist and how they can be known. Both Życiński and Heller directly express their awareness of the problem of how the human mind gets to know mathematics and the solutions they propose center on the evolutionary scenarios responsible for the acquisition of the mathematical knowledge
%\label{ref:RND673BjuCiKV}(Heller, 2010; Życiński, 2010).
\parencites[][]{heller_co_2010}[][]{heller_jak_2010}. %
 Of course, this is fully consistent with the ontological views of Xenocrates that they both adhere to in which mathematics is matched with ontology and no ontologically distinct platonic world is explicitly proposed. On such reading, however, the two difficulties mentioned above remain without explanation.

Interestingly enough, Życiński mentions the alternative solution in which the acceptance of the ontologically distinct world of mathematics is proposed and which is boldly marshaled by Penrose. In other words, the formal field differs from the field of rationality and the field of rationality constitutes but a~subsection of the formal field which governs the behavior of the Universe. While this swiftly explains the nature of mathematics, it generates two new problems: (1) how the human mind gains the quasi-mystic access to the platonic world of mathematical objects and structures and, most importantly, (2) it turns the mathematicity of the Universe into a~mystery because it is entirely unclear how the contingent physical world should emerge out of the acausal and eternal realm of mathematical forms. Moreover, Życiński not only devotes a~considerable portion of \textit{The World of Mathematics and Its Material Shades} to the survey of the typical arguments brought forth in support of the mathematical platonism but declares his open opposition to the purely naturalistic metaphorical approach to mathematics promoted by Lakoff and Nuñez
%\label{ref:RND8JDrxJx210}(Lakoff and Núñez, 2000; Życiński, 2013b, pp.48–49).
\parencites[][]{lakoff_where_2000}[][pp.48–49]{zycinski_swiat_2013}. %
 Paradoxically, this approach would better correspond with the ontology of the formal field which he confesses.

At this point one can get an impression that a~lot of effort have been devoted into making the concept of the field of rationality ``less fuzzy'', much fuzziness still remains. In a~way this fuzziness is enhanced by Heller himself who by considering the field of categories from the point of view of the ontology of Quine eschews the fundamental question of what objects and structures are actualized into existence and what mechanisms are responsible for this process by reducing this question to what is warranted by the discourse of the category theory only. Although the field of categories may indeed capture some of the hierarchical relations implied by the category theory, it works in this capacity at best as a~model and does not exhaust the full complexity of how a~given section of the field of rationality finds its application to a~given area of physical reality. However, the dependence of the distinction between the formal field and the field of rationality on the assumed philosophical position, namely that of Plato or Xenocrates, as well as the difficulties implied by both of them reveal a~deeper source of ``fuzziness'' of these concepts which at this point must remain a~mystery. After all, it is Penrose himself who saw mystery behind his ontology of the three worlds and claimed that now it is the time to study the nature of the human mind to shed some new light on what so far seems to resist our intellectual insight
%\label{ref:RND4ZG2wNtc03}(Brożek and Hohol, 2014).
\parencite[][]{brozek_umysl_2014}.%


\end{artengenv}
