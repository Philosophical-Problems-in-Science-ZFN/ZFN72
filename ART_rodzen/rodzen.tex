\begin{artengenv}{Jacek Rodzeń}
	{Stanisław Dunin-Borkowski and his views on Einstein's special theory of relativity}
	{Stanisław Dunin-Borkowski and his views\ldots}
	{Stanisław Dunin-Borkowski and his views on Einstein's special theory of relativity}
	{Jan Kochanowski University in Kielce}
	{The main purpose of this article is to discuss the views of the Jesuit Stanisław Dunin–Borkowski (1864–1934) about Albert Einstein's theory of relativity. These days, Dunin–Borkowski is a~rather obscure figure despite rising to fame in the interwar period as an outstanding expert in the philosophy of Baruch Spinoza. Thus, the secondary aim of this article is to remind ourselves of this somewhat forgotten scholar. As a~researcher, writer, and pedagogue, Dunin–Borkowski was interested in numerous fields of knowledge. Among these were the natural sciences, including physics and the influence that new physical theories had on philosophical thought. This present study therefore fills a~gap in the existing research about how Polish philosophers received Einstein's theories. The example of Dunin–Borkowski also serves as a~basis for discussing some of the fundamental problems of neo-scholasticism in receiving new mathematicised scientific theories.}
	{Stanisław Dunin-Borkowski, Spinoza, special theory of relativity, neo-scholastic philosophy, philosophy of nature, science and theology.}





\section{Introduction}
\lettrine[loversize=0.13,lines=2,lraise=-0.01,nindent=0em,findent=0.2pt]%
{I}{}n the history of contemporary philosophy, Stanisław Dunin–Borkowski (1864–1934) is rarely mentioned, and there is scant recognition of his writings. For example, two monumental works on the history of philosophy, one by F. Copleston
%\label{ref:RNDCO1JSVWxUm}(1994, p.209)
%
 and another by W. Tatarkiewicz 
%\label{ref:RND3ZnixW4UkW}(Tatarkiewicz, 1998, p.365),
%\parencite[][p.365]{tatarkiewicz_historia_1998}, %
 merely mention him
 \parencites[][p.209]{copleston_history_1994}[][p.365]{tatarkiewicz_historia_1998}. %
 Moreover, the recently published voluminous \textit{History of Polish Philosophy}, as edited by J. Skoczyński and J. Woleński 
%\label{ref:RNDoG8CXONF8Z}(2010),
\parencite*[][]{skoczynski_historia_2010}, %
 contains not even a~single word about Dunin–Borkowski. Only among the works of Jesuit biographers can one encounter descriptions of his life and academic achievements 
%\label{ref:RNDYRfDRIO8k0}(cf. Siwek, 1935; Pummerer, 1935; Darowski, 2001, pp.103–104).
\parencites[cf.][]{siwek_stanislaw_1935}[][]{pummerer_p_1935}[][pp.103–104]{darowski_filozofia_2001}.%
\footnote{Dunin-Borkowski and the first part of his comprehensive work on Spinoza were mentioned at the 1911 meeting of the Philosophical Society in Cracow by Rev. Stefan Pawlicki 
%\label{ref:RND9rZsPZHKih}(1912, pp.13–14).
\parencite*[][pp.13–14]{pawlicki_spinoza_1912}.%
}

Stanisław Dunin–Borkowski deserves a~secure place in the history of philosophy for at least one crucial reason, namely his seminal works on Baruch (Benedictus) Spinoza (1632–1677). Indeed, Dunin–Borkowski devoted four hefty volumes totaling around 2,000 pages to the life and works of this thinker from Amsterdam. In 1937, in his review of them, Richard P. McKeon (1900–1985), a~preeminent American philosopher and historian who is regarded as a~representative of \textit{The Chicago School} of literary criticism, wrote the following in \textit{The Journal of Philosophy}:

\myquote{
Data and explications important to the study of the seventeenth century and its intellectual background, bibliographies, recondite information concerning men, movements, and places, brief histories of concept, problems, and methods, frequently carried far beyond the immediate scope of Spinoza's usage or his knowledge, make Dunin Borkowski's work indispensable in the study of Spinoza or of the seventeenth century
%\label{ref:RND88cGJKV9Fu}(McKeon, 1937, p.381).
\parencite[][p.381]{mckeon_review_1937}.%
}
Dunin–Borkowski presented the results of his studies of Spinoza at prestigious international conferences. After one such event where he had delivered a~paper on Spinoza's physics
%\label{ref:RNDlXOvySca8O}(Dunin-Borkowski, 1933),
\parencite[][]{dunin-borkowski_aus_1933}, %
 León Brunschvicgs (1869–1944) referred to him as ``le grand historien de Spinoza'' 
%\label{ref:RND34owQT9HDx}(Brunschvicg, 1934, p.427).
\parencite[][p.427]{brunschvicg_septimana_1934}.%


However, Spinoza was not Dunin–Borkowski's only interest. He left a~sizeable legacy in the form of scientific and popular science literature on pedagogy, the history of religion (e.g., Buddhism), theology, and philosophy, but he was also familiar with issues connected with the contemporary natural sciences, including physics, and their relation to philosophy and theology. Several of his semi-popular research works bear significant traces of this interest
%\label{ref:RNDZamlsejc6A}(cf. Dunin-Borkowski, 1898; 1911).
\parencites[cf.][]{dunin-borkowski_popularer_1898}[][]{dunin-borkowski_wissenschaft_1911}. %
 Although Dunin–Borkowski was associated with Thomism 
%\label{ref:RNDK2y5cFUhpX}(Dunin-Borkowski, 1921a; 1921b; also Siwek, 1935, p.141; 1938, pp.306–307), %tu
\parencites[][]{dunin-borkowski_auf_1921}[][]{dunin-borkowski_neue_1921}[also][p.141]{siwek_stanislaw_1935}[][pp.306–307]{siwek_spinoza_1938}, %
 aside from some several-page articles, he did not leave any thorough work that represented this thought movement. His association with Thomism, or more generally with neo-scholasticism, should come as no surprise, however, given that he belonged to the Society of Jesus.

This present article is devoted to just one aspect of Dunin–Borkowski's work, however, namely this Polish thinker's appearance as a~commentator on the emergence of Albert Einstein's theory of relativity.\footnote{Dunin–Borkowski's comments concern only the special theory of relativity.} It is an episode that can be followed through two relatively short research works by Dunin–Borkowski
%\label{ref:RNDGG16vCZQuB}(1921a; 1921b),
\parencites*[][]{dunin-borkowski_auf_1921}[][]{dunin-borkowski_neue_1921}, %
 yet it appears these works have some relevance for broadening our perspective of how this physical theory was received in neo-scholastic circles in the early decades of the 20\textsuperscript{th} century.

There is a~moderate amount of literature on the early reception of relativity by neo-scholastics. In an article, S.L. ten Hagen
%\label{ref:RNDetiu7vO4I6}(2020, pp.238–239)
\parencite*[][pp.238–239]{hagen_local_2020} %
 briefly discussed the views of some Belgian adherents (mainly in Louvain) along this line of thought, including, among others, the Jesuit H. Dopp, P. Drumaux and D. Nys. In a~few paragraphs of his work, A.C. Flipse 
%\label{ref:RNDBEqkgKs137}(2010, pp.1148–1149)
\parencite*[][pp.1148–1149]{flipse_between_2010} %
 introduced the attitudes of P. Hoenen, a~Dutch Jesuit, toward relativity. T.~Glick 
%\label{ref:RNDe0oVkENqci}(1987, pp.240–242),
\parencite*[][pp.240–242]{glick_relativity_1987}, %
 in turn, discussed the views of neo-scholastics in Spain. Works have also been devoted to J. Maritain's views on Einstein's theory 
%\label{ref:RNDdk8oZbmINM}(Kłósak, 1980, pp.161–182; Wolak, 1991).
\parencites[][pp.161–182]{klosak_z_1980}[][]{wolak_filozofia_1991}. %
 The views of some Polish Catholic philosophers about this theory have also been discussed by P. Polak 
%\label{ref:RNDFJvOXSK3zq}(2016).
\parencite*[][]{polak_zmagania_2016}.%


Before we present Stanisław Dunin–Borkowski's views about the theory of relativity, let us first briefly discuss the life of this relatively unknown Polish philosopher, theologian, and pedagogue. Moreover, because he devoted so much of his activity as a~philosopher and historian to the philosophy of Spinoza, we will focus a~little more on this interest of the Polish Jesuit.

\section{A~count who became a~Jesuit and a~Spinoza specialist}
Stanisław Dunin–Borkowski, who is known chiefly as Stanislaus von Dunin–Borkowski in the literature, was the son of Count Witold Dunin–Borkowski, an owner of landed property in the village of Winniczki near Lvov, and Countess Kazimiera, née Fredro. Stanisław's grandfather, Aleksander, was an envoy to the Diet of Galicia and Lodomeria and a~founder and president of the Society of Fine Arts Enthusiasts in Lvov. His grandmother, Henryka, was a~head of the Lvov Educational Care Centre for the Blind. Suffused with the spirit of humanism and respect for art \textit{sensu largo}, a~family tradition of public activity left an undeniable imprint on Count Stanisław's character and his later interests.

Initially, Dunin–Borkowski's education was intended to prepare him to perform state administrative functions later in life, so the young count was sent to the famous \textit{Theresianische Akademie (Theresianum}) in Vienna. This episode was short-lived, however, and Stanisław soon moved to the Jesuit school \textit{Stella Matutina} in the Austrian town of Feldkirch. Sometime later, he entered the Jesuit novitiate before taking courses in classical languages and philosophy at the Jesuit study houses in the Netherlands. He completed a~four-year theology course among the English community at Ditton Hall, which had been established in the 1870s by German Jesuits who had fled Bismarck's \textit{Kulturkampf}. In 1889, between his philosophical and theological studies, Dunin–Borkowski began a~pastoral and teaching job at \textit{Stella Matutina}.

After taking holy orders in 1896, Dunin–Borkowski moved to the Jesuit writers' house in Limpertsberg (Lampertsbierg), a~district of Luxembourg City. According to the account of Rev. Paweł Siwek, S.J., he there fell under the friendly guidance of Rev. Erich Wasmann, S.J. (1859–1931), and the young Polish count and Jesuit was ``[…] trained for skillful use of the most effective weapon of today---the pen''
%\label{ref:RNDtSyOeEvW98}(Siwek, 1935, p.137).
\parencite[][p.137]{siwek_stanislaw_1935}. %
 Dunin–Borkowski was in daily contact with Wasmann, who was a~famous Jesuit entomologist, Catholic polemicist, and an advocate of the theistic interpretation of the theory of evolution, being dubbed ``the father of the ants'' (\textit{Ameisenpater}) 
%\label{ref:RND49bm4VmblP}(for more on Wasmann's activity see Baranzke, 1999; Polak, 2007).
\parencites[for more on Wasmann's activity see][]{baranzke_erich_1999}[][]{polak_spor_2007}.%
\footnote{For the subsequent years, Dunin-Borkowski's and Wasmann's publications would appear ``side by side'' in Jesuit journals \textit{Stimmen aus Maria Laach} and \textit{Stimmen der Zeit} (published by Herder).}

In 1920, Rev. Dunin–Borkowski, S.J. was appointed the spiritual director of the priestly monastery school in Wrocław, where he worked until 1931. He then moved to Koblenz and later Munich, where he died in 1934. The fruits of his activity can now only be found in his publications, because his unpublished manuscripts, notes, and other materials were destroyed when the Gestapo took control of the Munich Jesuit house in April 1941
%\label{ref:RND1G5wkxrOlb}(Stasiewski, 1959).
\parencite[][]{stasiewski_dunin-borkowski_1959}.%


To this day, Dunin–Borkowski's pastoral activity and pedagogical thought is best known in Germany and Austria. In the European and global dimensions, mainly in the interwar period, this Polish Jesuit was respected as an expert in the life and works of Baruch Spinoza
%\label{ref:RNDF112FFS6DE}(cf. Siwek, 1935, p.139).
\parencite[cf.][p.139]{siwek_stanislaw_1935}. %
 Dunin–Borkowski often tried to combine these areas of knowledge and practice, and he was convinced that ``[…] pedagogical theory that did not include the philosophy and psychology of love (\textit{die Philosophie und Psychologie der Liebe}) in the core of its content would never be able to find the road to pastoral deeds and success'' 
%\label{ref:RNDIrlXQwpUv3}(Dunin-Borkowski, 1926, p.43).
\parencite[][p.43]{dunin-borkowski_autobiographie_1926}.%


A~rather obvious question arises, though: Why did such a~well-educated Jesuit, educator, and theologian become interested in a~Jewish–Dutch thinker who created an extremely complex philosophical system? There is no simple answer to this question, because none of his works unequivocally explains Dunin–Borkowski's motivations for this interest. It is worth noting that he did not consider himself an advocate of Spinoza's views. In an early article about this thinker from Amsterdam, he declared that his mission was ``[…] just to understand Spinoza, but not admire or condemn him''
%\label{ref:RNDQz82g1doSy}(Dunin-Borkowski, 1902, p.126).
\parencite[][p.126]{dunin-borkowski_leben_1902}.%


On reading some of the remarks scattered among his works on Spinoza, one could conclude that Dunin–Borkowski was intrigued by the manifold interpretations of the oeuvre of this author, which included \textit{Ethics}
%\label{ref:RNDHsTmPJrxfc}(Dunin-Borkowski, 1910, p.30).
\parencite[][p.30]{dunin-borkowski_junge_1910}. %
 Moreover, interest in Spinoza was especially intense in the German cultural area in which Dunin–Borkowski lived and was active.\footnote{\textcolor{black}{There was also an interest in Spinoza among Jewish–German circles in the Weimar Republic} \label{ref:RNDIm3uwTIgBl}\textcolor{black}{(see, for example, Wertheim, 2011})\textcolor{black}{.}} Of the interpretations that developed in this area, two are particularly notable while also being essentially opposed to each other. The first reaches back to the 18\textsuperscript{th}-century German Romantic movement (e.g., F.H. Jacobi, J.W. Goethe) and then to the great idealist systems of F.W.J. Schelling and G.W.F. Hegel. F. Copleston wrote that the German Romantics ``found or thought they found in Spinoza a~kindred soul'' 
%\label{ref:RNDE8jRb7CNmU}(Copleston, 1994, p.261).
\parencite[][p.261]{copleston_history_1994}. %
 The second well-known pattern of thought that selected Spinoza as its guide was the monistic and materialistic movement of L. Büchner and K. Vogt 
%\label{ref:RNDRLm25c9wsk}(cf. Pawlicki, 1912).
\parencite[cf.][]{pawlicki_spinoza_1912}.%


It is hard to clearly establish which of the above two interpretations spurred Dunin–Borkowski into researching Spinoza's work. In his first volume of analyses (dated 1910), references to Jacobi and Goethe are as numerous as references to the monistic concept of psychophysical parallelism. At the same time, Dunin–Borkowski appeared to share the beliefs held by the materialistically oriented monists who saw Spinoza as their patron:\footnote{It is also noteworthy that the references to the author of Ethics in the context of the discussion of the soul-body problem and the parallelistic concepts came to feature quite prominently in the works by other Polish Jesuits active in the days of Dunin-Borkowski and beyond, e.g. Rev. Fryderyk Klimke (1878--1924) and Rev. Paweł Siwek (1893--1986)
%\label{ref:RNDFU3wI1RXuC}(cf. Klimke, 1906, pp.6, 16, 33; 1911, pp.312–332; Bremer and Poznański, 2020, pp.1310–1314).
\parencites*[cf.][pp.6]{klimke_teorya_1906}[][pp.312–332]{klimke_monismus_1911}[][pp.1310–1314]{bremer_philosophy_2020}.%
}

\myquote{
You will only have half of Spinoza if you emphasize his anti-Christian monism, but omit his silent struggle against the pestering, purely materialistic tendencies that undermined morality and religion back then
%\label{ref:RND031k4sAy2Q}(Dunin–Borkowski, 1910, p.XVI).
\parencite[][p.XVI]{dunin-borkowski_junge_1910}.%
}

It seems reasonable to assume that these prevalent anti-Christian interpretations of Spinoza to some extent motivated Dunin–Bukowski's studies of Spinoza, but it would go too far beyond the scope of this article to elaborate on this assumption. Anyway, in the face of the multiple interpretations of Spinoza's system, Dunin–Borkowski resolved to perform a~historical and critical analysis of the available early editions of Spinoza's works, as well as manuscripts and other materials related to this Dutch thinker. Employing philological and historical methods, he set out to capture a~proper sense of Spinoza's statements. He even sometimes tried to correct Spinoza's thoughts, believing that this would be what the author of \textit{Ethics} would have done to maintain consistent adherence to the premises of his own system
%\label{ref:RNDMMJH1ZuEKd}(Siwek, 1938, p.308).
\parencite[][p.308]{siwek_spinoza_1938}. %
 Indeed, the Polish Jesuit was perfectly prepared for such tasks. In addition to his fluency in Latin, he was also proficient in Dutch, among other things. For example, he had studied 17\textsuperscript{th}-century philosophical concepts and the specificity of the Dutch culture of that era. In Dunin–Borkowski's opinion, it was only from such a~research perspective that one could furnish explications that would enable a~proper reading of Spinoza's work 
%\label{ref:RNDKXwyzRIvZE}(Dunin-Borkowski, 1910, p.XI–XXIII).
\parencite[][p.XI–XXIII]{dunin-borkowski_junge_1910}.%


Having discussed Dunin–Borkowski's research interests, particularly his passion for analyzing the life and work of Spinoza, we will now focus on the Polish Jesuit's own views about the special theory of relativity proposed by Albert Einstein (1879–1955).\footnote{The convergence of Dunin-Borkowski's extensive studies on Spinoza and the widely known fascination on the part of Einstein with the life and works of this Dutch thinker, it seems, was entirely coincidental. There is no information either that the Polish Jesuit knew Einstein's interests in Spinoza, or that Einstein read Dunin-Borkowski's work \textit{Der junge De Spinoza}
%\label{ref:RNDsa26pIpsXY}(1910).
\parencite*[][]{dunin-borkowski_junge_1910}.%
}

\section{The theory of relativity as merely\\a~mathematical tool}
In 1921, the Jesuit journal \textit{Stimmen der Zeit} published an article of Dunin–Borkowski entitled ``Neue philosophische Strömungen [New philosophical currents]''
%\label{ref:RNDqqDiZYVXU8}(Dunin–Borkowski, 1921b).
\parencite*[][]{dunin-borkowski_neue_1921}. %
 In reality, he did not discuss any emergent thought currents but rather the philosophical repercussions associated with scientific ideas, such as the physical concept of force (in light of the classical concept of substance), the problem of the so-called psychophysical parallelism, and Einstein's special theory of relativity. Dunin–Borkowski paid most attention to the last of these matters. In his own words, he was interested in ``a certain philosophical line of thought'' that he believed was among ``the most important tasks of the current philosophy of nature (\textit{Naturphilosophie}).'' This is why he chose to ``take up several philosophical questions that this new theory [of relativity---J.R.] prompts'' 
%\label{ref:RNDD9HjrT4LE8}(Dunin–Borkowski, 1921b, p.211).
\parencite*[][p.211]{dunin-borkowski_neue_1921}.%


The special theory of relativity (STR) gives rise to conclusions about the contraction of object lengths and time dilation in moving systems (uniformly and rectilinearly) at high speeds (i.e., comparable with the speed of light) in relation to a~selected system. Even at the beginning of his commentary on these aspects of the STR, Dunin–Borkowski made a~surprising statement: ``[…] of course these differences cannot be proven experimentally''
%\label{ref:RNDJsvUoyeYTY}(Dunin-Borkowski, 1921b, p.211).
\parencite*[][p.211]{dunin-borkowski_neue_1921}. %
 For him, the conclusions that followed from the STR were ``purely mathematical, drawn thanks to the application of the Lorentz equations'' 
%\label{ref:RNDCzUqMHg2jq}(Dunin–Borkowski, 1921b, p.212).
\parencite*[][p.212]{dunin-borkowski_neue_1921}.%


A~similar opinion was expressed by the well-known Belgian neo-scholasticist Rev. Desiré Nys (1859–1927): ``The theory of relativity appears to us a~purely mathematical conception [...]. It would be reckless to regard the theory as a~representation of reality''
%\label{ref:RNDyEqoULB68k}(Nys, 1922, p.321).
\parencite*[][p.321]{nys_notion_1922}. %
 The view expressed by Rev. Konstantin Gutberlet (1837–1928)---a German theologian, philosopher, and friend of G. Cantor's---went in a~similar vein. In a~1913 work, he wrote that while the STR could be expressed in a~form of ``an even more adventurous fiction'' (\textit{noch abenteuerlichen Fiktion}) as a~four-dimensional spacetime, ``[...] this has nothing in common with reality'' 
%\label{ref:RNDM0ADaVpV8A}(Gutberlet, 1913, p.334).
\parencite*[][p.334]{gutberlet_streit_1913}.%


Dunin–Borkowski was, however, inclined to admit that the mathematical results related to the STR could significantly explain (\textit{erklären}) a~number of physical and astronomical facts in a~``simple and certain'' way
%\label{ref:RNDgmU1KW73M4}(Dunin-Borkowski, 1921b, p.212).
\parencite*[][p.212]{dunin-borkowski_neue_1921}. %
 Nevertheless, he believed achieving this kind of agreement provides explication (\textit{Erklärung}) but not comprehension (\textit{Erkenntnis}) of objective reality. Thus, the Polish thinker treated the STR as a~mere mathematical tool and physical hypothesis.

Interpreting the STR solely as a~mathematical tool and an experimentally unverifiable physical concept is reminiscent of the reaction by Catholic dignitaries and theologians to any questioning of the geocentric world system in the 16\textsuperscript{th} and 17\textsuperscript{th} centuries. Admittedly, as a~mathematical theory and a~physical hypothesis that afforded simple explanations and served certain practical purposes, Einstein's ``calculi'' were accepted to some extent by authors like Dunin–Borkowski, just like Copernicus' ``hypotheses'' in the past, but they were rejected outright as experimentally confirmed statements about reality
%\label{ref:RNDSSklyY9FMb}(cf. Benk, 2000, p.121).
\parencite[cf.][p.121]{benk_moderne_2000}. %
 An analogous situation involved the famous foreword (\textit{Ad Lectorem}) that unbeknownst to Nicolaus Copernicus had been included in the first edition of \textit{De revolutionibus}. In this, Andreas Osiander (1498–1552) tried to demonstrate that ``these hypotheses need not be true nor even probable. On the contrary, if they provide a~calculus consistent with the observations, that alone is enough'' 
%\label{ref:RNDQvTYwffmD2}(Osiander, 1978, p.XVI).
\parencite[][p.XVI]{osiander_foreword_1978}. %
 Similar advice was presented to Galileo by Cardinal Robert Bellarmine (1542–1621), where the motion of the planets in the solar system was presented as merely a~hypothesis in order to avoid conflicting with Catholic doctrine and its closely related geocentric cosmology\footnote{The fact that the Copernican system was treated by Catholic (and Protestant) theologians as a~hypothesis did not, however, stand in the way of the system being used as the foundation for the Gregorian calendar reform of 1582 
%\label{ref:RNDccEIJDFXLK}(cf. Crombie, 1953, p.326).
\parencite[cf.][p.326]{crombie_augustine_1953}.%
}  
%\label{ref:RNDTOhSkxo5Fm}(1968, pp.23–52).
\parencite[on the various ways of understanding the concept of ``hypothesis,'' from antiquity to Newton, see the work of][pp.23–52]{koyre_newtonian_1968}.%.

In addition, in light of the above seemingly surprising statement by the Polish Jesuit, whereby the conclusions that follow from the STR ``cannot of course be proven experimentally,'' the situation becomes clearer. In this case, Dunin–Borkowski was not referring to a~hypothesis that is empirically tested based on mathematical and natural methods but rather a~hypothesis as a~purely mathematical construct that reveals the existing knowledge (here: Lorentz's theory) in a~new way, thus enabling calculations. A~hypothesis understood this way cannot be confirmed (proven) experimentally, because in principle and in itself, it tells us nothing about reality. Such a~mindset clearly shows that Dunin–Borkowski had problems understanding the role of mathematics in modern natural sciences, and these likely resulted from his adoption of the current-specific Aristotelian concept of mathematics combined with the principle of \textit{metabasis}.\footnote{As formulated by Aristotle, this prohibited the use of methods and concepts from one area of knowledge (e.g., geometry) in another (e.g., the philosophy of nature). The areas of knowledge were classified by the Stagirite, with them having an immutable basis in an ontologically understood reality.}

With regard to the motion of physical systems in relation to one another, Dunin–Borkowski wrote the following in his 1921 article:

\myquote{
But according to Einstein, it is not motion in itself (\textit{die Bewegung an sich}) which exerts specific influence. The difference rather lies in the influence of purely \textbf{relative} motion---the only one he allows. […] In other words, relative motion influences (\textit{beeinflußt}) the length and time in exactly the same manner as every absolute does (\textit{wie eine etwaige absolute}) [emphasis---Dunin–Borkowski]
%\label{ref:RNDZLFCzgqKjZ}(Dunin-Borkowski, 1921b, p.212).
\parencite[][p.212]{dunin-borkowski_neue_1921}.%
}


What is notable in this passage is the distinction between ``purely relative motion'' and ``motion in itself,'' as well as the words about the ``influence'' of relative motion on length. The latter statement, along with its association with the notion of something ``absolute,'' may imply the presence of some real force that changes the length of an object as it moves. \textit{Nota bene}, the concept of the ``influence of motion'' appears very often in Dunin–Borkowski's article.

The mathematical structure of the STR does not require the introduction of the above distinctions in relation to motion. The phrase ``motion in itself'' rather shows that Dunin–Borkowski imposed a~conceptual pattern that was alien to the physical theory, and at the same time, he suggests an interpretation that attempts to impose a~specific philosophical character on the theory. The phrase ``the influence of motion'' for the change in length is essentially erroneous, because according to the STR, this length contraction is kinematic
%\label{ref:RNDYkwAQ0KBx8}(cf. Heller, 1992, p.216; 1995, p.40).
\parencites[cf.][p.216]{heller_materia_1992}[][p.40]{heller_henri_1995}.%


For Dunin–Borkowski, the STR was nothing more than ``a science of the system of relative motion''
%\label{ref:RNDDqjlJLYzWb}(1921b, p.213).
\parencite*[][p.213]{dunin-borkowski_neue_1921}. %
 Indeed, the Polish thinker wrote about ``the unacceptable \textbf{objectification of pure relation} [emphasis---Dunin–Borkowski]'' 
%\label{ref:RND5tj0tyg1eC}(1921b, pp.214–215).
\parencite*[][pp.214–215]{dunin-borkowski_neue_1921}.%
\footnote{His opposition to the ``objectification of pure relation'' might have been caused by Dunin-Borkowski's adoption of the Aristotelian concept of categories in which a~relation is only one of the nine accidental categories in relation to the overriding category of substance/object.} As he criticized such an approach, he concluded that what was being ``[…] introduced is the concept of objective changes instead of only relative shifts (\textit{relative Verschiebungen})'' 
%\label{ref:RNDT2Lnt6MQZl}(1921b, p.214).
\parencite*[][p.214]{dunin-borkowski_neue_1921}.%


Dunin–Borkowski by no means denied the fundamental possibility of acquiring objective knowledge about reality---he merely challenged the belief that such knowledge might be gleaned through the STR. ``You mustn't misunderstand Einstein,'' wrote the Polish Jesuit, ``he does not presuppose that the length of a~rigid rod has become \textbf{objectively} shorter through motion [emphasis---Dunin–Borkowski]''
%\label{ref:RNDHP2qLY7TQW}(1921b, p.212).
\parencite*[][p.212]{dunin-borkowski_neue_1921}. %
 Nevertheless, Dunin–Borkowski levelled the following accusation at Einstein: ``[\ldots] he and some of his supporters […] appear to be drawing philosophical conclusions that do not spring from [the STR---J.R.] by necessity'' (ibidem), and ``under closer scrutiny the surprising philosophical consequences\footnote{Among which Dunin-Borkowski reckoned also the so-called twin paradox.} do not take place'' 
%\label{ref:RNDFpLT9QRtGr}(1921b, p.215).
\parencite*[][p.215]{dunin-borkowski_neue_1921}.%


The Polish Jesuit disagreed with Einstein wherever he perceived the threat of ``the possibility of absolute definitions (\textit{Bestimmungen})''
%\label{ref:RNDsKUZbVUAFh}(Dunin-Borkowski, 1921b, p.214)
\parencite[][p.214]{dunin-borkowski_neue_1921} %
 in relation to the material outside (i.e., objective) world. In particular, he favored the notion of ``absolute motion'' 
%\label{ref:RNDshivUFghMK}(1921b, p.213).
\parencite*[][p.213]{dunin-borkowski_neue_1921}. %
 According to A.~Benk, who briefly referred to Dunin–Borkowski's views about the theory of relativity from the physical viewpoint, similar interpretative attempts can still, to some extent, be likened to the efforts of Hendrik A. Lorentz (1853–1928), the Dutch physicist who attempted to explain light propagation after the classical fashion. As he changed the frame of reference, apart from motion, he also transformed time, but he maintained the ``true'' or ``absolute'' time in relation to the local time in other frames of reference 
%\label{ref:RNDMkUkThmjQE}(Benk, 2000, p.123; cf. also Wróblewski, 2006, p.440).
\parencites[][p.123]{benk_moderne_2000}[cf. also][p.440]{wroblewski_historia_2006}. %
 Only the STR made it clear that from the physical aspect, no inertial frame of reference is favored, so by extension, this new physics does not justify the need to refer to absolute motion or absolute time. Still, in his 1921 article, Dunin–Borkowski unpretentiously stated that unlike Einstein and the STR, he ``did not see it necessary to be in favor of such conclusions'' 
%\label{ref:RNDRTLQmWuDbA}(Dunin-Borkowski, 1921b, p.214).
\parencite[][p.214]{dunin-borkowski_neue_1921}.%


In the same issue of \textit{Stimmen der Zeit} in which Dunin–Borkowski published his article, another Jesuit, Rev. Theodor Wulf (1868–1946), wrote a~text on the STR, expressing that philosophy ``certainly [teaches---J.R.] about the existence of absolute motion''
%\label{ref:RNDoDSCQDdiGp}(Wulf, 1921, p.115).
\parencite[][p.115]{wulf_pu1}. %
 This statement is interesting for at least one reason, namely that Wulf was an educated physicist who had studied under Walther Nernst (1864–1941) in Göttingen and authored some groundbreaking works on cosmic radiation. Neither Dunin–Borkowski nor Wulf provided a~more explicit justification for their assertion that in terms of philosophy, one can speak of absolute movement. However, at least one explanation can be given for this attitude, such that both authors understood movement as a~change that is, in the scholastic sense, the transition of being from a~potentiality to an act. The change in this case is objective and absolute, so accordingly, we can speak of absolute movements.\footnote{I~would like to thank one of the reviewers for this suggestion.} However, this is an ontological understanding of a~movement expressed in the context of a~specific philosophy. Such terms cannot be compared with those of physics, which are empirically operational, and one cannot judge physical theories based on a~specific philosophical theory. This has unfortunately been attempted by many neo-scholastic thinkers in relation to Einstein's theories.

\section{Dunin–Borkowski, Cardinal Bellarmine, and fictionalism}
In the literature, Stanisław Dunin–Borkowski's works, aside from those devoted to Spinoza, are associated with the \textit{sensu largo neo-scholastic current}, which chiefly refers to the intellectual legacy of Aristotle and Thomas Aquinas
%\label{ref:RND4FDw3AtDVV}(Bruehl, 1921, p.305; Benk, 2000, pp.119–123).
\parencites[][p.305]{bruehl_implications_1921}[][pp.119–123]{benk_moderne_2000}. %
 This current grew in strength in the 1880s following a~number of initiatives by Catholic Church authorities (mainly Pope Leo XIII) aimed at recommending a~revival in the studies and teachings of Aquinas' thoughts. A~crucial element of the neo-scholastic program was opening the research of Catholic scholars up to the contemporary currents of philosophy and science, including the natural sciences. Apparently, the neo-scholastic current could include Dunin–Borkowski's work ``\textit{Neue philosophische Strömungen,}''\footnote{As well as Dunin–Borkowski's article 
%\label{ref:RNDVvQ8dOkqiG}(1921a),
\parencite*[][]{dunin-borkowski_auf_1921}, %
 in which the author clearly confronts contemporary philosophical views and scientific concepts (and the theory of relativity, but only in one sentence) with the neo-scholastic viewpoint.} which more comprehensively addresses Einstein's special theory of relativity.

One might conclude from Dunin–Borkowski's arguments that to some extent, the Polish Jesuit accepted the special theory of relativity merely as a~mathematicised hypothesis or theory (i.e., he writes about the application of Lorentz equations) that allows the ``explication'' (i.e., achieving an agreement between the conclusions drawn and the observations) of certain physical and astronomical facts. At the same time, he levelled criticism against ``a certain philosophical line of thought'' that in his opinion accompanied the special theory of relativity as a~kind of interpretation.

Given Dunin–Borkowski's article, it is difficult to establish what source materials about the theory of relativity were used by the author. The semi-popular character of the work probably explains why there are no referenced quotations. The form of discourse used resembles descriptions of the physical consequences of this theory (e.g., length contraction, time dilation, relativity of motion) that are frequently presented, albeit for popularization purposes, by physicists themselves in everyday language. However, the Polish thinker intersperses his descriptions with analyses involving philosophical terminology (e.g., ``motion in itself,'' ``an objective absence of change,'' ``philosophical contradiction''). This seems unacceptable, especially in a~situation where one is trying to render the proper physical meaning of a~scientific theory's consequences.

Any presentation of a~physical theory like the special theory of relativity should, it seems, not ignore the properties of the mathematical apparatus in which it is being expressed or the methodological character of its references to physical reality, which serves as the testing ground for its reliability. Unfortunately, Dunin–Borkowski did not consider these basic elements of the cognitive character of mathematical physics. In his philosophical discourse, he engaged in polemics with paraphrases of popular (which, however, is not to be understood as erroneous!) descriptions of the consequences of the theory of relativity rather than referring directly to the mathematical presentation and sense. (For more on this subject see
%\label{ref:RNDWTwmi4p9o2}(Heller, 1999, pp.101–102)
\parencite[][pp.101–102]{heller_physics_1999}%
]). This is particularly evident in the Polish author's comparison of Einstein's alleged ``speculations'' about the relativity of time and space to, in his opinion, ``similar ideas'' by Jaime Balmes (1810–1848),\footnote{Balmes developed a~completely non-physical and thoroughly philosophical concept of a~relation in which time is a~factor secondary to changes taking place in reality. In particular, the relation was concerned with changes in the sequence of ``be'' after ``not be'' (\textit{non être}) 
%\label{ref:RNDZi7WGnTZnD}(cf. Balmes, 1852).
\parencite[cf.][]{balmes_philosophie_1852}. %
 For more on this concept see 
%\label{ref:RNDXLRrDnYJZt}(Wojciechowski, 1955, pp.678–683).
\parencite[][pp.678–683]{wojciechowski_teorie_1955}.%
} a~19\textsuperscript{th}-century Spanish Thomist 
%\label{ref:RNDg2wGC1XSEi}(Dunin–Borkowski, 1921a, p.275).
\parencite[][p.275]{dunin-borkowski_auf_1921}. %
 Being experienced in the milieu-related and literary analyses of Spinoza's works, Dunin–Borkowski probably did not fully realize that his skill set was not applicable to contemporary, mathematicised physical theories. His method for the exegesis of texts on theory rather than the exegesis of the structures of mathematical theories unfortunately meant that him and many other neo-scholastic thinkers were not in a~place where they could have a~responsible discussion about the philosophical aspects of new physical theories.

After reading this present paper, one may question whether Dunin–Borkowski's article should be placed within the interpretative tradition of factionalism, which was developed by Jesuits as early as the 16\textsuperscript{th} and 17\textsuperscript{th} centuries. According to William B. Ashworth, Jr. this tradition dates back to the works of Rev. Christoph Clavius, S.J., and its most prominent manifestation was the famous words of Cardinal R. Bellarmine, S.J. that were directed at Carmelite Fr. Paolo A. Foscarini, OCD in a~1615 letter:

\myquote{
I~say that it seems to me that Your Paternity and Mr. Galileo are proceeding prudently by limiting yourselves to speaking suppositionally (\textit{ex suppositione}) and not absolutely, as I~have always believed that Copernicus spoke
%\label{ref:RNDYiWaL1KQVh}(cit. by Finocchiaro, 1989, p.67).
\parencite[cit. by][p.67]{finocchiaro_galileo_1989}.%
}
Both Clavius and Bellarmine expressed the opinion that all astronomical systems are concepts (hypotheses) designed only to ``save phenomena,'' with them having nothing in common with reality. Reality is therefore the exclusive domain of philosophers and not astronomers
%\label{ref:RND5f95V2p91z}(Ashworth, Jr., 1986, p.158).
\parencite[][p.158]{numbers_catholicism_1986}. %
 The remarkable thing about the above quotation is the striking similarity between Bellarmine's suggestion that Foscarini and Galileo treat the Copernican system as a~hypothesis rather than an absolute formulation and Dunin–Borkowski's accusation that Einstein treated the concepts of relative motion, relative time, and relative length as something absolute (i.e., as if they referred to reality).\footnote{Interestingly, at the beginning of his article, as he briefly analyses the multitude of positions concerning the cognitive status of contemporary scientific concepts, Dunin-Borkowski repeatedly refers to Friedrich Lipsius (1873--1934), an advocate of Hans Vaihinger's views. Vaihinger (1852--1933) created a~thought current underpinned by contemporary positivism, and referred to as fictionalism (\textit{Philosophie des Als Ob}), which states that, \textit{inter alia}, mathematical and physical concepts are fictions.}

Stanisław Dunin–Borkowski's articles are not isolated examples given the number of works by other neo-scholastic (neo-Thomistic) authors who grappled intellectually with Einstein's theory of relativity in the interwar period 
%\label{ref:RNDR7nSKD1RuU}(Polak, 2016)
\parencites[cf.][]{polak_zmagania_2016}[and beyond the above-mentioned works][]{glick_relativity_1987}[][]{flipse_between_2010}[][]{hagen_local_2020}. %
 Of course, criticism of, and even opposition to, the new concepts of physics (they were also concerned with quantum mechanics) did not exert any influence on the further course of its history.\footnote{Georges Lemaître can be considered an exception in this circle. Although he was thoroughly educated in mathematics, physics, and also neo-Scholastic philosophy (i.a. at the Cardinal Mercier's Higher Institute for Philosophy), he was able to take his thoughts above the narrow interpretations of Einstein's theory and appreciate the importance of mathematics in understanding of nature 
%\label{ref:RNDuzJO8c7JiU}(Hagen, 2020, pp.241–242).
\parencite[][pp.241–242]{hagen_local_2020}.%
} On the other hand, part of the writing activity from this Polish count, Jesuit, and pedagogue revealed the crucial systemic difficulties faced by representatives of the neo-scholastic movement when engaging with the methods and language of the mathematicised natural sciences.\footnote{This state of affairs concerned not only neo-scholastics who were active in the area of interwar Germany and Austria, but also numerous representatives of this current of thought in France, Poland, and even Belgium (along with the still living tradition of studies originated by Cardinal D.-J. Mercier). In Poland the situation took a~qualitative turn only in the 1970s, which saw the beginning of collaboration between physicists, philosophers and theologians, most of whom were associated with the Cracow academic milieu 
%\label{ref:RNDUrY06lkRPN}(cf. Wolak, 1991; Polak and Rodzeń, 2021; Trombik, 2021).
\parencites[cf.][]{wolak_filozofia_1991}[][]{polak_science-religion_2021}[][]{trombik_koncepcje_2021}. %
 } It should be added that Dunin–Borkowski, despite his Polish origin, was associated with a~German-speaking (mainly Austrian) area of culture for most of his life, and he was largely shaped by this area. In any comparisons between his views and the works of other neo-scholastic authors, research should start with the relationship between his thoughts and the intellectual tendencies that were characteristic of circles in the mainly Austrian and German Catholic philosophy of that time.

\paragraph{Acknowledgments}
The author wishes to express his sincere gratitude to Paweł Polak and the anonymous reviewers for the immensely valuable remarks they made when this paper was being prepared.

\end{artengenv}
