\begin{newrevplenv}{Sebastian J. Szybka}
	{Mała książka o wielkim wszechświecie}
	{Mała książka o wielkim wszechświecie}
	{Mała książka o wielkim wszechświecie}
	{Uniwersytet Jagielloński}
	{Lyman Page, \textit{The Little Book of Cosmology}, Princeton University Press, 2020, polskie wydanie: \textit{Mała księga kosmologii}, tłum. J.J. Borkała, Copernicus Center Press, 2021.}


\lettrine[loversize=0.13,lines=2,lraise=-0.03,nindent=0em,findent=0.2pt]%
{L}{}yman Page, autor \textit{Małej księgi kosmologii}
%\label{ref:RNDqWKAYUb3ZT}(Page, 2021, 2020),
\parencite[][]{page_little_2020}, %
 profesor fizyki na Uniwersytecie Princeton, to znany ekspert z~zakresu kosmologii obserwacyjnej. Na początku XXI wieku Page współkierował zespołem analizującym anizotropie mikrofalowego promieniowania tła na podstawie danych z~satelity WMAP. Między innymi dzięki jego pracy udało się wyznaczyć, z~niespotykaną wcześniej dokładnością, wartości parametrów kosmologicznych. W~roku 2010 za swoje osiągnięcia otrzymał, wraz ze współpracownikami, Nagrodę Shawa. Osiem lat później został współlaureatem prestiżowej nagrody Fundamental Physics Breakthrough Prize.

Postęp technologiczny sprawił, iż na początku XXI wieku kosmologia przekształciła się w~naukę precyzyjną. Lyman Page brał czynny udział w~tej mini-rewolucji. Jego ,,mała księga'' to zwięzłe wprowadzenie do nowoczesnej kosmologii. Autor w~przystępny sposób przedstawia aktualne obserwacje i~objaśnia, jak na ich podstawie, przy użyciu teorii fizycznych, zbudować spójny obraz kosmosu w~jego największej skali. Książka ma charakter popularnonaukowy. Choć wymagana wiedza matematyczna nie wykracza poza podstawowe operacje algebraiczne, to autor odwołuje się do nich dosyć często i~tekst trzeba czytać uważnie. Nagrodą za wytrwałość jest zrozumienie związków pomiędzy omawianymi wielkościami. Jak przystało na kosmologa obserwacyjnego, Page skupia się na tym, co widzą astronomowie i~jak ich obserwacje pogodzić ze współczesną fizyką. Teoria grawitacji Einsteina odgrywa w~tej opowieści drugorzędną rolę. Szczególny nacisk jest położony na znaczenie i~interpretację obserwacji mikrofalowego promieniowania tła -- zagadnienie, z~którym związana jest kariera naukowa Page'a. Niemniej autor nie ogranicza się wyłącznie do swej ulubionej tematyki. Choć \textit{Mała księga kosmologii} rzeczywiście jest niewielką książeczką -- jej główna część ma mniej niż 150 stron -- to poruszono w~niej większość najważniejszych zagadnień współczesnej kosmologii. Czytelnik dowiaduje się, na czym polega ekspansja przestrzeni, dlaczego uważamy, że wszechświat był kiedyś bardzo gęsty i~gorący oraz dlaczego podejrzewamy, iż niewidoczne składniki wszechświata, takie jak ciemna materia i~energia, istnieją naprawdę. Page omawia również wiele innych tematów, takich jak: soczewkowanie grawitacyjne, krzywiznę przestrzeni, hipotezę kosmicznej inflacji, kwantowe fluktuacje jako zalążki kosmicznej struktury. Autor, na wzór odkrywcy elektronu Josepha Thompsona, wyznaje zasadę, iż mechaniczny model zjawiska znacznie ułatwia jego zrozumienie. W~książce znajdziemy co najmniej kilka tego typu analogii. Na przykład formowanie się kosmicznych struktur i~tzw. niestabilność grawitacyjna są objaśnione za pomocą uproszczonego modelu mechanicznego: nieskończenie długiego jednowymiarowego ciągu równo rozmieszczonych i~nieruchomych obiektów o~tej samej masie oddziałujących ze sobą grawitacyjnie. Intuicyjnie domyślamy się, że taka konfiguracja nie jest stabilna i~łatwo sobie wyobrazić, jak ewoluują drobne pierwotne zaburzenia w~rozkładzie tych obiektów.

Czym \textit{Mała księga kosmologii} wyróżnia się wśród wielu podobnych popularnonaukowych pozycji? Anton Czechow twierdził, że ,,sztuka pisania jest sztuką skracania, a~zwięzłość jest siostrą talentu''. Nie wiem, czy Czechow miał rację, ale kosmologiczna opowieść Page'a jest zwięzła. Nie ubarwiają jej postacie szalonych naukowców, opisy dyskusji przy kawie, chwil natchnienia, ludzkich porażek, sukcesów. Nie zawiera zabawnych anegdot. Nie znajdziemy w~niej głębokich przemyśleń na temat natury świata i~nauki, czy też intrygujących, lecz niepotwierdzonych obserwacyjnie spekulacji. To bezosobowa opowieść o~tym co wiemy, czego nie wiemy i~o tym, czego mamy nadzieję niedługo się dowiedzieć. Autor snując historie o~gwiazdach mocno stąpa po Ziemi. Być może wśród czytelników znajdą się i~tacy, których forma \textit{Małej księgi} zniechęci do lektury. Moim zdaniem, to właśnie prostota i~rzetelność wynikająca z~zakorzenienia w~,,twardej'' nauce czyni z~niej pozycję wartą uwagi. Lecz byłoby niesprawiedliwością twierdzić, iż oryginalność \textit{Małej księgi} bierze się w~głównej mierze z~literackiego minimalizmu i~suchości narracji. Dzieło Page'a nie jest podręcznikiem akademickim pozbawionym skomplikowanych wzorów.

Wszechświat wyzbyty ze swego metafizycznego znaczenia jest zwykłym układem fizycznym. Jak zauważa Page: ,,Niewiele jest układów badanych przez naukowców, które można opisać tak prosto, kompletnie i~z tak dużą dokładnością. Mamy szczęście, że obserwowalny wszechświat jest jednym z~nich''. Wielki Wybuch, ekspandująca przestrzeń wypełniona promieniowaniem reliktowym, tworzące się struktury, galaktyki i~ich gromady, olbrzymie kosmiczne pustki, parametry kosmologiczne i~ich związek z~ewolucją wszechświata. Po wielu latach studiów, w~głowie kosmologa wszystkie te elementy zaczynają ze sobą współgrać. Lyman Page dzieli się z~czytelnikiem swoją fizyczną intuicją, a~sposób, w~jaki to czyni, jest warty uwagi. Autor wykorzystuje analogie do zjawisk znanych z~codziennego życia. Ziemię od Księżyca oddziela około 400 000 km -- dystans odpowiadający typowemu przebiegowi dobrego samochodu, zanim się zepsuje. Rozmiary kątowe Ultragłębokiego Pola Hubble'a odpowiadają rozmiarom około jednej sześćdziesiątej tarczy Księżyca, czyli jednej studwudziestej kąta przesłanianego przez mały palec trzymany na wyciągnięcie ręki. Żarówka, fortepian, włos, mikrofalówka, klocki, szum na ekranie telewizora, pudełko M\&M-sów, policyjny radar, piłki plażowe, radio samochodowe, fale na oceanie, anteny satelitarne -- \textit{Mała księga kosmologii} jest pełna analogii, które niewyobrażalne astronomiczne liczby sprowadzają do znanych nam skal i~które pozwalają myśleć o~abstrakcyjnych procesach w~sposób strawny dla zwykłych śmiertelników. Page, niczym Enrico Fermi -- mistrz krótkich rachunków, podpowiada nam, jak na skrawku papieru w~prosty sposób wyliczyć takie wielkości jak wiek wszechświata czy też rozmiar jego obserwowalnej części -- i~to wszystko wyłącznie przy pomocy ,,szkolnej'' matematyki. W~książce poszczególne elementy kosmologicznej układanki łączą się w~całość, uwidaczniając głębokie powiązania istniejące w~naturze.

Lektura książki Page'a daje nam poczucie obcowania z~tekstem napisanym przez kompetentnego autora, eksperta w~swojej dziedzinie. Spojrzenie Page'a na kosmologię to aktualny raport z~pierwszej linii frontu -- sprawozdanie osoby, która jest bezpośrednio zaangażowana w~badania. Jednym z~najważniejszych źródeł wiedzy o~wszechświecie jest widmo mocy anizotropii mikrofalowego promieniowania tła. Wykres ten, kluczowy dla nowoczesnej kosmologii, jest trudny do objaśnienia bez odwoływania się do wiedzy specjalistycznej. Page radzi sobie z~tym problemem znakomicie. Bez zagłębiania się w~zbędne szczegóły opisuje, jak taki wykres samodzielnie przygotować na podstawie mapy mikrofalowego promieniowania tła. Dalsze wyjaśnienia stają się zbędne. Na uwagę zasługuje również rozdział dotyczący pomiarów mikrofalowego promieniowania tła. Page, kosmolog obserwacyjny, nie ogranicza się wypisania kto co zmierzył i~kiedy. Przedstawia problem z~punktu widzenia eksperymentatora i~opisuje w~jaki sposób wykonać taki pomiar, co jest źródłem zakłóceń, jak udoskonalić stosowaną metodę. Stawiane w~książce pytania trafnie uprzedzają te, które pojawią się w~głowie czytelnika. Skąd wiadomo, że promieniowanie tła ma charakter kosmologiczny? Dlaczego gęstsze obszary wszechświata są cieplejsze? Czym jest ,,promieniowanie ciała doskonale czarnego''?

Dla kogo Lyman Page napisał swoją książkę? Ascetyczna forma przekazu sprawia, że nie jest to książka dla wszystkich. Młodzi miłośnicy astronomii, studenci kierunków ścisłych, specjaliści z~innych dziedzin fizyki, filozofowie nauki -- inaczej mówiąc, osoby posiadające już podstawową wiedzę fizyczną i~zdeterminowane do jej poszerzenia o~zagadnienia kosmologiczne -- powinni entuzjastycznie podejść do tej pozycji. Dla nich będzie to lektura łatwa i~przyjemna -- niezastąpiona alternatywa względem zaawansowanej sześciusetstronicowej \textit{Kosmologii} Stevena Weinberga
%\label{ref:RND7AgCe7qBbS}(Weinberg, 2008).
\parencite*[][]{weinberg_cosmology_2008}. %
 Dzięki \textit{Małej Księdze} w jedno popołudnie można pobieżnie zapoznać się z~aktualnym stanem kosmologicznych badań. Jeśli jednak książka znajdzie się w~rękach niezdecydowanego czytelnika, który nie łaknie jeszcze kosmologicznej wiedzy, to nie rozbudzi ona jego apetytu. Nauka jest największą przygodą ludzkości i~jak każda przygoda składa się z~sukcesów, porażek, chwil niepewności. Nauka to proces dochodzenia do prawdy o~świecie, w~którym zakręty myśli wiążą się nierozerwalnie z~wirem ludzkich losów unoszonych podmuchami historii. To właśnie ta ludzka strona przygody ma swój nieodparty czar i~może zachęcić nieśmiałych podróżników, by wykonali pierwszy krok, od którego zaczyna się każda wędrówka. Doskonałym przykładem tego typu popularyzacji są książki Michała Hellera z~\textit{Ewolucją kosmosu i~kosmologii} \parencite*{heller_ewolucja} na czele, czy też niezrównane \textit{Gawędy o~sztuce} Bożeny Fabiani \parencite*[zob. np.][]{fabiani_gawedy}. Dostrzeżenie człowieka stojącego za dziełem sprawia, iż ,,nudna'' historia sztuki może zafascynować nawet małe dziecko. \textit{Mała księga kosmologii}, wskutek świadomej decyzji autora, nie posiada tego elementu. Nie jest to książka, która uwodzi, lecz zwięzłe źródło informacji podanych w~przystępnej formie.

Jeśli ktoś chciałby poszerzyć i~uaktualnić swoją wiedzę na temat wszechświata, to \textit{Mała księga kosmologii} Lymana Page'a jest pozycją wartą polecenia. Jest to relacja badacza, który bezpośrednio uczestniczy w~obecnym rozkwicie kosmologii, i~który potrafi podzielić się wiadomościami z~czytelnikiem. Page twardo stąpając po Ziemi na każdym kroku przypomina o~zasadniczej roli obserwacji i~pomiarów. Popularyzuje wiedzę pewną, wyraźnie rozdzielając to co wiadome, od tego co ciągle wymaga potwierdzenia. Ascetyczny styl nie przesłania zachwytu autora nad skutecznością metody naukowej. Choć wiele pytań pozostaje otwartych, to dzięki tej metodzie stworzyliśmy potężny i~predykcyjny model wszechświata. Jak zauważa Page: ,,Nie musiało tak być, ale natura okazała się łaskawa, pozwalając nam dowiedzieć się tak wiele''.

\nocite{page_mala_2021}

%-------------------------


\selectlanguage{english}
\vspace{5mm}%
\begin{flushright}
{\chaptitleeng\color{black!50}{The little book about the large universe}}
\end{flushright}

%\vspace{10mm}%
{\subsubsectit{\hfill Abstract}}\\
{We live in extraordinary times for cosmologists. A vast amount of new astronomical data is pushing our model of the universe to its limits. An interest in cosmology is growing. \textit{The Little Book of Cosmology} by Lyman Page offers a concise, up-to-date, and comprehensible introduction to the subject.
}\par%
\vspace{2mm}%
{\subsubsectit{\hfill Keywords}}\\%
{
cosmology, universe, cosmos, astronomy, physics, general relativity, cosmic microwave background, background radiation, CMB, WMAP, Albert Einstein.
}%

\selectlanguage{polish}

\end{newrevplenv}
