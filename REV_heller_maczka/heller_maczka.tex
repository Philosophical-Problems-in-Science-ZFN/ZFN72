\begin{newrevplenv2auth}{Michał Heller, Janusz Mączka}
	{Idee i~ideaty}
	{Idee i~ideaty}
	{Idee i~ideaty}
	{\flushright\subbold{Michał Heller}\\\subsubsectit\small{Centrum Kopernika Badań Interdyscyplinarnych}\par
	\flushright\subbold{Janusz Mączka}\\\subsubsectit\small{Uniwersytet Papieski Jana Pawła II w Krakowie}\par}
	{Józef Życiński, \textit{Struktura rewolucji metanaukowej. Studium rozwoju współczesnej nauki}, tłum. Michał Furman, Copernicus Center Press,
	\\Kraków 2013.}




\lettrine[loversize=0.13,lines=2,lraise=-0.01,nindent=0em,findent=0.2pt]%
{K}{}onferencja naukowa zwołana w~2021 roku w ramach obchodów dziesiątej rocznicy śmierci abpa Józefa Życińskiego nie tylko stworzyła okazję, lecz również zachęciła do ponownej lektury jego najważniejszych książek. Niewątpliwie należy do nich \textit{Struktura rewolucji metanaukowej}
%\label{ref:RNDhgBmBpe50H}(Życiński, 2013).
\parencite[][]{zycinski_struktura_2013}. %
 Oryginalnie została ona opublikowana w~języku angielskim 
%\label{ref:RNDSmpfCuQxPD}(Życiński, 1988),
\parencite[][]{zycinski_structure_1988}, %
 ale w~ostatnich latach znów stała się przedmiotem rozważań w~związku z~przetłumaczeniem jej na język polski 
%\label{ref:RNDyd473cX8Di}(zob. np. Liana, 2019, 2020).
\parencites[zob. np.][]{liana_nauka_2019}{liana_jozefa_2020}. %
 Książka ta zawiera przemyślenia Józefa Życińskiego dotyczące filozofii nauki istotne dla całego jego stylu filozofowania. Zostało w~niej także zapoczątkowanych szereg wątków, które i~dziś warto rozwijać. Niektóre z~nich przybliżymy w~tym niewielkim opracowaniu.

Tytuł tej książki, choć nawiązuje do znanej książki Thomasa Kuhna \textit{Struktura rewolucji naukowych}
%\label{ref:RNDOqtN4eO7Oa}(Kuhn, 1968),
\parencite[][]{kuhn_struktura_1968}, %
 nie jest jednak jego kopią. Różnica z~tytułem książki Kuhna sprowadza się do dwóch drobnych zmian, które jednak wprowadzają istotną różnicę. Te dwie zmiany to: mały przedrostek ,,meta'' i~liczba pojedyncza w~określeniu ,,metanaukowa rewolucja''. Przedrostek ,,meta'' określa perspektywę Życińskiego, która jest odmienna od perspektywy Kuhna. Kuhn poddawał analizie rewolucje, które dokonywały się w~nauce i~które powodowały istotne zmiany w~jej pojmowaniu. Przykładami takich rewolucji są przejście od nauki średniowiecznej do nowożytnej oraz przejście od fizyki klasycznej do mechaniki kwantowej i~relatywistycznej. Poddając analizie te i~inne rewolucje naukowe, Kuhn podważał obowiązujące dotychczas w~filozofii nauki poglądy, wiązane zwykle z~nazwiskiem Karla Poppera i~filozofią nauki Koła Wiedeńskiego. O~ile Popper i~neopozytywiści wiedeńscy dokonywali analiz naukowych przy użyciu narzędzi logicznych i~metodologicznych, o~tyle Kuhn posłużył się metodą analizy historycznej i~socjologicznej. Propozycja Kuhna odbiła się szerokim echem i~zapoczątkowała ciąg prac, których autorzy prześcigali się w~wynajdywaniu coraz to nowych ujęć rozwoju nauki. Zagadnienia te są zbyt dobrze znane, by je tutaj referować.

Oryginalne spojrzenie Życińskiego sprowadza się do spostrzeżenia -- głównie dzięki książce Kuhna -- iż niemal na naszych oczach dokonała się wielka rewolucja, ale nie w~nauce, lecz w~filozofii nauki, czyli rewolucja metanaukowa. Na niej to właśnie skupia się Życiński w~swojej książce. Warto zauważyć, że w~trakcie pisania książki rewolucja jeszcze się dokonywała. Autor dostrzega, że rewolucji metanaukowych w~historii nauki było więcej i~poświęca im nieco uwagi. Wspomina o~zmianach w~poglądach na rozumienie świata w~starożytnej Grecji, w~chrześcijańskim średniowieczu i~w renesansie.

W~trakcie dokonywania się rewolucji naukowych następuje stopniowa zmiana przedzałożeń, które obowiązywały w~dotychczasowym okresie. Zbiór twierdzeń filozoficznych, będących częścią przedzałożeń, często określa się mianem ,,ideologii''. Życiński pisze: ,,Zbiór twierdzeń ideologicznych uzależnionych od konkretnych interpretacji w~naukach przyrodniczych nie jest jednorodny epistemologicznie. Zawiera zasady metodologiczne oraz założenia epistemologiczne, heurystyczne wskazówki i~twierdzenia o~charakterze ontologicznym. Te same terminy, które tworzą «ideologiczne» twierdzenia, mogą denotować rzeczywistość, która jest odmienna w~innych programach badawczych''
%\label{ref:RNDIactpjxMjW}(Życiński, 2013, s.~33).
\parencite[][s.~33]{zycinski_struktura_2013}. %
 Celem ukonkretnienia swoich analiz, Życiński proponuje ,,pozanaukowe składniki teorii naukowych'' nazwać \textit{ideatami}. Dalej czytamy: ,,Ideaty są zarówno pojedynczymi, podstawowymi terminami w~konkretnych interpretacjach, na przykład atom, ewolucja czy stworzenie, jak i~asercjami inspirującymi określony sposób wyrażania rzeczywistości oraz wykonywania podstawowych funkcji metodami charakterystycznymi dla danych naukowych programów badawczych'' 
%\label{ref:RNDkY6huw8J2J}(Życiński, 2013, s.~34).
\parencite[][s.~34]{zycinski_struktura_2013}. %
 Według Życińskiego, ideaty ,,inspirują poszczególne programy badawcze oraz współtworzą ich twardy rdzeń'' 
%\label{ref:RNDKrR7So0QDp}(Życiński, 2013, s.~34).
\parencite[][s.~34]{zycinski_struktura_2013}. %
 Jednakże według niego, nie wszystkie pozanaukowe elementy, związane z~uprawianiem nauki, zasługują na miano ideatów. Niektóre z~nich są przypadkowymi wtrętami bez większego wpływu na całość programu badawczego.

Życiński wymienia ideaty o~charakterze ontologicznym, epistemologicznym i~metodologicznym. Tego rodzaju ideaty nie muszą wynikać z~jakichś głębszych przemyśleń filozoficznych; często ich treść ,,zależy wyłącznie od socjologicznych i~psychologicznych uwarunkowań, a~nie od bardziej podstawowej ontologii''
%\label{ref:RNDFdRToHVU8N}(Życiński, 2013, s.~51).
\parencite[][s.~51]{zycinski_struktura_2013}.%


Pojęcie ideatu, obok samej idei rewolucji metanaukowej, należy uznać za najbardziej oryginalny pomysł w~tej książce. Życiński przyznaje, że rola ideatów jest ,,taka sama'' jak rola tzw. \textit{themata} opisanych przez G. Holtona w~jego \textit{Thematic Origins of Scientific Thought}
%\footnotetext{ G. Holton, \textit{Thematic Origins of Scientific Thought: Kepler to Einstein}, Cambridge 1975.}
 \parencite*{holton_1988}.
Ponieważ jednak Życiński podał jedynie dość ogólnikowy zarys tego co powinno się rozumieć pod pojęciem ideatu, spróbujmy poszukać w~jego książce przykładów ideatów, na które się powołuje. Poniższa lista nie jest listą kompletną, ale daje ona wyobrażenie o~tym, co Życiński rozumiał pod pojęciem ideatu.

\begin{itemize}
\item I~tak zaraz po wprowadzeniu pojęcia ideatu, Życiński pisze o~ideatach bliskich filozofii Arystotelesa, które ,,kładą szczególny nacisk na kryteria zdrowego rozsądku i~racjonalności''
%\label{ref:RNDZzsuPriauM}(Życiński, 2013, s.~35).
\parencite[][s.~35]{zycinski_struktura_2013}.%

\item Natomiast ideatami bliskimi filozofii Platona są, według niego, ,,piękno teorii, jej prostota, symetria i~tym podobne''
%\label{ref:RNDARpSuhW9WM}(Życiński, 2013, s.~36).
(s.~36).

\item Życiński mówi również o~,,Euklidesowskich ideatach logicyzmu'', które ,,inspirowały nowe próby coraz bardziej wyrafinowanych poszukiwań głębszych podstaw'' (s.~167).
\item Wspomina także o~,,racjonalistycznych ideatach Euklidesa i~Kartezjusza'' (s.~183).
\item Ideaty funkcjonują również w~nauce nowożytnej. Na przykład: ,,W ramach dziewiętnastowiecznej nauki podobny racjonalistyczny redukcjonizm wyrażał się w~próbach wyjaśniania wszystkich procesów w~kategoriach mechaniki. Po upadku mechanicyzmu ideaty te przeniknęły w~obszar badań meta naukowych'' (s.~328).
\item Wedle Życińskiego istnieją ideaty, wykazujące niezwykłą trwałość na poziomie meta ,,niezależnie od lokalnych nieciągłości spowodowanych rewolucjami naukowymi''. ,,Jednym z~nich jest powszechnie rozumiany ideat racjonalności, którego różne warianty występują na różnych etapach rozwoju nauki'' (s.~334).
\item ,,Innym ważnym ideatem wyrażającym racjonalizm metanaukowy był ideat poznawalności, wyrażający przekonanie, że możliwe jest poznanie świata za pomocą środków dostępnych człowiekowi'' (s.~334).
\item Jeszcze innym ideatem jest ,,ideat pozytywizmu metodologicznego, który miał decydujący wpływ na sformułowanie zasad metodologicznych współczesnej nauki'' (s.~335--336).
\item Z~zasadami metodologicznymi wiąże się także ,,ideat istnienia w~nauce różnorodnych ograniczeń, które zostały dowiedzione na podstawie przesłanek dostarczonych przez nią samą'' (s.~336).
\item Zaprzeczenie ograniczeń poznania naukowego prowadzi do ,,ideatu totalizmu naukowego'', który wyraża slogan ,,Nie ma zbawienia poza nauką'' (s.~337).
\item Wreszcie Życiński zwięźle omawia ,,ideat unifikacji'', który przybierał różne formy: Od ,,poszukiwania jednej formuły stosowalnej do wszystkich zjawisk, typowej dla magii'', poprzez redukowanie wszystkiego do mechaniki, aż do ,,programu logicystycznego, w~którym cała matematyka miała zostać zredukowana do małej liczby aksjomatów i~reguł przekształcania'' (s.~340--341).
\end{itemize}
Należy przestrzec Czytelnika, że lista ta nie jest w~żadnym sensie systematyczna. Życiński wymienia te ideaty przy okazji omawiania innych tematów, zakładając, że pojęcie ideatu jest już znane. Jak widać z~tej listy, Życiński pojęcie ideatu rozumie szeroko i~w sposób rozmyty. Przypuszczalnie nie jest to efekt niezamierzony, gdyż, jak z~tych przykładów wynika, ideaty działają w~znacznej mierze dzięki swojemu niedookreśleniu, sterując niejako z~ukrycia sposobami myślenia w~danej epoce. Co więcej, ten sam ideat w~różnych epokach może przybierać inne treści zależnie od tego, w~jaki sposób jest wykorzystywany. Ideaty nie tylko kształtują myślenie w~danej epoce, lecz są również kształtowane przez to myślenie.

Życiński wspomina także ,,o niekrytycznych ideatach''. Pozytywizm i~scjentyzm XIX w. stworzył przekonanie, ,,że wyplenienie ideologicznych wtrętów z~nauki było tylko kwestią czasu''
%\label{ref:RND4x4FO1Ze0O}(Życiński, 2013, s.~49);
\parencite[][s.~49]{zycinski_struktura_2013}; %
 wkrótce jednak okazało się, że ,,znacznie łatwiej było krytykować ideologiczne programy badawcze szkodzące nauce w~przeszłości, niż unikać niekrytycznych ideatów w~nowych programach badawczych'' 
%\label{ref:RNDWCJ3bexo0Z}(Życiński, 2013, s.~49).
\parencite[][s.~49]{zycinski_struktura_2013}. %
 Tego rodzaju niekrytyczne ideaty niekiedy z~warstwy meta wpływają na ewolucję teorii naukowych. Próby ich wyeliminowania ,,prowadziły do wprowadzenia nowych, ukrytych ideatów, które bez uzasadnienia były podnoszone do rangi twierdzeń naukowych'' 
%\label{ref:RNDuRzQRCam2S}(Życiński, 2013, s.~59–60).
\parencite[][s.~59–60]{zycinski_struktura_2013}.%


W~kolejnych rozdziałach Życiński piętnuje nadużycia ideologiczne mechanicyzmu (rozdz. 2) i~fizykalizmu (rozdz. 3). W~obydwu tych rozdziałach pojawiają się wątki historyczne, ale historia jest traktowana bądź jako ilustracja tez stawianych przez Życińskiego, bądź w~charakterze argumentów przeciwko tezom zwalczanym przez niego. Należy pamiętać o~czasie, w~którym Życiński pisał tę książkę. Wówczas twierdzenia pozytywizmu logicznego były ciągle żywe, a~jedną z~cech pisarstwa Życińskiego jest to, że zawsze był wrażliwy na głos współczesności. Niejednokrotnie ta wrażliwość zamieniała się w~styl dziennikarski i~polemiczny i~często przejawia się to w~tej książce. Na przykład, polemizując ze zbyt pospiesznymi próbami ,,sformułowania uniwersalnej teorii rewolucji naukowych na podstawie wiedzy o~rewolucjach Galileusza -- Newtona oraz Einsteina -- Plancka''
%\label{ref:RNDVWSNb910YN}(Życiński, 2013, s.~45),
\parencite[][s.~45]{zycinski_struktura_2013}, %
 Życiński porównuje je do ,,opracowania teorii społeczeństwa na podstawie obserwacji dwóch rencistów'' 
%\label{ref:RNDZ57h120s7T}(Życiński, 2013, s.~45).
\parencite[][s.~45]{zycinski_struktura_2013}. %
 Albo polemizując z~dziewiętnastowiecznym materializmem, pisze: ,,w powyższym kontekście sama idea ‘obiektów matematycznych' wydaje się jedynie tytułem honorowym używanym grzecznościowo lub przez sentyment do tradycyjnej terminologii'' 
%\label{ref:RNDU2L2ZTuJ1y}(Życiński, 2013, s.~138).
\parencite[][s.~138]{zycinski_struktura_2013}. %
 Czytając tę książkę, niekiedy trudno oprzeć się wrażeniu, że jeżeli tylko nadarzy się okazja użycia zgrabnego sformułowania lub ciętej riposty, Życiński natychmiast ulegnie pokusie.

Życińskiego koncepcja rewolucji metanaukowej wyrosła z~jego fascynacji rewolucją, jaka dokonała się w~podstawach matematyki w~pierwszych dekadach dwudziestego wieku. Jego metodologiczna uwaga jest głównie skierowana na twierdzenia Gödla i~inne twierdzenia limitacyjne. Twierdzenia te zburzyły dotychczasowe przekonanie o~niewzruszalności podstaw matematyki i~tym samym przyczyniły się do zmiany paradygmatu w~matematyce z~\textit{episteme} na \textit{doxa}. Pojęcia te, nawiązujące do poglądów Platona i~Arystotelesa, odgrywają istotną rolę w~analizach Życińskiego. Przejście od \textit{episteme} do \textit{doxa} uważa on za główną charakterystykę rewolucji metanaukowej, jaka dokonała się XX wieku. Naukę typu \textit{episteme} charakteryzuje dążenie do sformułowania niepodważalnych podstaw dla wszelkich nauk oraz poszukiwanie uniwersalnej metody. Naukę typu \textit{doxa} charakteryzuje świadomość ograniczeń metodologicznych i~zadowalanie się argumentami probabilistycznymi, gdy inne są nieosiągalne. Przejście od \textit{episteme} do \textit{doxa} nie oznacza jednak rezygnacji z~racjonalności, lecz jedynie dostosowanie racjonalności do realnych możliwości metody. Temu wątkowi poświęcony jest rozdział 5, w~którym Życiński piętnuje poglądy radykalnego socjologizmu w~filozofii nauki, głoszącego koniec dotychczasowej racjonalności i~dominację pozaracjonalnych składników nauki.

Ważnym spostrzeżeniem książki jest to, że istotne przemiany dokonują się na poziomie meta, a~więc w~metarewolucji. Czy działają tam ideaty, czy \textit{themata}, czy metaparydagmaty, to już mniej istotne. O~tym, czy jakaś zmiana w~nauce zasługuje na miano rewolucji, decyduje to, czy przemiany, którym uległa, sięgały również do poziomu meta. Dlatego też akcja książki Życińskiego rozgrywa się głównie na tym poziomie, jednakże poziom meta nieustannie oddziaływuje z~tym, co dzieje się w~samej nauce.

Ponieważ Życiński szuka filozoficznych uzasadnień tego, co dzieje się w~nauce, na poziomie meta, książka miejscami sprawia wrażenie, jakby była opracowaniem z~historii nauki, ale historii nauki w~specyficznym wydaniu, gdyż nie jest to systematyczny wykład dziejów nauki, lecz raczej ,,próbkowanie'' historii licznymi przykładami. Jednak zbliżając się w~lekturze ku końcowi, czytelnik ma ogląd ,,całości'' dziejów nauki (choć jest to ogląd dość ogólny) od czasów starożytnych do polemik Życińskiego ze współczesnymi autorami.

Do książki Życińskiego warto powracać. Zawiera ona wiele ważkich myśli, które do dziś nie utraciły swojej doniosłości. Wśród tworów ludzkiej myśli trudno wskazać obszar, który zmieniałby się w~sposób bardziej dynamiczny niż właśnie nauka, ale spojrzenie na nią z~perspektywy meta pozwala uchwycić te jej cechy, które działają w~znacznie szerszej czasowej skali.




%-------------------------


\selectlanguage{english}
\vspace{5mm}%
\begin{flushright}
{\chaptitleeng\color{black!50}{Ideals and ideats}}
\end{flushright}

%\vspace{10mm}%
{\subsubsectit{\hfill Abstract}}\\
{
The original view of Joseph Życiński, presented in his book \textit{The Structure of the Metascientific Revolution} \parencite*[][]{zycinski_structure_1988}, boils down to the observation that almost before our eyes a great revolution took place, not in science, but in the philosophy of science, that is the meta-scientific revolution. His concept of the meta-scientific revolution grew out of his fascination with the revolution that took place in the foundations of mathematics in the first decades of the twentieth century. Whether a change in science deserves to be called a revolution is determined by whether the transformations it underwent also reached the meta-level. The set of presuppositions underlying transformations on the meta-level Życiński calls ideata. One of the aims of this article is to critically reconstruct the meaning of this term.

The action of Życiński's book takes place mainly on meta-level, but the meta-level constantly interacts with what is happening in science itself. The book sometimes makes an impression as if it were a study of the history of science, but history of science in a specific sense – something like a ``sampling'' of history with numerous examples. Among the creations of human thought, it is difficult to point to an area that changes more dynamically than science itself, but looking at it from a meta-perspective allows us to grasp those of its features that operate on a much broader scale.
}\par%
\vspace{2mm}%
{\subsubsectit{\hfill Keywords}}\\%
{
scientific revolution, meta-scientific revolution, ideats, themata, \textit{episteme}, \textit{doxa}.
}%

\selectlanguage{polish}

\end{newrevplenv2auth}
