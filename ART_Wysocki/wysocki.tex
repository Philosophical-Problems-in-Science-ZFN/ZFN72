\begin{artengenv}{Igor Wysocki}
	{The problem of indifference and homogeneity in Austrian economics: Nozick's challenge revisited}
	{The problem of indifference and homogeneity in Austrian economics\ldots}
	{The problem of indifference and homogeneity in Austrian economics: Nozick's challenge revisited}
	{Nicolaus Copernicus University}
	{The pivotal point in the Austrian literature on homogeneity, choice and indifference was constituted by Nozick's \textit{On Austrian Methodology}. Nozick provoked a~long debate on the above notions within Austrianism. The aim of this paper is to elaborate such an account of \textit{homogeneity} that would take the sting out of Nozick's challenge and allow for non-trivial formulation of the law of diminishing marginal utility. Hence, we shall first take a~closer look at the debate on indifference within the Austrian camp, while defending and building upon the Hoppean account vis-à-vis Block's criticism. Our justification of the Hoppean position shall consist in showing that his account of the correct description of an action is not an \textit{ad hoc} move aimed at solving just one problem of indifference but is highly intuitive and widely applicable. We conclude by restating the above-mentioned law, thus demonstrating that the Nozickian objection can be successfully addressed.}
	{indifference, choice, homogeneity, Nozick's challenge.}



\section{Introduction }
\lettrine[loversize=0.13,lines=2,lraise=-0.01,nindent=0em,findent=0.2pt]%
{I}{}n 1977, Nozick wrote his seminal paper \textit{On Austrian Methodology}
%\label{ref:RNDps0Zfg5ToD}(Nozick, 1977)
\parencite[][]{nozick_austrian_1977} %
 thereby levelling a~challenge at the entire Austrian school of economics. Nozick's critique pertained to all sorts of claims Austrians made, ranging from acting on strict preference vs weak preference, through the doctrine of sunk cost and Austrian avowed apriorism, to their theory of time preference. However, it was Nozick's claim\footnote{This claim was even dubbed as Nozick's challenge by Hudik 
%\label{ref:RNDxYMAZoNHH3}(2011).
\parencite*[][]{hudik_note_2011}.%
} that, logically speaking, Austrian's formulation of the law of diminishing marginal utility must rely on the notion of indifference that stirred a~long-lasting and still inconclusive debate on the role of the said concept in Austrian economics. This indictment by Nozick 
%\label{ref:RNDJay6TeejGy}(1977, pp.370–371)
\parencite*[][pp.370–371]{nozick_austrian_1977} %
 is so important for the entire Austrian edifice\footnote{As we are about to see, it is the very formulation of the law of diminishing marginal utility and the universal law of time preference that logically depends on the notion of indifference.} that it merits being quoted in full:

\myquote{
Indeed, the Austrian theorists need the notion of indifference to explain and mark off the notion of a~commodity, and of a~unit of a~commodity. If everyone or one person prefers one homogenous batch of stuff to another homogenous batch of the same shape of the same stuff (perhaps they like to choose the left-hand one, or the one mined first), these are not the same commodity. They will have different prices. Particular things $x$ and $y$ will be the same commodity (belong to the same commodity class) only if all persons are indifferent between $x$ and $y$. Without the notion of indifference, and, hence, of an equivalence class of things, we cannot have the notion of a~commodity, or of a~unit of a~commodity; without the notion of a~unit (‘an interchangeable unit') of a~commodity, we have no way to state the law of (diminishing) marginal utility.
}
Nozick's view is that the statement of the law of diminishing marginal utility presupposes the employment of the notion of indifference (or the one of the same commodity\footnote{The relation between the two is to be probed later. }). Therefore, it seems that homogeneity of economic goods is no mere epiphenomenon playing no role in praxeology as such. Quite the contrary, the notion of indifference is purportedly \textit{essential} for understanding the logic of human action. To appreciate the disagreement between Nozick and Austrians, it would suffice to realize that the Austrian dogma---adhered to by some of its most prominent figures, e.g. Rothbard and Block themselves---was that indifference is praxeologically irrelevant, and as such it cannot make any difference to human action. Or, in other words, man does not act on indifference. More specifically, Austrians of orthodox persuasion (Rothbard and Block\footnote{Block
%\label{ref:RNDaOh29qMnMU}(1980, pp.424–425, 1999, pp.22–24)
\parencite*[][pp.22–24]{block_austrian_1999} %
 conceives of indifference as ``vague, psychological category''.}) relegate indifference to the realm of mere psychology. Those insights were most tellingly captured by Rothbard 
%\label{ref:RNDkgop4g9o4w}(Rothbard, 2011, p.304)
\parencite*[][p.304]{rothbard_toward_2011} %
 in the following passage: ``Indifference can never be demonstrated by action. Quite the contrary. Every action necessarily signifies a~\textit{choice}, and every choice signifies a~definite preference. Action specifically implies the \textit{contrary} of indifference. The indifference concept is a~particularly unfortunate example of the psychologizing error.'' It is also Hoppe 
%\label{ref:RNDxufvb801ZV}(2005)
\parencite*[][]{hoppe_must_2005} %
 that wants to banish the concept of indifference out of the realm of human action. And it is precisely this relation of logical equivalence between indifference and no choice that constitutes the crux of the Hoppean 
%\label{ref:RNDou0qNn8Xuh}(2005)
\parencite*[][]{hoppe_must_2005} %
 solution. To put his point in still different terms, suppose we take a~set of various economic means and the equivalence relation (that of indifference) on this set. The relation of indifference would divide our original set into mutually disjunctive equivalence classes, which means that all the units in each class are the units between which the economic actor is indifferent. This in turn would mean that the actor cannot choose between the units within those equivalence classes. Conversely, if he can choose between any two units, these units must belong to distinct equivalence classes. Incidentally, we will revisit the original Hoppean solution when defending his account vis-à-vis Block's criticism in the forthcoming part of the present paper.

However, let us not precipitate things at that introductory stage. Instead, let us conclude the present section by setting the agenda for what is to follow. The present paper proceeds in this manner: section 2 takes the task of elucidating such critical concepts as \textit{indifference}, \textit{homogeneity} and \textit{same good}, with the relations between them being heeded too. Section 3 is dedicated to the detailed analysis of Nozick's challenge concerning the alleged necessity of the adoption of the concept of indifference within Austrianism. The burden of section 4 is to show the inadequacies of Block's attempt to deal with Nozick's problem. Section 5 tries to show the superiority of the Hoppean account of choice and indifference over Block's, while at the same time dispelling a~possible objection to the effect that the Hoppean solution is simply an ad hoc conceptual move designed to obviate one particular problem. Section 6 tries to extend Hoppe's conceptual framework in order to reformulate the law of diminishing marginal utility in a~way that apparently obviates Nozick's challenge. Section 7 concludes the paper.

\section{Mapping the conceptual terrain}
Before we move on to analyze the Nozick's challenge and to subsequently revisit the debate on indifference, as it unfolded within the Austrian camp, we need to clarify some critical concepts and straighten out possible misconceptions.

First and foremost, there is a~subtle terminological distinction that needs elucidating. So far, we have been using the words \textit{indifference}, \textit{homogeneity} or \textit{same commodity} (or \textit{same good} for that matter) in a~rather cavalier fashion and the inquisitive reader might wonder whether we treat them synonymously or there are possibly some more interesting relations between them. First, let us note that Austrians, as pretty much all economists, are concerned not with things (or physical objects) as such but with economic goods and the latter are only in the eye of a~beholder. The relation between things (the ones being able to satisfy potential human needs) and economic goods is that of inclusion. In other words, all economic goods are physical objects but not vice versa. What it takes then for a~physical object to count as an economic good is that it must be valued positively; or, it must be \textit{believed}\footnote{This very caveat related to the actor's \textit{beliefs} is of utmost importance here. It is because we maintain that Austrians, with their commitment to radical subjectivism, ought to reject the Mengerian
%\label{ref:RND2O00O2Xfht}(2007, p.52)
\parencite*[][p.52]{menger_principles_2007} %
 contention that a~thing can be ranked as a~good only when it has ``such properties as render the thing capable of being brought into a~causal connection with the satisfaction of this need.'' This is too strong a~requirement. If I~deal with some units that I~believe (even if falsely) satisfy the same list of ends, I~would be inclined to price them identically. Moreover, I~would believe that giving up one unit of this apparent supply would mean the resignation from the satisfaction of the least pressing need they are all believed to satisfy. Therefore, I~should not have any preference for giving up any particular (marginal) unit over any other. We contend that these implications are sufficient to find such units the ones of the same good. Incidentally, this radical subjectivism reflected in conceiving of things as means, that is in the contention that the sufficient condition for a~thing to become a~means is that the economic actor \textit{must merely believe} that it can serve his end, was aptly captured by Mises 
%\label{ref:RNDU4l21zwKmV}(1998, p.92):
\parencite*[][p.92]{mises_human_1998}: %
 ``Goods, commodities, and wealth and all the other notions of conduct are not elements of nature; they are elements of human meaning and conduct. He who wants to deal with them must not look at the external world; he must search for them in the meaning of acting men.''}
to be able to satisfy an \textit{actual human need} (or to use the parlance of neoclassicals: it must have positive utility). Therefore, economists are concerned with \textit{only this} subset of things which are economic goods. And for a~thing to constitute an economic good, what it takes is at least one economic actor that believes (falsely or not) that the physical object in question is able to satisfy at least one of his actual needs. Incidentally, note that given Austrian extreme subjectivism, no case can be made for any entailment between physical sameness and indifference (economic sameness). Machaj's
%\label{ref:RNDLusJVHw4CZ}(2007, p.232)
\parencite*[][p.232]{machaj_praxeological_2007} %
 celebrated example was that a~ring (of a~specific physical constitution) on one's fiancée's finger is not \textit{economically} identical with a~physically identical ring ``given to her by a~total stranger on the street.'' So, it looks as though physical sameness does not entail economic sameness. In other words, even if $x$~and $y$~are physically indistinguishable, $x$~and $y$~do not \textit{necessarily} constitute \textit{economically} homogeneous units. Or to use a~different jargon referring to \textit{the same fact}, even if $x$~and $y$~are physically identical (down to the level of particles), there may be some actor who may not be indifferent between the two. Instead, he may strictly prefer one over the other.\footnote{In other words, numerical identity might matter even in the absence of any qualitative differences between two objects.} Nor does indifference entail physical sameness. This statement is even more incontrovertible for it is readily imaginable that two units are \textit{slightly} physicallydifferent and yet, this difference cannot translate (by the lights of the economic actor) into an economic difference. In fact, we do not need to be so cautious with our examples here once we subscribe to the view that for units $x$~and $y$~to be subsumable under the rubric of the same economic good, it is enough that they are \textit{believed} to be equally serviceable in the eye of the economic actor doing the valuation. Suppose our actor believes (correctly or not) that---relative to his needs---an apple juice and mineral water are equally good; that is, he believes that there is no such end that an apple juice would satisfy but mineral water would not and vice versa. Granted, there are \textit{actually} many non-overlapping needs that apple juice and mineral water can satisfy but why should the economic actor care about it. These may not figure in his value scales either by virtue of the fact that the actor is unaware of these possible services the two goods might render or he might not value them at all. Such an actor would be prone to regarding apple juice and mineral water as \textit{economically} indistinguishable. If he were forced to give up a~unit of apple juice or the one of mineral water, he would be indifferent between the two. And crucially, given his beliefs, he would price them equally. Having established that physical sameness is logically independent of economic sameness, what is still left to explain is the relation between \textit{indifference} and \textit{homogeneity}. Here, following a~common parlance, we submit that one would be ill-advised to treat them synonymously. It appears to be intuitively clear that \textit{homogeneity} is a~relation between \textit{economic goods}, whereas \textit{indifference} is a~mental state (a belief) of an actor. Specifically, \textit{indifference} has such a~propositional content (believing \textit{that} $x$~and $y$ are economically identical) that it cannot motivate an actor to act on it. By contrast, \textit{homogeneity} is a~relation holding between \textit{economic goods}. However, remember that \textit{economic goods} are not mere physical goods. The former are in the eyes of an economic actor.\footnote{There are mind-boggling complications involved in counting the number of economic goods supervening on physical objects. Let us take the Rothbardian 
%\label{ref:RNDVGFaM8mwaJ}(2009, pp.73–74)
\parencite*[][pp.73–74]{rothbard_man_2009} %
 example with eggs and modify it slightly to illustrate our point. Suppose we have 3 eggs and let 1 egg serve the end of throwing it at our enemy's window (we have 3 of them, so we can throw one egg at one enemy's window). With 2 eggs we might already prepare scrambled eggs (which is our second most valued end) and with 3 eggs we might make an omelet. How many economic goods (given our value scale) do we have having 3 eggs? It looks as if we already have 3 of them since we can use all of them in order to annoy our enemies. Additionally, we have 3 distinct 2-egg combinations to make scrambled eggs (although there seems to be one \textit{type} of economic good here, with the marginal unit being \textit{a}~2-egg combination). So, do we already have 6 economic goods or four of them? The unit of 3 eggs put together constitutes a~separate economic good for it is only that large a~marginal unit that allows us to prepare an omelet. So in the end, how many economic goods do we have? Seven of them? The difficulty in counting seems to consist in two problems: a) do we count token economic goods or types of economic goods? and b) should we, when counting, add up \textit{all} marginal units (1 egg, 2 eggs and 3 eggs)? After all, all these units are not \textit{jointly} possible. In fact, if we decide to use all of our eggs to make an omelet, all other ends cannot be satisfied (no throwing eggs at windows, nor preparing scrambled eggs) as we would be left with no eggs. All in all, for the time being, we prefer to remain agnostic on these issues although they definitely merit a~separate paper.} So, the question arises: under what conditions would two physical units constitute the same economic goods? The answer seems all too obvious: only when an actor is indifferent between the two. So, the relation of equivalence appears to hold between an actor being indifferent between physical units $x$~and $y$~and these units being economically \textit{homogeneous}. In other words, when an actor is indifferent between physical units $x$~and $y$, this fact \textit{entails} that $x$~and $y$~are exemplary of \textit{the same commodity} (\textit{the same economic good} or \textit{economic homogeneity}). And vice versa, when physical units $x$~and $y$~are \textit{economically homogeneous}, this fact \textit{entails} that there is an economic actor who would be indifferent between these two. Note that Nozick's requirement for the same commodity is too strong. He demands that ``all persons are indifferent between $x$ and $y$.'' This however, given Austrian subjectivism, would be a~massive coincidence. To settle the issue that physical units $x$~and $y$ are a~part of the same supply, it would take establishing that literally all the persons are indifferent between the two---the sheer impossibility. Instead, Austrian economists must perceive the same supply as \textit{relative} to a~given economic actor. Physical objects $x$~and $y$~might be considered the same economic good by person A~but person B~might as well consider them economically distinct. What is more, person C~might find them both economic bads. Having said that, let us now proceed to interpret what putative formidability of Nozick's challenge consists in.

\section{The analysis of Nozick's challenge}
First and foremost, Nozick's challenge may be construed as a~\textit{purely logical} objection to Austrian repudiation of indifference. After all, remember, the gist of Nozick's objection was that without the concept of indifference, Austrians would be unable to formulate the law of diminishing marginal utility. Indubitably, in this respect Nozick is right. The law of diminishing marginal utility\footnote{Rothbard
%\label{ref:RNDKAEMqS2WIL}(2009, p.25)
\parencite*[][p.25]{rothbard_man_2009} %
 states the law of marginal utility very succinctly and very clearly indeed. His definition assumes the following form: ``Thus, for all human actions, as the quantity of the supply (stock) of a~good increases, the utility (value) of each additional unit decreases.''} has it that when we deal with a~supply of economically same units, each additional unit we value less; or, in other words, each additional unit is of lower utility. To put it formally, n+1\textsuperscript{th} unit of a~given supply is of lower utility than n\textsuperscript{th} unit. And conversely, n-1\textsuperscript{th} unit of a~given supply is of higher utility than n\textsuperscript{th} unit thereof. This in turn means that the utility of the marginal unit in a~smaller supply is higher than the utility of a~marginal unit in a~bigger supply of the same commodity.\footnote{In fact, this is the reason why Rothbard 
%\label{ref:RNDbP6IDCTIQ8}(2009, pp.21–23)
\parencite*[][pp.21–23]{rothbard_man_2009} %
 speaks of \textit{the law of marginal utility} instead of the law of \textit{diminishing} marginal utility. After all, as shown above, marginal utility may increase once the supply of a~good shrinks.} Hence, Nozick correctly notes that this law craves for an \textit{independent} understanding of the notion of homogeneity. For suppose, an Austrian proponent objects that there is no such logical requirement and that the notion in question may be defined \textit{within} the very law somehow along these lines: we can easily establish whether $x$, $y$ and $z$~are units of the same economic good and we would do so by checking whether these units obey the law of diminishing marginal utility. So, generally speaking, what an Austrian adherent would effectively say is that units of the same good are such units that obey the law of diminishing marginal utility. Incidentally, similar remarks would apply to Austrian formulation of the universal law of time preference.\footnote{Then again, we believe there is no clearer exposition of the said law than the following passage from Rothbard 
%\label{ref:RNDbHMVsaefRE}(2011, p.15):
\parencite*[][p.15]{rothbard_toward_2011}: %
 ``A fundamental and constant truth about human action is that \textit{man prefers his end to be achieved in the shortest possible time.} Given this specific satisfaction, the sooner it arrives, the better. This results from the fact that time is always scarce, and a~means to be economized. The sooner any end is attained, the better. Thus, with any \textit{given end} to be attained, the shorter the period of action, i.e., production, the more preferable for the actor. \textit{This is the universal fact of time preference}.''} Austrians hold that for one (and the same!) end,\footnote{Note that since we value means instrumentally (that is only as much as they contribute to the satisfaction of our ends), the universal law of time preference must \textit{derivatively} apply also to means. Since for any given end, we would rather achieve it sooner rather than later, we \textit{must} also prefer to employ (or come into possession of them) necessary means sooner rather than later.} each actor would prefer to achieve it sooner rather than later. Note, this law also presupposes the notion of the same good---but this time in a~sort of \textit{atemporal} way for it is the same economic good that is carried over time (we may obtain \textit{it} at t\textsubscript{1} or at t\textsubscript{2}). When asked how we should understand the concept of the same good presupposed by the universal law of time preference, an Austrian economist might reply\footnote{As brilliantly observed by an anonymous reviewer, the following analysis of the law of diminishing marginal utility does not imply that Austrians failed to formulate the law in non-trivial terms. This would indeed be uncharitable. Yet, our point is more modest. We claim that the law under consideration would be necessarily tautological unless we independently elaborate on the notion of homogeneity (same good), which is precisely what is going to be done in the forthcoming parts of the paper.} in a~similar fashion: we can easily learn whether $x_{1}$ (some economic good at t\textsubscript{1}) and $x_{2}$ (some economic good at t\textsubscript{2}) are the units of the same good. We would do so by checking whether an actor would \textit{now necessarily} prefer $x_{1}$ to $x_{2}$.\footnote{In fact, it was Rothbard 
%\label{ref:RNDGHTnn39SAL}(2011, pp.15–16)
\parencite*[][pp.15–16]{rothbard_toward_2011} %
 himself who resorted to this tautologous defense of the universal law of time preference, which is evidenced by the following passage: ``\textit{Time preference} may be called the preference for \textit{present satisfaction} over \textit{future satisfaction} or \textit{present good} over \textit{future good}, provided it is remembered that it is the \textit{same} satisfaction (or ``good'') that is being compared over the periods of time. Thus, a~common type of objection to the assertion of universal time preference is that, in the wintertime, a~man will prefer the delivery of ice the next summer (future) to delivery of ice in the present. This, however, confuses the concept ``good'' with the material properties of a~thing, whereas it actually refers to subjective satisfactions. Since ice-in-the-summer provides different (and greater) satisfaction than ice-in-the-winter, they are \textit{not} the same, but \textit{different} goods. In this case, it is different satisfactions that are being compared, despite the fact that \textit{physical} property of the thing may be the same.'' Whereas in the body of the text we considered the possible Austrian rejoinder in the form of saying that the same good can be conceptualized as the one that obeys the universal law of time preference, Rothbard merely contraposes by saying that if there is an apparent preference for a~future good (ice cream in summer) over a~present good (ice-cream in winter now), these two cannot constitute one and the same economic good. So, not only is the Rothbardian solution clearly circular, but also it gives the impression of fudging the notion of the same good. It may seem that whatever counterexamples to the law of time preference one may possibly come up with, Rothbard would rebut it by claiming that his critic invokes two distinct economic goods. This is yet another indication that an \textit{independent} concept of the same good is logically required.} But these two apodictically true statements come at a~price. For the consequence of the lack of the independent (of the laws in question) notion of the same good, would turn those laws into concealed tautologies. Consider yet again,

\begin{enumerate}[label=\arabic*), ref=\arabic*]
\item (\textit{the law of diminishing marginal utility}\footnote{We take the liberty of providing our own (and not Rothbardian) formulation of the law of diminishing marginal utility only because our version makes the ultimately tautologous character of the reasoning under consideration more conspicuous. Moreover, as conceded in footnote 12 above, the (explicitly) tautologous formulation of the law cannot be attributed to any particular Austrian.}): a~supply of economically same goods constitutes such a~collection of units that each additional unit therein is valued less than the previous unit, and
\item (\textit{the definition of a~supply of economically same goods}): what we here \textit{mean} by a~supply of economically same goods is such a~collection of units that each additional unit is valued less than the previous unit.
\end{enumerate}
Since any good definitions are equivalences and the \textit{definiens} may be substituted for \textit{definiendum} \textit{salva veritate}, let us substitute for ``a~supply of economically same goods'' in 1) our \textit{definiens} in 2). We would end up with

\begin{enumerate}[resume, label=\arabic*), ref=\arabic*]
\item A~collection of units that each successive unit is valued less than a~previous unit constitutes such a~collection of units that each successive unit therein is valued less than the previous unit.
\end{enumerate}
Now, it is clearly visible that 3) is a~tautology in its open form. Incidentally, if we were to understood the same good as the one that obeys the universal law of time preference, then this law in turn would be rendered equally uninformative. A~concealed tautology would turn into a~tautology in its open form via exactly the same reasoning (see: steps 1-3 above). So, the main thrust of Nozick's objection can be interpreted as saying that without an \textit{independent} notion of indifference, the law of diminishing marginal utility\footnote{As demonstrated in passing above, Nozick's objection would apply with the equal force to the universal fact of time preference too.} would be simply trivial. It would not state an interesting (and non-trivial) relation between \textit{two different properties} of units in question; 1) that of \textit{belonging to the class of economically same goods} and 2) that of \textit{being valued less and less on the margin} once the class in question has fewer and fewer members. By contrast, a~tautology would state a~trivial truth: a~property is identical with itself. In our case: the property of belonging to the class of economically same goods is the same as the property of belonging to the class of economically same goods. This, however, is a~far cry, to say the least, from stating a~meaningful economic law.

Second, we must also concede to Nozick that pricing of the commodity also seems to rest on the notion of indifference. Then, if Austrians fail to somehow accommodate indifference into their theory, this would have disastrous consequences for their entire conceptual edifice. After all, it must be borne in mind that the market (equilibrium) price of a~given product is a~function of supply and demand. And the demand curve is but a~reflection of the diminishing marginal utility of a~given product. That is, the demand curve---rather unsurprisingly---slopes downwards because the more we have (of a~given product), the less we value marginal units. And it is precisely why we are ready to buy \textit{more} (of pretty much anything) \textit{only when} the successive units of the product in question cost less and less. Therefore, it is clear to see that the demand curve reflects the logic of diminishing marginal utility. Hence, Nozick is right. If we fail to reconcile the notion of indifference with the law of diminishing marginal utility, then, while having a~distorted notion of the law, our idea of the demand curve would be flawed too. And this in turn would adversely affect the concept of the market prince since the market price is a~function of the demand curve. Yet, we believe that all these problems can be overcome once we make the concept of indifference a~function of the correct description of an action,\footnote{As suggested to me by an insightful anonymous reviewer, the word ``correct'' in the Hoppean
%\label{ref:RNDvdXSfND9UE}(2005)
\parencite*[][]{hoppe_must_2005} %
 phrase ``correct description of an action'' does not refer to any normative standard. Rather, it is ultimately a~matter of fact. Indeed, ``correct'' description of an action captures the \textit{mentalist} (or \textit{internal}) aspect of action; that is, \textit{how} the actor herself conceives of what she is doing. Still in other words, the correct description of action picks up only those elements which were chosen and which were thus strictly preferred to perceived alternatives.} very much in the vein of Hoppe 
%\label{ref:RNDs0HkMV8FOe}(2005).
\parencite*[][]{hoppe_must_2005}. %
 And it is the building upon this author's account (simultaneously defending it against Block's objections) that we shall now turn to.

\section{Why Block's account of indifference is inadequate }
The first Austrian to recognize the force of Nozick's challenge was Walter Block. This is evidenced in the way Block
%\label{ref:RNDyO2aE2QNkN}(Block, 1980, p.423)
\parencite*[][p.423]{block_robert_1980} %
 acknowledges the gravity thereof before he even tackles indifference: ``I consider Nozick's next attempt to show the necessity of indifference as one of the most brilliant and creative criticisms that has ever been levelled against any aspect of Austrian theory.'' But how does Block try to take the sting out of Nozick's objection? Since we are going to suggest a~more satisfactory solution than the ones hitherto ventured within the debate on indifference,\footnote{There were many contributors to the debate on indifference within Austrianism, regardless of whether they \textit{directly} address the Nozick's challenge or not. These include---among others---Block 
%\label{ref:RND3n5dmV3XEv}(2009a; 2009b);
\parencites*[][]{block_rejoinder_2009}[][]{block_rejoinder_2009-1}; %
 Block with Barnett 
%\label{ref:RNDWddJwwPCCa}(2010);
\parencite*[][]{block_rejoinder_2010}; %
 Hudik 
%\label{ref:RNDtovz4gvpAQ}(2011);
\parencite*[][]{hudik_note_2011}; %
 Rothbard 
%\label{ref:RNDOLaOHxWcv2}(Rothbard, 2011).
\parencite[][]{rothbard_toward_2011}. %
 Moreover, less characteristically within Austrianism, there is a~dissenting view to the effect that praxeology should embrace acting on \textit{weak preference} (rather than strict one), and thus possibly also on \textit{indifference}. This view is represented by, e.g., Machaj 
%\label{ref:RNDcRCVqlEmB4}(2007);
\parencite*[][]{machaj_praxeological_2007}; %
 O'Neill 
%\label{ref:RNDS9hQVdLlfc}(2010).
\parencite*[][]{oneill_choice_2010}. %
 The reason the present paper focuses on the discussion between Block and Hoppe is that, first, these two authors are particularly eloquent in presenting their respective (and contrasting) views; and second, they by far contributed most to the entire debate in question.} Block's 
%\label{ref:RNDhxhI3qux8E}(Block, 1980, pp.423–424)
\parencite*[][pp.423–424]{block_robert_1980} %
 wrestling with the challenge merits being quoted in full

\myquote{
Suppose that, for example, a~person has a~stock of some commodity. This means, of course, that he considers each unit equally useful, desirable, serviceable\ldots Let us presume that he has 100 lbs. of butter and now for some reason desires to give up one of these units of butter. And let us say, further, that he arbitrarily picks one such unit, say, the 72\textsuperscript{nd} one. Nozick would say that ‘the person does not prefer giving up this one to giving up another one' [\ldots]. But this interpretation is clearly unsatisfactory. For if the person didn't really prefer to give up this (72\textsuperscript{nd}) one, why did he pick it to be given. So. we are forced to conclude that the butter units were not really interchangeable from the point of view of an actor involved in the selection process. Thus, we seem to be forced to deny that there is ever any such thing as a~commodity, surely a~ludicrous position.
}
And we concur. Surely, it is a~ludicrous position; yet, it is precisely what Block's account is doomed to conclude for it is inherently unable to square the two apparent facts: 1) that those units of butter are \textit{really} (\textit{ex hypothesi}) ``equally useful, desirable, serviceable'' and 2) that 72\textsuperscript{nd} one was ‘picked up'.\footnote{Now, we must make a~slight concession in order to avoid begging any questions at this point. The concept of ‘picking up' does not conveniently play on the equivocation between preference and indifference since it unambiguously suggests the former. Our point, by contrast, is to say that somehow (in a~sense) 72\textsuperscript{nd} unit of butter was ‘picked up' but this notion of picking up is sort of \textit{non-preference implying} (or, positively speaking, indifference-implying). As noted, however, the notion of ‘picking up' (as commonly used) A~over B~implies that we prefer A~to B; after all, we want preferences to guide actual choices. As it will transpire, what captures the above scenario (with pounds of butter) much better is the description that 72\textsuperscript{nd} unit was not chosen (or picked up for that matter) at all. Yet, let us not precipitate things. We will come to this issue once we tackle Hoppe's account.} As we are about to see, the whole problem trades on the concept of ‘picking up'. Block seems to be lured into thinking that the imagined actor \textit{does} pick up the 72\textsuperscript{nd} unit where he says: ``For if the person didn't really prefer to give up this (72\textsuperscript{nd}) one, why did he pick it to be given''. Fair enough, if we \textit{assume} that he did pick up\footnote{Then again, the notion of ‘picking up' invoked here is the normal \textit{preference-implying} one.} this very unit, he must have preferred giving up this one to giving up any other, which simply logically follows from the concept of ‘picking up' employed herein. And yet, why should we beg any questions? It is to be established first that the actor \textit{does indeed} pick up the 72\textsuperscript{nd} unit. For settling this issue has a~bearing on whether he prefers giving this unit to any other or he does not. And this in turn determines whether the actor conceives of the 72\textsuperscript{nd} unit as the unit of the same supply (with all the other units of butter) or he conceives of the stock before him as consisting of two distinct classes: a) a~homogeneous class of 99 units (still intact) and b) a~singleton containing the very pound of butter given up.\footnote{We still hasten to add that ‘given up' here should not imply that the very item was dispreferred. The correct description of the action in case (and that is the point we shall press in the forthcoming part of the paper) all 100 units were perceived as equally serviceable is that the actor could not---logically speaking---choose between them.} Therefore, it seems that something has to give here: either 100 units were not in fact perceived as equally useful or they indeed were but the actor did not (in a~relevant sense) choose to give up the 72\textsuperscript{nd} unit. So, Block cannot have it both ways. But before we embark on further considerations, let us cite the apparent solution Block
%\label{ref:RNDMIRKfOUG10}(1980, p.424)
\parencite*[][p.424]{block_robert_1980} %
 offers:

\myquote{
I~think that this problem can be reconciled as follows. \textit{Before} the question of giving up one of the pounds of butter arose, they were all interchangeable units of the commodity, butter. They were all equally useful and valuable to the actor.

But then he decided to give up one pound. No longer did he hold, or can he be considered to have held, a~homogeneous commodity, consisting of butter pound units. Now there are really two commodities. Butter\textit{\textsubscript{a}}, on the one hand, consisting of 99 one-pound units, each (of the 99) equally valued, each interchangeable from the point of view of the actor with any of the other in the 99-pound set: on the other hand, butter\textit{\textsubscript{b}}, consisting of one pound of butler (the 72\textsuperscript{nd} unit out of the original 100 butter units, the one, as it happens, that he chose to give up when he desired to sell off one of his pounds of butter). In this case butter\textit{\textsubscript{a}} would be preferable to butter\textit{\textsubscript{b}}, as shown by the fact that when push came to shove butter\textit{\textsubscript{b}} was jettisoned and butter\textit{\textsubscript{a}} retained.
}
We will offer two interpretations of the above passage. One perusal will construe of what Block seems to mean literally, whereas the other will attempt to interpret him charitably, thus rendering Block's statement true but irrelevant. So, as hinted at above, Block seems to imply that \textit{the choice} constitutes a~sort of breaking point, after which there are no longer homogenous units but the formerly homogenous collection is now divided into two sets: in one of them we still have homogenous units and the other set is a~singleton, with the element not being homogenous with the remaining elements in the previous set. The problem with this contention is that Block must either invalidate his assumption that they were homogeneous before the choice in order to explain why the choice (i.e. picking up the least preferred unit of butter, as opposed to the remaining ones) took place. Alternatively, if he maintains that the units in question are indeed equally useful, then he cannot explain why this particular unit of butter was picked up because they were assumed to be equally valuable in the first place. Nozick's challenge comes with vengeance to Block and the reason is precisely that the latter author has a~distorted idea of choice.\footnote{This idea of choice is going to be remedied by Hoppe
%\label{ref:RNDysvIpjR6ZD}(2005),
\parencite*[][]{hoppe_must_2005}, %
 as we are about to see.} To appreciate this indictment of ours more clearly, let us press the problem of choice (and \textit{what exactly} is chosen) a~bit harder. What prompts Block to believe that it was the 72\textsuperscript{nd} unit---as opposed to just \textit{a}~unit---that was given up in the above-considered scenario? We would like to venture a~hypothesis that Block (however implicitly) could have believed that whatever made the sentence ``72\textsuperscript{nd} unit was exchanged for money'' true (with the truth-maker in question being the entire action-token in which all the details are provided: there was a~particular unit exchanged for a~particular banknote at a~particular time and space via particular bodily movements etc.) \textit{is the same} as the propositional content of the actor's intention. But this is highly improbable. This would predict that---at least in this case---there was only \textit{one possible} (and extensionally defined) \textit{state of affairs} which would satisfy the actor's intention. If Block were to think so, he would wind up advocating perfect heterogeneity of means, being left with no hope of intelligibly conceiving of \textit{the same commodity}. After all, choices reflect strict preferences and if everything (down to the level of the most minute details) is chosen, then at least for this actor, the supply of the same commodity is an empty category.\footnote{Strictly speaking, the equivalence classes of the same commodity would be always singletons.} However, it seems quite obvious that the actor's desire can be satisfied in an almost infinite number of ways. If we want to buy bread in a~local supermarket, it might be the case that we are indifferent even between supermarkets (because, say, they are equidistant and almost qualitatively identical) or between types of bread etc.---not to mention that it would be absurd to claim that we choose every single detail of our route to a~supermarket.\footnote{The same---rather commonsensical---point was pressed by Davidson 
%\label{ref:RNDGqeTcajk0A}(1963, p.688):
\parencite*[][p.688]{davidson_actions_1963}: %
 ``If I~turned on the light, then I~must have done it at a~precise moment, in a~particular way---every detail is fixed. But it makes no sense to demand that my want be directed at an action performed at any one moment or done in some unique manner. Any one of an indefinitely large number of actions would satisfy the want, and can be considered equally eligible as its object.''} So, there are infinitely many bodily behaviors and infinitely many routes that would do equally well from the actor's perspective. Now, combing these two infinities would yield a~Cartesian product, with every member thereof being equally good for that actor. In other words, any combination of a~particular route and a~particular bodily behavior under consideration would do as well as any other \textit{relative to the satisfaction of his particular intention}. Concluding, it would be a~fatal mistake to confuse \textit{a~particular state of affairs} (as specified in extensional terms) which actually occurred with \textit{a~content of the actor's intention}, with the latter being almost always intensionally specified. Granted, the content of the latter is propositional but the proposition (\textit{that} this or that happens) is normally satisfied by infinitely many particular states of affairs---but not by only one. And because the actor's intention can be satisfied in so many various ways, he \textit{must} be indifferent between some aspects of this multitude of states of affairs. And because he is indifferent between them, he does not choose between them. Having established that this possible retort Block might have availed himself of would not succeed either, let us move to the second perusal of the above-cited fragment from Block.

On the second reading, Block's position may be rendered true but then it would amount to the mere restatement of the law of marginal utility. In other words, what Block might mean is that \textit{after the choice}\footnote{But now, it is rather a~choice between having an exchange (of a~unit of butter for money) or refraining from it.} of a~particular unit of butter out of 100 of them we deal with a~new supply of 99 units thereof. Before any action was taken, the marginal value of each of those units was the least important goal each of them could satisfy. Now, whichever unit was gotten rid of, the marginal value of the remaining units must have increased. Hence, if Block ends up with 99 units of butter, it is a~matter of course that now the value of each of them (that is of a~marginal unit) is higher than what it was when he had 100 of them at his disposal. Yet, as indicated above, this is tantamount to the mere restatement of the law of marginal utility and therefore irrelevant.\footnote{Note that if Block's response is read as the mere restatement of the law of marginal utility, then it might be argued that it is not only irrelevant to the problem at stake but also question-begging: the response is called upon to vindicate the law of diminishing marginal utility against Nozick's challenge, yet it depends for its success on the restatement of the law as a~valid one. Furthermore, as we shall argue towards the end of the paper, the law in question is best understood, when grounded in the Hoppean correct description of an action and supplemented by sound counterfactual reasoning.} Specifically, it yet again fails to explain why the choice---as Block claims---of this particular unit took place. Having said that, it is about time to go on to consider the Hoppean account of choice and indifference.

\section{Hoppe's account as a~remedy for Block's shortcomings }
In this section, we are going to argue for two points: a) not only can a~choice be modelled in such a~way as to logically exclude the possibility of choice under indifference and b) there are additional reasons why we should endorse the Hoppean correct description of an action. We are going to stress on multiple occasions that it is not the case that the Hoppean account is just an \textit{ad hoc} proposal aimed at solving the problem of indifference, for if it were so, it might be claimed that Hoppe does not solve the problem of indifference and apparent choice among the units of the same supply but simply assumes it away: after all, Hoppe suggests understanding choice as such that it necessarily excludes indifference.

Let us now try to determine whether the Hoppean
%\label{ref:RNDtGo60RAdi2}(2005)
\parencite*[][]{hoppe_must_2005} %
 account fares any better when confronted with Nozick's challenge. First and foremost, it must be noted that---unlike Block's solution---it involves both doing justice to indifference\footnote{It should be constantly borne in mind that, after all, according to the majority of Austrians, indifference is not a~praxeological concept (see: footnote 17). This is due to the fact that indifference cannot be demonstrated in action, as is usually reiterated by Austrians 
%\label{ref:RND2SChO9KDOw}(see Block, 2009a; Block and Barnett, 2010; Rothbard, 2011).
\parencites[see][]{block_rejoinder_2009}[][]{block_rejoinder_2010}[][]{rothbard_toward_2011}. %
 By no means can we deduce from any actual choice whether we were confronted with the units of the same good or with the ones of distinct goods. } (at least admitting that a~man can be genuinely indifferent between some options) and barring it steadfastly from the realm of choice. Briefly speaking, Hoppe 
%\label{ref:RNDqcsL91XFnk}(2005)
\parencite*[][]{hoppe_must_2005} %
 maintains that one \textit{cannot} make a~choice under indifference. This ``cannot'' is definitely of logical nature and so, the truth of the proposition that a~man cannot choose when indifferent derives its truth solely from its constituent concepts. Specifically, Hoppe \textit{defines} choice in such a~way that it entails the lack of indifference. That is, if man chooses $x$ over $y$, he is not (and, logically speaking, \textit{cannot}) indifferent between the two. And conversely, he defines indifference in such a~way that the very fact that the actor is indifferent between $x$ and $y$ implies that he does not (and \textit{cannot}) choose between the two.\footnote{Then again, if this were all there is to the Hoppean account, it would hardly count as a~solution to Nozick's challenge. By contrast, Hoppe does indeed appeal to Searle's 
%\label{ref:RNDlGOI33r05g}(1984)
\parencite*[][]{searle_minds_1984} %
 distinction between internal-mentalist and external-behaviourist aspect of one's action, which definitely provides an independent reason counting in favour of the former's solution to the problem of indifference (vis-à-vis choice) in Austrian economics. In the forthcoming part of this section, we are going to build upon and thus sharpen Hoppe's (Searle's) insight by demonstrating that the there is a~deep distinction between \textit{a~description of one's action under such an aspect that makes it intentional} and \textit{the description of what one did} (whether intentionally or not). As we shall argue, the former captures not only the idea of what is chosen but also accounts for which maxim one acts on, thereby making it very useful in the assessment of the moral worth of one's actions. Briefly speaking, our agenda henceforth is to show that the Hoppean solution involving the correct description of an action is a~powerful explanatory device shedding light on many aspects of human life, while not being a~mere stipulative \textit{ad hoc} move (the definitional exclusion of indifference from the realm of choice) aimed at saving Austrian economics from Nozick's challenge.}

Based on the original Hoppean account just adduced, the following two relations must hold:

\begin{enumerate}
\item an actual choice between units $\rightarrow$ strict preference for one of the units;
\item indifference between units $\leftrightarrow$ no possible choice between the units.
\end{enumerate}
At this point, we would do best to obviate one possible objection that might be raised against Hoppe. Note that it might be claimed that it can surely be the case that one cannot choose $A$ over $B$ \textit{even if} one is not indifferent between the two and the reason might be that $A$ is unavailable. Then it would look as though it is only indifference that entails the impossibility of choice, whereas the impossibility of choice would fail to entail indifference. However, this merely psychological (in the absence of action) fact that an actor prefers $A$ over $B$ would be of no interest to Austrian economics with its commitment to the doctrine of demonstrated preference
%\label{ref:RNDZeuhzEUrWg}(see \textit{inter alia} Rothbard, 2011).
\parencite[see \textit{inter alia}][]{rothbard_toward_2011}. %
 Simply stated, the actor's preferences that cannot be demonstrated in action are not part and parcel of this school of thought. Therefore, whenever we speak of the possibility of choosing $A$ over $B$, this presupposes that both $A$ and $B$ are available. And that is why the only reason why an actor cannot choose (given our presupposition) between $A$ and $B$ is that he is indifferent between them. And this is why it is the relation of equivalence that holds between indifference between some units and an inability to choose between them. Having preempted this possible rejoinder, let us cite some textual support confirming that Hoppe 
%\label{ref:RNDXeoNeneUn0}(2005, p.91)
\parencite*[][p.91]{hoppe_must_2005} %
 indeed perceived the relation between choice and difference in the way reconstructed above:

\myquote{
Likewise, a~mother who sees her equally loved sons Peter and Paul drown and who can only rescue one does not demonstrate that she loves Peter more than Paul if she rescues the former. Instead, she demonstrates that she prefers~\textit{a}~(one) rescued child to none. On the other hand, if the correct (preferred) description is that she rescued Peter, then she was not indifferent as regards her sons.
}
Clearly then, if the mother chose (the preferred description) to save Peter, she thus demonstrated the strict preference for him over Paul, which exemplifies relation 1) cited above; whereas the relation 2) is most tellingly (however indirectly) elucidated with the proverbial Buridan's ass wavering over two identical bales of hay
%\label{ref:RND3KkT0Yhgia}(Hoppe, 2005, p.91):
\parencite[][p.91]{hoppe_must_2005}:%


\myquote{
Lastly, consider Buridan's ass standing between two identical and equidistant bales of hay. The ass is not indifferent and yet chooses one over the other, as Nozick would have it. Rather, it prefers~\textit{a}~bale of hay (whether it is the left or the right one is simply not part of the preferred choice description), and thus demonstrates its general preference of hay to death.
}
The relation 2) can easily be inferred therefrom: the ass being indifferent between these two bales of hay, did not \textit{choose} between them; rather, he chose \textit{a} hay over death, which, eventually, implies its preference for the former over the latter. Having said that, it is high time to ask what are the merits (or demerits?) of the Hoppean account? And in particular: why is the Hoppean account superior to Block's and how does former address Nozick's challenge? To test Hoppe's position, let us apply it to the scenario of giving up a~pound of butter cited above.

There are two logical possibilities here. If the actor views all 100 units of butter as genuinely equally serviceable, then all of them fall into the rubric of the same economic good. Then \textit{any} correct description of his action would not involve any choice \textit{between these units}. It is certainly the case that it is strict preference that guides the actor's choice; yet, this choice is not between the units assumed to be equally serviceable.\footnote{Remember, all these units would then fall into the same equivalence class (with indifference between the equivalence relation dividing all economic means into mutually disjoint classes) \textit{within which} an actor does not (and cannot) choose.} It is this very point that Block does not concede, thereby running into all the above-mentioned conceptual problems. So, positively speaking, how to account for the transaction that occurred? The solution seems fairly straightforward: since the actor \textit{did} indeed give up the 72\textsuperscript{nd} pound of butter (while holding all of them equally serviceable), he must have preferred giving up \textit{a}~unit of butter for some pecuniary equivalent. In other words, the actor preferred one unit of butter less, but some increment of money to retaining his entire stock of butter but depriving himself of an opportunity to earn this money. The second possibility is that the correct description of an action is that the actor really dispreferred that 72\textsuperscript{nd} unit that he actually gave up. If so, that unit was not the same economic good as all the other units in the first place and therefore, trivially, the original supply of 100 units was heterogeneous. At the very least, there were at least two classes of economic goods involved as the unit actually given up was \textit{ex hypothesi} (due to the correct description of the action) valued less than any other.

For the time being, let us return to the celebrated Hoppean thought experiment with mother saving either Peter and Paul from drowning and let us suppose that Block could still argue that the mother could not be indifferent between Peter and Paul under any circumstances once she saved Peter. The bone of contention then would be \textit{the act of saving Peter} and whether the fact that the mother (at least according to one description of her action) did save Peter in turn implies that the mother did indeed choose to save Peter. Let us analyze more closely this tack that Block might try. Block's point against Hoppe would be decisive if the act of \textit{saving a~particular child} (under this description) instead of another were inherently \textit{preference-implying}. That is, Block would succeed if we can infer from \textit{the fact} of saving a~particular child (or from bringing about the event of a~particular child being saved) that this particular child was preferred to the other. Yet, there is a~deep distinction favored by Davidson
%\label{ref:RNDQz9TpqEE9X}(Davidson, 2001)
\parencite*[][]{davidson_agency_2001} %
 between \textit{what an actor does} (including his primitive action consisting in his bodily movements up to everything they cause) and what he does \textit{intentionally}. As Davidson 
%\label{ref:RNDBeItpdWezQ}(2001, p.45)
\parencite*[][p.45]{davidson_agency_2001} %
 put it: ``[…] although intentionality implies agency, the converse does not hold.'' Therefore, it would simply beg the question to say that by the act of saving Peter the mother demonstrated her preference for Peter over Paul. As established above by alluding to the Davidsonian insight, from the event that the mother authored, we cannot infer \textit{which aspects thereof were informed by her preference}. Therefore, not to beg any questions, we should treat the act of saving Peter in \textit{the non-choice}---(and hence also non-preference)---\textit{implying} sense. Alternatively, just to remain neutral on whether the mother did actually choose to save Peter or chose to save \textit{a} child, we could say that what the mother \textit{in fact did} was to save Peter. After all, to say that the mother saved Pater is only \textit{to attribute her agency} to this event (in other words, it is to say that she \textit{authored} the event of Peter having been saved), which does not imply that she saved Peter \textit{intentionally.} And this is the key insight which, in our view, counts in favor of the Hoppean account. Just to reiterate, there is a~distinction to be drawn between the authored event (Peter being saved) and this description of the mother's action that makes it intentional (e.g. saving \textit{a}~child). It is only the latter description that accounts for what the mother intended to do, and hence chose. Therefore, we can easily conclude that it is only \textit{some aspects} of the authored event that an actor intentionally brings about or chooses to bring about. For example, assuming that the latter description is a~correct one, the mother was not choosing between her children, although it is true what she \textit{in fact did} was to save Peter. Finally, authored events are defined in extensional terms (with all minute details being fixed), whereas the actor's intentions (strictly speaking, their propositional content) is envisaged in intensional terms. And it is what Hoppe 
%\label{ref:RND2rGixACIkS}(2005)
\parencite*[][]{hoppe_must_2005} %
 hints at throughout his paper: it is the idea that what the actor genuinely \textit{chooses} is reflected in his (from his privileged first-person point of view) preferred description of the action.

To further reinforce the Hoppean point of the correct description of an action, we can also resort to Parfit's
%\label{ref:RNDnoqSHnRa1e}(2011, p.289)
\parencite*[][p.289]{parfit_what_2011} %
 incisive remarks (though literally located within the context of Kant's philosophy) related to the issues of adequately describing on what maxims people actually act:

\myquote{
Whether some act is wrong, Kant's formulas assume, depends on the agent's maxim. Of the maxims that Kant discusses, most involve some policy, which could be acted on in several cases. Two maxims may be different, though they involve the same policy, because they involve different underlying motives or aims. Two merchants, for example, may both act on the policy ‘Never cheat my customers'. But these merchants act on different maxims if one of them never cheats his customers because he believes this to be his duty, while the other's motive is to preserve his reputation and his profits.
}
This quote could aptly illustrate our (and Hoppean) intuition that two identical behaviors could then translate into two distinct actions, depending on the way we frame our goals. Or to use Parfit's language, the actor's observed particular behavior cannot unambiguously point to a~maxim he is acting upon for the former may be compatible with practically infinitely many varieties of the latter. After all, the relation between a~maxim and behavior is many-to-many. A~given maxim an actor is acting upon may be instantiated in infinitely many behaviors and vice versa: as we say, a~given behavior may translate into many maxims. And now, the way of getting to a~correct description of an action was brilliantly illuminated by Parfit
%\label{ref:RNDA4keeRzY8A}(2011, pp.289–290).
\parencite*[][pp.289–290]{parfit_what_2011}. %
 The author considered acting on the following highly specific maxim: stealing some wallet from some woman dressed in white who is eating strawberries while reading the last page of Spinoza's \textit{Ethics}. Ethical objections connected to acting on such rare maxims aside, the author suggested the following to determine which maxim is actually guiding our actor:

\myquote{
This objection can be partly answered. Just as it is a~factual question what someone believes, or wants, or intends, it is a~factual question on which maxim someone is acting. And real people seldom act on such highly specific maxims. When we describe someone's maxim, as O'Neill and others claim, we should not include any details whose absence would have made no difference to this person's decision to do whatever he is doing. In a~realistic version of my example, I~would have stolen from my victim even if she had been dressed in red, or had been eating blueberries, or had been reading the first page of \textit{Right Ho Jeeves}! My real maxim would be something like ‘Steal when that would benefit me.'
}
So now, do not the above considerations perfectly correspond with the Hoppean distinctions between choice, indifference and the correct description of an action? To put it more specifically, it should by now seem obvious that physical objects $A$ and $B$ cannot constitute two distinct economic goods when they do not figure in the correct description of an action. In other words, whether $A$ or $B$ is employed \textit{cannot make a~difference} to the actual maxim we are acting on. If our maxim (preferred description of an action) is to save \textit{a}~child, it simply follows that any child would do equally well. The mother cannot be rendered worse off when Peter (or Paul for that matter) is saved simply because both of these scenarios count as the satisfaction of the very same policy of ours. And that is the reason these two (only seemingly distinct) goods are actually the same economic good and it is precisely for the very same reason that we do not choose between them.\footnote{First, it appeared as if indifference between $A$ and $B$ analytically entailed the impossibility of choosing between $A$ and $B$ (what we christened Hoppe's stipulative move). Now it seems we found another reason why indifference between two units and the impossibility of choosing between them must go hand in hand.}

Finally, let us note that Nozick's challenge leaves the Hoppean position unscathed. Nozick's point is simply irrelevant once we subscribe to the Hoppean account of choice. To conceptualize a~supply, Austrians have to employ the notion of indifference---fair enough. Yet, whenever any two units are the units of the same commodity, they shall never figure in a~description of one and the same action. In other words, once any two items represent the same economic good, there is no choice between them. Therefore, a~choice under indifference---an anathema to Austrians---is rendered impossible now. We might also put the above point in the jargon of philosophers of actions, when two---economically identical---goods are at stake, our goal (maxim) is satisfied to the same degree regardless of whether one good or the other is employed. Since the correct description of an action might be mute on the employment of a~\textit{particular} good (as opposed to the use of a~\textit{type} of good), it follows that two \textit{numerically} distinct physical items being equally serviceable in the performance of an action in question must count as the same economic good simply because \textit{the satisfaction conditions} of our actions\footnote{These, of course, follow from the correct description of an action.} do not discriminate between these two units.

\section{Extending the Hoppean framework: stating the law of diminishing marginal utility }
Before we sharpen the formulation of the law of diminishing marginal utility, we need to take heed of one conceptual trap we might fall into. As we were pointing out throughout the paper, the meaningful (non-trivial) formulation of this law depends on the independent notion of the same economic good. Additionally, we posit that a~given stock of units may be considered by an economic actor as a~supply of \textit{the same commodity} only relative to a~given moment. Strictly speaking, it is a~matter of course that human action is sequential (in a~temporal sense) by nature; yet, an actor at t\textsubscript{1} may envisage the way he is going to employ consecutive units at later times. This \textit{double time indexation}---one standing for a~given moment in which an actor envisages the employment of his successive means and the other standing for the \textit{actual} time at which they are employed---is necessary. For suppose \textit{arguendo} that our only time indexation is the time of the \textit{actual} employment of the means for the satisfaction of our goals.Then, we submit, the hope of formulating the desired law would be forlorn. In fact, if we apply the said single index, we would observe that the marginal utility increases once we deal with fewer and fewer units. Certainly, it is impossible to still speak of the same commodity when the marginal utility varies. So, generally speaking, if we have n~units of apparently the same commodity, and once we employ the n\textsuperscript{th} one, we end up with the supply of n-1 units. The marginal utility of the latter supply is higher than in the original one. However, even the above statement is one not entirely correct. For, remember, to state that the marginal utility diminishes once the supply gets smaller and smaller, it must be \textit{the supply of the same economic good}. As we can see, \textit{the single indexation} would not enable us to formulate the law of diminishing marginal utility. Rather, it would depend on the very law we are trying to formulate. Note, our aim still is to develop a~robust notion of a~supply of the same commodity. Only then can we show that marginal utility would indeed increase once the said supply shrinks.

So, just to introduce our allegedly necessary double indexation, let us put forward the following notation. As promised, each unit is to be indexed for time twice in the following manner:

\begin{enumerate}
\item It is going to be indexed \textit{for the time of its actual employment}, with the time being indicated in the subscript. So, u2\textsubscript{3} is to be read as \textit{the second unit} employed at t\textsubscript{3} (time 3).
\item Additionally, it is going to be indexed \textit{for the moment in which an actor imagines its future employment}, with this moment being indicated in the superscript. So, adding to our previous example,
%u2\textsubscript{3}\textsuperscript{1}
u2$_{3}^{1}$ is to be read as how an actor imagines at t\textsubscript{1} how the second unit is to be employed at t\textsubscript{3}. Note, we allow the time variables in both indices to range from the present (t\textsubscript{1}) onwards up to the conceivable future. Yet, the time of envisaging the employment of the units must be earlier than the actual employment of the units. In other words, the natural number in the superscript must be lesser than the number in the subscript. After all, intuitively speaking, once a~means was utilized, there is nothing to economize any longer.
\end{enumerate}
So, armed with the above formal notation and having in mind the condition that given units can be viewed as constitutive of a~supply of the same commodity only \textit{relative to a~given moment}, we can now state what it is for a~given set of units to be perceived as economically identical. What would, for example, make u1 and u2 units of the same commodity, as viewed now (at t\textsubscript{1}) by an economic actor? Formally speaking, it would mean that for any t~(in the subscript, which is the time of the actual employments of these units), the actor is indifferent (\textit{now}) between 
%u1\textsubscript{t}\textsuperscript{1}
u1$_{\text{\,t}}^{1}$
and
%u2\textsubscript{t}\textsuperscript{1}
u2$_{\text{\,t}}^{1}$.
To put it verbally, at least as of now, the actor believes that he can swap these units in any time in the future without any loss of utility (or satisfaction for that matter). Still in other words, he \textit{now} believes that it is a~question of indifference whether he employs u1 at any time instead of u2 at that time. Note that we can easily understand that a~unit can preserve its \textit{economic} identity over time, which, incidentally, does not run counter to the universal fact of time preference. After all, we assume as a~correct description of our consecutive actions that a~given unit (say, u1) over certain time is equally serviceable as any other unit in our set. If an actor believes that u1 can be put to use at t\textsubscript{1} as well as at, say, t\textsubscript{8}, then there is no preference for the employment of this unit now to its employment later. By no means does that threaten the universal law of time preference. Quite the contrary, when we genuinely find (now) some set of units equally serviceable across a~given range of time, then, logically speaking, these units are viewed as economically identical \textit{across that time.} In other words, for any unit in that set, there is no preference for its use at any particular time over any other. When it comes to the satisfaction of ends, the situation is diametrically different. We do satisfy our ends in a~descending order of their importance over time. Yet, our means are \textit{believed} (correct description of an action) to be equally serviceable over that very time. By assumption then, any of the said units can be equally well employed at any time.

Let us represent our rather intuitive findings more rigorously and generally. Let S~be a~set of n~number of units, which are believed to be equally serviceable. Let e~be a~number of ends each of the units is \textit{believed} to be able to satisfy equally well. Let also n~${\leq}$ e. The last requirement is important for if n~were greater than e, then some of the units in S~would not count as economic goods (for a~proper subset of S~would already satisfy all the ends the means are supposed to be able to satisfy). Now as long as we consecutively allocate any of these units to less and less important ends (starting from the most important one), then there are e! number of scenarios an actor would be indifferent to.\footnote{The number of ends unsatisfied will be e-n. These will be the ends figuring at the bottom of the actor's value scale. Furthermore, as we can see, the Hoppean account can be given a~temporal dimension. Now we can say not only that saving Peter is as good as saving Paul \textit{now} but also that some (temporal) sequences of actions are considered as good as some other. In our case discussed above, there are e! of such equally good sequences of actions from the perspective of some economic actor.} Remember, the indifference relates to the means consecutively employed, but not the ends. The latter are obviously satisfied in the descending order of importance.

To conclude, let us show that the law of diminishing marginal utility firmly rests on correct description of (sequential) actions and does not depend on any \textit{actual employment} of the units of the same commodity in question. Let us consider a~set of units at time t\textsubscript{1}. Suppose an actor has at his disposal three eggs, which he finds equally serviceable. For the sake of simplicity, let us assume each of these eggs can equally well satisfy three needs (in the descending order of importance):

\begin{enumerate}
\item Throwing one at one's enemy window;
\item Eating one hard-boiled;
\item Eating one soft-boiled.\footnote{Also, for the sake of simplicity, let us assume that our marginal unit here is just one egg; that is, there no ends that are to be satisfied with either two or three eggs (put together) from the perspective of this actor. }
\end{enumerate}
As established above, if our three eggs are \textit{believed} to be able to equally satisfy these three needs, we would end up with 3! (which is 9) possible scenarios of satisfying these ends with our three economic goods among which our actor would be indifferent. The value of the marginal unit now is the third end since it is this end that one would not satisfy if one were to give up or lose one of his eggs. Now, we claim that the law of diminishing marginal utility (in a~truly Austrian spirit) does not depend on the actual employment of our eggs. Rather, the law should be conceived of \textit{counterfactually}. That is, holding an actor's correct description of his ends and his relative value rankings fixed, we should imagine how the same actor would value a~marginal unit of his shrunk supply. To illustrate, suppose an actor lost his third egg and is now (contrary to fact) left with only two of them. Then, the value of his marginal unit would be the second end (eating it hard-boiled) for if he were to lose either of the two remaining eggs, the need that would be then left unsatisfied would be eating an egg hard-boiled. So, we posit, while building upon the Hoppean correct description of an action, that the law of diminishing marginal utility can be derived solely from the Hoppean account coupled with purely counterfactual reasoning (by keeping the ends as envisaged at t\textsubscript{1} as well the relative ranking thereof equal). To summarize, it seems that the original Nozickian challenge can be adequately replied by the Hoppean account. What is more, the latter accommodates indifference and keeps it steadfastly from the realm of choice---very much in line with the demands of praxeology itself. Finally, after developing the notion of the same economic good, the sharpened Hoppean theory enabled us to clearly formulate the law of diminishing marginal utility.

\section{Conclusion}
The ultimate aim of this paper was to reply Nozick's challenge. In the meantime, we spelled out the implications of Nozick's criticism, which led us to the conclusion that the independent notion of the same economic good is very much needed. Then, on our way to sharpening the Hoppean account, we defended Hoppe vis-à-vis Block's criticism. We concluded that Block's position inherently fails to capture the notion of the same commodity, while Hoppe's fares very well in this respect.

Eventually, we developed a~formal notation to elucidate the notion of economic sameness, having built up on the Hoppean correct description of an action. Sticking to the Hoppean insight that there is no choice within the class of economically identical goods, we identified the number of possible scenarios (of sequentially employing the means to less and less important ends) among which an actor must be indifferent once he conceives of the units he is about to economize as equally serviceable. We concluded by claiming that the law of diminishing marginal utility can be derived solely from the Hoppean account, aided by counterfactual reasoning.

\paragraph{Acknowledgments}
The author wishes to thank two anonymous helpful referees whose insightful comments helped improve the quality of the present paper. Most certainly, if there are still some errors remaining, they are my own responsibility.



\end{artengenv}