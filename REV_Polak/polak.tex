\begin{newrevplenv}{Paweł Polak}
	{Dlaczego metoda krytyczna Poppera jest fascynująca?}
	{Dlaczego metoda krytyczna Poppera jest fascynująca?}
	{Dlaczego metoda krytyczna Poppera jest fascynująca?}
	{Uniwersytet Papieski Jana Pawła II w Krakowie}
	{Zbigniew Liana, \textit{Filozoficzne korzenie metody krytycznej K.R. Poppera: Metoda krytyczna we wczesnym ujęciu K.R. Poppera wobec metafilozoficznej tradycji neokantowskiej szkoły J.F. Friesa}, Copernicus Center Press, Kraków 2021, ss.~168.}





\lettrine[loversize=0.13,lines=2,lraise=-0.03,nindent=0em,findent=0.2pt]%
{N}{}akładem wydawnictwa Copernicus Center Press ukazała się długo oczekiwana książka poświęcona analizie metody filozofowania Poppera. Autorem jej jest krakowski filozof nauki Zbigniew Liana, wykładowca Wydziału Filozoficznego PAT i~UPJPII w~Krakowie, wychowanek Józefa Życińskiego, znany czytelnikom ZFN choćby z~ostatnich cennych i~wnikliwych analiz poglądów swego mistrza
%\label{ref:RND3bYSUddjsO}(Liana, 2019, 2020).
\parencites[][]{liana_nauka_2019}[][]{liana_jozefa_2020}. %
 Tym razem powraca on do fascynacji myślą Poppera, a~dokładnie wczesnym jej okresem, który znalazł swój wyraz w~Popperowskim manuskrypcie \textit{Die beiden Grundprobleme der Erkenntnistheorie} powstałym w~latach 1930--1933.

\textit{Filozoficzne korzenie metody krytycznej}\ldots jest starannie opracowaną analizą tytułowego zagadnienia. Już od pierwszych stron zwraca uwagę erudycją, skrupulatnością i~precyzją. Niemalże każde zdanie nasycone jest treścią i~osadzone w~szerokim kontekście dotychczasowych badań myśli Poppera oraz Firesa i~jego szkoły. Na pozór niewielka objętość książki niech więc nie będzie zwodnicza. W~naszych czasach, gdy łatwość pisania i~publikowania tekstów prowadzi do nadmiernego wydłużania się prac obciążonych redundancjami i~wypełnionych często miałkimi stwierdzeniami, warto sięgnąć po pracę pisaną klasycznym, skondensowanym stylem. Styl ten pod wieloma względami przypomina żywo sposób pisania wybitnych współczesnych filozofów anglojęzycznych, w~zasadzie może więc dziwić fakt, że praca opublikowana została w~języku polskim. Tym bardziej, że ze względu na znaczenie prowadzonych w~niej analiz, powinna zostać udostępniona czytelnikom spoza naszego kraju. Postaram się tutaj uzasadnić tę tezę.

Książka składa się z~czterech rozdziałów, zakończenia oraz, \textit{last not but not least}, obszernej bibliografii; książka posiada również indeks nazwisk. W~pierwszym rozdziale została przedstawiona sytuacja problemowa związana z~interpretacjami poglądów Poppera, zwłaszcza we wczesnym okresie jego twórczości. Wprawnie ukazany został również spór o~Poppera i~zarysowane zostały najważniejsze zagadnienia, które posłużą dalszym analizom, jak np. rozumienie krytycyzmu, kantyzmu i~neokantyzmu w~kontekście filozofii Poppera, krytyka neopozytywizujących interpretacji myśli Poppera, ukazanie tła historycznego wczesnych publikacji wiedeńskiego filozofa. W~tej części nakreślony został też cel pracy, którym jest ,,całościowe i~systematyczne ujęcie koncepcji metody krytycznej Kanta, Friesa i~Nelsona, ukazujące relacje logiczne, jakie zachodzą pomiędzy jej elementami [... oraz zamierza] ukazać relacje logiczne pomiędzy metodą krytyczną wczesnego Poppera a~metodą krytyczną Kanta w~ujęciu szkoły Friesa w~perspektywie systematycznego ujęcia metody krytycznej jako takiej''
%\label{ref:RNDsTNhlwIys5}(Liana, 2021, s.~35).
\parencite[][s.~35]{liana_filozoficzne_2021}. %
 Zatem to właśnie analiza ewolucji koncepcji metody krytycznej stanowi oś niniejszej pracy. Autora interesują różnorodne aspekty metody krytycznej, choć najważniejsze są właściwie przemiany w~rozumieniu tej metody, a~zwłaszcza twórcze przemiany dokonane przez Poppera.

W~ujęciu Liany Popper jest wybitnym kontynuatorem neokantowskiej tradycji szkoły friesowskiej, choć rozwija ją w~tak oryginalny sposób, że powstaje w~zasadzie nowa jakość. Tę istotną zmianę Autor określa metaforą przejścia ,,nieliniowego'', zaczerpniętą z~teorii układów dynamicznych, o~tyle szczęśliwą, że w~układach dynamicznych tego typu przejścia odpowiadają za powstawanie nowych jakościowo stanów, mamy więc do czynienia z~analogią. Pytanie, czy Autor nie rozszerza tu w~pewnej mierze pomysłu Michała Hellera postulującego, że rozwój nauki jest kierowany swoistym systemem dynamicznym
%\label{ref:RNDeTDYZ8VLIQ}(zob. P. Polak, 2008, 2004)
\parencites[zob.][]{polak_nieprzewidywalnosc_2008}[][]{polak_dynamika_2004} %
 -- tutaj idea byłaby rozszerzona na wyjaśnienia historii rozwoju myśli filozoficznej. Jeśli taka była intencja Autora, warto by rozwinął tę intrygującą propozycję historiograficzną w~osobnym opracowaniu.

Wracając do struktury dalszych części pracy to jest ona czytelnie podzielona na trzy kolejne rozdziały podejmujące kolejno zagadnienie metody krytycznej Kanta w~ujęciu J.F. Friesa, następnie w~ujęciu B.~Nelsona, a~w końcu w~ujęciu młodego Poppera w~\textit{Die beiden Grundprobleme der Erkenntnistheorie}. Trudno szczegółowo przekazać tutaj bogactwo różnych wątków podejmowanych w~analizie, zainteresowanych zachęcam po prostu do lektury opracowania. Dość powiedzieć, że czytelnik otrzymuje cenne studium porównawcze myśli Poppera i~Friesa oraz Poppera i~Nelsona. Chciałbym tylko zaznaczyć, że książka daje bardzo głęboki wgląd w~metodę filozoficzną Poppera, mamy więc doskonały klucz do filozofii Poppera (wczesnego okresu). Dla osób, które zajmowały się poglądami wiedeńskiego filozofa, znajdzie się tam z~pewnością wiele oryginalnych analiz i~spostrzeżeń oraz cennych odniesień do prac źródłowych. Innymi słowy, książka ta pomaga w~znaczącym pogłębieniu rozumienia wczesnego Poppera, ale i~neokantyzmu szkoły friesowskiej. Trzeba jednak lojalnie uprzedzić, że wymaga ona biegłego poruszania się w~literaturze popperowskiej i~o neokantyzmie, aby być w~stanie podążać za Autorem. Z~pewnością nie jest to książka dla początkujących filozofów nauki. Specjaliści będą zapewne również zainteresowani analizami realizmu metodologicznego Poppera -- w~niniejszej książce zarysowanymi w~paragrafie~4.6., który wyjaśnia budzący nieraz kontrowersje realizm Poppera. Miejmy nadzieję, że doczekamy się odrębnego, równie wnikliwego całościowego studium realizmu Poppera. Książka pełna jest interesujących i~cennych wątków pobocznych, dla przykładu wspomnę tu tylko o~kwestii metody nauczania filozofii w~szkole Friesa
%\label{ref:RNDMUGYbRYwpm}(Liana, 2021, s.~87),
\parencite[][s.~87]{liana_filozoficzne_2021}, %
 którą miała być metoda Sokratejska.

Wspominałem już, że praca opracowana została niezwykle starannie. Udało mi się znaleźć zaledwie jedną literówkę na s.~145 w~niemieckim tytule pracy. W~zasadzie więc jest to książka praktycznie bez błędów, a~cytowane są w~niej prace w~5 językach. Staranność opracowania przejawia się również w~bardzo precyzyjnym i~celowym doborze pojęć stosowanych w~analizach. Dzięki tej książce możemy zwrócić uwagę np. na interesujące zmiany pojęciowe zachodzące w~tradycji Friesowskiej. Duża zręczność w~ukazywaniu tych przemian pozwoliła zachować jednocześnie spójność prezentacji głównych zagadnień. Sądzę więc, że osoby, które czytały już wiele o~neokantyzmie i~o Popperze, znajdą wiele interesujących uwag dotyczących aparatu pojęciowego stosowanego przez tych filozofów. W~naturalny sposób kwestia ta rzutuje też na autorskie przekłady niemieckich filozofów dokonywane przez Z. Lianę. Tłumaczenia są więc bardzo dobre i~językowo, i~filozoficznie, a~komentarze w~nawiasach ukazują bogactwo oryginalnych pojęć, jednocześnie kierując uwagę czytelnika ku interpretacjom prowadzonym w~książce.

Książka Z. Liany, choć jest niezwykle klarowna, precyzyjna i~doskonale logicznie ułożona, nie należy z~pewnością jednak do prac łatwych. Wymaga dużej wiedzy zarówno o~filozofii Poppera, o~kantyzmie i~neokantyzmie. Autor zakłada niejawnie, że czytelnik orientuje się doskonale w~zagadnieniach filozofii nauki, a~\textit{Logik der Forschung} ma wciąż w~świeżej pamięci. Obserwacje te tłumaczą, dlaczego książka jest zaskakująco mała -- ogromna jest bowiem wiedza tła, którą trzeba się posługiwać, aby ją w~pełni zrozumieć. Zapewne już studenci filozofii nauki odniosą z~niej duży pożytek, o~ile zdobędą się na poważny wysiłek samodzielnego poszerzania wiedzy. Sądzę jednak, że jest to prostu książka kierowana do specjalistów. I~jako taka spełnia z~nawiązką swe zadanie. I~znawcy filozofii Poppera mogą być wdzięczni, że nie muszą się zmagać z~mało interesującymi wprowadzeniami, a~mogą się skupić na tym, co rzeczywiście może posunąć do przodu rozumienie myśli Poppera.

Na zakończenie chciałbym raz jeszcze życzyć Autorowi, aby opublikował wyniki swych badań również w~języku angielskim, najlepiej w~czasopismach o~międzynarodowym zasięgu. Cenny wkład w~rozwój rozumienia metody Poppera będzie interesujący dla filozofów nauki, a~szczególnie powinien zainteresować współczesnych kontynuatorów tradycji szkoły Friesa. Wszak niezbyt często możemy się przekonać, jak fascynująca może być metoda krytyczna sama w~sobie.




%-------------------------


\selectlanguage{english}
\vspace{5mm}%
\begin{flushright}
{\chaptitleeng\color{black!50}{Why is Popper's critical method fascinating?}}
\end{flushright}

%\vspace{10mm}%
{\subsubsectit{\hfill Abstract}}\\
{This review article presents an important, newly published study of Popper's critical method by Zbigniew Liana. The review emphasizes the very high level of the study, points to its originality, and explains why the book is recommended mainly to specialists of Popper's thought. It is also explained how the book manages to contain so many original and valuable analyses in a small volume.
}\par%
\vspace{2mm}%
{\subsubsectit{\hfill Keywords}}\\%
{Fries's trilemma, critical philosophy, critical method, transcendental method, philosophy of science, metaphilosophy, Karl Popper, Leonard Nelson, Jakob Friedrich Fries, Immanuel Kant.}%

\selectlanguage{polish}

\end{newrevplenv}